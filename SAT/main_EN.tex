\section{\ac{SAT}-solvers}

SMT vs. SAT is like high level \ac{PL} vs. assembly language.
The latter can be much more efficient, but it's hard to program in it.

\subsection{CNF form}

\ac{CNF}\footnote{\url{https://en.wikipedia.org/wiki/Conjunctive_normal_form}} is a \textit{normal form}.

% TODO recheck
% TODO write abt it!
%\textit{normal form} is somewhat similar to polynomials in algebra. 
%What is polynomial?
%It is a standard way to express unsystematic equations like $2x \cdot x$ as $3x$ polynomial, 
%and so you will be able to apply some operations to polynomials like summing, etc.

Any boolean expression can be converted to \textit{normal form} and \ac{CNF} is one of them.
The \ac{CNF} expression is a bunch of clauses (sub-expressions) constisting of terms (variables), ORs and NOTs, 
all of which are then glueled together with AND into a full expression.
There is a way to memorize it: \ac{CNF} is ``AND of ORs'' (or ``product of sums'') and \ac{DNF} is ``OR of ANDs'' (or ``sum of products'').

Example is: $(\neg A \vee B) \wedge (C \vee \neg D)$.

$\vee$ stands for OR (logical disjunction\footnote{\url{https://en.wikipedia.org/wiki/Logical_disjunction}}), 
``+'' sign is also sometimes used for OR.

$\wedge$ stands for AND (logical conjunction\footnote{\url{https://en.wikipedia.org/wiki/Logical_conjunction}}).
It is easy to memorize: $\wedge$ looks like ``A'' letter.
``$\cdot$'' is also sometimes used for AND.

$\neg$ is negation (NOT).

% TODO A/B is the first clause, C/D is second

\subsection{Example: 2-bit adder}
\label{adder}

\ac{SAT}-solver is merely a solver of huge boolean equations in CNF form.
It just gives the answer, if there is a set of input values which can satisfy CNF expression, and what input values must be.

Here is a 2-bit adder for example:

\begin{figure}[ht!]
\centering
\includegraphics[scale=0.75]{SAT/adder_logisim.png}
\caption{2-bit adder circuit}
\end{figure}

The adder in its simplest form: it has no carry-in and carry-out, and it has 3 XOR gates and one AND gate.
Let's try to figure out, which sets of input values will force adder to set both two output bits?
By doing quick memory calculation, we can see that there are 4 ways to do so: $0+3=3$, $1+2=3$, $2+1=3$, $3+0=3$.
Here is also truth table, with these rows highlighted:

\newcommand{\HLcell}{\cellcolor{blue!25}}

\begin{center}
\begin{doublespace}
\noindent\(\begin{array}{l|llllll}
  & \text{aH} & \text{aL} & \text{bH} & \text{bL} & \text{qH} & \text{qL} \\
\hline
 \text{3+3 = 6 $\equiv $ 2 (mod 4)} & 1 & 1 & 1 & 1 & 1 & 0 \\
 \text{3+2 = 5 $\equiv $ 1 (mod 4)} & 1 & 1 & 1 & 0 & 0 & 1 \\
 \text{3+1 = 4 $\equiv $ 0 (mod 4)} & 1 & 1 & 0 & 1 & 0 & 0 \\
 \text{\HLcell{}3+0 = 3 $\equiv $ 3 (mod 4)} & \HLcell{}1 & \HLcell{}1 & \HLcell{}0 & \HLcell{}0 & \HLcell{}1 & \HLcell{}1 \\
 \text{2+3 = 5 $\equiv $ 1 (mod 4)} & 1 & 0 & 1 & 1 & 0 & 1 \\
 \text{2+2 = 4 $\equiv $ 0 (mod 4)} & 1 & 0 & 1 & 0 & 0 & 0 \\
 \text{\HLcell{}2+1 = 3 $\equiv $ 3 (mod 4)} & \HLcell{}1 & \HLcell{}0 & \HLcell{}0 & \HLcell{}1 & \HLcell{}1 & \HLcell{}1 \\
 \text{2+0 = 2 $\equiv $ 2 (mod 4)} & 1 & 0 & 0 & 0 & 1 & 0 \\
 \text{1+3 = 4 $\equiv $ 0 (mod 4)} & 0 & 1 & 1 & 1 & 0 & 0 \\
 \text{\HLcell{}1+2 = 3 $\equiv $ 3 (mod 4)} & \HLcell{}0 & \HLcell{}1 & \HLcell{}1 & \HLcell{}0 & \HLcell{}1 & \HLcell{}1 \\
 \text{1+1 = 2 $\equiv $ 2 (mod 4)} & 0 & 1 & 0 & 1 & 1 & 0 \\
 \text{1+0 = 1 $\equiv $ 1 (mod 4)} & 0 & 1 & 0 & 0 & 0 & 1 \\
 \text{\HLcell{}0+3 = 3 $\equiv $ 3 (mod 4)} & \HLcell{}0 & \HLcell{}0 & \HLcell{}1 & \HLcell{}1 & \HLcell{}1 & \HLcell{}1 \\
 \text{0+2 = 2 $\equiv $ 2 (mod 4)} & 0 & 0 & 1 & 0 & 1 & 0 \\
 \text{0+1 = 1 $\equiv $ 1 (mod 4)} & 0 & 0 & 0 & 1 & 0 & 1 \\
 \text{0+0 = 0 $\equiv $ 0 (mod 4)} & 0 & 0 & 0 & 0 & 0 & 0 \\
\end{array}\)
\end{doublespace}
\end{center}


Let's find, what \ac{SAT}-solver can say about it?

First, we should represent our 2-bit adder as \ac{CNF} expression.

Using Wolfram Mathematica, we can express 1-bit expression for both adder outputs:\\
\\
\textbf{\texttt{In[]:=AdderQ0[aL$\_$,bL$\_$]=Xor[aL,bL]}} \\
\textbf{\texttt{Out[]:=aL $\veebar$ bL}} \\
\\
\textbf{\texttt{In[]:=AdderQ1[aL$\_$,aH$\_$,bL$\_$,bH$\_$]=Xor[And[aL,bL],Xor[aH,bH]]}} \\
\textbf{\texttt{Out[]:=aH $\veebar$ bH $\veebar$ (aL \&\& bL)}} \\
\\
We need such expression, where both parts will generate 1's.
Let's use Wolfram Mathematica find all instances of such expression (I glueled both parts with And): \\
\\
\textbf{\texttt{In[]:=Boole[SatisfiabilityInstances[And[AdderQ0[aL,bL],AdderQ1[aL,aH,bL,bH]],\{aL,aH,bL,bH\},4]]}} \\
\textbf{\texttt{Out[]:=\{1,1,0,0\},\{1,0,0,1\},\{0,1,1,0\},\{0,0,1,1\}}} \\
\\
Yes, indeed, Mathematica says, there are 4 inputs which will lead to the result we need.
So, Mathematica can also be used as \ac{SAT} solver.

Nevertheless, let's proceed to \ac{CNF} form. Using Mathematica again, let's convert our expression to \ac{CNF} form:\\
\\
\textbf{\texttt{In[]:=cnf=BooleanConvert[And[AdderQ0[aL,bL],AdderQ1[aL,aH,bL,bH]],``CNF'']}} \\
\textbf{\texttt{Out[]:=(!aH $\|$ !bH) \&\& (aH $\|$ bH) \&\& (!aL $\|$ !bL) \&\& (aL $\|$ bL)}} \\
\\
Looks more complex. The reason of such verbosity is that \ac{CNF} form doesn't allow XOR operations.
% FIXME: TeX form of the expression!

\subsubsection{MiniSat}

For the starters, we can try MiniSat\footnote{\url{http://minisat.se/MiniSat.html}}.
The standard way to encode \ac{CNF} expression for MiniSat is to enumerate all OR parts at each line.
Also, MiniSat doesn't support variable names, just numbers.
Let's enumerate our variables: 1 will be aH, 2 -- aL, 3 -- bH, 4 -- bL.

Here is what I've got when I converted Mathematica expression to the MiniSat input file:

\begin{lstlisting}
p cnf 4 4
-1 -3 0
1 3 0
-2 -4 0
2 4 0
\end{lstlisting}

Two 4's at the first lines are number of variables and number of clauses respectively.
There are 4 lines then, each for each OR clause.
Minus before variable number meaning that the variable is negated.
Absence of minus -- not negated.
Zero at the end is just terminating zero, meaning end of the clause.

In other words, each line is OR-clause with optional negations,
and the task of MiniSat is to find such set of input, which can satisfy all lines in the input file.

That file I named as \textit{adder.cnf} and now let's try MiniSat:

\begin{lstlisting}
% minisat -verb=0 adder.cnf results.txt
SATISFIABLE
\end{lstlisting}

The results are in \textit{results.txt} file:

\begin{lstlisting}
SAT
-1 -2 3 4 0
\end{lstlisting}

This means, if the first two variables (aH and aL) will be \textit{false}, and the last two variables (bH and bL) will be set to \textit{true},
the whole \ac{CNF} expression is satisfiable.
Seems to be true: if bH and bL are the only inputs set to \textit{true}, both resulting bits are also has \textit{true} states.

Now how to get other instances? \ac{SAT}-solvers, like \ac{SMT} solvers, produce only one solution (or \textit{instance}).

MiniSat uses \ac{PRNG} and its initial seed can be set explicitely. I tried different values, but result is still the same.
Nevertheless, CryptoMiniSat in this case was able to show all possible 4 instances, in chaotic order, though.
So this is not very robust way.

Perhaps, the only known way is to negate solution clause and add it to the input expression.
We've got \TT{-1 -2 3 4}, 
now we can negate all values in it (just toggle minuses: \TT{1 2 -3 -4}) and add it to the end of the input file:

\begin{lstlisting}
p cnf 4 5
-1 -3 0
1 3 0
-2 -4 0
2 4 0
1 2 -3 -4
\end{lstlisting}

Now we've got another result:

\begin{lstlisting}
SAT
1 2 -3 -4 0
\end{lstlisting}

This means, aH and aL must be both \textit{true} and bH and bL must be \textit{false}, to satisfy the input expression.
Let's negate this clause and add it again:

\begin{lstlisting}
p cnf 4 6
-1 -3 0
1 3 0
-2 -4 0
2 4 0
1 2 -3 -4
-1 -2 3 4 0
\end{lstlisting}

The result is:

\begin{lstlisting}
SAT
-1 2 3 -4 0
\end{lstlisting}

aH=false, aL=true, bH=true, bL=false. This is also correct, according to our truth table.

Let's add it again:

\begin{lstlisting}
p cnf 4 7
-1 -3 0
1 3 0
-2 -4 0
2 4 0
1 2 -3 -4
-1 -2 3 4 0
1 -2 -3 4 0
\end{lstlisting}

\begin{lstlisting}
SAT
1 -2 -3 4 0
\end{lstlisting}

\textit{aH=true, aL=false, bH=false, bL=true.} This is also correct.

This is fourth result. There are shouldn't be more. What if to add it?

\begin{lstlisting}
p cnf 4 8
-1 -3 0
1 3 0
-2 -4 0
2 4 0
1 2 -3 -4
-1 -2 3 4 0
1 -2 -3 4 0
-1 2 3 -4 0
\end{lstlisting}

Now MiniSat just says ``UNSATISFIABLE'' without any additional information in the resulting file.

Our example is tiny, but MiniSat can work with huge \ac{CNF} expressions.

\subsubsection{CryptoMiniSat}

XOR operation is absent in \ac{CNF} form, but crucial in cryptographical algorithms.
Simplest possible way to represent single XOR operation in \ac{CNF} form is:
$(\neg x \vee \neg y) \wedge (x \vee y)$ -- not that small expression, 
though, many XOR operations in single expression can be optimized better.

One significant difference between MiniSat and CryptoMiniSat is that
the latter supports clauses with XOR operations instead of ORs,
because CryptoMiniSat has aim to analyze crypto algorithms\footnote{\url{http://www.msoos.org/xor-clauses/}}.
XOR clauses are handled by CryptoMiniSat in a special way without translating to OR clauses.

You need just to prepend a clause with ``x'' in \ac{CNF} file and OR clause is then treated as XOR clause by CryptoMiniSat.
As of 2-bit adder, this smallest possible XOR-CNF expression can be used to find all inputs where both output adder bits are set:

$(aH \oplus bH) \wedge (aL \oplus bL)$

This is \TT{.cnf} file for CryptoMiniSat:

\begin{lstlisting}
p cnf 4 2
x1 3 0
x2 4 0
\end{lstlisting}

Now I run CryptoMiniSat with various random values to initialize its \ac{PRNG} \dots

\begin{lstlisting}
% cryptominisat4 --verb 0 --random 0 XOR_adder.cnf
s SATISFIABLE
v 1 2 -3 -4 0
% cryptominisat4 --verb 0 --random 1 XOR_adder.cnf
s SATISFIABLE
v -1 -2 3 4 0
% cryptominisat4 --verb 0 --random 2 XOR_adder.cnf
s SATISFIABLE
v 1 -2 -3 4 0
% cryptominisat4 --verb 0 --random 3 XOR_adder.cnf
s SATISFIABLE
v 1 2 -3 -4 0
% cryptominisat4 --verb 0 --random 4 XOR_adder.cnf
s SATISFIABLE
v -1 2 3 -4 0
% cryptominisat4 --verb 0 --random 5 XOR_adder.cnf
s SATISFIABLE
v -1 2 3 -4 0
% cryptominisat4 --verb 0 --random 6 XOR_adder.cnf
s SATISFIABLE
v -1 -2 3 4 0
% cryptominisat4 --verb 0 --random 7 XOR_adder.cnf
s SATISFIABLE
v 1 -2 -3 4 0
% cryptominisat4 --verb 0 --random 8 XOR_adder.cnf
s SATISFIABLE
v 1 2 -3 -4 0
% cryptominisat4 --verb 0 --random 9 XOR_adder.cnf
s SATISFIABLE
v 1 2 -3 -4 0
\end{lstlisting}

Nevertheless, all 4 possible solutions are:

\begin{lstlisting}
v -1 -2 3 4 0
v -1 2 3 -4 0
v 1 -2 -3 4 0
v 1 2 -3 -4 0
\end{lstlisting}

\dots the same as reported by MiniSat.

% subsections:
\subsection{Cracking Minesweeper with SAT solver}
\label{minesweeper_SAT}

See also about cracking it using Z3: \ref{minesweeper_SMT}.

SAT solvers are very different in that sense that they are at low-level, and can take only CNF expressions on input.

\subsubsection{Simple population count function}

First of all, somehow we need to count neighbour bombs.
The counting function is very similar to \textit{population count} function.

We can try to make CNF expression in Wolfram Mathematica.
This will be a function, returning True if any of 2 bits of 8 inputs bits are True and others are False.
First, we make truth table of such function:

\begin{lstlisting}
In[]:= tbl2 = 
 Table[PadLeft[IntegerDigits[i, 2], 8] -> 
   If[Equal[DigitCount[i, 2][[1]], 2], 1, 0], {i, 0, 255}]

Out[]= {{0, 0, 0, 0, 0, 0, 0, 0} -> 0, {0, 0, 0, 0, 0, 0, 0, 1} -> 0, 
{0, 0, 0, 0, 0, 0, 1, 0} -> 0, {0, 0, 0, 0, 0, 0, 1, 1} -> 1, 
{0, 0, 0, 0, 0, 1, 0, 0} -> 0, {0, 0, 0, 0, 0, 1, 0, 1} -> 1, 
{0, 0, 0, 0, 0, 1, 1, 0} -> 1, {0, 0, 0, 0, 0, 1, 1, 1} -> 0, 
{0, 0, 0, 0, 1, 0, 0, 0} -> 0, {0, 0, 0, 0, 1, 0, 0, 1} -> 1, 
{0, 0, 0, 0, 1, 0, 1, 0} -> 1, {0, 0, 0, 0, 1, 0, 1, 1} -> 0, 
...
{1, 1, 1, 1, 1, 0, 1, 0} -> 0, {1, 1, 1, 1, 1, 0, 1, 1} -> 0, 
{1, 1, 1, 1, 1, 1, 0, 0} -> 0, {1, 1, 1, 1, 1, 1, 0, 1} -> 0, 
{1, 1, 1, 1, 1, 1, 1, 0} -> 0, {1, 1, 1, 1, 1, 1, 1, 1} -> 0}
\end{lstlisting}

Now we can make CNF expression using this truth table:

\begin{lstlisting}
In[]:= BooleanConvert[
 BooleanFunction[tbl2, {a, b, c, d, e, f, g, h}], "CNF"]

Out[]= (! a || ! b || ! c) && (! a || ! b || ! d) && (! a || ! 
    b || ! e) && (! a || ! b || ! f) && (! a || ! b || ! g) && (! 
    a || ! b || ! h) && (! a || ! c || ! d) && (! a || ! c || ! 
    e) && (! a || ! c || ! f) && (! a || ! c || ! g) && (! a || ! 
    c || ! h) && (! a || ! d || ! e) && (! a || ! d || ! f) && (! 
    a || ! d || ! g) && (! a || ! d || ! h) && (! a || ! e || ! 
    f) && (! a || ! e || ! g) && (! a || ! e || ! h) && (! a || ! 
    f || ! g) && (! a || ! f || ! h) && (! a || ! g || ! h) && (a || 
   b || c || d || e || f || g) && (a || b || c || d || e || f || 
   h) && (a || b || c || d || e || g || h) && (a || b || c || d || f ||
    g || h) && (a || b || c || e || f || g || h) && (a || b || d || 
   e || f || g || h) && (a || c || d || e || f || g || 
   h) && (! b || ! c || ! d) && (! b || ! c || ! e) && (! b || ! 
    c || ! f) && (! b || ! c || ! g) && (! b || ! c || ! h) && (! 
    b || ! d || ! e) && (! b || ! d || ! f) && (! b || ! d || ! 
    g) && (! b || ! d || ! h) && (! b || ! e || ! f) && (! b || ! 
    e || ! g) && (! b || ! e || ! h) && (! b || ! f || ! g) && (! 
    b || ! f || ! h) && (! b || ! g || ! h) && (b || c || d || e || 
   f || g || 
   h) && (! c || ! d || ! e) && (! c || ! d || ! f) && (! c || ! 
    d || ! g) && (! c || ! d || ! h) && (! c || ! e || ! f) && (! 
    c || ! e || ! g) && (! c || ! e || ! h) && (! c || ! f || ! 
    g) && (! c || ! f || ! h) && (! c || ! g || ! h) && (! d || ! 
    e || ! f) && (! d || ! e || ! g) && (! d || ! e || ! h) && (! 
    d || ! f || ! g) && (! d || ! f || ! h) && (! d || ! g || ! 
    h) && (! e || ! f || ! g) && (! e || ! f || ! h) && (! e || ! 
    g || ! h) && (! f || ! g || ! h)
\end{lstlisting}

The syntax is similar to C/C++.
Let's check it.

I wrote a Python function to convert Mathematica's output into CNF file which can be feeded to SAT solver:

\lstinputlisting{SAT/minesweeper/tst.py}

It replaces a/b/c/... variables to the variable names passed (1/2/3...), reworks syntax, etc.
Here is a result:

\lstinputlisting{SAT/minesweeper/tst1.cnf}

I can run it:

\begin{lstlisting}
% minisat -verb=0 tst1.cnf results.txt
WARNING: for repeatability, setting FPU to use double precision
SATISFIABLE

% cat results.txt
SAT
1 -2 -3 -4 -5 -6 -7 8 0
\end{lstlisting}

The variable name in results lacking minus sign is "True".
Variable name with minus sign is "False".
We see there are just two variables are "True": 1 and 8.
This is indeed correct: MiniSat solver found a condition, for which our function returns "True".
Zero at the end is just a terminal symbol which means nothing.

We can ask MiniSat for another solution, by adding current solution to the input CNF file, but with all variables negated:

\begin{lstlisting}
...
-5 -6 -8 0
-5 -7 -8 0
-6 -7 -8 0
-1 2 3 4 5 6 7 -8 0
\end{lstlisting}

In plain English language, this means "give me ANY solution which can satisfy all clauses, but also not equal to the last clause we've just added".

MiniSat, indeed, found another solution, again, with only 2 variables equal to "True":

\begin{lstlisting}
% minisat -verb=0 tst2.cnf results.txt
WARNING: for repeatability, setting FPU to use double precision
SATISFIABLE

% cat results.txt
SAT
1 2 -3 -4 -5 -6 -7 -8 0
\end{lstlisting}

By the way, <i>population count</i> function for 8 neighbours in CNF form is simplest:

\begin{lstlisting}
a&&b&&c&&d&&e&&f&&g&&h
\end{lstlisting}

Indeed: it's true if all 8 input bits are "True".

The function for 0 neighbours is also simple:

\begin{lstlisting}
!a&&!b&&!c&&!d&&!e&&!f&&!g&&!h
\end{lstlisting}

It means, it will return "True", if all input variables are "False".

By the way, function for POPCNT1 is also simple:

\begin{lstlisting}
(!a||!b)&&(!a||!c)&&(!a||!d)&&(!a||!e)&&(!a||!f)&&(!a||!g)&&(!a||!h)&&(a||b||c||d||e||f||g||h)&&
(!b||!c)&&(!b||!d)&&(!b||!e)&&(!b||!f)&&(!b||!g)&&(!b||!h)&&(!c||!d)&&(!c||!e)&&(!c||!f)&&(!c||!g)&&
(!c||!h)&&(!d||!e)&&(!d||!f)&&(!d||!g)&&(!d||!h)&&(!e||!f)&&(!e||!g)&&(!e||!h)&&(!f||!g)&&(!f||!h)&&(!g||!h)
\end{lstlisting}

It just enumerates all possible pairs of 8 variables (a/b, a/c, a/d, etc) and says: no two bits must be present
simultaneously in each possible pair.
And there is another clause: "(a||b||c||d||e||f||g||h)", which says: at least one bit must be present among 8 variables.

And yes, you can ask Mathematica for finding CNF expressions for any other truth table.

\subsubsection{Minesweeper}

Now we can use Mathematica to get all \textit{population count} functions for 0..8 neighbours.

For 9*9 Minesweeper grid including invisible border, there will be 11*11=121 variables, mapped to Minesweeper grid like this:

\begin{lstlisting}
 1    2   3   4   5   6   7   8   9  10  11
12   13  14  15  16  17  18  19  20  21  22
23   24  25  26  27  28  29  30  31  32  33
34   35  36  37  38  39  40  41  42  43  44

...

100 101 102 103 104 105 106 107 108 109 110
111 112 113 114 115 116 117 118 119 120 121
\end{lstlisting}

Then we write a Python script which stacks all \textit{population count} functions: each function for each known number of neighbours (digit on Minesweeper field).
Each POPCNTx() function takes list of variable numbers and outputs list of clauses to be added to the final CNF file.

As of empty cells, we also add them as clauses, but with minus sign, which means, the variable must be False.
Whenever we try to place bomb, we add its variable as clause without minus sign, this means the variable must be True.

Then we execute external minisat process.
The only thing we need from it is exit code.
If an input CNF is UNSAT, it returns 20:

\lstinputlisting{SAT/minesweeper/minesweeper_SAT.py}

The output CNF file can be large, up to ~2000 clauses, or more, here is an example: URL sample.cnf

Anyway, it works just like my previous Z3Py script:

\begin{lstlisting}
row=1, col=3, unsat!
row=6, col=2, unsat!
row=6, col=3, unsat!
row=7, col=4, unsat!
row=7, col=9, unsat!
row=8, col=9, unsat!
\end{lstlisting}

... but it runs way faster, even considering overhead of executing external program.
Perhaps, Z3Py version could be optimized much better?

The files, including Wolfram Mathematica notebook: URL.


\section{KLEE}

% subsections:
\subsection{School-level equation}

Let's revisit school-level system of equations from (\ref{eq2_SMT}).

We will force KLEE to find a path, where all the constraints are satisfied:

\lstinputlisting{KLEE/klee_eq1.c}

% FIXME:
\begin{lstlisting}
\$ clang -emit-llvm -c -g klee_eq.c
...

\$ klee klee_eq.bc
KLEE: output directory is "/home/klee/klee-out-93"
KLEE: WARNING: undefined reference to function: klee_assert
KLEE: WARNING ONCE: calling external: klee_assert(0)
KLEE: ERROR: /home/klee/klee_eq.c:18: failed external call: klee_assert
KLEE: NOTE: now ignoring this error at this location

KLEE: done: total instructions = 32
KLEE: done: completed paths = 1
KLEE: done: generated tests = 1
\end{lstlisting}

Let's find out, where \TT{klee\_assert()} has been triggered:

% FIXME:
\begin{lstlisting}
\$ ls klee-last | grep err
test000001.external.err

\$ ktest-tool --write-ints klee-last/test000001.ktest
ktest file : 'klee-last/test000001.ktest'
args       : ['klee_eq.bc']
num objects: 3
object    0: name: b'circle'
object    0: size: 4
object    0: data: 5
object    1: name: b'square'
object    1: size: 4
object    1: data: 2
object    2: name: b'triangle'
object    2: size: 4
object    2: data: 1
\end{lstlisting}

This is indeed correct solution to the system of equations.

KLEE has intrinsic \TT{klee\_assume()} which tells KLEE to cut path if some constraint is not true.
So we can rewrite our example in such cleaner way:

\lstinputlisting{KLEE/klee_eq2.c}



\subsection{Zebra puzzle (\ac{AKA} Einstein puzzle)}
\label{zebra_SMT}

Zebra puzzle is a popular puzzle, defined as follows:

% FIXME remove paragraph at first line
\begin{framed}
\begin{quotation}
1.There are five houses.\\
2.The Englishman lives in the red house.\\
3.The Spaniard owns the dog.\\
4.Coffee is drunk in the green house.\\
5.The Ukrainian drinks tea.\\
6.The green house is immediately to the right of the ivory house.\\
7.The Old Gold smoker owns snails.\\
8.Kools are smoked in the yellow house.\\
9.Milk is drunk in the middle house.\\
10.The Norwegian lives in the first house.\\
11.The man who smokes Chesterfields lives in the house next to the man with the fox.\\
12.Kools are smoked in the house next to the house where the horse is kept.\\
13.The Lucky Strike smoker drinks orange juice.\\
14.The Japanese smokes Parliaments.\\
15.The Norwegian lives next to the blue house.\\
\\
Now, who drinks water? Who owns the zebra?\\
\\
In the interest of clarity, it must be added that each of the five houses is painted a different color, and their inhabitants are of different national extractions, own different pets, drink different beverages and smoke different brands of American cigarets [sic]. One other thing: in statement 6, right means your right.
\end{quotation}
\end{framed}
( \url{https://en.wikipedia.org/wiki/Zebra_Puzzle} ) \\
\\
It's a very good example of constraint satisfaction problem (CSP). % FIXME \ac

We would encode each entity as integer variable, representing number of house.

Then, to define that Englishman lives in red house, we will define this constraint: \TT{Englishman == Red}, meaning that number of a house where Englishmen resides and where tea is drunk is the same.

To define that Norwegian lives next to the blue house, we don't realy know, if it is at left side of blue house or at right side, but we know that house numbers are different by just 1.
So we will define this constraint: \TT{Norwegian==Blue-1 OR Norwegian==Blue+1}.

We will also need to limit all house numbers, so they will be in range of 1..5.

We will also use \TT{Distinct} to show that all various entities of the same type are all has different house numbers.

\lstinputlisting{SMT/zebra.py}

When we run it, we got correct result:

\begin{lstlisting}
sat
[Snails = 3,
 Blue = 2,
 Ivory = 4,
 OrangeJuice = 4,
 Parliament = 5,
 Yellow = 1,
 Fox = 1,
 Zebra = 5,
 Horse = 2,
 Dog = 4,
 Tea = 2,
 Water = 1,
 Chesterfield = 2,
 Red = 3,
 Japanese = 5,
 LuckyStrike = 4,
 Norwegian = 1,
 Milk = 3,
 Kools = 1,
 OldGold = 3,
 Ukrainian = 2,
 Coffee = 5,
 Green = 5,
 Spaniard = 4,
 Englishman = 3]
 \end{lstlisting}


\subsection{Sudoku}

I've also rewritten Sudoku example (\ref{sudoku_SMT}) for KLEE:

\lstinputlisting[numbers=left]{KLEE/klee_sudoku_or1.c}

Let's run it:

% FIXME:
\begin{lstlisting}
\$ clang -emit-llvm -c -g klee_sudoku_or1.c
...

\$ time klee klee_sudoku_or1.bc
KLEE: output directory is "/home/klee/klee-out-98"
KLEE: WARNING: undefined reference to function: klee_assert
KLEE: WARNING ONCE: calling external: klee_assert(0)
KLEE: ERROR: /home/klee/klee_sudoku_or1.c:93: failed external call: klee_assert
KLEE: NOTE: now ignoring this error at this location

KLEE: done: total instructions = 7512
KLEE: done: completed paths = 161
KLEE: done: generated tests = 161

real    3m44.111s
user    3m43.319s
sys     0m0.951s
\end{lstlisting}

Now this is really slower (on my Intel Core i3-3110M 2.4GHz notebook) in comparison to Z3Py solution (\ref{sudoku_SMT}).

But the answer is correct:

% FIXME:
\begin{lstlisting}
\$ ls klee-last | grep err
test000161.external.err

\$ ktest-tool --write-ints klee-last/test000161.ktest
ktest file : 'klee-last/test000161.ktest'
args       : ['klee_sudoku_or1.bc']
num objects: 1
object    0: name: b'cells'
object    0: size: 81
object    0: data: b'\x01\x04\x05\x03\x02\x07\x06\t\x08\x08\x03\t\x06\x05\x04\x01\x02\x07\x06\x07\x02\t\x01\x08\x05\x04\x03\x04\t\x06\x01\x08\x05\x03\x07\x02\x02\x01\x08\x04\x07\x03\t\x05\x06\x07\x05\x03\x02\t\x06\x04\x08\x01\x03\x06\x07\x05\x04\x02\x08\x01\t\t\x08\x04\x07\x06\x01\x02\x03\x05\x05\x02\x01\x08\x03\t\x07\x06\x04'
\end{lstlisting}

% FIXME backslash
\TT{\\t} is character with ASCII code of 9 in C/C++, and KLEE attempts to treat byte array as C/C++ string, so it shows some values in such way.
We can just remember that there is 9 at the each place where we see \TT{\\t}.
The solution, while not properly formatted, correct indeed. \\
\\
By the way, at lines 42, 43 you may see how we tell to KLEE that all array elements must be within some limits.
If we comment these lines out, we've got this:

% FIXME:
\begin{lstlisting}
\$ time klee klee_sudoku_or1.bc
KLEE: output directory is "/home/klee/klee-out-100"
KLEE: WARNING: undefined reference to function: klee_assert
KLEE: ERROR: /home/klee/klee_sudoku_or1.c:51: overshift error
KLEE: NOTE: now ignoring this error at this location
KLEE: ERROR: /home/klee/klee_sudoku_or1.c:51: overshift error
KLEE: NOTE: now ignoring this error at this location
KLEE: ERROR: /home/klee/klee_sudoku_or1.c:51: overshift error
KLEE: NOTE: now ignoring this error at this location
...
\end{lstlisting}

KLEE warns us that shift value at line 51 is too big.
Indeed, KLEE may try all byte values up to 255 (0xFF), which are pointless to use there, and may indicate error or bug, so KLEE warns about it.\\
\\
Now let's use \TT{klee\_assume()} again:

\lstinputlisting{KLEE/klee_sudoku_or2.c}

% FIXME:
\begin{lstlisting}
\$ time klee klee_sudoku_or2.bc
KLEE: output directory is "/home/klee/klee-out-99"
KLEE: WARNING: undefined reference to function: klee_assert
KLEE: WARNING ONCE: calling external: klee_assert(0)
KLEE: ERROR: /home/klee/klee_sudoku_or2.c:93: failed external call: klee_assert
KLEE: NOTE: now ignoring this error at this location

KLEE: done: total instructions = 7119
KLEE: done: completed paths = 1
KLEE: done: generated tests = 1

real    0m35.312s
user    0m34.945s
sys     0m0.318s
\end{lstlisting}

That works much faster: perhaps KLEE indeed handle this intrinsic in a special way.
And, as we see, the only one path is generated (one we actually interesting in it) instead of 161.

It's still much slower than Z3Py solution, though.


\input{KLEE/UNIXdatetime.tex}
\input{KLEE/base64.tex}
\subsection{CRC} % FIXME full name

\subsubsection{Buffer alteration case \#1}

Sometimes, you need to alter a piece of data which is \textit{protected} by some kind of checksum or \ac{CRC}, and you can't change checksum or CRC value, but can alter piece of data so that checksum will remain the same.

Let's pretend, we've got a piece of data with ``Hello, world!'' string at the beginning and ``and goodbye'' string at the end.
We can alter 14 characters at the middle, but for some reason, they must be in \textit{a..z} limits, but we can put any characters there.
CRC64 of the whole block must be \TT{0x12345678abcdef12}.

Let's see\footnote{There are several slightly different CRC64 implementations, the one I use here can also be different from popular ones.}:

\lstinputlisting{KLEE/klee_CRC64.c}

Since our code uses memcmp() standard C/C++ function, we need to add \TT{--libc=uclibc} switch, so KLEE will use its own uClibc % FIXME check spelling
implementation. % \ref{} -> closed programs

% FIXME:
\begin{lstlisting}
\$ clang -emit-llvm -c -g klee_CRC64.c

\$ time klee --libc=uclibc klee_CRC64.bc
\end{lstlisting}

It takes about 1 minute (on my XXX) and we getting this:

% FIXME:
\begin{lstlisting}
...
real    0m52.643s
user    0m51.232s
sys     0m0.239s
...
\$ ls klee-last | grep err
test000001.user.err
test000002.user.err
test000003.user.err
test000004.external.err

\$ ktest-tool --write-ints klee-last/test000004.ktest
ktest file : 'klee-last/test000004.ktest'
args       : ['klee_CRC64.bc']
num objects: 1
object    0: name: b'buf'
object    0: size: 46
object    0: data: b'Hello, world!.. qqlicayzceamyw ... and goodbye'
\end{lstlisting}

Maybe it's slow, but definitely faster than bruteforce.
Indeed, $log_2{26^{14}} \approx 65.8$
which is close to 64 bits.
In other words, one need $\approx 14$ latin characters to encode 64 bits.
And KLEE + \ac{SMT} solver needs 64 bits at some place it can alter to make final CRC64 value equal to what we defined.

I tried to reduce length of the \textit{middle block} to 13 characters: no luck for KLEE then, it has no space enough.

\subsubsection{Buffer alteration case \#2}

I went sadistic: what if the buffer must contain the CRC64 value which, after calculation of CRC64, will result in the same value?
Fascinately, % FIXME check spelling
KLEE can solve this.
The buffer will have the following format:

% FIXME:
\begin{lstlisting}
Hello, world! <8-bytes (64-bit value)> and goodbye <6 more bytes>
\end{lstlisting}

% FIXME:
\begin{lstlisting}
int main()
{
#define HEAD_STR "Hello, world!.. "
#define HEAD_SIZE strlen(HEAD_STR)
#define TAIL_STR " ... and goodbye"
#define TAIL_SIZE strlen(TAIL_STR)
// 8 bytes for 64-bit value:
#define MID_SIZE 8
#define BUF_SIZE HEAD_SIZE+TAIL_SIZE+MID_SIZE+6

	char buf[BUF_SIZE];
  
	klee_make_symbolic(buf, sizeof buf, "buf");

	klee_assume (memcmp (buf, HEAD_STR, HEAD_SIZE)==0);

	klee_assume (memcmp (buf+HEAD_SIZE+MID_SIZE, TAIL_STR, TAIL_SIZE)==0);
	
	uint64_t mid_value=*(uint64_t*)(buf+HEAD_SIZE);
	klee_assume (crc64 (0, buf, BUF_SIZE)==mid_value);

	klee_assert(0);

	return 0;
}
\end{lstlisting}

It works:

% FIXME:
\begin{lstlisting}
\$ time klee --libc=uclibc klee_CRC64.bc
...
real    5m17.081s
user    5m17.014s
sys     0m0.319s

\$ ls klee-last | grep err
test000001.user.err
test000002.user.err
test000003.external.err

\$ ktest-tool --write-ints klee-last/test000003.ktest
ktest file : 'klee-last/test000003.ktest'
args       : ['klee_CRC64.bc']
num objects: 1
object    0: name: b'buf'
object    0: size: 46
object    0: data: b'Hello, world!.. T+]\xb9A\x08\x0fq ... and goodbye\xb6\x8f\x9c\xd8\xc5\x00'
\end{lstlisting}

8 bytes between two strings is 64-bit value which equals to CRC64 of this whole block.
Again, it's faster than brute-force way to find it.
If to decrease last spare 6-byte buffer to 4 bytes or less, KLEE works so long so I've stopped it.

\subsubsection{Recovering input data for given CRC32 value of it}

I've always wanted to do so, but everyone knows this is impossible for input buffers larger than 4 bytes.
As my experiments show, it's still possible for tiny input buffers of data, constrained in some way.

The CRC32 value of 6-byte ``SILVER'' string is known: \TT{0xDFA3DFDD}.
KLEE can find this 6-byte string, if it knows that each byte of input buffer is in \textit{A..Z} limits:

\lstinputlisting[numbers=left]{KLEE/klee_SILVER.c}

% FIXME:
\begin{lstlisting}
\$ clang -emit-llvm -c -g klee_SILVER.c
...

\$ klee klee_SILVER.bc
...

\$ ls klee-last | grep err
test000013.external.err

\$ ktest-tool --write-ints klee-last/test000013.ktest
ktest file : 'klee-last/test000013.ktest'
args       : ['klee_SILVER.bc']
num objects: 1
object    0: name: b'str'
object    0: size: 6
object    0: data: b'SILVER'
\end{lstlisting}

Still, it's no magic: if to remove condition at lines 23..25 (i.e., if to relax constraints),
KLEE will produce some other string, which will be still correct for the CRC32 value given.

It works, because 6 Latin characters in \textit{A..Z} limits contain $\approx 28.2$ bits:
$log_2{26^6} \approx 28.2$, which is even smaller value than 32.
In other words, the final CRC32 value holds enough bits to recover $\approx 28.2$ bits of input.

The input buffer can be even bigger, if each byte of it will be in even tighter % FIXME spelling
constraints (decimal digits, binary digits, etc).

\subsubsection{In comparison with other hashing algorithms}

Things are that easy for some other hashing algorithms like \textit{fletcher checksum}, % FIXME URL
but not for cryptographically secure ones (like MD5, SHA1, etc), they are protected from such simple cryptoanalysis. % FIXME \ref{} -> am.crypto


\input{KLEE/LZSS.tex}

\subsection{Exercise}

Here is my crackme/keygenme, which may be tricky, but easy to solve using KLEE:
\url{http://challenges.re/74/}.





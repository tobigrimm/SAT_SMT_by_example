% TODO translate src
\subsection{Головоломка Зебры как SAT-проблема}
\label{Zebra_SAT}

Попробуем решить головоломку Зебры (\ref{zebra_SMT}) в SAT.

Я определю каждую переменную как вектор из пяти переменных, как я делал это раннее в солвере Судоку: \ref{Sudoku_SAT}.

Я также использую ф-цию \TT{POPCNT1}, но в отличие от предыдущего примера,
я использовал Wolfram Mathematica для генерирования её в CNF-форме:

\begin{lstlisting}
In[]:= tbl1=Table[PadLeft[IntegerDigits[i,2],5] ->If[Equal[DigitCount[i,2][[1]],1],1,0],{i,0,63}]
Out[]= {{0,0,0,0,0}->0,
{0,0,0,0,1}->1,
{0,0,0,1,0}->1,
{0,0,0,1,1}->0,
{0,0,1,0,0}->1,
{0,0,1,0,1}->0,

...

{1,1,1,1,0}->0,
{1,1,1,1,1}->0}

In[]:= BooleanConvert[BooleanFunction[tbl1,{a,b,c,d,e}],"CNF"]
Out[]= (!a||!b)&&(!a||!c)&&(!a||!d)&&(!a||!e)&&(a||b||c||d||e)&&(!b||!c)&&(!b||!d)&&(!b||!e)&&(!c||!d)&&(!c||!e)&&(!d||!e)
\end{lstlisting}

Также, как я предлагал раньше (\ref{OR_in_POPCNT1}), я использовал операцию \textit{ИЛИ} для второго шага.

\begin{lstlisting}
def mathematica_to_CNF (s, d):
    for k in d.keys():
        s=s.replace(k, d[k])
    s=s.replace("!", "-").replace("||", " ").replace("(", "").replace(")", "")
    s=s.split ("&&")
    return s

def add_popcnt1(v1, v2, v3, v4, v5):
    global clauses
    s="(!a||!b)&&" \
      "(!a||!c)&&" \
      "(!a||!d)&&" \
      "(!a||!e)&&" \
      "(!b||!c)&&" \
      "(!b||!d)&&" \
      "(!b||!e)&&" \
      "(!c||!d)&&" \
      "(!c||!e)&&" \
      "(!d||!e)&&" \
      "(a||b||c||d||e)"

    clauses=clauses+mathematica_to_CNF(s, {"a":v1, "b":v2, "c":v3, "d":v4, "e":v5})

...

# k=tuple: ("high-level" variable name, number of bit (0..4))
# v=variable number in CNF
vars={}
vars_last=1

...

def alloc_distinct_variables(names):
    global vars
    global vars_last
    for name in names:
        for i in range(5):
            vars[(name,i)]=str(vars_last)
            vars_last=vars_last+1

        add_popcnt1(vars[(name,0)], vars[(name,1)], vars[(name,2)], vars[(name,3)], vars[(name,4)])

    # make them distinct:
    for i in range(5):
        clauses.append(vars[(names[0],i)] + " " + vars[(names[1],i)] + " " + vars[(names[2],i)] + " " + vars[(names[3],i)] + " " + vars[(names[4],i)])

...

alloc_distinct_variables(["Yellow", "Blue", "Red", "Ivory", "Green"])
alloc_distinct_variables(["Norwegian", "Ukrainian", "Englishman", "Spaniard", "Japanese"])
alloc_distinct_variables(["Water", "Tea", "Milk", "OrangeJuice", "Coffee"])
alloc_distinct_variables(["Kools", "Chesterfield", "OldGold", "LuckyStrike", "Parliament"])
alloc_distinct_variables(["Fox", "Horse", "Snails", "Dog", "Zebra"])

...

\end{lstlisting}

Теперь у нас пять булевых переменных для каждой \textit{высокоуровневной} переменной,
и каждая группа переменных гарантированно будет иметь разные значения.

Теперь перечитаем условие головоломки: ``2. Англичанин живёт в красном доме.''.
Это легко.
В моих примерах на Z3 и KLEE я просто написал ``Englishman==Red''.
Та же история и здесь: мы просто добавляем клозы, показывающие, что 5 булевых переменных для ``Englishman''
должны равняться пяти переменных для ``Red''.

На самом низком уровне CNF, если мы хотим сказать, что две переменных должны равняться друг другу,
мы добавляем два клоза:

$(var1 \vee \neg var2) \wedge (\neg var1 \vee var2)$

Это означает что значения обоих \textit{var1} и \textit{var2} должны быть или \textit{Ложно} или \textit{Истинно},
но они не могут быть разными.

\begin{lstlisting}
def add_eq_clauses(var1, var2):
    global clauses
    clauses.append(var1 + " -" + var2)
    clauses.append("-"+var1 + " " + var2)

def add_eq (n1, n2):
    for i in range(5):
        add_eq_clauses(vars[(n1,i)], vars[(n2, i)])

...

# 2.The Englishman lives in the red house.
add_eq("Englishman","Red")

# 3.The Spaniard owns the dog.
add_eq("Spaniard","Dog")

# 4.Coffee is drunk in the green house.
add_eq("Coffee","Green")

...

\end{lstlisting}

Теперь следующие условия:
``9. В центральном доме пьют молоко.'' (т.е., в третьем доме), ``10. Норвежец живёт в первом доме.''
Мы можем присвоить булевы значения напрямую:

\begin{lstlisting}
# n=1..5
def add_eq_var_n (name, n):
    global clauses
    global vars
    for i in range(5):
        if i==n-1:
            clauses.append(vars[(name,i)]) # always True
        else:
            clauses.append("-"+vars[(name,i)]) # always False

...

# 9.Milk is drunk in the middle house.
add_eq_var_n("Milk",3) # i.e., 3rd house

# 10.The Norwegian lives in the first house.
add_eq_var_n("Norwegian",1)
\end{lstlisting}

Для ``Milk'' у нас значение ``0 0 1 0 0'', для ``Norwegian'': ``1 0 0 0 0''.

Что делать с этим?
``6. Зелёный дом стоит сразу справа от белого дома.''
Я могу сконструировать такое условие:

\begin{lstlisting}
    Ivory      Green
AND(1 0 0 0 0  0 1 0 0 0)
.. OR ..
AND(0 1 0 0 0  0 0 1 0 0)
.. OR ..
AND(0 0 1 0 0  0 0 0 1 0)
.. OR ..
AND(0 0 0 1 0  0 0 0 0 1)
\end{lstlisting}

Для ``белого/ivory'' тут нет ``0 0 0 0 1'', потому что он не может быть последним.
Теперь я конвертирую эти условия в CNF при помощи Wolfram Mathematica:

\begin{lstlisting}
In[]:= BooleanConvert[(a1&& !b1&&!c1&&!d1&&!e1&&!a2&& b2&&!c2&&!d2&&!e2) ||
(!a1&& b1&&!c1&&!d1&&!e1&&!a2&& !b2&&c2&&!d2&&!e2) ||
(!a1&& !b1&&c1&&!d1&&!e1&&!a2&& !b2&&!c2&&d2&&!e2) ||
(!a1&& !b1&&!c1&&d1&&!e1&&!a2&& !b2&&!c2&&!d2&&e2) ,"CNF"]

Out[]= (!a1||!b1)&&(!a1||!c1)&&(!a1||!d1)&&(a1||b1||c1||d1)&&!a2&&(!b1||!b2)&&(!b1||!c1)&&
(!b1||!d1)&&(b1||b2||c1||d1)&&(!b2||!c1)&&(!b2||!c2)&&(!b2||!d1)&&(!b2||!d2)&&(!b2||!e2)&&
(b2||c1||c2||d1)&&(b2||c2||d1||d2)&&(b2||c2||d2||e2)&&(!c1||!c2)&&(!c1||!d1)&&(!c2||!d1)&&
(!c2||!d2)&&(!c2||!e2)&&(!d1||!d2)&&(!d2||!e2)&&!e1
\end{lstlisting}

И вот фрагмент моего кода на Питоне:

\begin{lstlisting}
def add_right (n1, n2):
    global clauses
    s="(!a1||!b1)&&(!a1||!c1)&&(!a1||!d1)&&(a1||b1||c1||d1)&&!a2&&(!b1||!b2)&&(!b1||!c1)&&(!b1||!d1)&&" \
      "(b1||b2||c1||d1)&&(!b2||!c1)&&(!b2||!c2)&&(!b2||!d1)&&(!b2||!d2)&&(!b2||!e2)&&(b2||c1||c2||d1)&&" \
      "(b2||c2||d1||d2)&&(b2||c2||d2||e2)&&(!c1||!c2)&&(!c1||!d1)&&(!c2||!d1)&&(!c2||!d2)&&(!c2||!e2)&&" \
      "(!d1||!d2)&&(!d2||!e2)&&!e1"

    clauses=clauses+mathematica_to_CNF(s, {
	"a1": vars[(n1,0)], "b1": vars[(n1,1)], "c1": vars[(n1,2)], "d1": vars[(n1,3)], "e1": vars[(n1,4)],
	"a2": vars[(n2,0)], "b2": vars[(n2,1)], "c2": vars[(n2,2)], "d2": vars[(n2,3)], "e2": vars[(n2,4)]})

...

# 6.The green house is immediately to the right of the ivory house.
add_right("Ivory", "Green")
\end{lstlisting}

Что мы будем делать с этим?
``11. Сосед того, кто курит Chesterfield, держит лису.''
``12. В доме по соседству с тем, в котором держат лошадь, курят Kool.''

Мы не знаем с какой стороны, слева или справа, но знаем что они отличаются на единицу.
Вот какие клозы я добавлю:

\begin{lstlisting}
    Chesterfield  Fox
AND(0 0 0 0 1     0 0 0 1 0)
.. OR ..
AND(0 0 0 1 0     0 0 0 0 1)
AND(0 0 0 1 0     0 0 1 0 0)
.. OR ..
AND(0 0 1 0 0     0 1 0 0 0)
AND(0 0 1 0 0     0 0 0 1 0)
.. OR ..
AND(0 1 0 0 0     1 0 0 0 0)
AND(0 1 0 0 0     0 0 1 0 0)
.. OR ..
AND(1 0 0 0 0     0 1 0 0 0)
\end{lstlisting}

И снова могу сконвертировать это всё в CNF при помощи Mathematica:

\begin{lstlisting}
In[]:= BooleanConvert[(a1&& !b1&&!c1&&!d1&&!e1&&!a2&& b2&&!c2&&!d2&&!e2) ||

(!a1&& b1&&!c1&&!d1&&!e1&&a2&& !b2&&!c2&&!d2&&!e2) ||
(!a1&& b1&&!c1&&!d1&&!e1&&!a2&& !b2&&c2&&!d2&&!e2) ||

(!a1&& !b1&&c1&&!d1&&!e1&&!a2&& b2&&!c2&&!d2&&!e2) ||
(!a1&& !b1&&c1&&!d1&&!e1&&!a2&& !b2&&!c2&&d2&&!e2) ||

(!a1&& !b1&&!c1&&d1&&!e1&&!a2&& !b2&&c2&&!d2&&!e2) ||
(!a1&& !b1&&!c1&&d1&&!e1&&!a2&& !b2&&!c2&&!d2&&e2) ||

(!a1&& !b1&&!c1&&!d1&&e1&&!a2&& !b2&&!c2&&d2&&!e2) ,"CNF"]

Out[]= (!a1||!b1)&&(!a1||!c1)&&(!a1||!d1)&&(!a1||!e1)&&(a1||b1||c1||d1||e1)&&(!a2||b1)&&(!a2||!b2)&&
(!a2||!c2)&&(!a2||!d2)&&(!a2||!e2)&&(a2||b2||c1||c2||d1||e1)&&(a2||b2||c2||d1||d2)&&(a2||b2||c2||d2||e2)&&
(!b1||!b2)&&(!b1||!c1)&&(!b1||!d1)&&(!b1||!e1)&&(b1||b2||c1||d1||e1)&&(!b2||!c2)&&(!b2||!d1)&&(!b2||!d2)&&
(!b2||!e1)&&(!b2||!e2)&&(!c1||!c2)&&(!c1||!d1)&&(!c1||!e1)&&(!c2||!d2)&&(!c2||!e1)&&(!c2||!e2)&&
(!d1||!d2)&&(!d1||!e1)&&(!d2||!e2)
\end{lstlisting}

И вот мой код:

\begin{lstlisting}
def add_right_or_left (n1, n2):
    global clauses
    s="(!a1||!b1)&&(!a1||!c1)&&(!a1||!d1)&&(!a1||!e1)&&(a1||b1||c1||d1||e1)&&(!a2||b1)&&" \
      "(!a2||!b2)&&(!a2||!c2)&&(!a2||!d2)&&(!a2||!e2)&&(a2||b2||c1||c2||d1||e1)&&(a2||b2||c2||d1||d2)&&" \
       "(a2||b2||c2||d2||e2)&&(!b1||!b2)&&(!b1||!c1)&&(!b1||!d1)&&(!b1||!e1)&&(b1||b2||c1||d1||e1)&&" \
       "(!b2||!c2)&&(!b2||!d1)&&(!b2||!d2)&&(!b2||!e1)&&(!b2||!e2)&&(!c1||!c2)&&(!c1||!d1)&&(!c1||!e1)&&" \
       "(!c2||!d2)&&(!c2||!e1)&&(!c2||!e2)&&(!d1||!d2)&&(!d1||!e1)&&(!d2||!e2)"
    
    clauses=clauses+mathematica_to_CNF(s, {
	"a1": vars[(n1,0)], "b1": vars[(n1,1)], "c1": vars[(n1,2)], "d1": vars[(n1,3)], "e1": vars[(n1,4)],
	"a2": vars[(n2,0)], "b2": vars[(n2,1)], "c2": vars[(n2,2)], "d2": vars[(n2,3)], "e2": vars[(n2,4)]})

...

# 11.The man who smokes Chesterfields lives in the house next to the man with the fox.
add_right_or_left("Chesterfield","Fox") # left or right

# 12.Kools are smoked in the house next to the house where the horse is kept.
add_right_or_left("Kools","Horse") # left or right
\end{lstlisting}

Вот и всё!
Полный исходный код: \url{https://github.com/DennisYurichev/SAT_SMT_article/blob/master/SAT/zebra/zebra_SAT.py}.

Итоговая CNF-проблема имеет 125 булевых переменных и 511 клозов: \\
\url{https://github.com/DennisYurichev/SAT_SMT_article/blob/master/SAT/zebra/1.cnf}.
Это очень легкая задача для любого SAT-солвера.
Даже мой игрушечный SAT-солвер (\ref{SAT_backtrack}) может решить её за \textasciitilde{}1 секунду на моем древнем
нетбуке с Intel Atom.

И конечно же, тут только одно решение, что и подтверждается при помощи Picosat.

\begin{lstlisting}
% python zebra_SAT.py
Yellow 1
Blue 2
Red 3
Ivory 4
Green 5
Norwegian 1
Ukrainian 2
Englishman 3
Spaniard 4
Japanese 5
Water 1
Tea 2
Milk 3
OrangeJuice 4
Coffee 5
Kools 1
Chesterfield 2
OldGold 3
LuckyStrike 4
Parliament 5
Fox 1
Horse 2
Snails 3
Dog 4
Zebra 5
\end{lstlisting}


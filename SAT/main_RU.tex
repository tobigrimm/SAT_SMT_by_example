\section{\ac{SAT}-солверы}

Сравнивать SMT с SAT, это как сравнивать высокоуровневый ЯП с языком ассемблера.
Последний может быть куда более эффективным, но на нем труднее программировать.

\subsection{CNF форма}

\ac{CNF}\footnote{\url{https://en.wikipedia.org/wiki/Conjunctive_normal_form}} это так называемая \textit{нормальная форма}.

% TODO recheck
% TODO write abt it!
%\textit{normal form} is somewhat similar to polynomials in algebra. 
%What is polynomial?
%It is a standard way to express unsystematic equations like $2x \cdot x$ as $3x$ polynomial, 
%and so you will be able to apply some operations to polynomials like summing, etc.

Любое булево выражение может быть сконвертировано в \textit{нормальную форму}, и \ac{CNF} это одна из них.
\ac{CNF}-выражение это пачка клозов (подвыражений) состоящих их литералов (или термов, переменных), операций ИЛИ и НЕ,
все из которых склеены друг с другом в полное выражение операцией И.
Вот способ запомнить: \ac{CNF} это ``И всех ИЛИ'' (или ``произведение всех сумм'')
и \ac{DNF} это ``ИЛИ всех И'' (или ``сумма всех произведений'').

Пример: $(\neg A \vee B) \wedge (C \vee \neg D)$.

$\vee$ означает ИЛИ (логическая дизьюнкция\footnote{\url{https://en.wikipedia.org/wiki/Logical_disjunction}}), 
знак ``+'' также иногда используется для ИЛИ.

$\wedge$ означает ИЛИ (логическая коньюнкция\footnote{\url{https://en.wikipedia.org/wiki/Logical_conjunction}}).
Легко запомнить: $\wedge$ выглядит как буква ``A''.
Знак ``$\cdot$'' также иногда используется для И.

$\neg$ это отрицание (НЕ).

% TODO A/B is the first clause, C/D is second

\subsection{Пример: двухбитный сумматор}
\label{adder}

В сущности, \ac{SAT}-солвер это солвер огромных булевых уравнений в CNF-форме.
Он просто выдает ответ, есть ли набор входных значений, удовлетворяющий CNF-выражению, и какие это значения должны быть.

Вот для примера двухбитный сумматор:

\begin{figure}[ht!]
\centering
\includegraphics[scale=0.75]{SAT/adder_logisim.png}
\caption{Схема двухбитного сумматора}
\end{figure}

Сумматор здесь в самом простом возможном виде: у него нет входных и выходных переносов, и тут только 3 XOR-гейта
и один AND-гейт.
Попробуем разобраться, какой набор входных переменных заставит сумматор выставить оба выходных бита?
Просто подсчитав в уме, мы можем увидеть, что таких способа 4: $0+3=3$, $1+2=3$, $2+1=3$, $3+0=3$.
Вот также таблица истинности, с подсвеченными соответствующими рядами:

\newcommand{\HLcell}{\cellcolor{blue!25}}

\begin{center}
\begin{doublespace}
\noindent\(\begin{array}{l|llllll}
  & \text{aH} & \text{aL} & \text{bH} & \text{bL} & \text{qH} & \text{qL} \\
\hline
 \text{3+3 = 6 $\equiv $ 2 (mod 4)} & 1 & 1 & 1 & 1 & 1 & 0 \\
 \text{3+2 = 5 $\equiv $ 1 (mod 4)} & 1 & 1 & 1 & 0 & 0 & 1 \\
 \text{3+1 = 4 $\equiv $ 0 (mod 4)} & 1 & 1 & 0 & 1 & 0 & 0 \\
 \text{\HLcell{}3+0 = 3 $\equiv $ 3 (mod 4)} & \HLcell{}1 & \HLcell{}1 & \HLcell{}0 & \HLcell{}0 & \HLcell{}1 & \HLcell{}1 \\
 \text{2+3 = 5 $\equiv $ 1 (mod 4)} & 1 & 0 & 1 & 1 & 0 & 1 \\
 \text{2+2 = 4 $\equiv $ 0 (mod 4)} & 1 & 0 & 1 & 0 & 0 & 0 \\
 \text{\HLcell{}2+1 = 3 $\equiv $ 3 (mod 4)} & \HLcell{}1 & \HLcell{}0 & \HLcell{}0 & \HLcell{}1 & \HLcell{}1 & \HLcell{}1 \\
 \text{2+0 = 2 $\equiv $ 2 (mod 4)} & 1 & 0 & 0 & 0 & 1 & 0 \\
 \text{1+3 = 4 $\equiv $ 0 (mod 4)} & 0 & 1 & 1 & 1 & 0 & 0 \\
 \text{\HLcell{}1+2 = 3 $\equiv $ 3 (mod 4)} & \HLcell{}0 & \HLcell{}1 & \HLcell{}1 & \HLcell{}0 & \HLcell{}1 & \HLcell{}1 \\
 \text{1+1 = 2 $\equiv $ 2 (mod 4)} & 0 & 1 & 0 & 1 & 1 & 0 \\
 \text{1+0 = 1 $\equiv $ 1 (mod 4)} & 0 & 1 & 0 & 0 & 0 & 1 \\
 \text{\HLcell{}0+3 = 3 $\equiv $ 3 (mod 4)} & \HLcell{}0 & \HLcell{}0 & \HLcell{}1 & \HLcell{}1 & \HLcell{}1 & \HLcell{}1 \\
 \text{0+2 = 2 $\equiv $ 2 (mod 4)} & 0 & 0 & 1 & 0 & 1 & 0 \\
 \text{0+1 = 1 $\equiv $ 1 (mod 4)} & 0 & 0 & 0 & 1 & 0 & 1 \\
 \text{0+0 = 0 $\equiv $ 0 (mod 4)} & 0 & 0 & 0 & 0 & 0 & 0 \\
\end{array}\)
\end{doublespace}
\end{center}


Посмотрим, что об этом скажет SAT-солвер?

В начале, нам нужно представить наш двухбитный сумматор как CNF-выражение.

Используя Wolfram Mathematica, можно выразить 1-битное выражение для обоих выходов сумматоров:\\
\\
\textbf{\texttt{In[]:=AdderQ0[aL$\_$,bL$\_$]=Xor[aL,bL]}} \\
\textbf{\texttt{Out[]:=aL $\veebar$ bL}} \\
\\
\textbf{\texttt{In[]:=AdderQ1[aL$\_$,aH$\_$,bL$\_$,bH$\_$]=Xor[And[aL,bL],Xor[aH,bH]]}} \\
\textbf{\texttt{Out[]:=aH $\veebar$ bH $\veebar$ (aL \&\& bL)}} \\
\\
Нам нужно такое выражение, где обе части выдадут единицы.
Используя Wolfram Mathematica, найдем все возможные входы такого выражения (я склеил обе части при помощи And): \\
\\
\textbf{\texttt{In[]:=Boole[SatisfiabilityInstances[And[AdderQ0[aL,bL],AdderQ1[aL,aH,bL,bH]],\{aL,aH,bL,bH\},4]]}} \\
\textbf{\texttt{Out[]:=\{1,1,0,0\},\{1,0,0,1\},\{0,1,1,0\},\{0,0,1,1\}}} \\
\\
Да, действительно, Mathematica говорит, что здесь 4 входа, которые приведут к нужному нам результату.
Так что, Mathematica тоже может использоваться как \ac{SAT}-солвер.

Тем не менее, перейдем к CNF-форме. Используя Mathematica, сконвертируем наше выражение в CNF-форму:\\
\\
\textbf{\texttt{In[]:=cnf=BooleanConvert[And[AdderQ0[aL,bL],AdderQ1[aL,aH,bL,bH]],``CNF'']}} \\
\textbf{\texttt{Out[]:=(!aH $\|$ !bH) \&\& (aH $\|$ bH) \&\& (!aL $\|$ !bL) \&\& (aL $\|$ bL)}} \\
\\
Выглядит посложнее. Причина такой многословности в том, что \ac{CNF}-форма не поддерживает операцию исключающего
ИЛИ.
% FIXME: TeX form of the expression!

\subsubsection{MiniSat}

Для начала, попробуем MiniSat\footnote{\url{http://minisat.se/MiniSat.html}}.
Стандартный способ закодировать \ac{CNF}-выражение для MiniSat это перечислить все части ИЛИ в каждой строке.
Также, MiniSat не поддерживает имена переменных, только числа.
Перечислим наши переменные: 1 будет aH, 2 -- aL, 3 -- bH, 4 -- bL.

Вот что получилось, когда я сконвертировал выражение из Mathematica во входной файл для MiniSat:

\lstinputlisting{SAT/adder.cnf}

Две четверки в первой строке это, соответственно, число переменных и число клозов.
Так что тут 4 строки, каждая для каждого клоза ИЛИ.
Минус перед номером переменной означает что переменная инвертирована.
Отсутствие минуса -- не инвертирована.
Ноль в конце это просто оконечивающий ноль, означающий конец клоза.

Другими словами, каждая строка это ИЛИ-клоз с возможными инвертированиями,
и задача MiniSat в том, чтобы найти такой набор входных переменных, который удовлетворит все строки во входном файле.

Этот файл я назвал \textit{adder.cnf} и теперь попробуем MiniSat:

\begin{lstlisting}
% minisat -verb=0 adder.cnf results.txt
SATISFIABLE
\end{lstlisting}

Результаты в файле \textit{results.txt}:

\begin{lstlisting}
SAT
-1 -2 3 4 0
\end{lstlisting}

Это означает, что если первые две переменных (aH и aL) будут \textit{false},
и две последние переменные (bH и bL) будут \textit{true},
все \ac{CNF}-выражение будет истинно (satisfiable).
Похоже на правду: если bH и bL выставить в \textit{true}, оба бита результата также будут \textit{true}.

Как получить другие решения (instances)?
\ac{SAT}-солверы, как и \ac{SMT}-солверы, выдают только одно решение (или \textit{instance}).

MiniSat использует \ac{PRNG}, и его изначальное состояние (seed) можно задать явно.
Я попробовал разные значения, но результат всё тот же.
Тем не менее, CryptoMiniSat в этом случае может показать все возможные 4 решения, хотя и в хаотичном порядке.
Так что это не очень надежный способ.

Видимо, единственный способ, это инвертировать клоз решения и добавить его во входное выражение.
Мы получили \TT{-1 -2 3 4}, 
теперь мы можем инвертировать все значения в нем (просто поменяйте минусы: \TT{1 2 -3 -4}),
и добавим это в конец входного файла:

\begin{lstlisting}
p cnf 4 5
-1 -3 0
1 3 0
-2 -4 0
2 4 0
1 2 -3 -4
\end{lstlisting}

Получаем другой результат:

\begin{lstlisting}
SAT
1 2 -3 -4 0
\end{lstlisting}

Это означает что обе aH и aL должны быть \textit{true} и bH и bL должны быть \textit{false}, чтобы удовлетворить
входное выражение.
Снова инвертируем это решение и снова добавим:

\begin{lstlisting}
p cnf 4 6
-1 -3 0
1 3 0
-2 -4 0
2 4 0
1 2 -3 -4
-1 -2 3 4 0
\end{lstlisting}

Результат:

\begin{lstlisting}
SAT
-1 2 3 -4 0
\end{lstlisting}

aH=false, aL=true, bH=true, bL=false. Это также корректно, в соответствии с таблицей истинности.

Добавим снова:

\begin{lstlisting}
p cnf 4 7
-1 -3 0
1 3 0
-2 -4 0
2 4 0
1 2 -3 -4
-1 -2 3 4 0
1 -2 -3 4 0
\end{lstlisting}

\begin{lstlisting}
SAT
1 -2 -3 4 0
\end{lstlisting}

\textit{aH=true, aL=false, bH=false, bL=true.} Это тоже верно.

Это четвертый результат. Больше быть не должно. Что если добавим и это?

\begin{lstlisting}
p cnf 4 8
-1 -3 0
1 3 0
-2 -4 0
2 4 0
1 2 -3 -4
-1 -2 3 4 0
1 -2 -3 4 0
-1 2 3 -4 0
\end{lstlisting}

Теперь MiniSat просто говорит ``UNSATISFIABLE'' без всякой дополнительной информации в файле результатов.

Нам пример крохотный, но MiniSat может работать с огромными \ac{CNF}-выражениями.

\subsubsection{CryptoMiniSat}

Операция исключающего ИЛИ (XOR) отсутствует в CNF-форме, но она очень важна в криптографических алгоритмах.
Простейший способ представить одну единственную XOR-операцию в CNF-форме, это:
$(\neg x \vee \neg y) \wedge (x \vee y)$ -- не очень короткое выражение,
хотя, множество XOR-операций в одном выражении могут оптимизироваться лучше.

Одна значительная разница между MiniSat и CryptoMiniSat в том, что последний поддерживает
клозы с операцией XOR вместо ИЛИ,
потому что CryptoMiniSat предназначен больше для анализа криптоалгоритмов\footnote{\url{http://www.msoos.org/xor-clauses/}}.
XOR-клозы поддерживаются в CryptoMiniSat специальным образом, без трансляции в клозы ИЛИ.

Вам нужно просто прибавить ``x'' к клозу в \ac{CNF}-файле и CryptoMiniSat затем считает обычный ИЛИ-клоз как XOR-клоз.
Что до двухбитного сумматора, вот самое короткое из возможных XOR-CNF выражений, которое можно использовать
для поиска всех входных значений, где оба выходных бита выставлены:

$(aH \oplus bH) \wedge (aL \oplus bL)$

Это \TT{.cnf}-файл CryptoMiniSat:

\begin{lstlisting}
p cnf 4 2
x1 3 0
x2 4 0
\end{lstlisting}

Запускаю CryptoMiniSat с разными значениями для инициализации его \ac{PRNG} \dots

\begin{lstlisting}
% cryptominisat4 --verb 0 --random 0 XOR_adder.cnf
s SATISFIABLE
v 1 2 -3 -4 0
% cryptominisat4 --verb 0 --random 1 XOR_adder.cnf
s SATISFIABLE
v -1 -2 3 4 0
% cryptominisat4 --verb 0 --random 2 XOR_adder.cnf
s SATISFIABLE
v 1 -2 -3 4 0
% cryptominisat4 --verb 0 --random 3 XOR_adder.cnf
s SATISFIABLE
v 1 2 -3 -4 0
% cryptominisat4 --verb 0 --random 4 XOR_adder.cnf
s SATISFIABLE
v -1 2 3 -4 0
% cryptominisat4 --verb 0 --random 5 XOR_adder.cnf
s SATISFIABLE
v -1 2 3 -4 0
% cryptominisat4 --verb 0 --random 6 XOR_adder.cnf
s SATISFIABLE
v -1 -2 3 4 0
% cryptominisat4 --verb 0 --random 7 XOR_adder.cnf
s SATISFIABLE
v 1 -2 -3 4 0
% cryptominisat4 --verb 0 --random 8 XOR_adder.cnf
s SATISFIABLE
v 1 2 -3 -4 0
% cryptominisat4 --verb 0 --random 9 XOR_adder.cnf
s SATISFIABLE
v 1 2 -3 -4 0
\end{lstlisting}

Тем не менее, все 4 возможных решения, это:

\begin{lstlisting}
v -1 -2 3 4 0
v -1 2 3 -4 0
v 1 -2 -3 4 0
v 1 2 -3 -4 0
\end{lstlisting}

\dots то же, что и выдал MiniSat.

\subsection{Picosat}

По крайней мере Picosat может перечислить все возможные решения без тех костылей, которые я только что показывал:

\begin{lstlisting}
% picosat --all adder.cnf
s SATISFIABLE
v -1 -2 3 4 0
s SATISFIABLE
v -1 2 3 -4 0
s SATISFIABLE
v 1 2 -3 -4 0
s SATISFIABLE
v 1 -2 -3 4 0
s SOLUTIONS 4
\end{lstlisting}

% subsections:
\subsection{Головоломка о восьми ферзях}
\label{EightQueens}

Восемь ферзей это популярная головоломка, и она часто используется для измерения скорости работы SAT-солверов.
Нужно расставить на шахматной доске 8 ферзей так, чтобы они не атаковали друг друга.
Например:

\begin{lstlisting}
| | | |*| | | | |
| | | | | | |*| |
| | | | |*| | | |
| |*| | | | | | |
| | | | | |*| | |
|*| | | | | | | |
| | |*| | | | | |
| | | | | | | |*|
\end{lstlisting}

Попробуем разобраться, как её решить.

\subsubsection{POPCNT1}
\label{POPCNTOne}

Одна важная ф-ция, которую мы будем (часто) использовать это \TT{POPCNT1}.
Это ф-ция, которая возвращает \textit{Истинно}, если один из входов истинен, остальные ложны.
Она вернет \textit{Ложно} в остальных случаях.

В моих других примерах, я использовал Wolfram Mathematica для генерирования CNF-клозов для этого, например: \ref{minesweeper_SAT}.
Какое выражение сгенерирует Mathematica для ф-ции \TT{POPCNT1} для 8-и входов?

\begin{lstlisting}
(!a||!b)&&(!a||!c)&&(!a||!d)&&(!a||!e)&&(!a||!f)&&(!a||!g)&&(!a||!h)&&(a||b||c||d||e||f||g||h)&&
(!b||!c)&&(!b||!d)&&(!b||!e)&&(!b||!f)&&(!b||!g)&&(!b||!h)&&(!c||!d)&&(!c||!e)&&(!c||!f)&&(!c||!g)&&
(!c||!h)&&(!d||!e)&&(!d||!f)&&(!d||!g)&&(!d||!h)&&(!e||!f)&&(!e||!g)&&(!e||!h)&&(!f||!g)&&(!f||!h)&&(!g||!h)
\end{lstlisting}

Мы можем ясно увидеть что выражение состоит из всех возможных пар переменных (инвертированных) плюс
перечисление всех переменных (не инвертированных).
На обычном русском языке это означает: ``ни одна пара не должна быть равна двум \textit{Истинно} \textit{И}
по крайней мере одно \textit{Истинно} должно
присутствовать среди переменных''.

Вот как это работает: если две переменных будут \textit{Истино}, инвертированными они обе будут \textit{Ложно},
и этот клоз не будет
вычислен как \textit{Истинно}, а это наша конечная цель.
Если одна из переменных будет \textit{Истинно}, инвертированными, одна будет \textit{Истинно},
вторая \textit{Ложно} (хорошо).
Если обе переменных будут \textit{Ложно}, инвертированными, они обе будут \textit{Истинно} (тоже хорошо).

Вот как мы можем сгенерировать клозы для этой ф-ции используя модуль \textit{itertools} из Питона,
который также содержит много важных ф-ций из комбинаторики:

\begin{lstlisting}
    # naive/pairwise encoding   
    def AtMost1(self, lst):
        for pair in itertools.combinations(lst, r=2):
            self.add_clause([self.neg(pair[0]), self.neg(pair[1])])
        
    def POPCNT1(self, lst):
        self.AtMost1(lst)
        self.OR_always(lst)
\end{lstlisting}

Ф-ция \TT{AtMost1()} перечисляет все возможные пары используя ф-цию \textit{combinations()} из
\textit{itertools}.

Ф-ция \TT{POPCNT1()} делает то же самое, только добавляет последний клоз, который заставляет иметь хотя бы одну
переменную, равную Истинно.

Какие клозы будут сгенерированы для 5-и переменных (1..5)?

\lstinputlisting{SAT/8queens/popcnt1.cnf}

Да, это все возможные пары чисел 1..5 + все 5 чисел.

Можем посмотреть все решения используя Picosat:

\begin{lstlisting}
% picosat --all popcnt1.cnf
s SATISFIABLE
v -1 -2 -3 -4 5 0
s SATISFIABLE
v -1 -2 -3 4 -5 0
s SATISFIABLE
v -1 -2 3 -4 -5 0
s SATISFIABLE
v -1 2 -3 -4 -5 0
s SATISFIABLE
v 1 -2 -3 -4 -5 0
s SOLUTIONS 5
\end{lstlisting}

Действительно, 5 возможных решений.

\subsubsection{Восемь ферзей}

Теперь вернемся назад к восьми ферзям.

Мы можем назначить 64 переменных для $8 \cdot 8=64$ клеток.
Клетка на которой есть ферзь будет равна \textit{Истинно}, пустая клетка будет \textit{Ложно}.

Проблема расположения неатакующих (друг друга) ферзей на шахматной доске (любого размера), на обычном русском
языке может быть выражена так:

\begin{itemize}
\item один единственный ферзь должен присутствовать в каждом ряду;

\item один единственный ферзь должен присутствовать в каждом столбце;

\item или один ферзь должен присутствовать на каждой диагонали, или вовсе отсутствовать (пустые диагонали могут быть
и в правильном решении).
\end{itemize}

Эти правила можно перевести так:

\begin{itemize}
\item POPCNT1(каждый ряд)==\textit{Истинно}

\item POPCNT1(каждый столбец)==\textit{Истинно}

\item AtMost1(каждая диагональ)==\textit{Истинно}
\end{itemize}

Теперь мы должны перечислить ряды, столбцы и диагонали, и собрать все клозы:

\lstinputlisting{SAT/8queens/8queens.py}
( \url{https://github.com/dennis714/SAT_SMT_article/blob/master/SAT/8queens/8queens.py} )

Возможно, ф-ция \TT{gen\_diagonal()} выглядит не очень эстетично:
она перечисляет также поддиагонали более длинных диагоналей, которые уже были раннее.
Чтобы не было повторяющихся клозов, глобальная переменная \textit{clauses} это не список, а множество,
которое может содержать в себе только уникальные данные.

Также, я использовал ф-цию \TT{AtMost1} для каждого столбца, это поможет генерировать чуть меньше клозов.
Каждый столбец будет содержать ферзя в любом случае, это следует из первого правила (\TT{POPCNT1} для каждого ряда).

После запуска, получаем CNF-файл с 64-я переменными и 736-я клозами (\url{https://github.com/dennis714/SAT_SMT_article/blob/master/SAT/8queens/8queens.cnf}).
Вот одно из решений:

\begin{lstlisting}
% python 8queens.py
len(clauses)= 736
| | | |*| | | | |
| | | | | | |*| |
| | | | |*| | | |
| |*| | | | | | |
| | | | | |*| | |
|*| | | | | | | |
| | |*| | | | | |
| | | | | | | |*|
\end{lstlisting}

Как много здесь возможных решений?
Picosat говорит что 92, что действительно корректное число решений (\url{https://oeis.org/A000170}).

Скорость Picosat не очень впечатляет, вероятно потому что ему приходится выводить все решения.
Моему древнему нетбуку с Intel Atom 1.66GHz, понадобилось 34 для перечисления всех решений для шахматной доски
$11 \cdot 11$ 
(2680 решения),
что намного медленнее, чем моя прямолинейная программа полного перебора: \url{https://yurichev.com/blog/8queens/}.
Тем не менее, для поиска первого решения, Picosat работает крайне быстро (как и другие SAT-солверы).

Эта SAT-задача также достаточно проста, чтобы её можно было легко решить при помощи моего простейшего
SAT-солвера, работающего на базе поиска с возвратом (\textit{backtracking}):
\ref{SAT_backtrack}.

\subsubsection{Подсчет всех решений}

Мы получаем решение, инвертируем его и добавляем как новый констрайнт.
На обычном русском языке это звучит ``найди решение, котороые также не ровно тому, что мы только что нашли/добавили''.
Мы добавляем их последовательно, и процесс замедляется --- потому что размер проблемы (\textit{instance}) растет 
и SAT-солверу всё труднее находить новое решение.

\subsubsection{Пропуск симметрических решений}

Мы также можем добавлять повернутое и отраженное (горизонтально) решение, чтобы пропускать симметрические решения.
Делая так, мы получаем 12 решений для доски 8*8, 46 для 9*9, итд.
Это \url{https://oeis.org/A002562}.


\subsection{Sudoku в SAT}
\label{Sudoku_SAT}

Кто-то может подумать, что мы можем закодировать каждое число 1..9 в двоичном виде: 5 бит или переменных было бы достаточно.
Но есть даже еще более простой способ: выделить 9 бит, где только один бит будет \textit{Истинен}.
Число 1 может быт закодировано как [1, 0, 0, 0, 0, 0, 0, 0, 0], число 3 как [0, 0, 1, 0, 0, 0, 0, 0, 0], итд.
Выглядит неэкономично? Да, но другие операции будут проще.

Прежде всего, мы будем снова использовать важную ф-цию \TT{POPCNT1}, которую я описывал раннее: \ref{POPCNTOne}.

Вторая нужная нам важная операция, которую нам нужно придумать, это как сделать 9 чисел уникальными.
Если каждое число закодировано как 9-битный вектор, 9 чисел могут сформировать матрицу, вроде:

\begin{lstlisting}
0 0 0 0 0 0 1 0 0 <- §1-е§ число
0 0 0 0 0 1 0 0 0 <- §2-е§ число
0 1 0 0 0 0 0 0 0 <- ...
0 0 1 0 0 0 0 0 0 <- ...
0 0 0 0 0 0 0 0 1 <- ...
0 0 0 0 1 0 0 0 0 <- ...
0 0 0 0 0 0 0 1 0 <- ...
1 0 0 0 0 0 0 0 0 <- ...
0 0 0 1 0 0 0 0 0 <- §9-е§ число
\end{lstlisting}

Теперь будем использовать ф-цию \TT{POPCNT1} чтобы сделать каждый ряд в матрице содержащим только один бит \textit{Истина},
и это будет сохранять корректность нашего способа кодирования, т.к., вектор не может иметь более одного бита \textit{Истина},
либо не иметь битов \textit{Истина} вообще.
Затем мы будем использовать ф-цию \TT{POPCNT1} снова чтобы сделать так, чтобы каждый столбец в матрице имел только
один единственный бит \textit{Истина}.
Это сделает все ряды в матрице уникальными, другими словами, все 9 закодированных чисел всегда будут уникальными.

После применения ф-ции \TT{POPCNT1} 9+9=18 раз, у нас будет 9 уникальных чисел в пределах 1..9.

Используя эту операцию мы можем сделать каждый ряд в головоломке Судоку уникальным, каждый столбец уникальным,
и каждый квадрат $3 \cdot 3=9$ тоже уникальным.

\lstinputlisting{SAT/sudoku/sudoku_SAT.py}
( \url{https://github.com/DennisYurichev/SAT_SMT_article/blob/master/SAT/sudoku/sudoku_SAT.py} )

Ф-ция \TT{make\_distinct\_bits\_in\_vector()} сохраняет корректность кодирования.\\
Ф-ция \TT{make\_distinct\_vectors()} делает 9 чисел уникальными.\\
Ф-ция \TT{cvt\_vector\_to\_number()} декодирует вектор в число.\\
Ф-ция \TT{number\_to\_vector()} кодирует число в вектор.\\
Ф-ция \TT{main()} содержит все необходимые вызовы, чтобы сделать уникальными ряды/столбцы и квадраты $3\cdot 3$.

Работает:

\begin{lstlisting}
% python sudoku_SAT.py
len(clauses)= 12195
1 4 5 3 2 7 6 9 8
8 3 9 6 5 4 1 2 7
6 7 2 9 1 8 5 4 3
4 9 6 1 8 5 3 7 2
2 1 8 4 7 3 9 5 6
7 5 3 2 9 6 4 8 1
3 6 7 5 4 2 8 1 9
9 8 4 7 6 1 2 3 5
5 2 1 8 3 9 7 6 4
\end{lstlisting}

Такое же решение как и раннее: \ref{sudoku_SMT}.

Picosat говорит что эта SAT-проблема имеет только одно решение.
Действительно, как говорят, настоящая головоломка Судоку может иметь только одно решение.

\subsubsection{Избавление от одного вызова POPCNT1}
\label{OR_in_POPCNT1}

Чтобы сделать 9 уникальный чисел 1..9 мы можем использовать ф-цию \TT{POPCNT1}, чтобы сделать уникальным каждый ряд
в матрице, и использовать операцию \textit{ИЛИ} для всех столбцов.
Это будет иметь такой же эффект: все ряды должны быть уникальны, чтобы каждый столбец вычислялся в \textit{Истино}
после применения операции \textit{ИЛИ} ко всем переменным в столбце.
(Я буду делать так в следующем примере: \ref{Zebra_SAT}.)

Это приведет к тому, что будет 3447 клозов вместо 12195, но почему-то, SAT-солверы работают медленнее. Не знаю, почему.


\subsection{Взлом Сапёра при помощи SAT}
\label{minesweeper_SAT}

См.также о взломе оного при помощи Z3: \ref{minesweeper_SMT}.

\subsubsection{Простая ф-ция подсчета бит (\textit{population count})}

Прежде всего, нам нужно как считать количество соседних бомб.
Ф-ция подсчета та же, что и ф-ция подсчета бит (\textit{population count}).

Мы можем создать \ac{CNF}-выражение используя Wolfram Mathematica.
Это будет ф-ция, возвращающая \textit{True} если любые из двух бит 8-битного входа равняются \textit{True},
а остальные --- \textit{False}.
В начале, сделаем таблицу истинности для такой ф-ции:

\begin{lstlisting}
In[]:= tbl2 = 
 Table[PadLeft[IntegerDigits[i, 2], 8] -> 
   If[Equal[DigitCount[i, 2][[1]], 2], 1, 0], {i, 0, 255}]

Out[]= {{0, 0, 0, 0, 0, 0, 0, 0} -> 0, {0, 0, 0, 0, 0, 0, 0, 1} -> 0, 
{0, 0, 0, 0, 0, 0, 1, 0} -> 0, {0, 0, 0, 0, 0, 0, 1, 1} -> 1, 
{0, 0, 0, 0, 0, 1, 0, 0} -> 0, {0, 0, 0, 0, 0, 1, 0, 1} -> 1, 
{0, 0, 0, 0, 0, 1, 1, 0} -> 1, {0, 0, 0, 0, 0, 1, 1, 1} -> 0, 
{0, 0, 0, 0, 1, 0, 0, 0} -> 0, {0, 0, 0, 0, 1, 0, 0, 1} -> 1, 
{0, 0, 0, 0, 1, 0, 1, 0} -> 1, {0, 0, 0, 0, 1, 0, 1, 1} -> 0, 
...
{1, 1, 1, 1, 1, 0, 1, 0} -> 0, {1, 1, 1, 1, 1, 0, 1, 1} -> 0, 
{1, 1, 1, 1, 1, 1, 0, 0} -> 0, {1, 1, 1, 1, 1, 1, 0, 1} -> 0, 
{1, 1, 1, 1, 1, 1, 1, 0} -> 0, {1, 1, 1, 1, 1, 1, 1, 1} -> 0}
\end{lstlisting}

Теперь можем сделать \ac{CNF}-выражение используя эту таблицу истинности:

\begin{lstlisting}
In[]:= BooleanConvert[
 BooleanFunction[tbl2, {a, b, c, d, e, f, g, h}], "CNF"]

Out[]= (! a || ! b || ! c) && (! a || ! b || ! d) && (! a || ! 
    b || ! e) && (! a || ! b || ! f) && (! a || ! b || ! g) && (! 
    a || ! b || ! h) && (! a || ! c || ! d) && (! a || ! c || ! 
    e) && (! a || ! c || ! f) && (! a || ! c || ! g) && (! a || ! 
    c || ! h) && (! a || ! d || ! e) && (! a || ! d || ! f) && (! 
    a || ! d || ! g) && (! a || ! d || ! h) && (! a || ! e || ! 
    f) && (! a || ! e || ! g) && (! a || ! e || ! h) && (! a || ! 
    f || ! g) && (! a || ! f || ! h) && (! a || ! g || ! h) && (a || 
   b || c || d || e || f || g) && (a || b || c || d || e || f || 
   h) && (a || b || c || d || e || g || h) && (a || b || c || d || f ||
    g || h) && (a || b || c || e || f || g || h) && (a || b || d || 
   e || f || g || h) && (a || c || d || e || f || g || 
   h) && (! b || ! c || ! d) && (! b || ! c || ! e) && (! b || ! 
    c || ! f) && (! b || ! c || ! g) && (! b || ! c || ! h) && (! 
    b || ! d || ! e) && (! b || ! d || ! f) && (! b || ! d || ! 
    g) && (! b || ! d || ! h) && (! b || ! e || ! f) && (! b || ! 
    e || ! g) && (! b || ! e || ! h) && (! b || ! f || ! g) && (! 
    b || ! f || ! h) && (! b || ! g || ! h) && (b || c || d || e || 
   f || g || 
   h) && (! c || ! d || ! e) && (! c || ! d || ! f) && (! c || ! 
    d || ! g) && (! c || ! d || ! h) && (! c || ! e || ! f) && (! 
    c || ! e || ! g) && (! c || ! e || ! h) && (! c || ! f || ! 
    g) && (! c || ! f || ! h) && (! c || ! g || ! h) && (! d || ! 
    e || ! f) && (! d || ! e || ! g) && (! d || ! e || ! h) && (! 
    d || ! f || ! g) && (! d || ! f || ! h) && (! d || ! g || ! 
    h) && (! e || ! f || ! g) && (! e || ! f || ! h) && (! e || ! 
    g || ! h) && (! f || ! g || ! h)
\end{lstlisting}

Синтаксис такой же как и в Си/Си++
Проверим.

Я написал Питоновскую ф-цию для конвертирования вывода Mathematica в \ac{CNF}-файл, который можно подать на вход
SAT-солверу:

\lstinputlisting{SAT/minesweeper/tst.py}

Она заменяет переменные a/b/c/... на переданные имена переменных (1/2/3...), перерабатыает синтаксис, итд.
Here is a result:

\lstinputlisting{SAT/minesweeper/tst1.cnf}

Могу запустить:

\begin{lstlisting}
% minisat -verb=0 tst1.cnf results.txt
SATISFIABLE

% cat results.txt
SAT
1 -2 -3 -4 -5 -6 -7 8 0
\end{lstlisting}

Имя переменной в результате без знака минуса, это \textit{True}.
Имя переменной со знаком минус, это \textit{False}.
Мы здесь видим только две переменных \textit{True}: 1 и 8.
Это действительно корректно: солвер MiniSat нашел условие, для которого наша ф-ция возвращает \textit{True}.
Ноль в конце это просто терминирующий символ, который ничего не означает.

Мы можем попросить MiniSat найти еще одно решение, добавив текущее решение во входной CNF-файл,
но где все переменные инвертированы:

\begin{lstlisting}
...
-5 -6 -8 0
-5 -7 -8 0
-6 -7 -8 0
-1 2 3 4 5 6 7 -8 0
\end{lstlisting}

В обычном русском языке, это означает ``дайте ЛЮБОЕ решение, которые удовлетворяет все клозы, но также не равно
последнему клозу, которое мы только что добавили''.

MiniSat, действительно, находит еще одно решение, и снова, только с двумя переменными, равными \textit{True}:

\begin{lstlisting}
% minisat -verb=0 tst2.cnf results.txt
SATISFIABLE

% cat results.txt
SAT
1 2 -3 -4 -5 -6 -7 -8 0
\end{lstlisting}

Кстати, ф-ция \textit{population count} для 8-и соседей (POPCNT8) в CNF-форме, самая простая:

\begin{lstlisting}
a&&b&&c&&d&&e&&f&&g&&h
\end{lstlisting}

Действительно: она истинна, если все 8 входных бит тоже истинны.

Ф-ция для отсутствия соседей (POPCNT0) тоже очень простая:

\begin{lstlisting}
!a&&!b&&!c&&!d&&!e&&!f&&!g&&!h
\end{lstlisting}

Это означает, что она вернет \textit{True}, если все входные переменные \textit{False}.

Кстати, ф-ция POPCNT1 тоже простая:

\begin{lstlisting}
(!a||!b)&&(!a||!c)&&(!a||!d)&&(!a||!e)&&(!a||!f)&&(!a||!g)&&(!a||!h)&&(a||b||c||d||e||f||g||h)&&
(!b||!c)&&(!b||!d)&&(!b||!e)&&(!b||!f)&&(!b||!g)&&(!b||!h)&&(!c||!d)&&(!c||!e)&&(!c||!f)&&(!c||!g)&&
(!c||!h)&&(!d||!e)&&(!d||!f)&&(!d||!g)&&(!d||!h)&&(!e||!f)&&(!e||!g)&&(!e||!h)&&(!f||!g)&&(!f||!h)&&(!g||!h)
\end{lstlisting}

Здесь просто перечисление всех возможных пар 8-и переменных
(a/b, a/c, a/d, итд), что подразумевает: не должно присутствовать одновременно двух бит в каждой возможной паре.
И еще один клоз: ``(a||b||c||d||e||f||g||h)'', что подразумевает: минимум один бит должен присутствовать
среди 8-и переменных.

И да, вы можете использовать Mathematica для поиска \ac{CNF}-выражения для любой другой таблицы истинности.

\subsubsection{Сапёр}

Теперь можем использовать Mathematica для генерации всех ф-ций \textit{population count} для количества соседей 0..8.

Для Сапёра с матрицей $9 \cdot 9$ включая невидимую рамку, здесь будет $11 \cdot 11=121$ переменных,
связанных с матрицей Сапёра вот так:

\begin{lstlisting}
 1    2   3   4   5   6   7   8   9  10  11
12   13  14  15  16  17  18  19  20  21  22
23   24  25  26  27  28  29  30  31  32  33
34   35  36  37  38  39  40  41  42  43  44

...

100 101 102 103 104 105 106 107 108 109 110
111 112 113 114 115 116 117 118 119 120 121
\end{lstlisting}

Потом мы пишем Питоновский скрипт, складывающий все ф-ции \textit{population count}:
каждая ф-ция для каждого известного числа соседей (число на поле Сапёра).
Каждая ф-ция POPCNTx() берет на вход список переменных и выдает список клозов, которые будут добавлены
в итоговый \ac{CNF}-файл.

Что до пустых клеток, мы тоже добавляем их как клозы, но со знаком минус, что означает, что переменная
должна быть \textit{False}.
А когда мы пытаемся поместить бомбу, мы добавляем её переменную как клоз без знака минуса, что означает
что переменная должна быть \textit{True}.

Затем запускаем внешний процесс minisat.
Всё что нам от него нужно, это код возврата.
Если входной \ac{CNF} это \TT{UNSAT}, он возвращает 20:

Мы также используем здесь информацию из предыдущего решения Сапёра: \ref{minesweeper_SMT}.

\lstinputlisting{SAT/minesweeper/minesweeper_SAT.py}

( \url{https://github.com/dennis714/SAT_SMT_article/blob/master/SAT/minesweeper/minesweeper_SAT.py} ) \\
\\
Выходной \ac{CNF}-файл большой, вплоть до $\approx 2000$ клозов, и даже больше, вот, например: \url{https://github.com/dennis714/SAT_SMT_article/blob/master/SAT/minesweeper/sample.cnf}.

Так или иначе, это работает так же, как мой предыдущий скрипт для Z3Py:

\begin{lstlisting}
row=1, col=3, unsat!
row=6, col=2, unsat!
row=6, col=3, unsat!
row=7, col=4, unsat!
row=7, col=9, unsat!
row=8, col=9, unsat!
\end{lstlisting}

\dots но работает намного быстрее, даже учитывая запуск внешней программы.
Вероятно, версию для Z3Py можно было бы оптимизировать получше?

Файлы, включая файл для Wolfram Mathematica: \url{https://github.com/dennis714/SAT_SMT_article/tree/master/SAT/minesweeper}.


% TODO translate src
\subsection{Головоломка Зебры как SAT-проблема}
\label{Zebra_SAT}

Попробуем решить головоломку Зебры (\ref{zebra_SMT}) в SAT.

Я определю каждую переменную как вектор из пяти переменных, как я делал это раннее в солвере Судоку: \ref{Sudoku_SAT}.

Я также использую ф-цию \TT{POPCNT1}, но в отличие от предыдущего примера,
я использовал Wolfram Mathematica для генерирования её в CNF-форме:

\begin{lstlisting}
In[]:= tbl1=Table[PadLeft[IntegerDigits[i,2],5] ->If[Equal[DigitCount[i,2][[1]],1],1,0],{i,0,63}]
Out[]= {{0,0,0,0,0}->0,
{0,0,0,0,1}->1,
{0,0,0,1,0}->1,
{0,0,0,1,1}->0,
{0,0,1,0,0}->1,
{0,0,1,0,1}->0,

...

{1,1,1,1,0}->0,
{1,1,1,1,1}->0}

In[]:= BooleanConvert[BooleanFunction[tbl1,{a,b,c,d,e}],"CNF"]
Out[]= (!a||!b)&&(!a||!c)&&(!a||!d)&&(!a||!e)&&(a||b||c||d||e)&&(!b||!c)&&(!b||!d)&&(!b||!e)&&(!c||!d)&&(!c||!e)&&(!d||!e)
\end{lstlisting}

Также, как я предлагал раньше (\ref{OR_in_POPCNT1}), я использовал операцию \textit{ИЛИ} для второго шага.

\begin{lstlisting}
def mathematica_to_CNF (s, d):
    for k in d.keys():
        s=s.replace(k, d[k])
    s=s.replace("!", "-").replace("||", " ").replace("(", "").replace(")", "")
    s=s.split ("&&")
    return s

def add_popcnt1(v1, v2, v3, v4, v5):
    global clauses
    s="(!a||!b)&&" \
      "(!a||!c)&&" \
      "(!a||!d)&&" \
      "(!a||!e)&&" \
      "(!b||!c)&&" \
      "(!b||!d)&&" \
      "(!b||!e)&&" \
      "(!c||!d)&&" \
      "(!c||!e)&&" \
      "(!d||!e)&&" \
      "(a||b||c||d||e)"

    clauses=clauses+mathematica_to_CNF(s, {"a":v1, "b":v2, "c":v3, "d":v4, "e":v5})

...

# k=tuple: ("high-level" variable name, number of bit (0..4))
# v=variable number in CNF
vars={}
vars_last=1

...

def alloc_distinct_variables(names):
    global vars
    global vars_last
    for name in names:
        for i in range(5):
            vars[(name,i)]=str(vars_last)
            vars_last=vars_last+1

        add_popcnt1(vars[(name,0)], vars[(name,1)], vars[(name,2)], vars[(name,3)], vars[(name,4)])

    # make them distinct:
    for i in range(5):
        clauses.append(vars[(names[0],i)] + " " + vars[(names[1],i)] + " " + vars[(names[2],i)] + " " + vars[(names[3],i)] + " " + vars[(names[4],i)])

...

alloc_distinct_variables(["Yellow", "Blue", "Red", "Ivory", "Green"])
alloc_distinct_variables(["Norwegian", "Ukrainian", "Englishman", "Spaniard", "Japanese"])
alloc_distinct_variables(["Water", "Tea", "Milk", "OrangeJuice", "Coffee"])
alloc_distinct_variables(["Kools", "Chesterfield", "OldGold", "LuckyStrike", "Parliament"])
alloc_distinct_variables(["Fox", "Horse", "Snails", "Dog", "Zebra"])

...

\end{lstlisting}

Теперь у нас пять булевых переменных для каждой \textit{высокоуровневной} переменной,
и каждая группа переменных гарантированно будет иметь разные значения.

Теперь перечитаем условие головоломки: ``2. Англичанин живёт в красном доме.''.
Это легко.
В моих примерах на Z3 и KLEE я просто написал ``Englishman==Red''.
Та же история и здесь: мы просто добавляем клозы, показывающие, что 5 булевых переменных для ``Englishman''
должны равняться пяти переменных для ``Red''.

На самом низком уровне CNF, если мы хотим сказать, что две переменных должны равняться друг другу,
мы добавляем два клоза:

$(var1 \vee \neg var2) \wedge (\neg var1 \vee var2)$

Это означает что значения обоих \textit{var1} и \textit{var2} должны быть или \textit{Ложно} или \textit{Истинно},
но они не могут быть разными.

\begin{lstlisting}
def add_eq_clauses(var1, var2):
    global clauses
    clauses.append(var1 + " -" + var2)
    clauses.append("-"+var1 + " " + var2)

def add_eq (n1, n2):
    for i in range(5):
        add_eq_clauses(vars[(n1,i)], vars[(n2, i)])

...

# 2.The Englishman lives in the red house.
add_eq("Englishman","Red")

# 3.The Spaniard owns the dog.
add_eq("Spaniard","Dog")

# 4.Coffee is drunk in the green house.
add_eq("Coffee","Green")

...

\end{lstlisting}

Теперь следующие условия:
``9. В центральном доме пьют молоко.'' (т.е., в третьем доме), ``10. Норвежец живёт в первом доме.''
Мы можем присвоить булевы значения напрямую:

\begin{lstlisting}
# n=1..5
def add_eq_var_n (name, n):
    global clauses
    global vars
    for i in range(5):
        if i==n-1:
            clauses.append(vars[(name,i)]) # always True
        else:
            clauses.append("-"+vars[(name,i)]) # always False

...

# 9.Milk is drunk in the middle house.
add_eq_var_n("Milk",3) # i.e., 3rd house

# 10.The Norwegian lives in the first house.
add_eq_var_n("Norwegian",1)
\end{lstlisting}

Для ``Milk'' у нас значение ``0 0 1 0 0'', для ``Norwegian'': ``1 0 0 0 0''.

Что делать с этим?
``6. Зелёный дом стоит сразу справа от белого дома.''
Я могу сконструировать такое условие:

\begin{lstlisting}
    Ivory      Green
AND(1 0 0 0 0  0 1 0 0 0)
.. OR ..
AND(0 1 0 0 0  0 0 1 0 0)
.. OR ..
AND(0 0 1 0 0  0 0 0 1 0)
.. OR ..
AND(0 0 0 1 0  0 0 0 0 1)
\end{lstlisting}

Для ``белого/ivory'' тут нет ``0 0 0 0 1'', потому что он не может быть последним.
Теперь я конвертирую эти условия в CNF при помощи Wolfram Mathematica:

\begin{lstlisting}
In[]:= BooleanConvert[(a1&& !b1&&!c1&&!d1&&!e1&&!a2&& b2&&!c2&&!d2&&!e2) ||
(!a1&& b1&&!c1&&!d1&&!e1&&!a2&& !b2&&c2&&!d2&&!e2) ||
(!a1&& !b1&&c1&&!d1&&!e1&&!a2&& !b2&&!c2&&d2&&!e2) ||
(!a1&& !b1&&!c1&&d1&&!e1&&!a2&& !b2&&!c2&&!d2&&e2) ,"CNF"]

Out[]= (!a1||!b1)&&(!a1||!c1)&&(!a1||!d1)&&(a1||b1||c1||d1)&&!a2&&(!b1||!b2)&&(!b1||!c1)&&
(!b1||!d1)&&(b1||b2||c1||d1)&&(!b2||!c1)&&(!b2||!c2)&&(!b2||!d1)&&(!b2||!d2)&&(!b2||!e2)&&
(b2||c1||c2||d1)&&(b2||c2||d1||d2)&&(b2||c2||d2||e2)&&(!c1||!c2)&&(!c1||!d1)&&(!c2||!d1)&&
(!c2||!d2)&&(!c2||!e2)&&(!d1||!d2)&&(!d2||!e2)&&!e1
\end{lstlisting}

И вот фрагмент моего кода на Питоне:

\begin{lstlisting}
def add_right (n1, n2):
    global clauses
    s="(!a1||!b1)&&(!a1||!c1)&&(!a1||!d1)&&(a1||b1||c1||d1)&&!a2&&(!b1||!b2)&&(!b1||!c1)&&(!b1||!d1)&&" \
      "(b1||b2||c1||d1)&&(!b2||!c1)&&(!b2||!c2)&&(!b2||!d1)&&(!b2||!d2)&&(!b2||!e2)&&(b2||c1||c2||d1)&&" \
      "(b2||c2||d1||d2)&&(b2||c2||d2||e2)&&(!c1||!c2)&&(!c1||!d1)&&(!c2||!d1)&&(!c2||!d2)&&(!c2||!e2)&&" \
      "(!d1||!d2)&&(!d2||!e2)&&!e1"

    clauses=clauses+mathematica_to_CNF(s, {
	"a1": vars[(n1,0)], "b1": vars[(n1,1)], "c1": vars[(n1,2)], "d1": vars[(n1,3)], "e1": vars[(n1,4)],
	"a2": vars[(n2,0)], "b2": vars[(n2,1)], "c2": vars[(n2,2)], "d2": vars[(n2,3)], "e2": vars[(n2,4)]})

...

# 6.The green house is immediately to the right of the ivory house.
add_right("Ivory", "Green")
\end{lstlisting}

Что мы будем делать с этим?
``11. Сосед того, кто курит Chesterfield, держит лису.''
``12. В доме по соседству с тем, в котором держат лошадь, курят Kool.''

Мы не знаем с какой стороны, слева или справа, но знаем что они отличаются на единицу.
Вот какие клозы я добавлю:

\begin{lstlisting}
    Chesterfield  Fox
AND(0 0 0 0 1     0 0 0 1 0)
.. OR ..
AND(0 0 0 1 0     0 0 0 0 1)
AND(0 0 0 1 0     0 0 1 0 0)
.. OR ..
AND(0 0 1 0 0     0 1 0 0 0)
AND(0 0 1 0 0     0 0 0 1 0)
.. OR ..
AND(0 1 0 0 0     1 0 0 0 0)
AND(0 1 0 0 0     0 0 1 0 0)
.. OR ..
AND(1 0 0 0 0     0 1 0 0 0)
\end{lstlisting}

И снова могу сконвертировать это всё в CNF при помощи Mathematica:

\begin{lstlisting}
In[]:= BooleanConvert[(a1&& !b1&&!c1&&!d1&&!e1&&!a2&& b2&&!c2&&!d2&&!e2) ||

(!a1&& b1&&!c1&&!d1&&!e1&&a2&& !b2&&!c2&&!d2&&!e2) ||
(!a1&& b1&&!c1&&!d1&&!e1&&!a2&& !b2&&c2&&!d2&&!e2) ||

(!a1&& !b1&&c1&&!d1&&!e1&&!a2&& b2&&!c2&&!d2&&!e2) ||
(!a1&& !b1&&c1&&!d1&&!e1&&!a2&& !b2&&!c2&&d2&&!e2) ||

(!a1&& !b1&&!c1&&d1&&!e1&&!a2&& !b2&&c2&&!d2&&!e2) ||
(!a1&& !b1&&!c1&&d1&&!e1&&!a2&& !b2&&!c2&&!d2&&e2) ||

(!a1&& !b1&&!c1&&!d1&&e1&&!a2&& !b2&&!c2&&d2&&!e2) ,"CNF"]

Out[]= (!a1||!b1)&&(!a1||!c1)&&(!a1||!d1)&&(!a1||!e1)&&(a1||b1||c1||d1||e1)&&(!a2||b1)&&(!a2||!b2)&&
(!a2||!c2)&&(!a2||!d2)&&(!a2||!e2)&&(a2||b2||c1||c2||d1||e1)&&(a2||b2||c2||d1||d2)&&(a2||b2||c2||d2||e2)&&
(!b1||!b2)&&(!b1||!c1)&&(!b1||!d1)&&(!b1||!e1)&&(b1||b2||c1||d1||e1)&&(!b2||!c2)&&(!b2||!d1)&&(!b2||!d2)&&
(!b2||!e1)&&(!b2||!e2)&&(!c1||!c2)&&(!c1||!d1)&&(!c1||!e1)&&(!c2||!d2)&&(!c2||!e1)&&(!c2||!e2)&&
(!d1||!d2)&&(!d1||!e1)&&(!d2||!e2)
\end{lstlisting}

И вот мой код:

\begin{lstlisting}
def add_right_or_left (n1, n2):
    global clauses
    s="(!a1||!b1)&&(!a1||!c1)&&(!a1||!d1)&&(!a1||!e1)&&(a1||b1||c1||d1||e1)&&(!a2||b1)&&" \
      "(!a2||!b2)&&(!a2||!c2)&&(!a2||!d2)&&(!a2||!e2)&&(a2||b2||c1||c2||d1||e1)&&(a2||b2||c2||d1||d2)&&" \
       "(a2||b2||c2||d2||e2)&&(!b1||!b2)&&(!b1||!c1)&&(!b1||!d1)&&(!b1||!e1)&&(b1||b2||c1||d1||e1)&&" \
       "(!b2||!c2)&&(!b2||!d1)&&(!b2||!d2)&&(!b2||!e1)&&(!b2||!e2)&&(!c1||!c2)&&(!c1||!d1)&&(!c1||!e1)&&" \
       "(!c2||!d2)&&(!c2||!e1)&&(!c2||!e2)&&(!d1||!d2)&&(!d1||!e1)&&(!d2||!e2)"
    
    clauses=clauses+mathematica_to_CNF(s, {
	"a1": vars[(n1,0)], "b1": vars[(n1,1)], "c1": vars[(n1,2)], "d1": vars[(n1,3)], "e1": vars[(n1,4)],
	"a2": vars[(n2,0)], "b2": vars[(n2,1)], "c2": vars[(n2,2)], "d2": vars[(n2,3)], "e2": vars[(n2,4)]})

...

# 11.The man who smokes Chesterfields lives in the house next to the man with the fox.
add_right_or_left("Chesterfield","Fox") # left or right

# 12.Kools are smoked in the house next to the house where the horse is kept.
add_right_or_left("Kools","Horse") # left or right
\end{lstlisting}

Вот и всё!
Полный исходный код: \url{https://github.com/DennisYurichev/SAT_SMT_article/blob/master/SAT/zebra/zebra_SAT.py}.

Итоговая CNF-проблема имеет 125 булевых переменных и 511 клозов: \\
\url{https://github.com/DennisYurichev/SAT_SMT_article/blob/master/SAT/zebra/1.cnf}.
Это очень легкая задача для любого SAT-солвера.
Даже мой игрушечный SAT-солвер (\ref{SAT_backtrack}) может решить её за \textasciitilde{}1 секунду на моем древнем
нетбуке с Intel Atom.

И конечно же, тут только одно решение, что и подтверждается при помощи Picosat.

\begin{lstlisting}
% python zebra_SAT.py
Yellow 1
Blue 2
Red 3
Ivory 4
Green 5
Norwegian 1
Ukrainian 2
Englishman 3
Spaniard 4
Japanese 5
Water 1
Tea 2
Milk 3
OrangeJuice 4
Coffee 5
Kools 1
Chesterfield 2
OldGold 3
LuckyStrike 4
Parliament 5
Fox 1
Horse 2
Snails 3
Dog 4
Zebra 5
\end{lstlisting}


\subsection{Простейший SAT-солвер в \textasciitilde{}120 строках}
\label{SAT_backtrack}

Это простейший SAT-солвер работающий на базе поиска с возвратом (\textit{backtracking}) (не \ac{DPLL}), написанный
на Питоне.
Он использует тот же поиск с возвратом, который можно найти в простейших солверах Судоку и задачи о восьми ферзях.
Он работает значительно медленнее, но, из-за предельной простоты, он также может подсчитывать количество решений.
Например, он может подсчитать все решения для задачи о восьми ферзях (\ref{EightQueens}).

Также, имеется 70 решений для ф-ции POPCNT4
\footnote{\url{https://github.com/DennisYurichev/SAT_SMT_article/blob/master/SAT/backtrack/POPCNT4.cnf}}
(ф-ция истинна, если любые из её 4-х входов из 8-и истинны):

\begin{lstlisting}
SAT
-1 -2 -3 -4 5 6 7 8 0
SAT
-1 -2 -3 4 -5 6 7 8 0
SAT
-1 -2 -3 4 5 -6 7 8 0
SAT
-1 -2 -3 4 5 6 -7 8 0
...

SAT
1 2 3 -4 -5 6 -7 -8 0
SAT
1 2 3 -4 5 -6 -7 -8 0
SAT
1 2 3 4 -5 -6 -7 -8 0
UNSAT
solutions= 70
\end{lstlisting}

Солвер также тестировался на моем взломщике Сапёра основанном на SAT (\ref{minesweeper_SAT}),
и заканчивает работу в разумное время (хотя и медленнее чем MiniSat раз в \textasciitilde{}10).

На б\'{о}льших \ac{CNF}-задачах он зависает.

Исходный код:
% TODO: translate to RU:
\lstinputlisting{SAT/backtrack/SAT_backtrack.py}

Как вы видите, всё что он делает, это перечисляет все возможные решения, но отсекает поисковое дерево настолько рано,
насколько это возможно.
Это и есть поиск с возвратом (\textit{backtracking}).

Файлы: \url{https://github.com/DennisYurichev/SAT_SMT_article/tree/master/SAT/backtrack}.

Некоторые комментарии: \url{https://www.reddit.com/r/compsci/comments/6jn3th/simplest_sat_solver_in_120_lines/}.




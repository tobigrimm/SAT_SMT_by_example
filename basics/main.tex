\section{Basics}

\subsection{One-hot encoding}

Throughout this book, we'll often use so-called ``one-hot encoding''.
In short, this is:

\begin{center}
\begin{longtable}{ | l | l | }                                                                    
\hline
Decimal & One-hot \\
\hline
0	& 00000001 \\
1	& 00000010 \\
2	& 00000100 \\
3	& 00001000 \\
4	& 00010000 \\
5	& 00100000 \\
6	& 01000000 \\
7	& 10000000 \\
\hline
\end{longtable}
\end{center}

Or in reversed form:

\begin{center}
\begin{longtable}{ | l | l | }                                                                    
\hline
Decimal & One-hot \\
\hline
0	& 10000000 \\
1	& 01000000 \\
2	& 00100000 \\
3	& 00010000 \\
4	& 00001000 \\
5	& 00000100 \\
6	& 00000010 \\
7	& 00000001 \\
\hline
\end{longtable}
\end{center}

It has several advantages and disadvantages as well.
See also: \url{https://en.wikipedia.org/wiki/One-hot}.

% subsections
\input{basics/SMT}
\section{KLEE}

% subsections:
\subsection{School-level equation}

Let's revisit school-level system of equations from (\ref{eq2_SMT}).

We will force KLEE to find a path, where all the constraints are satisfied:

\lstinputlisting{KLEE/klee_eq1.c}

% FIXME:
\begin{lstlisting}
\$ clang -emit-llvm -c -g klee_eq.c
...

\$ klee klee_eq.bc
KLEE: output directory is "/home/klee/klee-out-93"
KLEE: WARNING: undefined reference to function: klee_assert
KLEE: WARNING ONCE: calling external: klee_assert(0)
KLEE: ERROR: /home/klee/klee_eq.c:18: failed external call: klee_assert
KLEE: NOTE: now ignoring this error at this location

KLEE: done: total instructions = 32
KLEE: done: completed paths = 1
KLEE: done: generated tests = 1
\end{lstlisting}

Let's find out, where \TT{klee\_assert()} has been triggered:

% FIXME:
\begin{lstlisting}
\$ ls klee-last | grep err
test000001.external.err

\$ ktest-tool --write-ints klee-last/test000001.ktest
ktest file : 'klee-last/test000001.ktest'
args       : ['klee_eq.bc']
num objects: 3
object    0: name: b'circle'
object    0: size: 4
object    0: data: 5
object    1: name: b'square'
object    1: size: 4
object    1: data: 2
object    2: name: b'triangle'
object    2: size: 4
object    2: data: 1
\end{lstlisting}

This is indeed correct solution to the system of equations.

KLEE has intrinsic \TT{klee\_assume()} which tells KLEE to cut path if some constraint is not true.
So we can rewrite our example in such cleaner way:

\lstinputlisting{KLEE/klee_eq2.c}



\subsection{Zebra puzzle (\ac{AKA} Einstein puzzle)}
\label{zebra_SMT}

Zebra puzzle is a popular puzzle, defined as follows:

% FIXME remove paragraph at first line
\begin{framed}
\begin{quotation}
1.There are five houses.\\
2.The Englishman lives in the red house.\\
3.The Spaniard owns the dog.\\
4.Coffee is drunk in the green house.\\
5.The Ukrainian drinks tea.\\
6.The green house is immediately to the right of the ivory house.\\
7.The Old Gold smoker owns snails.\\
8.Kools are smoked in the yellow house.\\
9.Milk is drunk in the middle house.\\
10.The Norwegian lives in the first house.\\
11.The man who smokes Chesterfields lives in the house next to the man with the fox.\\
12.Kools are smoked in the house next to the house where the horse is kept.\\
13.The Lucky Strike smoker drinks orange juice.\\
14.The Japanese smokes Parliaments.\\
15.The Norwegian lives next to the blue house.\\
\\
Now, who drinks water? Who owns the zebra?\\
\\
In the interest of clarity, it must be added that each of the five houses is painted a different color, and their inhabitants are of different national extractions, own different pets, drink different beverages and smoke different brands of American cigarets [sic]. One other thing: in statement 6, right means your right.
\end{quotation}
\end{framed}
( \url{https://en.wikipedia.org/wiki/Zebra_Puzzle} ) \\
\\
It's a very good example of constraint satisfaction problem (CSP). % FIXME \ac

We would encode each entity as integer variable, representing number of house.

Then, to define that Englishman lives in red house, we will define this constraint: \TT{Englishman == Red}, meaning that number of a house where Englishmen resides and where tea is drunk is the same.

To define that Norwegian lives next to the blue house, we don't realy know, if it is at left side of blue house or at right side, but we know that house numbers are different by just 1.
So we will define this constraint: \TT{Norwegian==Blue-1 OR Norwegian==Blue+1}.

We will also need to limit all house numbers, so they will be in range of 1..5.

We will also use \TT{Distinct} to show that all various entities of the same type are all has different house numbers.

\lstinputlisting{SMT/zebra.py}

When we run it, we got correct result:

\begin{lstlisting}
sat
[Snails = 3,
 Blue = 2,
 Ivory = 4,
 OrangeJuice = 4,
 Parliament = 5,
 Yellow = 1,
 Fox = 1,
 Zebra = 5,
 Horse = 2,
 Dog = 4,
 Tea = 2,
 Water = 1,
 Chesterfield = 2,
 Red = 3,
 Japanese = 5,
 LuckyStrike = 4,
 Norwegian = 1,
 Milk = 3,
 Kools = 1,
 OldGold = 3,
 Ukrainian = 2,
 Coffee = 5,
 Green = 5,
 Spaniard = 4,
 Englishman = 3]
 \end{lstlisting}


\subsection{Sudoku}

I've also rewritten Sudoku example (\ref{sudoku_SMT}) for KLEE:

\lstinputlisting[numbers=left]{KLEE/klee_sudoku_or1.c}

Let's run it:

% FIXME:
\begin{lstlisting}
\$ clang -emit-llvm -c -g klee_sudoku_or1.c
...

\$ time klee klee_sudoku_or1.bc
KLEE: output directory is "/home/klee/klee-out-98"
KLEE: WARNING: undefined reference to function: klee_assert
KLEE: WARNING ONCE: calling external: klee_assert(0)
KLEE: ERROR: /home/klee/klee_sudoku_or1.c:93: failed external call: klee_assert
KLEE: NOTE: now ignoring this error at this location

KLEE: done: total instructions = 7512
KLEE: done: completed paths = 161
KLEE: done: generated tests = 161

real    3m44.111s
user    3m43.319s
sys     0m0.951s
\end{lstlisting}

Now this is really slower (on my Intel Core i3-3110M 2.4GHz notebook) in comparison to Z3Py solution (\ref{sudoku_SMT}).

But the answer is correct:

% FIXME:
\begin{lstlisting}
\$ ls klee-last | grep err
test000161.external.err

\$ ktest-tool --write-ints klee-last/test000161.ktest
ktest file : 'klee-last/test000161.ktest'
args       : ['klee_sudoku_or1.bc']
num objects: 1
object    0: name: b'cells'
object    0: size: 81
object    0: data: b'\x01\x04\x05\x03\x02\x07\x06\t\x08\x08\x03\t\x06\x05\x04\x01\x02\x07\x06\x07\x02\t\x01\x08\x05\x04\x03\x04\t\x06\x01\x08\x05\x03\x07\x02\x02\x01\x08\x04\x07\x03\t\x05\x06\x07\x05\x03\x02\t\x06\x04\x08\x01\x03\x06\x07\x05\x04\x02\x08\x01\t\t\x08\x04\x07\x06\x01\x02\x03\x05\x05\x02\x01\x08\x03\t\x07\x06\x04'
\end{lstlisting}

% FIXME backslash
\TT{\\t} is character with ASCII code of 9 in C/C++, and KLEE attempts to treat byte array as C/C++ string, so it shows some values in such way.
We can just remember that there is 9 at the each place where we see \TT{\\t}.
The solution, while not properly formatted, correct indeed. \\
\\
By the way, at lines 42, 43 you may see how we tell to KLEE that all array elements must be within some limits.
If we comment these lines out, we've got this:

% FIXME:
\begin{lstlisting}
\$ time klee klee_sudoku_or1.bc
KLEE: output directory is "/home/klee/klee-out-100"
KLEE: WARNING: undefined reference to function: klee_assert
KLEE: ERROR: /home/klee/klee_sudoku_or1.c:51: overshift error
KLEE: NOTE: now ignoring this error at this location
KLEE: ERROR: /home/klee/klee_sudoku_or1.c:51: overshift error
KLEE: NOTE: now ignoring this error at this location
KLEE: ERROR: /home/klee/klee_sudoku_or1.c:51: overshift error
KLEE: NOTE: now ignoring this error at this location
...
\end{lstlisting}

KLEE warns us that shift value at line 51 is too big.
Indeed, KLEE may try all byte values up to 255 (0xFF), which are pointless to use there, and may indicate error or bug, so KLEE warns about it.\\
\\
Now let's use \TT{klee\_assume()} again:

\lstinputlisting{KLEE/klee_sudoku_or2.c}

% FIXME:
\begin{lstlisting}
\$ time klee klee_sudoku_or2.bc
KLEE: output directory is "/home/klee/klee-out-99"
KLEE: WARNING: undefined reference to function: klee_assert
KLEE: WARNING ONCE: calling external: klee_assert(0)
KLEE: ERROR: /home/klee/klee_sudoku_or2.c:93: failed external call: klee_assert
KLEE: NOTE: now ignoring this error at this location

KLEE: done: total instructions = 7119
KLEE: done: completed paths = 1
KLEE: done: generated tests = 1

real    0m35.312s
user    0m34.945s
sys     0m0.318s
\end{lstlisting}

That works much faster: perhaps KLEE indeed handle this intrinsic in a special way.
And, as we see, the only one path is generated (one we actually interesting in it) instead of 161.

It's still much slower than Z3Py solution, though.


\input{KLEE/UNIXdatetime.tex}
\input{KLEE/base64.tex}
\subsection{CRC} % FIXME full name

\subsubsection{Buffer alteration case \#1}

Sometimes, you need to alter a piece of data which is \textit{protected} by some kind of checksum or \ac{CRC}, and you can't change checksum or CRC value, but can alter piece of data so that checksum will remain the same.

Let's pretend, we've got a piece of data with ``Hello, world!'' string at the beginning and ``and goodbye'' string at the end.
We can alter 14 characters at the middle, but for some reason, they must be in \textit{a..z} limits, but we can put any characters there.
CRC64 of the whole block must be \TT{0x12345678abcdef12}.

Let's see\footnote{There are several slightly different CRC64 implementations, the one I use here can also be different from popular ones.}:

\lstinputlisting{KLEE/klee_CRC64.c}

Since our code uses memcmp() standard C/C++ function, we need to add \TT{--libc=uclibc} switch, so KLEE will use its own uClibc % FIXME check spelling
implementation. % \ref{} -> closed programs

% FIXME:
\begin{lstlisting}
\$ clang -emit-llvm -c -g klee_CRC64.c

\$ time klee --libc=uclibc klee_CRC64.bc
\end{lstlisting}

It takes about 1 minute (on my XXX) and we getting this:

% FIXME:
\begin{lstlisting}
...
real    0m52.643s
user    0m51.232s
sys     0m0.239s
...
\$ ls klee-last | grep err
test000001.user.err
test000002.user.err
test000003.user.err
test000004.external.err

\$ ktest-tool --write-ints klee-last/test000004.ktest
ktest file : 'klee-last/test000004.ktest'
args       : ['klee_CRC64.bc']
num objects: 1
object    0: name: b'buf'
object    0: size: 46
object    0: data: b'Hello, world!.. qqlicayzceamyw ... and goodbye'
\end{lstlisting}

Maybe it's slow, but definitely faster than bruteforce.
Indeed, $log_2{26^{14}} \approx 65.8$
which is close to 64 bits.
In other words, one need $\approx 14$ latin characters to encode 64 bits.
And KLEE + \ac{SMT} solver needs 64 bits at some place it can alter to make final CRC64 value equal to what we defined.

I tried to reduce length of the \textit{middle block} to 13 characters: no luck for KLEE then, it has no space enough.

\subsubsection{Buffer alteration case \#2}

I went sadistic: what if the buffer must contain the CRC64 value which, after calculation of CRC64, will result in the same value?
Fascinately, % FIXME check spelling
KLEE can solve this.
The buffer will have the following format:

% FIXME:
\begin{lstlisting}
Hello, world! <8-bytes (64-bit value)> and goodbye <6 more bytes>
\end{lstlisting}

% FIXME:
\begin{lstlisting}
int main()
{
#define HEAD_STR "Hello, world!.. "
#define HEAD_SIZE strlen(HEAD_STR)
#define TAIL_STR " ... and goodbye"
#define TAIL_SIZE strlen(TAIL_STR)
// 8 bytes for 64-bit value:
#define MID_SIZE 8
#define BUF_SIZE HEAD_SIZE+TAIL_SIZE+MID_SIZE+6

	char buf[BUF_SIZE];
  
	klee_make_symbolic(buf, sizeof buf, "buf");

	klee_assume (memcmp (buf, HEAD_STR, HEAD_SIZE)==0);

	klee_assume (memcmp (buf+HEAD_SIZE+MID_SIZE, TAIL_STR, TAIL_SIZE)==0);
	
	uint64_t mid_value=*(uint64_t*)(buf+HEAD_SIZE);
	klee_assume (crc64 (0, buf, BUF_SIZE)==mid_value);

	klee_assert(0);

	return 0;
}
\end{lstlisting}

It works:

% FIXME:
\begin{lstlisting}
\$ time klee --libc=uclibc klee_CRC64.bc
...
real    5m17.081s
user    5m17.014s
sys     0m0.319s

\$ ls klee-last | grep err
test000001.user.err
test000002.user.err
test000003.external.err

\$ ktest-tool --write-ints klee-last/test000003.ktest
ktest file : 'klee-last/test000003.ktest'
args       : ['klee_CRC64.bc']
num objects: 1
object    0: name: b'buf'
object    0: size: 46
object    0: data: b'Hello, world!.. T+]\xb9A\x08\x0fq ... and goodbye\xb6\x8f\x9c\xd8\xc5\x00'
\end{lstlisting}

8 bytes between two strings is 64-bit value which equals to CRC64 of this whole block.
Again, it's faster than brute-force way to find it.
If to decrease last spare 6-byte buffer to 4 bytes or less, KLEE works so long so I've stopped it.

\subsubsection{Recovering input data for given CRC32 value of it}

I've always wanted to do so, but everyone knows this is impossible for input buffers larger than 4 bytes.
As my experiments show, it's still possible for tiny input buffers of data, constrained in some way.

The CRC32 value of 6-byte ``SILVER'' string is known: \TT{0xDFA3DFDD}.
KLEE can find this 6-byte string, if it knows that each byte of input buffer is in \textit{A..Z} limits:

\lstinputlisting[numbers=left]{KLEE/klee_SILVER.c}

% FIXME:
\begin{lstlisting}
\$ clang -emit-llvm -c -g klee_SILVER.c
...

\$ klee klee_SILVER.bc
...

\$ ls klee-last | grep err
test000013.external.err

\$ ktest-tool --write-ints klee-last/test000013.ktest
ktest file : 'klee-last/test000013.ktest'
args       : ['klee_SILVER.bc']
num objects: 1
object    0: name: b'str'
object    0: size: 6
object    0: data: b'SILVER'
\end{lstlisting}

Still, it's no magic: if to remove condition at lines 23..25 (i.e., if to relax constraints),
KLEE will produce some other string, which will be still correct for the CRC32 value given.

It works, because 6 Latin characters in \textit{A..Z} limits contain $\approx 28.2$ bits:
$log_2{26^6} \approx 28.2$, which is even smaller value than 32.
In other words, the final CRC32 value holds enough bits to recover $\approx 28.2$ bits of input.

The input buffer can be even bigger, if each byte of it will be in even tighter % FIXME spelling
constraints (decimal digits, binary digits, etc).

\subsubsection{In comparison with other hashing algorithms}

Things are that easy for some other hashing algorithms like \textit{fletcher checksum}, % FIXME URL
but not for cryptographically secure ones (like MD5, SHA1, etc), they are protected from such simple cryptoanalysis. % FIXME \ref{} -> am.crypto


\input{KLEE/LZSS.tex}

\subsection{Exercise}

Here is my crackme/keygenme, which may be tricky, but easy to solve using KLEE:
\url{http://challenges.re/74/}.





\subsection{Register allocation using graph coloring}

\renewcommand{\CURPATH}{color/reg_alloc}

This is an implementation of nuth-Morris-Pratt algorithm, it searches for a substring in a string
\footnote{Copypasted from \url{http://cprogramming.com/snippets/source-code/knuthmorrispratt-kmp-string-search-algorithm}}.

\lstinputlisting[style=customc]{\CURPATH/1.c}

... as you can see, I simplified it a bit, there are no more calls to malloc/free and T[] array is now global.

Then I compiled it using GCC 7.3 x64 and reworked assembly listing a little, now there are no registers, but rather vXX variables, each one is assigned only once,
in a \ac{SSA} manner. No variable assigned more than once. This is AT\&T syntax.

%FIXME:
%\lstinputlisting[style=customasm]{\CURPATH/KMP_template.s}
\lstinputlisting[basicstyle=\footnotesize]{\CURPATH/KMP_template.s}

Dangling "noodles" you see at right are "live ranges" of each vXX variable. "D" means "defined", "U" - "used" or "used and then defined again".
Whenever live range is started, we need to allocate variable (in a register or a local stack).
When it's ending, we do not need to keep it somewhere in storage (in a register or a local stack).

As you can see, the function has two parts: preparation and processing.
You can clearly see how live ranges are divided by two groups, except of first 4 variables, which are function arguments.

You see, there are 16 variables. But we want to use as small number of registers, as possible.
If several live ranges are present at some address or point of time, these variables cannot be allocated in the same register.

This is how we can assign a register to each live range using Z3 SMT-solver:

\lstinputlisting[style=custompy]{\CURPATH/solver.py}

What we've got for 12, 11 and 10 registers:

\lstinputlisting{\CURPATH/solver.txt}

It's not possible to assign 9 or less registers. 10 is a minimum.

Now all I do is replacing vXX variables to registers the SMT-solver offered:

\lstinputlisting[basicstyle=\footnotesize]{\CURPATH/KMP.s}

That works and it's almost the same as GCC does.

The problem of register allocation as a graph coloring problem: each live range is a vertex.
It a live range must coexist with another live range, put an edge between vertices, that means, vertices cannot share same color.
Color reflecting register number.

Almost all compilers (except simplest) do this in code generator. They use simpler algorithms, though, instead of such a heavy machinery as SAT/SMT solvers.

See also: \url{https://en.wikipedia.org/wiki/Register_allocation}, \url{https://en.wikipedia.org/wiki/Live_variable_analysis}.

Since graph coloring can have many solutions, you can probably hide some information
in "coloring": \url{https://yurichev.com/blog/lehmer/}.


\subsection{Zebra puzzle as a SAT problem}
\label{Zebra_SAT}

I would define each variable as vector of 5 variables, as I did before in Sudoku solver: \ref{Sudoku_SAT}.

I also use \TT{POPCNT1} function, but unlike previous example, I used Wolfram Mathematica to generate it in CNF form:

\begin{lstlisting}
In[]:= tbl1=Table[PadLeft[IntegerDigits[i,2],5] ->If[Equal[DigitCount[i,2][[1]],1],1,0],{i,0,63}]
Out[]= {{0,0,0,0,0}->0,
{0,0,0,0,1}->1,
{0,0,0,1,0}->1,
{0,0,0,1,1}->0,
{0,0,1,0,0}->1,
{0,0,1,0,1}->0,

...

{1,1,1,1,0}->0,
{1,1,1,1,1}->0}

In[]:= BooleanConvert[BooleanFunction[tbl1,{a,b,c,d,e}],"CNF"]
Out[]= (!a||!b)&&(!a||!c)&&(!a||!d)&&(!a||!e)&&(a||b||c||d||e)&&(!b||!c)&&(!b||!d)&&(!b||!e)&&(!c||!d)&&(!c||!e)&&(!d||!e)
\end{lstlisting}

Also, as I suggested before (\ref{OR_in_POPCNT1}), I used \textit{OR} operation as the second step.

\begin{lstlisting}
def mathematica_to_CNF (s, d):
    for k in d.keys():
        s=s.replace(k, d[k])
    s=s.replace("!", "-").replace("||", " ").replace("(", "").replace(")", "")
    s=s.split ("&&")
    return s

def add_popcnt1(v1, v2, v3, v4, v5):
    global clauses
    s="(!a||!b)&&" \
      "(!a||!c)&&" \
      "(!a||!d)&&" \
      "(!a||!e)&&" \
      "(!b||!c)&&" \
      "(!b||!d)&&" \
      "(!b||!e)&&" \
      "(!c||!d)&&" \
      "(!c||!e)&&" \
      "(!d||!e)&&" \
      "(a||b||c||d||e)"

    clauses=clauses+mathematica_to_CNF(s, {"a":v1, "b":v2, "c":v3, "d":v4, "e":v5})

...

# k=tuple: ("high-level" variable name, number of bit (0..4))
# v=variable number in CNF
vars={}
vars_last=1

...

def alloc_distinct_variables(names):
    global vars
    global vars_last
    for name in names:
        for i in range(5):
            vars[(name,i)]=str(vars_last)
            vars_last=vars_last+1

        add_popcnt1(vars[(name,0)], vars[(name,1)], vars[(name,2)], vars[(name,3)], vars[(name,4)])

    # make them distinct:
    for i in range(5):
        clauses.append(vars[(names[0],i)] + " " + vars[(names[1],i)] + " " + vars[(names[2],i)] + " " + vars[(names[3],i)] + " " + vars[(names[4],i)])

...

alloc_distinct_variables(["Yellow", "Blue", "Red", "Ivory", "Green"])
alloc_distinct_variables(["Norwegian", "Ukrainian", "Englishman", "Spaniard", "Japanese"])
alloc_distinct_variables(["Water", "Tea", "Milk", "OrangeJuice", "Coffee"])
alloc_distinct_variables(["Kools", "Chesterfield", "OldGold", "LuckyStrike", "Parliament"])
alloc_distinct_variables(["Fox", "Horse", "Snails", "Dog", "Zebra"])

...

\end{lstlisting}

Now we have 5 boolean variables for each \textit{high-level} variable,
and each group of variables will always have distinct values.

Now let's reread puzzle description: ``2.The Englishman lives in the red house.''.
That's easy.
In my Z3 and KLEE examples I just wrote ``Englishman==Red''.
Same story here: we just add a clauses showing that 5 boolean variables for ``Englishman''
must be equal to 5 booleans for ``Red''.

On a lowest CNF level, if we want to say that two variables must be equal to each other, we add two clauses:

$(var1 \vee \neg var2) \wedge (\neg var1 \vee var2)$

That means, both \textit{var1} and \textit{var2} values must be \textit{False} or \textit{True},
but they cannot be different.

\begin{lstlisting}
def add_eq_clauses(var1, var2):
    global clauses
    clauses.append(var1 + " -" + var2)
    clauses.append("-"+var1 + " " + var2)

def add_eq (n1, n2):
    for i in range(5):
        add_eq_clauses(vars[(n1,i)], vars[(n2, i)])

...

# 2.The Englishman lives in the red house.
add_eq("Englishman","Red")

# 3.The Spaniard owns the dog.
add_eq("Spaniard","Dog")

# 4.Coffee is drunk in the green house.
add_eq("Coffee","Green")

...

\end{lstlisting}

Now the next conditions:
``9.Milk is drunk in the middle house.'' (i.e., 3rd house), ``10.The Norwegian lives in the first house.''
We can just assign boolean values directly:

\begin{lstlisting}
# n=1..5
def add_eq_var_n (name, n):
    global clauses
    global vars
    for i in range(5):
        if i==n-1:
            clauses.append(vars[(name,i)]) # always True
        else:
            clauses.append("-"+vars[(name,i)]) # always False

...

# 9.Milk is drunk in the middle house.
add_eq_var_n("Milk",3) # i.e., 3rd house

# 10.The Norwegian lives in the first house.
add_eq_var_n("Norwegian",1)
\end{lstlisting}

For ``Milk'' we will have ``0 0 1 0 0'' value, for ``Norwegian'': ``1 0 0 0 0''.

What to do with this?
``6.The green house is immediately to the right of the ivory house.''
I can construct the following condition:

\begin{lstlisting}
    Ivory      Green
AND(1 0 0 0 0  0 1 0 0 0)
.. OR ..
AND(0 1 0 0 0  0 0 1 0 0)
.. OR ..
AND(0 0 1 0 0  0 0 0 1 0)
.. OR ..
AND(0 0 0 1 0  0 0 0 0 1)
\end{lstlisting}

There is no ``0 0 0 0 1'' for ``Ivory'', because it cannot be the last one.
Now I can convert these conditions to CNF using Wolfram Mathematica:

\begin{lstlisting}
In[]:= BooleanConvert[(a1&& !b1&&!c1&&!d1&&!e1&&!a2&& b2&&!c2&&!d2&&!e2) ||
(!a1&& b1&&!c1&&!d1&&!e1&&!a2&& !b2&&c2&&!d2&&!e2) ||
(!a1&& !b1&&c1&&!d1&&!e1&&!a2&& !b2&&!c2&&d2&&!e2) ||
(!a1&& !b1&&!c1&&d1&&!e1&&!a2&& !b2&&!c2&&!d2&&e2) ,"CNF"]

Out[]= (!a1||!b1)&&(!a1||!c1)&&(!a1||!d1)&&(a1||b1||c1||d1)&&!a2&&(!b1||!b2)&&(!b1||!c1)&&
(!b1||!d1)&&(b1||b2||c1||d1)&&(!b2||!c1)&&(!b2||!c2)&&(!b2||!d1)&&(!b2||!d2)&&(!b2||!e2)&&
(b2||c1||c2||d1)&&(b2||c2||d1||d2)&&(b2||c2||d2||e2)&&(!c1||!c2)&&(!c1||!d1)&&(!c2||!d1)&&
(!c2||!d2)&&(!c2||!e2)&&(!d1||!d2)&&(!d2||!e2)&&!e1
\end{lstlisting}

And here is a piece of my Python code:

\begin{lstlisting}
def add_right (n1, n2):
    global clauses
    s="(!a1||!b1)&&(!a1||!c1)&&(!a1||!d1)&&(a1||b1||c1||d1)&&!a2&&(!b1||!b2)&&(!b1||!c1)&&(!b1||!d1)&&" \
      "(b1||b2||c1||d1)&&(!b2||!c1)&&(!b2||!c2)&&(!b2||!d1)&&(!b2||!d2)&&(!b2||!e2)&&(b2||c1||c2||d1)&&" \
      "(b2||c2||d1||d2)&&(b2||c2||d2||e2)&&(!c1||!c2)&&(!c1||!d1)&&(!c2||!d1)&&(!c2||!d2)&&(!c2||!e2)&&" \
      "(!d1||!d2)&&(!d2||!e2)&&!e1"

    clauses=clauses+mathematica_to_CNF(s, {
	"a1": vars[(n1,0)], "b1": vars[(n1,1)], "c1": vars[(n1,2)], "d1": vars[(n1,3)], "e1": vars[(n1,4)],
	"a2": vars[(n2,0)], "b2": vars[(n2,1)], "c2": vars[(n2,2)], "d2": vars[(n2,3)], "e2": vars[(n2,4)]})

...

# 6.The green house is immediately to the right of the ivory house.
add_right("Ivory", "Green")
\end{lstlisting}

What we will do with that?
``11.The man who smokes Chesterfields lives in the house next to the man with the fox.''
``12.Kools are smoked in the house next to the house where the horse is kept.''

We don't know side, left or right, but we know that they are differ in one.
Here is a clauses I would add:

\begin{lstlisting}
    Chesterfield  Fox
AND(0 0 0 0 1     0 0 0 1 0)
.. OR ..
AND(0 0 0 1 0     0 0 0 0 1)
AND(0 0 0 1 0     0 0 1 0 0)
.. OR ..
AND(0 0 1 0 0     0 1 0 0 0)
AND(0 0 1 0 0     0 0 0 1 0)
.. OR ..
AND(0 1 0 0 0     1 0 0 0 0)
AND(0 1 0 0 0     0 0 1 0 0)
.. OR ..
AND(1 0 0 0 0     0 1 0 0 0)
\end{lstlisting}

I can convert this into CNF using Mathematica again:

\begin{lstlisting}
In[]:= BooleanConvert[(a1&& !b1&&!c1&&!d1&&!e1&&!a2&& b2&&!c2&&!d2&&!e2) ||

(!a1&& b1&&!c1&&!d1&&!e1&&a2&& !b2&&!c2&&!d2&&!e2) ||
(!a1&& b1&&!c1&&!d1&&!e1&&!a2&& !b2&&c2&&!d2&&!e2) ||

(!a1&& !b1&&c1&&!d1&&!e1&&!a2&& b2&&!c2&&!d2&&!e2) ||
(!a1&& !b1&&c1&&!d1&&!e1&&!a2&& !b2&&!c2&&d2&&!e2) ||

(!a1&& !b1&&!c1&&d1&&!e1&&!a2&& !b2&&c2&&!d2&&!e2) ||
(!a1&& !b1&&!c1&&d1&&!e1&&!a2&& !b2&&!c2&&!d2&&e2) ||

(!a1&& !b1&&!c1&&!d1&&e1&&!a2&& !b2&&!c2&&d2&&!e2) ,"CNF"]

Out[]= (!a1||!b1)&&(!a1||!c1)&&(!a1||!d1)&&(!a1||!e1)&&(a1||b1||c1||d1||e1)&&(!a2||b1)&&(!a2||!b2)&&
(!a2||!c2)&&(!a2||!d2)&&(!a2||!e2)&&(a2||b2||c1||c2||d1||e1)&&(a2||b2||c2||d1||d2)&&(a2||b2||c2||d2||e2)&&
(!b1||!b2)&&(!b1||!c1)&&(!b1||!d1)&&(!b1||!e1)&&(b1||b2||c1||d1||e1)&&(!b2||!c2)&&(!b2||!d1)&&(!b2||!d2)&&
(!b2||!e1)&&(!b2||!e2)&&(!c1||!c2)&&(!c1||!d1)&&(!c1||!e1)&&(!c2||!d2)&&(!c2||!e1)&&(!c2||!e2)&&
(!d1||!d2)&&(!d1||!e1)&&(!d2||!e2)
\end{lstlisting}

And here is my code:

\begin{lstlisting}
def add_right_or_left (n1, n2):
    global clauses
    s="(!a1||!b1)&&(!a1||!c1)&&(!a1||!d1)&&(!a1||!e1)&&(a1||b1||c1||d1||e1)&&(!a2||b1)&&" \
      "(!a2||!b2)&&(!a2||!c2)&&(!a2||!d2)&&(!a2||!e2)&&(a2||b2||c1||c2||d1||e1)&&(a2||b2||c2||d1||d2)&&" \
       "(a2||b2||c2||d2||e2)&&(!b1||!b2)&&(!b1||!c1)&&(!b1||!d1)&&(!b1||!e1)&&(b1||b2||c1||d1||e1)&&" \
       "(!b2||!c2)&&(!b2||!d1)&&(!b2||!d2)&&(!b2||!e1)&&(!b2||!e2)&&(!c1||!c2)&&(!c1||!d1)&&(!c1||!e1)&&" \
       "(!c2||!d2)&&(!c2||!e1)&&(!c2||!e2)&&(!d1||!d2)&&(!d1||!e1)&&(!d2||!e2)"
    
    clauses=clauses+mathematica_to_CNF(s, {
	"a1": vars[(n1,0)], "b1": vars[(n1,1)], "c1": vars[(n1,2)], "d1": vars[(n1,3)], "e1": vars[(n1,4)],
	"a2": vars[(n2,0)], "b2": vars[(n2,1)], "c2": vars[(n2,2)], "d2": vars[(n2,3)], "e2": vars[(n2,4)]})

...

# 11.The man who smokes Chesterfields lives in the house next to the man with the fox.
add_right_or_left("Chesterfield","Fox") # left or right

# 12.Kools are smoked in the house next to the house where the horse is kept.
add_right_or_left("Kools","Horse") # left or right
\end{lstlisting}

This is it!
The full source code: \url{https://github.com/DennisYurichev/SAT_SMT_by_example/blob/master/puzzles/zebra/SAT/zebra_SAT.py}.

Resulting CNF instance has 125 boolean variables and 511 clauses: \\
\url{https://github.com/DennisYurichev/SAT_SMT_by_example/blob/master/puzzles/zebra/SAT/1.cnf}.
It is a piece of cake for any SAT solver.
Even my toy-level SAT-solver (\ref{SAT_backtrack}) can solve it in \textasciitilde{}1 second on my ancient Intel Atom netbook.

And of course, there is only one possible solution, what is acknowledged by Picosat.

\begin{lstlisting}
% python zebra_SAT.py
Yellow 1
Blue 2
Red 3
Ivory 4
Green 5
Norwegian 1
Ukrainian 2
Englishman 3
Spaniard 4
Japanese 5
Water 1
Tea 2
Milk 3
OrangeJuice 4
Coffee 5
Kools 1
Chesterfield 2
OldGold 3
LuckyStrike 4
Parliament 5
Fox 1
Horse 2
Snails 3
Dog 4
Zebra 5
\end{lstlisting}


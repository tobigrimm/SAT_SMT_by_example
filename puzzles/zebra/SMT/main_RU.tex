\subsection{SMT}
\label{zebra_SMT}

Головоломка зебры это популярная головоломка, определяется так:

% FIXME remove paragraph at first line
\begin{framed}
\begin{quotation}
1. На улице стоят пять домов.\\
2. Англичанин живёт в красном доме.\\
3. У испанца есть собака.\\
4. В зелёном доме пьют кофе.\\
5. Украинец пьёт чай.\\
6. Зелёный дом стоит сразу справа от белого дома.\\
7. Тот, кто курит Old Gold, разводит улиток.\\
8. В жёлтом доме курят Kool.\\
9. В центральном доме пьют молоко.\\
10. Норвежец живёт в первом доме.\\
11. Сосед того, кто курит Chesterfield, держит лису.\\
12. В доме по соседству с тем, в котором держат лошадь, курят Kool.\\
13. Тот, кто курит Lucky Strike, пьёт апельсиновый сок.\\
14. Японец курит Parliament.\\
15. Норвежец живёт рядом с синим домом.\\
\\
Кто пьёт воду? Кто держит зебру?\\
\\
В целях ясности следует добавить, что каждый из пяти домов окрашен в свой цвет, а их жители — разных национальностей, владеют разными животными, пьют разные напитки и курят разные марки американских сигарет. Ещё одно замечание: в утверждении 6 справа означает справа относительно вас.
\end{quotation}
\end{framed}
( \url{http://bit.ly/2oUNBhc} (Wikipedia) ) \\
\\
Это очень хороший пример \ac{CSP}.

Мы можем закодировать каждый объект как целочисленную переменную, определяющую номер дома.

Тогда, чтобы определить, что англичанин живет в красном доме, мы добавим этот констрайнт: \TT{Englishman == Red}, означающий, что номер дома, где живет англичанин, и номер дома окрашенный в красный --- один и тот же.

Мы определяем что норвежец живет в соседнем доме с синим домом, но мы точно не знаем, слева от синего дома, или справа,
но мы знаем что номер дома будет отличается на 1.
Так что определим такой констрайнт: \TT{Norwegian==Blue-1 OR Norwegian==Blue+1}.

Также нужно ограничить номера всех домов, чтобы они были в пределах 1..5.

Мы также будем использовать \TT{Distinct}, чтобы показать, что все различные объекты одного типа должны находиться в домах
с разными номерами.

\lstinputlisting[style=custompy]{puzzles/zebra/SMT/zebra.py}

Запускаем и получаем корректный результат:

\begin{lstlisting}
sat
[Snails = 3,
 Blue = 2,
 Ivory = 4,
 OrangeJuice = 4,
 Parliament = 5,
 Yellow = 1,
 Fox = 1,
 Zebra = 5,
 Horse = 2,
 Dog = 4,
 Tea = 2,
 Water = 1,
 Chesterfield = 2,
 Red = 3,
 Japanese = 5,
 LuckyStrike = 4,
 Norwegian = 1,
 Milk = 3,
 Kools = 1,
 OldGold = 3,
 Ukrainian = 2,
 Coffee = 5,
 Green = 5,
 Spaniard = 4,
 Englishman = 3]
\end{lstlisting}


\subsection{Альфаметика}

Согласно Дональду Кнуту, термин ``Альфаметика'' был придуман Дж. Эйч. Аш. Хантером.
Это головоломка: какие десятичные цифры в пределах 0..9 нужно присвоить каждой букве, чтобы это уравнение было справедливо?

\begin{lstlisting}
  SEND
+ MORE
 -----
 MONEY
\end{lstlisting}

Для Z3 это легко:

\lstinputlisting{puzzles/alphametics/alpha.py}

Вывод:

\begin{lstlisting}
sat
[E, = 5,
 S, = 9,
 M, = 1,
 N, = 6,
 D, = 7,
 R, = 8,
 O, = 0,
 Y = 2]
\end{lstlisting}

Вот еще одна, из \ac{TAOCP} том IV (\url{http://www-cs-faculty.stanford.edu/~uno/fasc2b.ps.gz}):

\lstinputlisting{puzzles/alphametics/alpha2.py}

\begin{lstlisting}
sat
[L, = 6,
 S, = 7,
 N, = 2,
 T, = 1,
 I, = 5,
 V = 3,
 A, = 8,
 R, = 9,
 O, = 4,
 TRIO = 1954,
 SONATA, = 742818,
 VIOLA, = 35468,
 VIOLIN, = 354652]
\end{lstlisting}

% TODO URL
Эту головоломку я нашел в примерах pySMT:

\lstinputlisting{puzzles/alphametics/alpha3.py}

\begin{lstlisting}
sat
[D = 5, R = 4, O = 3, E = 8, L = 6, W = 7, H = 2]
\end{lstlisting}

%%% 

Это упражнение Q209 из
[Companion to the Papers of Donald Knuth]\footnote{\url{http://www-cs-faculty.stanford.edu/~knuth/cp.html}}.

\begin{lstlisting}
 KNIFE
  FORK
 SPOON
  SOUP
------
SUPPER
\end{lstlisting}

В целях упрощения, я добавил ф-цию (list\_to\_expr()):

\lstinputlisting{puzzles/alphametics/alpha4.py}

\begin{lstlisting}
sat
[K = 7,
 N = 4,
 R = 9,
 I = 1,
 E = 6,
 S = 0,
 O = 3,
 F = 5,
 U = 8,
 P = 2,
 SUPPER = 82269,
 SOUP = 382,
 SPOON = 2334,
 FORK = 5397,
 KNIFE = 74156]
\end{lstlisting}

S это 0, так что значение SUPPER начинается с (убранного) нуля. Скажем так, нам это не нравится.
Добавим это, чтобы это исправить:

\begin{lstlisting}
s.add(S!=0)
\end{lstlisting}

\begin{lstlisting}
sat
[K = 8,
 N = 4,
 R = 3,
 I = 7,
 E = 6,
 S = 1,
 O = 9,
 F = 2,
 U = 0,
 P = 5,
 SUPPER = 105563,
 SOUP = 1905,
 SPOON = 15994,
 FORK = 2938,
 KNIFE = 84726]
\end{lstlisting}

\paragraph{Создание своей собственной головоломки}

Вот проблема: есть только 10 букв, но как их выбрать из числа слов?
Мы можем использовать Z3 для этого:

\lstinputlisting{puzzles/alphametics/gen.py}

Это первая сгенерированная головоломка:

\begin{lstlisting}
sat
EGGS
JELLY
LUNCH
C 5
E 6
G 3
H 7
J 0
L 1
N 4
S 8
U 2
Y 9
\end{lstlisting}

Что если мы хотим, чтобы слово ``CAKE'' присутствовало в числе ``слагаемых''?

Добавим это:

\begin{lstlisting}
s.add(word_used[words.index('CAKE')])
\end{lstlisting}

\begin{lstlisting}
sat
CAKE
TEA
LUNCH
A 8
C 3
E 1
H 9
J 6
K 2
L 0
N 5
T 7
U 4
\end{lstlisting}

Добавим это:

\begin{lstlisting}
s.add(word_used[words.index('EGGS')])
\end{lstlisting}

Теперь оно может найти пару к EGGS:

\begin{lstlisting}
sat
EGGS
HONEY
LUNCH
C 6
E 7
G 9
H 4
L 5
N 8
O 2
S 3
U 0
Y 1
\end{lstlisting}

\paragraph{Файлы}

\url{https://github.com/DennisYurichev/...}




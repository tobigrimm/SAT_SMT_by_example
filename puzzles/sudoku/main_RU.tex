\subsection{Судоку}

Головоломка Судоку это решетка 9*9, некоторые ячейки заполнены значениями, некоторые пустые:

% copypasted from http://www.texample.net/tikz/examples/sudoku/
\newcounter{row}
\newcounter{col}

\newcommand\setrow[9]{
  \setcounter{col}{1}
  \foreach \n in {#1, #2, #3, #4, #5, #6, #7, #8, #9} {
    \edef\x{\value{col} - 0.5}
    \edef\y{9.5 - \value{row}}
    \node[anchor=center] at (\x, \y) {\n};
    \stepcounter{col}
  }
  \stepcounter{row}
}

\begin{center}
\begin{tikzpicture}[scale=.7]
  \begin{scope}
    \draw (0, 0) grid (9, 9);
    \draw[very thick, scale=3] (0, 0) grid (3, 3);

    \setcounter{row}{1}
    \setrow { }{ }{5}  {3}{ }{ }  { }{ }{ }
    \setrow {8}{ }{ }  { }{ }{ }  { }{2}{ }
    \setrow { }{7}{ }  { }{1}{ }  {5}{ }{ }

    \setrow {4}{ }{ }  { }{ }{5}  {3}{ }{ }
    \setrow { }{1}{ }  { }{7}{ }  { }{ }{6}
    \setrow { }{ }{3}  {2}{ }{ }  { }{8}{ }

    \setrow { }{6}{ }  {5}{ }{ }  { }{ }{9}
    \setrow { }{ }{4}  { }{ }{ }  { }{3}{ }
    \setrow { }{ }{ }  { }{ }{9}  {7}{ }{ }

    \node[anchor=center] at (4.5, -0.5) {Нерешенная Судоку};
  \end{scope}
\end{tikzpicture}
\end{center}

Числа в каждом ряду должны быть уникальными, т.е., каждый ряд должен содержать 9 чисел в пределах 1..9 без повторений.
Та же история и для каждого столбца и каждого квадрата 3*3.

Головоломка представляет собой хороший кандидат, на котором можно попробовать \ac{SMT}-солвер, потому что это,
в общем-то, просто нерешенная система уравнений.

\subsubsection{Простая Судоку и SMT}
\label{sudoku_SMT}

\paragraph{Первая идея}

Всё что нужно решить, это как определять в одном выражении, содержат ли 9 переменных все 9 уникальных чисел?
Они ведь не упорядочены и не отсортированы, все-таки.

Из школьной арифметики, мы можем найти такую идею:

\begin{equation}
\underbrace{10^{i_1} + 10^{i_2} + \cdots + 10^{i_9}}_9 = 1111111110
\end{equation}

Берете каждую входную переменную, вычисляете $10^i$ и суммируете.
Если все входные значения уникальны, каждая найдет свое собственное место.
И даже более того: не будет дыр, т.е., не будет пропущенных значений.
Так что, в случае Судоку, число 1111111110 будет конечным результатом, означающим, что все входные
9 переменных уникальны, в пределах 1..9.

Возведение в степень это тяжелая операция, можно ли использовать двоичные операции? Да, просто замените 10 на 2:

\begin{equation}
\underbrace{2^{i_1} + 2^{i_2} + \cdots + 2^{i_9}}_9 = 1111111110_2
\end{equation}

Эффект тот же, но результат будет в двоичной системе а не в десятичной.

Вот рабочий пример:

\lstinputlisting{puzzles/sudoku/1/sudoku_plus.py}
( \url{https://github.com/DennisYurichev/SAT_SMT_article/.../sudoku_plus.py} )

\begin{lstlisting}
% time python sudoku_plus.py
1 4 5 3 2 7 6 9 8
8 3 9 6 5 4 1 2 7
6 7 2 9 1 8 5 4 3
4 9 6 1 8 5 3 7 2
2 1 8 4 7 3 9 5 6
7 5 3 2 9 6 4 8 1
3 6 7 5 4 2 8 1 9
9 8 4 7 6 1 2 3 5
5 2 1 8 3 9 7 6 4

real    0m11.717s
user    0m10.896s
sys     0m0.068s
\end{lstlisting}

И даже более того, можно заменить суммирование на логическое ИЛИ:

\begin{equation}
\underbrace{2^{i_1} \vee 2^{i_2} \vee \cdots \vee 2^{i_9}}_9 = 1111111110_2
\end{equation}

% FIXME: только часть исходника
\lstinputlisting{puzzles/sudoku/1/sudoku_or.py}
( \url{https://github.com/DennisYurichev/SAT_SMT_article/.../sudoku_or.py} )

Теперь работает намного быстрее. Наверное, Z3 лучше поддерживает операцию ИЛИ над битовыми векторами, чем сложение?

\begin{lstlisting}
% time python sudoku_or.py
1 4 5 3 2 7 6 9 8
8 3 9 6 5 4 1 2 7
6 7 2 9 1 8 5 4 3
4 9 6 1 8 5 3 7 2
2 1 8 4 7 3 9 5 6
7 5 3 2 9 6 4 8 1
3 6 7 5 4 2 8 1 9
9 8 4 7 6 1 2 3 5
5 2 1 8 3 9 7 6 4

real    0m1.429s
user    0m1.393s
sys     0m0.036s
\end{lstlisting}

Головоломка, которую я использовал как пример, известна как самая трудная из известных
\footnote{\url{http://www.mirror.co.uk/news/weird-news/worlds-hardest-sudoku-can-you-242294}} (по крайней мере для людей).
Для решения понадобилось $\approx 1.4$ секунды на моем ноутбуке Intel Core i3-3110M 2.4GHz.

\paragraph{Вторая идея}

Мой первый подход далек от эффективного, я сделал то что первым пришло в голову, и оно заработало.
Другой подход это использовать команду \TT{distinct} в SMT-LIB, которая говорит Z3, что некоторые переменные
должны быть отличны друг от друга (или уникальны).
Эта команда также имеется в питоновском интерфейсе к Z3.

Я переписал мой первый солвер Судоку, теперь он работает над \textit{sort}-ом 
\textit{Int}, и имеет команды \TT{distinct} вместо битовых операций,
и еще один констрайнт добавлен: значение каждой ячейки должно быть в пределах 1..9, потому что, иначе, Z3 предложит
(хотя и корректное) решение с очень большими и/или отрицательными числами.

% FIXME: только часть исходника
\lstinputlisting{puzzles/sudoku/1/sudoku2.py}
( \url{https://github.com/DennisYurichev/SAT_SMT_article/.../sudoku2.py} )

\begin{lstlisting}
% time python sudoku2.py
1 4 5 3 2 7 6 9 8
8 3 9 6 5 4 1 2 7
6 7 2 9 1 8 5 4 3
4 9 6 1 8 5 3 7 2
2 1 8 4 7 3 9 5 6
7 5 3 2 9 6 4 8 1
3 6 7 5 4 2 8 1 9
9 8 4 7 6 1 2 3 5
5 2 1 8 3 9 7 6 4

real    0m0.382s
user    0m0.346s
sys     0m0.036s
\end{lstlisting}

Это намного быстрее.

\paragraph{Вывод}

\ac{SMT}-солверы настолько удобны, что в нашем солвере Судоку нет ничего больше ничего, мы просто определили
отношения между переменными (ячейками).

\paragraph{Домашная работа}

Как видно, настоящая головоломка Судоку это та, для которой имеется только одно решение.
Фрагмент кода, который приведен здесь, показывает только первое.
Использая метод описанный раннее (\ref{SMTEnumerate}, также называемый ``подсчет моделей (model counting)''), 
попытайтесь найти больше решений, или доказать, что решение, которое вы нашли, единственное возможное.

\paragraph{Дальнейшее чтение}

\url{http://www.norvig.com/sudoku.html}

\paragraph{Судоку как \ac{SAT}-проблема}

Головоломку Судоку можно также представить как огромное \ac{CNF}-уравнение и использовать \ac{SAT}-солвер для поиска решения,
но это просто сложнее.

Некоторые статьи об этом:
\textit{Building a Sudoku Solver with SAT}\footnote{\url{http://ocw.mit.edu/courses/electrical-engineering-and-computer-science/6-005-elements-of-software-construction-fall-2011/assignments/MIT6_005F11_ps4.pdf}},
Tjark Weber, \textit{A SAT-based Sudoku Solver}\footnote{\url{https://www.lri.fr/~conchon/mpri/weber.pdf}},
Ines Lynce, Joel Ouaknine, \textit{Sudoku as a SAT Problem}\footnote{\url{http://sat.inesc-id.pt/~ines/publications/aimath06.pdf}},
Gihwon Kwon, Himanshu Jain, \textit{Optimized CNF Encoding for Sudoku Puzzles}\footnote{\url{http://www.cs.cmu.edu/~hjain/papers/sudoku-as-SAT.pdf}}.

\ac{SMT}-солвер также может использовать \ac{SAT}-солвер в своем ядре, так что он делает всю эту скучную работу
по трансляции.
Хотя и, как и компилятор, он может это делать не самым эффективным способом.


%% TODO translate src
\subsection{Головоломка Зебры как SAT-проблема}
\label{Zebra_SAT}

Попробуем решить головоломку Зебры (\ref{zebra_SMT}) в SAT.

Я определю каждую переменную как вектор из пяти переменных, как я делал это раннее в солвере Судоку: \ref{Sudoku_SAT}.

Я также использую ф-цию \TT{POPCNT1}, но в отличие от предыдущего примера,
я использовал Wolfram Mathematica для генерирования её в CNF-форме:

\begin{lstlisting}
In[]:= tbl1=Table[PadLeft[IntegerDigits[i,2],5] ->If[Equal[DigitCount[i,2][[1]],1],1,0],{i,0,63}]
Out[]= {{0,0,0,0,0}->0,
{0,0,0,0,1}->1,
{0,0,0,1,0}->1,
{0,0,0,1,1}->0,
{0,0,1,0,0}->1,
{0,0,1,0,1}->0,

...

{1,1,1,1,0}->0,
{1,1,1,1,1}->0}

In[]:= BooleanConvert[BooleanFunction[tbl1,{a,b,c,d,e}],"CNF"]
Out[]= (!a||!b)&&(!a||!c)&&(!a||!d)&&(!a||!e)&&(a||b||c||d||e)&&(!b||!c)&&(!b||!d)&&(!b||!e)&&(!c||!d)&&(!c||!e)&&(!d||!e)
\end{lstlisting}

Также, как я предлагал раньше (\ref{OR_in_POPCNT1}), я использовал операцию \textit{ИЛИ} для второго шага.

\begin{lstlisting}
def mathematica_to_CNF (s, d):
    for k in d.keys():
        s=s.replace(k, d[k])
    s=s.replace("!", "-").replace("||", " ").replace("(", "").replace(")", "")
    s=s.split ("&&")
    return s

def add_popcnt1(v1, v2, v3, v4, v5):
    global clauses
    s="(!a||!b)&&" \
      "(!a||!c)&&" \
      "(!a||!d)&&" \
      "(!a||!e)&&" \
      "(!b||!c)&&" \
      "(!b||!d)&&" \
      "(!b||!e)&&" \
      "(!c||!d)&&" \
      "(!c||!e)&&" \
      "(!d||!e)&&" \
      "(a||b||c||d||e)"

    clauses=clauses+mathematica_to_CNF(s, {"a":v1, "b":v2, "c":v3, "d":v4, "e":v5})

...

# k=tuple: ("high-level" variable name, number of bit (0..4))
# v=variable number in CNF
vars={}
vars_last=1

...

def alloc_distinct_variables(names):
    global vars
    global vars_last
    for name in names:
        for i in range(5):
            vars[(name,i)]=str(vars_last)
            vars_last=vars_last+1

        add_popcnt1(vars[(name,0)], vars[(name,1)], vars[(name,2)], vars[(name,3)], vars[(name,4)])

    # make them distinct:
    for i in range(5):
        clauses.append(vars[(names[0],i)] + " " + vars[(names[1],i)] + " " + vars[(names[2],i)] + " " + vars[(names[3],i)] + " " + vars[(names[4],i)])

...

alloc_distinct_variables(["Yellow", "Blue", "Red", "Ivory", "Green"])
alloc_distinct_variables(["Norwegian", "Ukrainian", "Englishman", "Spaniard", "Japanese"])
alloc_distinct_variables(["Water", "Tea", "Milk", "OrangeJuice", "Coffee"])
alloc_distinct_variables(["Kools", "Chesterfield", "OldGold", "LuckyStrike", "Parliament"])
alloc_distinct_variables(["Fox", "Horse", "Snails", "Dog", "Zebra"])

...

\end{lstlisting}

Теперь у нас пять булевых переменных для каждой \textit{высокоуровневной} переменной,
и каждая группа переменных гарантированно будет иметь разные значения.

Теперь перечитаем условие головоломки: ``2. Англичанин живёт в красном доме.''.
Это легко.
В моих примерах на Z3 и KLEE я просто написал ``Englishman==Red''.
Та же история и здесь: мы просто добавляем клозы, показывающие, что 5 булевых переменных для ``Englishman''
должны равняться пяти переменных для ``Red''.

На самом низком уровне CNF, если мы хотим сказать, что две переменных должны равняться друг другу,
мы добавляем два клоза:

$(var1 \vee \neg var2) \wedge (\neg var1 \vee var2)$

Это означает что значения обоих \textit{var1} и \textit{var2} должны быть или \textit{Ложно} или \textit{Истинно},
но они не могут быть разными.

\begin{lstlisting}
def add_eq_clauses(var1, var2):
    global clauses
    clauses.append(var1 + " -" + var2)
    clauses.append("-"+var1 + " " + var2)

def add_eq (n1, n2):
    for i in range(5):
        add_eq_clauses(vars[(n1,i)], vars[(n2, i)])

...

# 2.The Englishman lives in the red house.
add_eq("Englishman","Red")

# 3.The Spaniard owns the dog.
add_eq("Spaniard","Dog")

# 4.Coffee is drunk in the green house.
add_eq("Coffee","Green")

...

\end{lstlisting}

Теперь следующие условия:
``9. В центральном доме пьют молоко.'' (т.е., в третьем доме), ``10. Норвежец живёт в первом доме.''
Мы можем присвоить булевы значения напрямую:

\begin{lstlisting}
# n=1..5
def add_eq_var_n (name, n):
    global clauses
    global vars
    for i in range(5):
        if i==n-1:
            clauses.append(vars[(name,i)]) # always True
        else:
            clauses.append("-"+vars[(name,i)]) # always False

...

# 9.Milk is drunk in the middle house.
add_eq_var_n("Milk",3) # i.e., 3rd house

# 10.The Norwegian lives in the first house.
add_eq_var_n("Norwegian",1)
\end{lstlisting}

Для ``Milk'' у нас значение ``0 0 1 0 0'', для ``Norwegian'': ``1 0 0 0 0''.

Что делать с этим?
``6. Зелёный дом стоит сразу справа от белого дома.''
Я могу сконструировать такое условие:

\begin{lstlisting}
    Ivory      Green
AND(1 0 0 0 0  0 1 0 0 0)
.. OR ..
AND(0 1 0 0 0  0 0 1 0 0)
.. OR ..
AND(0 0 1 0 0  0 0 0 1 0)
.. OR ..
AND(0 0 0 1 0  0 0 0 0 1)
\end{lstlisting}

Для ``белого/ivory'' тут нет ``0 0 0 0 1'', потому что он не может быть последним.
Теперь я конвертирую эти условия в CNF при помощи Wolfram Mathematica:

\begin{lstlisting}
In[]:= BooleanConvert[(a1&& !b1&&!c1&&!d1&&!e1&&!a2&& b2&&!c2&&!d2&&!e2) ||
(!a1&& b1&&!c1&&!d1&&!e1&&!a2&& !b2&&c2&&!d2&&!e2) ||
(!a1&& !b1&&c1&&!d1&&!e1&&!a2&& !b2&&!c2&&d2&&!e2) ||
(!a1&& !b1&&!c1&&d1&&!e1&&!a2&& !b2&&!c2&&!d2&&e2) ,"CNF"]

Out[]= (!a1||!b1)&&(!a1||!c1)&&(!a1||!d1)&&(a1||b1||c1||d1)&&!a2&&(!b1||!b2)&&(!b1||!c1)&&
(!b1||!d1)&&(b1||b2||c1||d1)&&(!b2||!c1)&&(!b2||!c2)&&(!b2||!d1)&&(!b2||!d2)&&(!b2||!e2)&&
(b2||c1||c2||d1)&&(b2||c2||d1||d2)&&(b2||c2||d2||e2)&&(!c1||!c2)&&(!c1||!d1)&&(!c2||!d1)&&
(!c2||!d2)&&(!c2||!e2)&&(!d1||!d2)&&(!d2||!e2)&&!e1
\end{lstlisting}

И вот фрагмент моего кода на Питоне:

\begin{lstlisting}
def add_right (n1, n2):
    global clauses
    s="(!a1||!b1)&&(!a1||!c1)&&(!a1||!d1)&&(a1||b1||c1||d1)&&!a2&&(!b1||!b2)&&(!b1||!c1)&&(!b1||!d1)&&" \
      "(b1||b2||c1||d1)&&(!b2||!c1)&&(!b2||!c2)&&(!b2||!d1)&&(!b2||!d2)&&(!b2||!e2)&&(b2||c1||c2||d1)&&" \
      "(b2||c2||d1||d2)&&(b2||c2||d2||e2)&&(!c1||!c2)&&(!c1||!d1)&&(!c2||!d1)&&(!c2||!d2)&&(!c2||!e2)&&" \
      "(!d1||!d2)&&(!d2||!e2)&&!e1"

    clauses=clauses+mathematica_to_CNF(s, {
	"a1": vars[(n1,0)], "b1": vars[(n1,1)], "c1": vars[(n1,2)], "d1": vars[(n1,3)], "e1": vars[(n1,4)],
	"a2": vars[(n2,0)], "b2": vars[(n2,1)], "c2": vars[(n2,2)], "d2": vars[(n2,3)], "e2": vars[(n2,4)]})

...

# 6.The green house is immediately to the right of the ivory house.
add_right("Ivory", "Green")
\end{lstlisting}

Что мы будем делать с этим?
``11. Сосед того, кто курит Chesterfield, держит лису.''
``12. В доме по соседству с тем, в котором держат лошадь, курят Kool.''

Мы не знаем с какой стороны, слева или справа, но знаем что они отличаются на единицу.
Вот какие клозы я добавлю:

\begin{lstlisting}
    Chesterfield  Fox
AND(0 0 0 0 1     0 0 0 1 0)
.. OR ..
AND(0 0 0 1 0     0 0 0 0 1)
AND(0 0 0 1 0     0 0 1 0 0)
.. OR ..
AND(0 0 1 0 0     0 1 0 0 0)
AND(0 0 1 0 0     0 0 0 1 0)
.. OR ..
AND(0 1 0 0 0     1 0 0 0 0)
AND(0 1 0 0 0     0 0 1 0 0)
.. OR ..
AND(1 0 0 0 0     0 1 0 0 0)
\end{lstlisting}

И снова могу сконвертировать это всё в CNF при помощи Mathematica:

\begin{lstlisting}
In[]:= BooleanConvert[(a1&& !b1&&!c1&&!d1&&!e1&&!a2&& b2&&!c2&&!d2&&!e2) ||

(!a1&& b1&&!c1&&!d1&&!e1&&a2&& !b2&&!c2&&!d2&&!e2) ||
(!a1&& b1&&!c1&&!d1&&!e1&&!a2&& !b2&&c2&&!d2&&!e2) ||

(!a1&& !b1&&c1&&!d1&&!e1&&!a2&& b2&&!c2&&!d2&&!e2) ||
(!a1&& !b1&&c1&&!d1&&!e1&&!a2&& !b2&&!c2&&d2&&!e2) ||

(!a1&& !b1&&!c1&&d1&&!e1&&!a2&& !b2&&c2&&!d2&&!e2) ||
(!a1&& !b1&&!c1&&d1&&!e1&&!a2&& !b2&&!c2&&!d2&&e2) ||

(!a1&& !b1&&!c1&&!d1&&e1&&!a2&& !b2&&!c2&&d2&&!e2) ,"CNF"]

Out[]= (!a1||!b1)&&(!a1||!c1)&&(!a1||!d1)&&(!a1||!e1)&&(a1||b1||c1||d1||e1)&&(!a2||b1)&&(!a2||!b2)&&
(!a2||!c2)&&(!a2||!d2)&&(!a2||!e2)&&(a2||b2||c1||c2||d1||e1)&&(a2||b2||c2||d1||d2)&&(a2||b2||c2||d2||e2)&&
(!b1||!b2)&&(!b1||!c1)&&(!b1||!d1)&&(!b1||!e1)&&(b1||b2||c1||d1||e1)&&(!b2||!c2)&&(!b2||!d1)&&(!b2||!d2)&&
(!b2||!e1)&&(!b2||!e2)&&(!c1||!c2)&&(!c1||!d1)&&(!c1||!e1)&&(!c2||!d2)&&(!c2||!e1)&&(!c2||!e2)&&
(!d1||!d2)&&(!d1||!e1)&&(!d2||!e2)
\end{lstlisting}

И вот мой код:

\begin{lstlisting}
def add_right_or_left (n1, n2):
    global clauses
    s="(!a1||!b1)&&(!a1||!c1)&&(!a1||!d1)&&(!a1||!e1)&&(a1||b1||c1||d1||e1)&&(!a2||b1)&&" \
      "(!a2||!b2)&&(!a2||!c2)&&(!a2||!d2)&&(!a2||!e2)&&(a2||b2||c1||c2||d1||e1)&&(a2||b2||c2||d1||d2)&&" \
       "(a2||b2||c2||d2||e2)&&(!b1||!b2)&&(!b1||!c1)&&(!b1||!d1)&&(!b1||!e1)&&(b1||b2||c1||d1||e1)&&" \
       "(!b2||!c2)&&(!b2||!d1)&&(!b2||!d2)&&(!b2||!e1)&&(!b2||!e2)&&(!c1||!c2)&&(!c1||!d1)&&(!c1||!e1)&&" \
       "(!c2||!d2)&&(!c2||!e1)&&(!c2||!e2)&&(!d1||!d2)&&(!d1||!e1)&&(!d2||!e2)"
    
    clauses=clauses+mathematica_to_CNF(s, {
	"a1": vars[(n1,0)], "b1": vars[(n1,1)], "c1": vars[(n1,2)], "d1": vars[(n1,3)], "e1": vars[(n1,4)],
	"a2": vars[(n2,0)], "b2": vars[(n2,1)], "c2": vars[(n2,2)], "d2": vars[(n2,3)], "e2": vars[(n2,4)]})

...

# 11.The man who smokes Chesterfields lives in the house next to the man with the fox.
add_right_or_left("Chesterfield","Fox") # left or right

# 12.Kools are smoked in the house next to the house where the horse is kept.
add_right_or_left("Kools","Horse") # left or right
\end{lstlisting}

Вот и всё!
Полный исходный код: \url{https://github.com/DennisYurichev/SAT_SMT_article/blob/master/SAT/zebra/zebra_SAT.py}.

Итоговая CNF-проблема имеет 125 булевых переменных и 511 клозов: \\
\url{https://github.com/DennisYurichev/SAT_SMT_article/blob/master/SAT/zebra/1.cnf}.
Это очень легкая задача для любого SAT-солвера.
Даже мой игрушечный SAT-солвер (\ref{SAT_backtrack}) может решить её за \textasciitilde{}1 секунду на моем древнем
нетбуке с Intel Atom.

И конечно же, тут только одно решение, что и подтверждается при помощи Picosat.

\begin{lstlisting}
% python zebra_SAT.py
Yellow 1
Blue 2
Red 3
Ivory 4
Green 5
Norwegian 1
Ukrainian 2
Englishman 3
Spaniard 4
Japanese 5
Water 1
Tea 2
Milk 3
OrangeJuice 4
Coffee 5
Kools 1
Chesterfield 2
OldGold 3
LuckyStrike 4
Parliament 5
Fox 1
Horse 2
Snails 3
Dog 4
Zebra 5
\end{lstlisting}


%% TODO translate src
\subsection{Головоломка Зебры как SAT-проблема}
\label{Zebra_SAT}

Попробуем решить головоломку Зебры (\ref{zebra_SMT}) в SAT.

Я определю каждую переменную как вектор из пяти переменных, как я делал это раннее в солвере Судоку: \ref{Sudoku_SAT}.

Я также использую ф-цию \TT{POPCNT1}, но в отличие от предыдущего примера,
я использовал Wolfram Mathematica для генерирования её в CNF-форме:

\begin{lstlisting}
In[]:= tbl1=Table[PadLeft[IntegerDigits[i,2],5] ->If[Equal[DigitCount[i,2][[1]],1],1,0],{i,0,63}]
Out[]= {{0,0,0,0,0}->0,
{0,0,0,0,1}->1,
{0,0,0,1,0}->1,
{0,0,0,1,1}->0,
{0,0,1,0,0}->1,
{0,0,1,0,1}->0,

...

{1,1,1,1,0}->0,
{1,1,1,1,1}->0}

In[]:= BooleanConvert[BooleanFunction[tbl1,{a,b,c,d,e}],"CNF"]
Out[]= (!a||!b)&&(!a||!c)&&(!a||!d)&&(!a||!e)&&(a||b||c||d||e)&&(!b||!c)&&(!b||!d)&&(!b||!e)&&(!c||!d)&&(!c||!e)&&(!d||!e)
\end{lstlisting}

Также, как я предлагал раньше (\ref{OR_in_POPCNT1}), я использовал операцию \textit{ИЛИ} для второго шага.

\begin{lstlisting}
def mathematica_to_CNF (s, d):
    for k in d.keys():
        s=s.replace(k, d[k])
    s=s.replace("!", "-").replace("||", " ").replace("(", "").replace(")", "")
    s=s.split ("&&")
    return s

def add_popcnt1(v1, v2, v3, v4, v5):
    global clauses
    s="(!a||!b)&&" \
      "(!a||!c)&&" \
      "(!a||!d)&&" \
      "(!a||!e)&&" \
      "(!b||!c)&&" \
      "(!b||!d)&&" \
      "(!b||!e)&&" \
      "(!c||!d)&&" \
      "(!c||!e)&&" \
      "(!d||!e)&&" \
      "(a||b||c||d||e)"

    clauses=clauses+mathematica_to_CNF(s, {"a":v1, "b":v2, "c":v3, "d":v4, "e":v5})

...

# k=tuple: ("high-level" variable name, number of bit (0..4))
# v=variable number in CNF
vars={}
vars_last=1

...

def alloc_distinct_variables(names):
    global vars
    global vars_last
    for name in names:
        for i in range(5):
            vars[(name,i)]=str(vars_last)
            vars_last=vars_last+1

        add_popcnt1(vars[(name,0)], vars[(name,1)], vars[(name,2)], vars[(name,3)], vars[(name,4)])

    # make them distinct:
    for i in range(5):
        clauses.append(vars[(names[0],i)] + " " + vars[(names[1],i)] + " " + vars[(names[2],i)] + " " + vars[(names[3],i)] + " " + vars[(names[4],i)])

...

alloc_distinct_variables(["Yellow", "Blue", "Red", "Ivory", "Green"])
alloc_distinct_variables(["Norwegian", "Ukrainian", "Englishman", "Spaniard", "Japanese"])
alloc_distinct_variables(["Water", "Tea", "Milk", "OrangeJuice", "Coffee"])
alloc_distinct_variables(["Kools", "Chesterfield", "OldGold", "LuckyStrike", "Parliament"])
alloc_distinct_variables(["Fox", "Horse", "Snails", "Dog", "Zebra"])

...

\end{lstlisting}

Теперь у нас пять булевых переменных для каждой \textit{высокоуровневной} переменной,
и каждая группа переменных гарантированно будет иметь разные значения.

Теперь перечитаем условие головоломки: ``2. Англичанин живёт в красном доме.''.
Это легко.
В моих примерах на Z3 и KLEE я просто написал ``Englishman==Red''.
Та же история и здесь: мы просто добавляем клозы, показывающие, что 5 булевых переменных для ``Englishman''
должны равняться пяти переменных для ``Red''.

На самом низком уровне CNF, если мы хотим сказать, что две переменных должны равняться друг другу,
мы добавляем два клоза:

$(var1 \vee \neg var2) \wedge (\neg var1 \vee var2)$

Это означает что значения обоих \textit{var1} и \textit{var2} должны быть или \textit{Ложно} или \textit{Истинно},
но они не могут быть разными.

\begin{lstlisting}
def add_eq_clauses(var1, var2):
    global clauses
    clauses.append(var1 + " -" + var2)
    clauses.append("-"+var1 + " " + var2)

def add_eq (n1, n2):
    for i in range(5):
        add_eq_clauses(vars[(n1,i)], vars[(n2, i)])

...

# 2.The Englishman lives in the red house.
add_eq("Englishman","Red")

# 3.The Spaniard owns the dog.
add_eq("Spaniard","Dog")

# 4.Coffee is drunk in the green house.
add_eq("Coffee","Green")

...

\end{lstlisting}

Теперь следующие условия:
``9. В центральном доме пьют молоко.'' (т.е., в третьем доме), ``10. Норвежец живёт в первом доме.''
Мы можем присвоить булевы значения напрямую:

\begin{lstlisting}
# n=1..5
def add_eq_var_n (name, n):
    global clauses
    global vars
    for i in range(5):
        if i==n-1:
            clauses.append(vars[(name,i)]) # always True
        else:
            clauses.append("-"+vars[(name,i)]) # always False

...

# 9.Milk is drunk in the middle house.
add_eq_var_n("Milk",3) # i.e., 3rd house

# 10.The Norwegian lives in the first house.
add_eq_var_n("Norwegian",1)
\end{lstlisting}

Для ``Milk'' у нас значение ``0 0 1 0 0'', для ``Norwegian'': ``1 0 0 0 0''.

Что делать с этим?
``6. Зелёный дом стоит сразу справа от белого дома.''
Я могу сконструировать такое условие:

\begin{lstlisting}
    Ivory      Green
AND(1 0 0 0 0  0 1 0 0 0)
.. OR ..
AND(0 1 0 0 0  0 0 1 0 0)
.. OR ..
AND(0 0 1 0 0  0 0 0 1 0)
.. OR ..
AND(0 0 0 1 0  0 0 0 0 1)
\end{lstlisting}

Для ``белого/ivory'' тут нет ``0 0 0 0 1'', потому что он не может быть последним.
Теперь я конвертирую эти условия в CNF при помощи Wolfram Mathematica:

\begin{lstlisting}
In[]:= BooleanConvert[(a1&& !b1&&!c1&&!d1&&!e1&&!a2&& b2&&!c2&&!d2&&!e2) ||
(!a1&& b1&&!c1&&!d1&&!e1&&!a2&& !b2&&c2&&!d2&&!e2) ||
(!a1&& !b1&&c1&&!d1&&!e1&&!a2&& !b2&&!c2&&d2&&!e2) ||
(!a1&& !b1&&!c1&&d1&&!e1&&!a2&& !b2&&!c2&&!d2&&e2) ,"CNF"]

Out[]= (!a1||!b1)&&(!a1||!c1)&&(!a1||!d1)&&(a1||b1||c1||d1)&&!a2&&(!b1||!b2)&&(!b1||!c1)&&
(!b1||!d1)&&(b1||b2||c1||d1)&&(!b2||!c1)&&(!b2||!c2)&&(!b2||!d1)&&(!b2||!d2)&&(!b2||!e2)&&
(b2||c1||c2||d1)&&(b2||c2||d1||d2)&&(b2||c2||d2||e2)&&(!c1||!c2)&&(!c1||!d1)&&(!c2||!d1)&&
(!c2||!d2)&&(!c2||!e2)&&(!d1||!d2)&&(!d2||!e2)&&!e1
\end{lstlisting}

И вот фрагмент моего кода на Питоне:

\begin{lstlisting}
def add_right (n1, n2):
    global clauses
    s="(!a1||!b1)&&(!a1||!c1)&&(!a1||!d1)&&(a1||b1||c1||d1)&&!a2&&(!b1||!b2)&&(!b1||!c1)&&(!b1||!d1)&&" \
      "(b1||b2||c1||d1)&&(!b2||!c1)&&(!b2||!c2)&&(!b2||!d1)&&(!b2||!d2)&&(!b2||!e2)&&(b2||c1||c2||d1)&&" \
      "(b2||c2||d1||d2)&&(b2||c2||d2||e2)&&(!c1||!c2)&&(!c1||!d1)&&(!c2||!d1)&&(!c2||!d2)&&(!c2||!e2)&&" \
      "(!d1||!d2)&&(!d2||!e2)&&!e1"

    clauses=clauses+mathematica_to_CNF(s, {
	"a1": vars[(n1,0)], "b1": vars[(n1,1)], "c1": vars[(n1,2)], "d1": vars[(n1,3)], "e1": vars[(n1,4)],
	"a2": vars[(n2,0)], "b2": vars[(n2,1)], "c2": vars[(n2,2)], "d2": vars[(n2,3)], "e2": vars[(n2,4)]})

...

# 6.The green house is immediately to the right of the ivory house.
add_right("Ivory", "Green")
\end{lstlisting}

Что мы будем делать с этим?
``11. Сосед того, кто курит Chesterfield, держит лису.''
``12. В доме по соседству с тем, в котором держат лошадь, курят Kool.''

Мы не знаем с какой стороны, слева или справа, но знаем что они отличаются на единицу.
Вот какие клозы я добавлю:

\begin{lstlisting}
    Chesterfield  Fox
AND(0 0 0 0 1     0 0 0 1 0)
.. OR ..
AND(0 0 0 1 0     0 0 0 0 1)
AND(0 0 0 1 0     0 0 1 0 0)
.. OR ..
AND(0 0 1 0 0     0 1 0 0 0)
AND(0 0 1 0 0     0 0 0 1 0)
.. OR ..
AND(0 1 0 0 0     1 0 0 0 0)
AND(0 1 0 0 0     0 0 1 0 0)
.. OR ..
AND(1 0 0 0 0     0 1 0 0 0)
\end{lstlisting}

И снова могу сконвертировать это всё в CNF при помощи Mathematica:

\begin{lstlisting}
In[]:= BooleanConvert[(a1&& !b1&&!c1&&!d1&&!e1&&!a2&& b2&&!c2&&!d2&&!e2) ||

(!a1&& b1&&!c1&&!d1&&!e1&&a2&& !b2&&!c2&&!d2&&!e2) ||
(!a1&& b1&&!c1&&!d1&&!e1&&!a2&& !b2&&c2&&!d2&&!e2) ||

(!a1&& !b1&&c1&&!d1&&!e1&&!a2&& b2&&!c2&&!d2&&!e2) ||
(!a1&& !b1&&c1&&!d1&&!e1&&!a2&& !b2&&!c2&&d2&&!e2) ||

(!a1&& !b1&&!c1&&d1&&!e1&&!a2&& !b2&&c2&&!d2&&!e2) ||
(!a1&& !b1&&!c1&&d1&&!e1&&!a2&& !b2&&!c2&&!d2&&e2) ||

(!a1&& !b1&&!c1&&!d1&&e1&&!a2&& !b2&&!c2&&d2&&!e2) ,"CNF"]

Out[]= (!a1||!b1)&&(!a1||!c1)&&(!a1||!d1)&&(!a1||!e1)&&(a1||b1||c1||d1||e1)&&(!a2||b1)&&(!a2||!b2)&&
(!a2||!c2)&&(!a2||!d2)&&(!a2||!e2)&&(a2||b2||c1||c2||d1||e1)&&(a2||b2||c2||d1||d2)&&(a2||b2||c2||d2||e2)&&
(!b1||!b2)&&(!b1||!c1)&&(!b1||!d1)&&(!b1||!e1)&&(b1||b2||c1||d1||e1)&&(!b2||!c2)&&(!b2||!d1)&&(!b2||!d2)&&
(!b2||!e1)&&(!b2||!e2)&&(!c1||!c2)&&(!c1||!d1)&&(!c1||!e1)&&(!c2||!d2)&&(!c2||!e1)&&(!c2||!e2)&&
(!d1||!d2)&&(!d1||!e1)&&(!d2||!e2)
\end{lstlisting}

И вот мой код:

\begin{lstlisting}
def add_right_or_left (n1, n2):
    global clauses
    s="(!a1||!b1)&&(!a1||!c1)&&(!a1||!d1)&&(!a1||!e1)&&(a1||b1||c1||d1||e1)&&(!a2||b1)&&" \
      "(!a2||!b2)&&(!a2||!c2)&&(!a2||!d2)&&(!a2||!e2)&&(a2||b2||c1||c2||d1||e1)&&(a2||b2||c2||d1||d2)&&" \
       "(a2||b2||c2||d2||e2)&&(!b1||!b2)&&(!b1||!c1)&&(!b1||!d1)&&(!b1||!e1)&&(b1||b2||c1||d1||e1)&&" \
       "(!b2||!c2)&&(!b2||!d1)&&(!b2||!d2)&&(!b2||!e1)&&(!b2||!e2)&&(!c1||!c2)&&(!c1||!d1)&&(!c1||!e1)&&" \
       "(!c2||!d2)&&(!c2||!e1)&&(!c2||!e2)&&(!d1||!d2)&&(!d1||!e1)&&(!d2||!e2)"
    
    clauses=clauses+mathematica_to_CNF(s, {
	"a1": vars[(n1,0)], "b1": vars[(n1,1)], "c1": vars[(n1,2)], "d1": vars[(n1,3)], "e1": vars[(n1,4)],
	"a2": vars[(n2,0)], "b2": vars[(n2,1)], "c2": vars[(n2,2)], "d2": vars[(n2,3)], "e2": vars[(n2,4)]})

...

# 11.The man who smokes Chesterfields lives in the house next to the man with the fox.
add_right_or_left("Chesterfield","Fox") # left or right

# 12.Kools are smoked in the house next to the house where the horse is kept.
add_right_or_left("Kools","Horse") # left or right
\end{lstlisting}

Вот и всё!
Полный исходный код: \url{https://github.com/DennisYurichev/SAT_SMT_article/blob/master/SAT/zebra/zebra_SAT.py}.

Итоговая CNF-проблема имеет 125 булевых переменных и 511 клозов: \\
\url{https://github.com/DennisYurichev/SAT_SMT_article/blob/master/SAT/zebra/1.cnf}.
Это очень легкая задача для любого SAT-солвера.
Даже мой игрушечный SAT-солвер (\ref{SAT_backtrack}) может решить её за \textasciitilde{}1 секунду на моем древнем
нетбуке с Intel Atom.

И конечно же, тут только одно решение, что и подтверждается при помощи Picosat.

\begin{lstlisting}
% python zebra_SAT.py
Yellow 1
Blue 2
Red 3
Ivory 4
Green 5
Norwegian 1
Ukrainian 2
Englishman 3
Spaniard 4
Japanese 5
Water 1
Tea 2
Milk 3
OrangeJuice 4
Coffee 5
Kools 1
Chesterfield 2
OldGold 3
LuckyStrike 4
Parliament 5
Fox 1
Horse 2
Snails 3
Dog 4
Zebra 5
\end{lstlisting}


\subsubsection{Судоку}

Я также переписал пример с Судоку (\ref{sudoku_SMT}) для KLEE:

\lstinputlisting[numbers=left]{puzzles/sudoku/KLEE/klee_sudoku_or1.c}

Запустим:

\begin{lstlisting}
% clang -emit-llvm -c -g klee_sudoku_or1.c
...

\$ time klee klee_sudoku_or1.bc
KLEE: output directory is "/home/klee/klee-out-98"
KLEE: WARNING: undefined reference to function: klee_assert
KLEE: WARNING ONCE: calling external: klee_assert(0)
KLEE: ERROR: /home/klee/klee_sudoku_or1.c:93: failed external call: klee_assert
KLEE: NOTE: now ignoring this error at this location

KLEE: done: total instructions = 7512
KLEE: done: completed paths = 161
KLEE: done: generated tests = 161

real    3m44.111s
user    3m43.319s
sys     0m0.951s
\end{lstlisting}

Это работает медленнее (на моем ноутбуке Intel Core i3-3110M 2.4GHz) в сравнении с решением на Z3Py (\ref{sudoku_SMT}).

Но ответ верный:

\begin{lstlisting}
% ls klee-last | grep err
test000161.external.err

% ktest-tool --write-ints klee-last/test000161.ktest
ktest file : 'klee-last/test000161.ktest'
args       : ['klee_sudoku_or1.bc']
num objects: 1
object    0: name: b'cells'
object    0: size: 81
object    0: data: b'\x01\x04\x05\x03\x02\x07\x06\t\x08\x08\x03\t\x06\x05\x04\x01\x02\x07\x06\x07\x02\t\x01\x08\x05\x04\x03\x04\t\x06\x01\x08\x05\x03\x07\x02\x02\x01\x08\x04\x07\x03\t\x05\x06\x07\x05\x03\x02\t\x06\x04\x08\x01\x03\x06\x07\x05\x04\x02\x08\x01\t\t\x08\x04\x07\x06\x01\x02\x03\x05\x05\x02\x01\x08\x03\t\x07\x06\x04'
\end{lstlisting}

Символ \TT{\textbackslash{}t} в Си/Си++ имеет код 9,
а KLEE выводит массив байт как строку в Си/Си++, так что он показывает некоторые значения в таком виде.
Мы может просто помнить, что здесь 9 во всех местах, где мы видим \TT{\textbackslash{}t}.
Решение, хотя и не отформатировано должным образом, тем не мнее, корректно.

Кстати, в строках 42 и 43 вы можете увидеть, как мы говорим KLEE, что все элементы массива должны быть в некоторых
пределах.
Если закомментируем эти строки, получим это:

\begin{lstlisting}
% time klee klee_sudoku_or1.bc
KLEE: output directory is "/home/klee/klee-out-100"
KLEE: WARNING: undefined reference to function: klee_assert
KLEE: ERROR: /home/klee/klee_sudoku_or1.c:51: overshift error
KLEE: NOTE: now ignoring this error at this location
KLEE: ERROR: /home/klee/klee_sudoku_or1.c:51: overshift error
KLEE: NOTE: now ignoring this error at this location
KLEE: ERROR: /home/klee/klee_sudoku_or1.c:51: overshift error
KLEE: NOTE: now ignoring this error at this location
...
\end{lstlisting}

KLEE предупреждает нас, что значение сдвига на строке 51 слишком большое.
Действительно, KLEE может пробовать все значения байт вплоть до 255 (0xFF), что, в свою очередь, здесь бессмысленно,
и может быть симптомом ошибки, так что KLEE предупреждает об этом.

Снова попробуем использовать \TT{klee\_assume()}:

\lstinputlisting{puzzles/sudoku/KLEE/klee_sudoku_or2.c}

\begin{lstlisting}
% time klee klee_sudoku_or2.bc
KLEE: output directory is "/home/klee/klee-out-99"
KLEE: WARNING: undefined reference to function: klee_assert
KLEE: WARNING ONCE: calling external: klee_assert(0)
KLEE: ERROR: /home/klee/klee_sudoku_or2.c:93: failed external call: klee_assert
KLEE: NOTE: now ignoring this error at this location

KLEE: done: total instructions = 7119
KLEE: done: completed paths = 1
KLEE: done: generated tests = 1

real    0m35.312s
user    0m34.945s
sys     0m0.318s
\end{lstlisting}

Это работает намного быстрее: наверное, KLEE работает с этой \textit{intrinsic} специальным образом.
И, как мы видим, только один путь был найден (тот, который нам действительно интересен) вместо 161.

Но это всё еще намного медленнее, чем решение на Z3Py.


% TODO translate src
\subsection{Головоломка Зебры как SAT-проблема}
\label{Zebra_SAT}

Попробуем решить головоломку Зебры (\ref{zebra_SMT}) в SAT.

Я определю каждую переменную как вектор из пяти переменных, как я делал это раннее в солвере Судоку: \ref{Sudoku_SAT}.

Я также использую ф-цию \TT{POPCNT1}, но в отличие от предыдущего примера,
я использовал Wolfram Mathematica для генерирования её в CNF-форме:

\begin{lstlisting}
In[]:= tbl1=Table[PadLeft[IntegerDigits[i,2],5] ->If[Equal[DigitCount[i,2][[1]],1],1,0],{i,0,63}]
Out[]= {{0,0,0,0,0}->0,
{0,0,0,0,1}->1,
{0,0,0,1,0}->1,
{0,0,0,1,1}->0,
{0,0,1,0,0}->1,
{0,0,1,0,1}->0,

...

{1,1,1,1,0}->0,
{1,1,1,1,1}->0}

In[]:= BooleanConvert[BooleanFunction[tbl1,{a,b,c,d,e}],"CNF"]
Out[]= (!a||!b)&&(!a||!c)&&(!a||!d)&&(!a||!e)&&(a||b||c||d||e)&&(!b||!c)&&(!b||!d)&&(!b||!e)&&(!c||!d)&&(!c||!e)&&(!d||!e)
\end{lstlisting}

Также, как я предлагал раньше (\ref{OR_in_POPCNT1}), я использовал операцию \textit{ИЛИ} для второго шага.

\begin{lstlisting}
def mathematica_to_CNF (s, d):
    for k in d.keys():
        s=s.replace(k, d[k])
    s=s.replace("!", "-").replace("||", " ").replace("(", "").replace(")", "")
    s=s.split ("&&")
    return s

def add_popcnt1(v1, v2, v3, v4, v5):
    global clauses
    s="(!a||!b)&&" \
      "(!a||!c)&&" \
      "(!a||!d)&&" \
      "(!a||!e)&&" \
      "(!b||!c)&&" \
      "(!b||!d)&&" \
      "(!b||!e)&&" \
      "(!c||!d)&&" \
      "(!c||!e)&&" \
      "(!d||!e)&&" \
      "(a||b||c||d||e)"

    clauses=clauses+mathematica_to_CNF(s, {"a":v1, "b":v2, "c":v3, "d":v4, "e":v5})

...

# k=tuple: ("high-level" variable name, number of bit (0..4))
# v=variable number in CNF
vars={}
vars_last=1

...

def alloc_distinct_variables(names):
    global vars
    global vars_last
    for name in names:
        for i in range(5):
            vars[(name,i)]=str(vars_last)
            vars_last=vars_last+1

        add_popcnt1(vars[(name,0)], vars[(name,1)], vars[(name,2)], vars[(name,3)], vars[(name,4)])

    # make them distinct:
    for i in range(5):
        clauses.append(vars[(names[0],i)] + " " + vars[(names[1],i)] + " " + vars[(names[2],i)] + " " + vars[(names[3],i)] + " " + vars[(names[4],i)])

...

alloc_distinct_variables(["Yellow", "Blue", "Red", "Ivory", "Green"])
alloc_distinct_variables(["Norwegian", "Ukrainian", "Englishman", "Spaniard", "Japanese"])
alloc_distinct_variables(["Water", "Tea", "Milk", "OrangeJuice", "Coffee"])
alloc_distinct_variables(["Kools", "Chesterfield", "OldGold", "LuckyStrike", "Parliament"])
alloc_distinct_variables(["Fox", "Horse", "Snails", "Dog", "Zebra"])

...

\end{lstlisting}

Теперь у нас пять булевых переменных для каждой \textit{высокоуровневной} переменной,
и каждая группа переменных гарантированно будет иметь разные значения.

Теперь перечитаем условие головоломки: ``2. Англичанин живёт в красном доме.''.
Это легко.
В моих примерах на Z3 и KLEE я просто написал ``Englishman==Red''.
Та же история и здесь: мы просто добавляем клозы, показывающие, что 5 булевых переменных для ``Englishman''
должны равняться пяти переменных для ``Red''.

На самом низком уровне CNF, если мы хотим сказать, что две переменных должны равняться друг другу,
мы добавляем два клоза:

$(var1 \vee \neg var2) \wedge (\neg var1 \vee var2)$

Это означает что значения обоих \textit{var1} и \textit{var2} должны быть или \textit{Ложно} или \textit{Истинно},
но они не могут быть разными.

\begin{lstlisting}
def add_eq_clauses(var1, var2):
    global clauses
    clauses.append(var1 + " -" + var2)
    clauses.append("-"+var1 + " " + var2)

def add_eq (n1, n2):
    for i in range(5):
        add_eq_clauses(vars[(n1,i)], vars[(n2, i)])

...

# 2.The Englishman lives in the red house.
add_eq("Englishman","Red")

# 3.The Spaniard owns the dog.
add_eq("Spaniard","Dog")

# 4.Coffee is drunk in the green house.
add_eq("Coffee","Green")

...

\end{lstlisting}

Теперь следующие условия:
``9. В центральном доме пьют молоко.'' (т.е., в третьем доме), ``10. Норвежец живёт в первом доме.''
Мы можем присвоить булевы значения напрямую:

\begin{lstlisting}
# n=1..5
def add_eq_var_n (name, n):
    global clauses
    global vars
    for i in range(5):
        if i==n-1:
            clauses.append(vars[(name,i)]) # always True
        else:
            clauses.append("-"+vars[(name,i)]) # always False

...

# 9.Milk is drunk in the middle house.
add_eq_var_n("Milk",3) # i.e., 3rd house

# 10.The Norwegian lives in the first house.
add_eq_var_n("Norwegian",1)
\end{lstlisting}

Для ``Milk'' у нас значение ``0 0 1 0 0'', для ``Norwegian'': ``1 0 0 0 0''.

Что делать с этим?
``6. Зелёный дом стоит сразу справа от белого дома.''
Я могу сконструировать такое условие:

\begin{lstlisting}
    Ivory      Green
AND(1 0 0 0 0  0 1 0 0 0)
.. OR ..
AND(0 1 0 0 0  0 0 1 0 0)
.. OR ..
AND(0 0 1 0 0  0 0 0 1 0)
.. OR ..
AND(0 0 0 1 0  0 0 0 0 1)
\end{lstlisting}

Для ``белого/ivory'' тут нет ``0 0 0 0 1'', потому что он не может быть последним.
Теперь я конвертирую эти условия в CNF при помощи Wolfram Mathematica:

\begin{lstlisting}
In[]:= BooleanConvert[(a1&& !b1&&!c1&&!d1&&!e1&&!a2&& b2&&!c2&&!d2&&!e2) ||
(!a1&& b1&&!c1&&!d1&&!e1&&!a2&& !b2&&c2&&!d2&&!e2) ||
(!a1&& !b1&&c1&&!d1&&!e1&&!a2&& !b2&&!c2&&d2&&!e2) ||
(!a1&& !b1&&!c1&&d1&&!e1&&!a2&& !b2&&!c2&&!d2&&e2) ,"CNF"]

Out[]= (!a1||!b1)&&(!a1||!c1)&&(!a1||!d1)&&(a1||b1||c1||d1)&&!a2&&(!b1||!b2)&&(!b1||!c1)&&
(!b1||!d1)&&(b1||b2||c1||d1)&&(!b2||!c1)&&(!b2||!c2)&&(!b2||!d1)&&(!b2||!d2)&&(!b2||!e2)&&
(b2||c1||c2||d1)&&(b2||c2||d1||d2)&&(b2||c2||d2||e2)&&(!c1||!c2)&&(!c1||!d1)&&(!c2||!d1)&&
(!c2||!d2)&&(!c2||!e2)&&(!d1||!d2)&&(!d2||!e2)&&!e1
\end{lstlisting}

И вот фрагмент моего кода на Питоне:

\begin{lstlisting}
def add_right (n1, n2):
    global clauses
    s="(!a1||!b1)&&(!a1||!c1)&&(!a1||!d1)&&(a1||b1||c1||d1)&&!a2&&(!b1||!b2)&&(!b1||!c1)&&(!b1||!d1)&&" \
      "(b1||b2||c1||d1)&&(!b2||!c1)&&(!b2||!c2)&&(!b2||!d1)&&(!b2||!d2)&&(!b2||!e2)&&(b2||c1||c2||d1)&&" \
      "(b2||c2||d1||d2)&&(b2||c2||d2||e2)&&(!c1||!c2)&&(!c1||!d1)&&(!c2||!d1)&&(!c2||!d2)&&(!c2||!e2)&&" \
      "(!d1||!d2)&&(!d2||!e2)&&!e1"

    clauses=clauses+mathematica_to_CNF(s, {
	"a1": vars[(n1,0)], "b1": vars[(n1,1)], "c1": vars[(n1,2)], "d1": vars[(n1,3)], "e1": vars[(n1,4)],
	"a2": vars[(n2,0)], "b2": vars[(n2,1)], "c2": vars[(n2,2)], "d2": vars[(n2,3)], "e2": vars[(n2,4)]})

...

# 6.The green house is immediately to the right of the ivory house.
add_right("Ivory", "Green")
\end{lstlisting}

Что мы будем делать с этим?
``11. Сосед того, кто курит Chesterfield, держит лису.''
``12. В доме по соседству с тем, в котором держат лошадь, курят Kool.''

Мы не знаем с какой стороны, слева или справа, но знаем что они отличаются на единицу.
Вот какие клозы я добавлю:

\begin{lstlisting}
    Chesterfield  Fox
AND(0 0 0 0 1     0 0 0 1 0)
.. OR ..
AND(0 0 0 1 0     0 0 0 0 1)
AND(0 0 0 1 0     0 0 1 0 0)
.. OR ..
AND(0 0 1 0 0     0 1 0 0 0)
AND(0 0 1 0 0     0 0 0 1 0)
.. OR ..
AND(0 1 0 0 0     1 0 0 0 0)
AND(0 1 0 0 0     0 0 1 0 0)
.. OR ..
AND(1 0 0 0 0     0 1 0 0 0)
\end{lstlisting}

И снова могу сконвертировать это всё в CNF при помощи Mathematica:

\begin{lstlisting}
In[]:= BooleanConvert[(a1&& !b1&&!c1&&!d1&&!e1&&!a2&& b2&&!c2&&!d2&&!e2) ||

(!a1&& b1&&!c1&&!d1&&!e1&&a2&& !b2&&!c2&&!d2&&!e2) ||
(!a1&& b1&&!c1&&!d1&&!e1&&!a2&& !b2&&c2&&!d2&&!e2) ||

(!a1&& !b1&&c1&&!d1&&!e1&&!a2&& b2&&!c2&&!d2&&!e2) ||
(!a1&& !b1&&c1&&!d1&&!e1&&!a2&& !b2&&!c2&&d2&&!e2) ||

(!a1&& !b1&&!c1&&d1&&!e1&&!a2&& !b2&&c2&&!d2&&!e2) ||
(!a1&& !b1&&!c1&&d1&&!e1&&!a2&& !b2&&!c2&&!d2&&e2) ||

(!a1&& !b1&&!c1&&!d1&&e1&&!a2&& !b2&&!c2&&d2&&!e2) ,"CNF"]

Out[]= (!a1||!b1)&&(!a1||!c1)&&(!a1||!d1)&&(!a1||!e1)&&(a1||b1||c1||d1||e1)&&(!a2||b1)&&(!a2||!b2)&&
(!a2||!c2)&&(!a2||!d2)&&(!a2||!e2)&&(a2||b2||c1||c2||d1||e1)&&(a2||b2||c2||d1||d2)&&(a2||b2||c2||d2||e2)&&
(!b1||!b2)&&(!b1||!c1)&&(!b1||!d1)&&(!b1||!e1)&&(b1||b2||c1||d1||e1)&&(!b2||!c2)&&(!b2||!d1)&&(!b2||!d2)&&
(!b2||!e1)&&(!b2||!e2)&&(!c1||!c2)&&(!c1||!d1)&&(!c1||!e1)&&(!c2||!d2)&&(!c2||!e1)&&(!c2||!e2)&&
(!d1||!d2)&&(!d1||!e1)&&(!d2||!e2)
\end{lstlisting}

И вот мой код:

\begin{lstlisting}
def add_right_or_left (n1, n2):
    global clauses
    s="(!a1||!b1)&&(!a1||!c1)&&(!a1||!d1)&&(!a1||!e1)&&(a1||b1||c1||d1||e1)&&(!a2||b1)&&" \
      "(!a2||!b2)&&(!a2||!c2)&&(!a2||!d2)&&(!a2||!e2)&&(a2||b2||c1||c2||d1||e1)&&(a2||b2||c2||d1||d2)&&" \
       "(a2||b2||c2||d2||e2)&&(!b1||!b2)&&(!b1||!c1)&&(!b1||!d1)&&(!b1||!e1)&&(b1||b2||c1||d1||e1)&&" \
       "(!b2||!c2)&&(!b2||!d1)&&(!b2||!d2)&&(!b2||!e1)&&(!b2||!e2)&&(!c1||!c2)&&(!c1||!d1)&&(!c1||!e1)&&" \
       "(!c2||!d2)&&(!c2||!e1)&&(!c2||!e2)&&(!d1||!d2)&&(!d1||!e1)&&(!d2||!e2)"
    
    clauses=clauses+mathematica_to_CNF(s, {
	"a1": vars[(n1,0)], "b1": vars[(n1,1)], "c1": vars[(n1,2)], "d1": vars[(n1,3)], "e1": vars[(n1,4)],
	"a2": vars[(n2,0)], "b2": vars[(n2,1)], "c2": vars[(n2,2)], "d2": vars[(n2,3)], "e2": vars[(n2,4)]})

...

# 11.The man who smokes Chesterfields lives in the house next to the man with the fox.
add_right_or_left("Chesterfield","Fox") # left or right

# 12.Kools are smoked in the house next to the house where the horse is kept.
add_right_or_left("Kools","Horse") # left or right
\end{lstlisting}

Вот и всё!
Полный исходный код: \url{https://github.com/DennisYurichev/SAT_SMT_article/blob/master/SAT/zebra/zebra_SAT.py}.

Итоговая CNF-проблема имеет 125 булевых переменных и 511 клозов: \\
\url{https://github.com/DennisYurichev/SAT_SMT_article/blob/master/SAT/zebra/1.cnf}.
Это очень легкая задача для любого SAT-солвера.
Даже мой игрушечный SAT-солвер (\ref{SAT_backtrack}) может решить её за \textasciitilde{}1 секунду на моем древнем
нетбуке с Intel Atom.

И конечно же, тут только одно решение, что и подтверждается при помощи Picosat.

\begin{lstlisting}
% python zebra_SAT.py
Yellow 1
Blue 2
Red 3
Ivory 4
Green 5
Norwegian 1
Ukrainian 2
Englishman 3
Spaniard 4
Japanese 5
Water 1
Tea 2
Milk 3
OrangeJuice 4
Coffee 5
Kools 1
Chesterfield 2
OldGold 3
LuckyStrike 4
Parliament 5
Fox 1
Horse 2
Snails 3
Dog 4
Zebra 5
\end{lstlisting}




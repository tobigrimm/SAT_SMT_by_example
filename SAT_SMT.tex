\documentclass{article}

\usepackage{cmap}

\usepackage[russian,english]{babel}
\usepackage[T2A]{fontenc}
\usepackage[default]{sourcesanspro}

\usepackage{amsmath, amssymb, graphics, setspace}
\usepackage[table]{xcolor}% http://ctan.org/pkg/xcolor

\usepackage{listings}
\usepackage{longtable}
\usepackage{pmboxdraw}
\usepackage{url}
\usepackage[cm]{fullpage}
\usepackage{graphicx}
\usepackage{framed}
\usepackage{color}
\usepackage{float}
\usepackage{tikz}
\usepackage[margin=0.5in,headheight=12.5pt]{geometry}
\usepackage[footnote,printonlyused,withpage]{acronym}
\usetikzlibrary{arrows}
\usepackage[nottoc]{tocbibind}
%\usepackage{esint}
%\usepackage{charter}
%\usepackage[charter]{mathdesign}
\usepackage[]{hyperref} % should be last

\definecolor{lstbgcolor}{rgb}{0.94,0.94,0.94}
\definecolor{light-gray}{gray}{0.87}

%\newcommand{\TT}[1]{\texttt{#1}}
\newcommand*{\TT}[1]{\colorbox{light-gray}{\texttt{#1}}}

\newcommand{\TITLE}{Quick introduction into SAT/SMT solvers and symbolic execution}
\newcommand{\AUTHOR}{Dennis Yurichev \TT{<dennis(a)yurichev.com>}}

\hypersetup{
    pdftex,
    colorlinks=true,
    allcolors=blue,
    pdfauthor={\AUTHOR},
    pdftitle={\TITLE},
    pdfpagemode=None
}

\lstset{
    %backgroundcolor=\color{lstbgcolor},
    %backgroundcolor=\color{light-gray},
    basicstyle=\ttfamily\small, 
    literate={~} {$\sim$}{1},
    breaklines=true,
    frame=single,
    columns=fullflexible,keepspaces,
    escapeinside=§§,
    inputencoding=utf8
}

\author{\AUTHOR}
\title{\TITLE}
\date{December 2015 -- April 2017}

\begin{document}

\maketitle

\tableofcontents

% sections
\EN{\section{This is a draft!}

This is very early draft, but still can be interesting for someone.

Latest version is always available at \url{http://yurichev.com/writings/SAT_SMT_draft-EN.pdf}.
Russian version is at \url{http://yurichev.com/writings/SAT_SMT_draft-RU.pdf}.

New parts are appearing here from time to time, see: \url{https://github.com/dennis714/SAT_SMT_article/blob/master/ChangeLog}.

For news about updates, you may subscribe my 
twitter\footnote{\url{https://twitter.com/yurichev}}, 
facebook\footnote{\url{https://www.facebook.com/dennis.yurichev.5}}, 
or github repo\footnote{\url{https://github.com/dennis714/SAT_SMT_article}}.

\section{Thanks}

Leonardo Mendonça de Moura\footnote{\url{https://www.microsoft.com/en-us/research/people/leonardo/}},
Nikolaj Bjørner\footnote{\url{https://www.microsoft.com/en-us/research/people/nbjorner/}},
Armin Biere\footnote{\url{http://fmv.jku.at/biere/}} and
Mate Soos\footnote{\url{https://www.msoos.org/}},
for help.

Alex ``clayrat'' Gryzlov found a bug.

\section{Praise}

``An excellent source of well-worked through and motivating examples of using Z3's python interface.''
\footnote{\url{https://github.com/Z3Prover/z3/wiki}}
(Nikolaj Bjorner, one of Z3's authors).

``Impressive collection of fun examples!''
(Pascal Fontaine\footnote{\url{https://members.loria.fr/PFontaine/}}, one of veriT solver's authors.)

\section{Introduction}

\ac{SAT}/\ac{SMT} solvers can be viewed as solvers of huge systems of equations.
The difference is that \ac{SMT} solvers takes systems in arbitrary format,
while \ac{SAT} solvers are limited to boolean equations in \ac{CNF} form.

A lot of real world problems can be represented as problems of solving system of equations.

\section{Is it a hype? Yet another fad?}

Some people say, this is just another hype.
No, \ac{SAT} is old enough and fundamental to \ac{CS}.
The reason of increased interest to it is that computers gets faster over the last couple decades,
so there are attempts to solve old problems using \ac{SAT}/\ac{SMT}, which were inaccessible in past.

}
\RU{\section{Отказ от ответственности}

Автор этих строк ни в какой мере не является экспертом в SAT/SMT.
Это не сколько книга, сколько студенческий конспект.
Воспринимайте её с разумной долей сомнения...

\section{Последняя версия}

Последняя версия всегда доступна на \url{http://yurichev.com/writings/SAT_SMT_draft-RU.pdf}.
Англоязычная версия: \url{http://yurichev.com/writings/SAT_SMT_draft-EN.pdf}.

Время от времени здесь появляется что-то новое, см: \url{https://github.com/DennisYurichev/SAT_SMT_article/blob/master/ChangeLog}.

\section{Благодарности}

Armin Biere\footnote{\url{http://fmv.jku.at/biere/}} долго и терпеливо отвечал на мои скучные вопросы.

Также помогали:
Leonardo Mendonça de Moura\footnote{\url{https://www.microsoft.com/en-us/research/people/leonardo/}},
Nikolaj Bjørner\footnote{\url{https://www.microsoft.com/en-us/research/people/nbjorner/}},
Mate Soos\footnote{\url{https://www.msoos.org/}}.

Masahiro Sakai\footnote{\url{https://twitter.com/masahiro_sakai}} помог с головоломкой numberlink.
% TODO \ref

Алекс ``clayrat'' Грызлов и @mztropics в твиттере нашли пару ошибок.

\section{Отзывы}

``An excellent source of well-worked through and motivating examples of using Z3's python interface.''
\footnote{\url{https://github.com/Z3Prover/z3/wiki}}
(Nikolaj Bjorner, один из авторов Z3).

``Impressive collection of fun examples!''
(Pascal Fontaine\footnote{\url{https://members.loria.fr/PFontaine/}}, один из авторов veriT solver.)

\section{Введение}

\ac{SAT}/\ac{SMT} солверы можно рассматривать как солверы огромных систем уравнений.
Разница в том, что \ac{SMT}-солверы берут системы в произвольном формате,
в то время как \ac{SAT}-солверы ограничены булевыми уравнениями вида \ac{CNF}.

Огромное количество проблем их практики можно представить как проблемы решения систем уравнений.

\section{Это хайп? Очередная мода?}

Некоторые люди говорят, что это очередной хайп.
Нет, \ac{SAT} достаточно стар, чтобы быть фундаментальным в \ac{CS}.
Причина повышенного интереса в том, что компьютеры стали работать быстрее,
так что теперь больше попыток решать старые проблемы используя 
\ac{SAT}/\ac{SMT}, которые раннее были недоступны.

}


\section{\ac{SMT}-solvers}

\subsection{School-level system of equations}

I've got this school-level system of equations copypasted from Wikipedia\footnote{\url{https://en.wikipedia.org/wiki/System_of_linear_equations}}:

\begin{alignat*}{7}
3x &&\; + \;&& 2y             &&\; - \;&& z  &&\; = \;&& 1 & \\
2x &&\; - \;&& 2y             &&\; + \;&& 4z &&\; = \;&& -2 & \\
-x &&\; + \;&& \tfrac{1}{2} y &&\; - \;&& z  &&\; = \;&& 0 &
\end{alignat*}

Will it be possible to solve it using Z3? Here it is:

\begin{lstlisting}
#!/usr/bin/python
from z3 import *

x = Real('x')
y = Real('y')
z = Real('z')
s = Solver()
s.add(3*x + 2*y - z == 1)
s.add(2*x - 2*y + 4*z == -2)
s.add(-x + 0.5*y - z == 0)
print s.check()
print s.model()
\end{lstlisting}

We see this after run:

\begin{lstlisting}
sat
[z = -2, y = -2, x = 1]
\end{lstlisting}

If we change equation in some way so it will have no solution, s.check() will return ``unsat''.

I've used ``Real'' \textit{sort} (some kind of data type in \ac{SMT}-solvers) because the last expression has $\frac{1}{2}$, which is, of course, a real number.
For the integer system of equations, ``Int'' \textit{sort} would work fine.

Python (and other high-level PLs like C\#) interface is highly popular, because it's practical, but in fact, 
there is a standard language for \ac{SMT}-solvers named SMT-LIB\footnote{\url{http://smtlib.cs.uiowa.edu/papers/smt-lib-reference-v2.5-r2015-06-28.pdf}}.

Our example rewritten to it looks like this:

\begin{lstlisting}
(declare-const x Real)
(declare-const y Real)
(declare-const z Real)
(assert (=(-(+(* 3 x) (* 2 y)) z) 1))
(assert (=(+(-(* 2 x) (* 2 y)) (* 4 z)) -2))
(assert (=(-(+ (- 0 x) (* 0.5 y)) z) 0))
(check-sat)
(get-model)
\end{lstlisting}

This language is very close to LISP, but hard to read for untrained eyes.

Now we run it:

% FIXME:
\begin{lstlisting}
\$ z3 -smt2 example.smt
sat
(model
  (define-fun z () Real
    (- 2.0))
  (define-fun y () Real
    (- 2.0))
  (define-fun x () Real
    1.0)
)
\end{lstlisting}

So when you look back to my Python code, you may feel that these 3 expressions could be executed.
This is not true: Z3Py API offers overloaded operators, so expressions are constructed and passed into the guts of Z3 without any execution
\footnote{\url{https://github.com/Z3Prover/z3/blob/6e852762baf568af2aad1e35019fdf41189e4e12/src/api/python/z3.py}}.
I would call it ``embedded \ac{DSL}''.

% FIXME \tt, etc
Same thing for Z3 C++ API, you may find there ``operator+'' declarations and many more
\footnote{\url{https://github.com/Z3Prover/z3/blob/6e852762baf568af2aad1e35019fdf41189e4e12/src/api/c\%2B\%2B/z3\%2B\%2B.h}}.

Z3 APIs for Java, ML and .NET are also exist\footnote{\url{https://github.com/Z3Prover/z3/tree/6e852762baf568af2aad1e35019fdf41189e4e12/src/api}}.\\
\\
Z3Py tutorial: \url{https://github.com/ericpony/z3py-tutorial}.

Z3 tutorial which uses SMT-LIB language: \url{http://rise4fun.com/Z3/tutorial/guide}.

\subsection{Another school-level system of equations}

I've found this somewhere at Facebook:

\begin{figure}[H]
\centering
\includegraphics[scale=0.3]{SMT/equation.jpg}
\caption{System of equations}
\end{figure}

It's that easy to solve it in Z3 \ac{SMT}-solver:

\begin{lstlisting}
#!/usr/bin/python
from z3 import *

circle, square, triangle = Ints('circle square triangle')
s = Solver()
s.add(circle+circle==10)
s.add(circle*square+square==12)
s.add(circle*square-triangle*circle==circle)
print s.check()
print s.model()
\end{lstlisting}

\begin{lstlisting}
sat
[triangle = 1, square = 2, circle = 5]
\end{lstlisting}

\subsection{Finding modular inverse}

And now one useful equation.

It's possible to replace division operation (slow) by multiplication (faster).
Read \href{http://beginners.re/15-Nov-2015/Reverse_Engineering_for_Beginners-en.pdf#page=490&zoom=auto,-107,806}{here} or \href{http://yurichev.com/blog/modulo/}{here} about it.

In shortest possible terms, here is how it works.

If you need to divide integer by 13, this is the same as multiplication it by $\frac{1}{13}$.
But you operate on integer numbers (\ac{CPU} registers) and can't use ratios.

Given the fact that the division by $2^n$ on \ac{CPU} occurring at lightning speed, we can reformulate the problem as:

\begin{center}
{\Large $\frac{x}{13} = x \cdot \frac{1}{13} = x \cdot \frac{M}{2^n} = \frac{x \cdot M}{2^n}$ }
\end{center}

... where $M$ is \textit{magic coefficient} and $n$ is usually started at register width in bits (32 or 64).

But how to find \textit{magic coefficient}?
This equation is to be solved:

\begin{center}
{\Large $17 \cdot x = 1+k \cdot 2^{64}$ }
\end{center}

This is a little overkill to use SMT-solver to solve this equation
\footnote{\url{https://en.wikipedia.org/wiki/Extended_Euclidean_algorithm} is usually used:
\url{http://rosettacode.org/wiki/Modular\_inverse}}, but still possible:

\begin{lstlisting}
(declare-const x Int)
(declare-const k Int)
(assert (> x 0))
(assert (> k 0))
(assert (=
		(* 17 x)
		(+ 1 (* k (^ 2 64)))
	)
)
(check-sat)
(get-model)
\end{lstlisting}

Important note: we use \textit{Int} sort here, because this is Diophantine equation (i.e., only integer solutions are allowed).
Resulting code will run on integer \ac{CPU} registers, after all.

Z3 gave this:

% FIXME:
\begin{lstlisting}
\$ z3 -smt2 1.smt
sat
(model
  (define-fun k () Int
    16)
  (define-fun x () Int
    17361641481138401521)
)
\end{lstlisting}

Let's check:

% FIXME:
\begin{lstlisting}
#include <stdint.h>
#include <stdio.h>

uint64_t div_by_17 (uint64_t x)
{
	__int128 result = (__int128)x * (__int128)17361641481138401521UL;
	return result >> 64 >> 4;
};

int main()
{
	printf ("%lld\n", 123456789/17);
	printf ("%lld\n", div_by_17(123456789));
};
\end{lstlisting}

... and it works.

GCC doesn't use 128-bit registers, it just exploit the fact that x86 multiplication instruction produces 128-bit result, lower 64-bit part in RAX register,
and higher 64-bit part in RDX register.
In standard C/C++ it's not possible to access higher part, so this hack is helpful.

This is what optimizing GCC 4.8.4 generates:

\begin{lstlisting}
div_by_17:
        movabs  rax, -1085102592571150095
        mul     rdi
        mov     rax, rdx
        sar     rax, 4
        ret
\end{lstlisting}



\subsection{Connection between \ac{SAT} and \ac{SMT} solvers}

Early \ac{SMT}-solvers were frontends to \ac{SAT} solvers, i.e., they translating input SMT expressions into \ac{CNF} and feed SAT-solver with it.
Conversion process is sometimes called ``bit blasting''.
Some \ac{SMT}-solvers still works in that way: STP uses MiniSAT or CryptoMiniSAT as backend SAT-solver.
Some other \ac{SMT}-solvers are more advanced (like Z3), so they use something even more complex.

% subsections
\subsection{Zebra puzzle (\ac{AKA} Einstein puzzle)}
\label{zebra_SMT}

Zebra puzzle is a popular puzzle, defined as follows:

% FIXME remove paragraph at first line
\begin{framed}
\begin{quotation}
1.There are five houses.\\
2.The Englishman lives in the red house.\\
3.The Spaniard owns the dog.\\
4.Coffee is drunk in the green house.\\
5.The Ukrainian drinks tea.\\
6.The green house is immediately to the right of the ivory house.\\
7.The Old Gold smoker owns snails.\\
8.Kools are smoked in the yellow house.\\
9.Milk is drunk in the middle house.\\
10.The Norwegian lives in the first house.\\
11.The man who smokes Chesterfields lives in the house next to the man with the fox.\\
12.Kools are smoked in the house next to the house where the horse is kept.\\
13.The Lucky Strike smoker drinks orange juice.\\
14.The Japanese smokes Parliaments.\\
15.The Norwegian lives next to the blue house.\\
\\
Now, who drinks water? Who owns the zebra?\\
\\
In the interest of clarity, it must be added that each of the five houses is painted a different color, and their inhabitants are of different national extractions, own different pets, drink different beverages and smoke different brands of American cigarets [sic]. One other thing: in statement 6, right means your right.
\end{quotation}
\end{framed}
( \url{https://en.wikipedia.org/wiki/Zebra_Puzzle} ) \\
\\
It's a very good example of constraint satisfaction problem (CSP). % FIXME \ac

We would encode each entity as integer variable, representing number of house.

Then, to define that Englishman lives in red house, we will define this constraint: \TT{Englishman == Red}, meaning that number of a house where Englishmen resides and where tea is drunk is the same.

To define that Norwegian lives next to the blue house, we don't realy know, if it is at left side of blue house or at right side, but we know that house numbers are different by just 1.
So we will define this constraint: \TT{Norwegian==Blue-1 OR Norwegian==Blue+1}.

We will also need to limit all house numbers, so they will be in range of 1..5.

We will also use \TT{Distinct} to show that all various entities of the same type are all has different house numbers.

\lstinputlisting{SMT/zebra.py}

When we run it, we got correct result:

\begin{lstlisting}
sat
[Snails = 3,
 Blue = 2,
 Ivory = 4,
 OrangeJuice = 4,
 Parliament = 5,
 Yellow = 1,
 Fox = 1,
 Zebra = 5,
 Horse = 2,
 Dog = 4,
 Tea = 2,
 Water = 1,
 Chesterfield = 2,
 Red = 3,
 Japanese = 5,
 LuckyStrike = 4,
 Norwegian = 1,
 Milk = 3,
 Kools = 1,
 OldGold = 3,
 Ukrainian = 2,
 Coffee = 5,
 Green = 5,
 Spaniard = 4,
 Englishman = 3]
 \end{lstlisting}


\subsection{Sudoku puzzle}

Sudoku puzzle is a 9*9 grid with some cells filled, some are left to be found:

\newcounter{row}
\newcounter{col}

\newcommand\setrow[9]{
  \setcounter{col}{1}
  \foreach \n in {#1, #2, #3, #4, #5, #6, #7, #8, #9} {
    \edef\x{\value{col} - 0.5}
    \edef\y{9.5 - \value{row}}
    \node[anchor=center] at (\x, \y) {\n};
    \stepcounter{col}
  }
  \stepcounter{row}
}

\begin{center}
\begin{tikzpicture}[scale=.7]
  \begin{scope}
    \draw (0, 0) grid (9, 9);
    \draw[very thick, scale=3] (0, 0) grid (3, 3);

    \setcounter{row}{1}
    \setrow { }{ }{5}  {3}{ }{ }  { }{ }{ }
    \setrow {8}{ }{ }  { }{ }{ }  { }{2}{ }
    \setrow { }{7}{ }  { }{1}{ }  {5}{ }{ }

    \setrow {4}{ }{ }  { }{ }{5}  {3}{ }{ }
    \setrow { }{1}{ }  { }{7}{ }  { }{ }{6}
    \setrow { }{ }{3}  {2}{ }{ }  { }{8}{ }

    \setrow { }{6}{ }  {5}{ }{ }  { }{ }{9}
    \setrow { }{ }{4}  { }{ }{ }  { }{3}{ }
    \setrow { }{ }{ }  { }{ }{9}  {7}{ }{ }

    \node[anchor=center] at (4.5, -0.5) {Unsolved Sudoku};
  \end{scope}
\end{tikzpicture}
\end{center}

Numbers of each row must be unique, i.e., must contain all 9 numbers in range of 1..9 without repetition.
Same story for each column and also for each 3*3 square.

This puzzle is good candidate to try \ac{SMT} solver on, because it's essentially an unsolved system of equations.

\subsubsection{The first idea}

The only thing we must solve is that how to determine in one expression, if the input 9 variables has all 9 unique numbers?
They are not ordered or sorted, after all.

From the school-level mathematics, we can devise this idea:

\begin{equation}
\underbrace{10^{i_1} + 10^{i_2} + \cdots + 10^{i_9}}_9 = 1111111110
\end{equation}

Take each input variable, calculate $10^i$ and sum them all.
If all input values are unique, each will be settled at its own place.
Even more than that: there will be no holes, i.e., no skipped values.
So, in case of Sudoku, 1111111110 number will be final result, indicating that all 9 input variables are unique, in range of 1..9.

Exponentiation is heavy operation, can we use binary operations? Yes, just replace 10 with 2:

\begin{equation}
\underbrace{2^{i_1} + 2^{i_2} + \cdots + 2^{i_9}}_9 = 1111111110_2
\end{equation}

The effect is just the same, but the final value is in base 2 instead of 10.

Now a working example:

\lstinputlisting{SMT/sudoku_plus.py}

\begin{lstlisting}
\$ time python sudoku_plus.py
1 4 5 3 2 7 6 9 8
8 3 9 6 5 4 1 2 7
6 7 2 9 1 8 5 4 3
4 9 6 1 8 5 3 7 2
2 1 8 4 7 3 9 5 6
7 5 3 2 9 6 4 8 1
3 6 7 5 4 2 8 1 9
9 8 4 7 6 1 2 3 5
5 2 1 8 3 9 7 6 4

real    0m11.717s
user    0m10.896s
sys     0m0.068s
\end{lstlisting}

Even more, we can replace summing operation to logical OR:

\begin{equation}
\underbrace{2^{i_1} \vee 2^{i_2} \vee \cdots \vee 2^{i_9}}_9 = 1111111110_2
\end{equation}

% FIXME: только часть исходника + ссылка на github
\lstinputlisting{SMT/sudoku_or.py}

Now it works much faster. Z3 handles OR operation over bit vectors better than addition?

\begin{lstlisting}
\$ time python sudoku_or.py
1 4 5 3 2 7 6 9 8
8 3 9 6 5 4 1 2 7
6 7 2 9 1 8 5 4 3
4 9 6 1 8 5 3 7 2
2 1 8 4 7 3 9 5 6
7 5 3 2 9 6 4 8 1
3 6 7 5 4 2 8 1 9
9 8 4 7 6 1 2 3 5
5 2 1 8 3 9 7 6 4

real    0m1.429s
user    0m1.393s
sys     0m0.036s
\end{lstlisting}

The puzzle I used as example is dubbed as one of the hardest known
\footnote{\url{http://www.mirror.co.uk/news/weird-news/worlds-hardest-sudoku-can-you-242294}} (well, for humans).
It took ~1.4 seconds on my Intel Core i3-3110M 2.4GHz notebook to solve it. % FIXME tilde

\subsubsection{The second idea}

My first approach is far from effective, I did what first came to my mind and worked.
Another approach is to use \texttt{distinct} command from SMTLIB, which tells Z3 that some variables (of unspecific number) must be distinct (or unique).
This command is also available in Z3 Python interface.

I've rewritten my first Sudoku solver, now it operates over \textit{Int} \textit{sort}, it has \texttt{distinct} commands instead of bit operations,
and now also other constaint added: each cell value must be in 1..9 range, because, otherwise, Z3 will offer (although correct) solution with too big
and/or negative numbers.

% FIXME: только часть исходника + ссылка на github
\lstinputlisting{SMT/sudoku2.py}

\begin{lstlisting}
\$ time python sudoku2.py
1 4 5 3 2 7 6 9 8
8 3 9 6 5 4 1 2 7
6 7 2 9 1 8 5 4 3
4 9 6 1 8 5 3 7 2
2 1 8 4 7 3 9 5 6
7 5 3 2 9 6 4 8 1
3 6 7 5 4 2 8 1 9
9 8 4 7 6 1 2 3 5
5 2 1 8 3 9 7 6 4

real    0m0.382s
user    0m0.346s
sys     0m0.036s
\end{lstlisting}

That's much faster.

\subsubsection{Conclusion}

The awesomeness of \ac{SMT}-solvers is that our Sudoku solver has nothing else, we have just defined relationships between variables (cells).

\subsubsection{Homework}

As it seems, true Sudoku puzzle is the one which has only one solution.
The piece of code I've included in this article shows only the first one.
Using the method described earlier (\ref{SMTEnumerate}, also called ``model counting''), 
try to find more solutions, or prove that the solution you have just found is the only one possible.

\subsubsection{Further reading}

\url{http://www.norvig.com/sudoku.html}

\subsubsection{Sudoku as a \ac{SAT} problem}

It's also possible to represend Sudoku puzzle as a huge \ac{CNF} equation and use \ac{SAT}-solver to find solution, but it's just trickier.

Some articles about it\footnote{Another one: \url{http://ocw.mit.edu/courses/electrical-engineering-and-computer-science/6-005-elements-of-software-construction-fall-2011/assignments/MIT6_005F11_ps4.pdf}}: \cite{Sudoku1}, \cite{Sudoku2}, \cite{Sudoku3}.

\ac{SMT}-solver can also use \ac{SAT}-solver in its core, so it does all mundane translating work.
As a ``compiler'', it may not do this in the most efficient way, though.


\input{SMT/pe31.tex}
\subsection{Using Z3 theorem prover to prove equivalence of some weird alternative to XOR operation}

(The article was first published in my blog at April 2015: \url{http://blog.yurichev.com/node/86}).

There is a "A Hacker's Assistant" program\footnote{\url{http://www.hackersdelight.org/}} (\textit{Aha!}) written by Henry Warren,
who is also the author of the great "Hacker's Delight" book.

The \textit{Aha!} program is essentially \textit{superoptimizer}\footnote{\url{http://en.wikipedia.org/wiki/Superoptimization}},
which blindly brute-force a list of some generic RISC CPU instructions to achieve shortest possible (and jumpless or branch-free) 
CPU code sequence for desired operation.
For example, \textit{Aha!} can find jumpless version of abs() function easily.

Compiler developers use superoptimization to find shortest possible (and/or jumpless) code, but I tried to do otherwise -- to find longest code for some basic operation.
I tried \textit{Aha!} to find equivalent of basic XOR operation without usage of the actual XOR instruction, and the most bizarre example \textit{Aha!} gave is:

\begin{lstlisting}
Found a 4-operation program:
   add   r1,ry,rx
   and   r2,ry,rx
   mul   r3,r2,-2
   add   r4,r3,r1
   Expr: (((y & x)*-2) + (y + x))
\end{lstlisting}

And it's hard to say, why/where we can use it, maybe for obfuscation, I'm not sure.
I would call this \textit{suboptimization} (as opposed to \textit{superoptimization}). Or maybe \textit{superdeoptimization}.

But my another question was also, is it possible to prove that this is correct formula at all?
The \textit{Aha!} checking some intput/output values against XOR operation, but of course, not all the possible values.
It is 32-bit code, so it may take very long time to try all possible 32-bit inputs to test it.

We can try Z3 theorem prover for the job. It's called ``prover'', after all.

So I wrote this:

\begin{lstlisting}
#!/usr/bin/python
from z3 import *

x = BitVec('x', 32)
y = BitVec('y', 32)
output = BitVec('output', 32)
s = Solver()
s.add(x^y==output)
s.add(((y & x)*0xFFFFFFFE) + (y + x)!=output)
print s.check()
\end{lstlisting}

In plain English language, this means "are there any case for x and y where $x \oplus y$ doesn't equals to $((y \& x)*-2) + (y + x)$?"
... and Z3 prints "unsat", meaning, it can't find any counterexample to the equation.
So this \textit{Aha!} output is proved to be working just like XOR operation.

Oh, I also tried to extend the formula to 64-bit:

\begin{lstlisting}
#!/usr/bin/python
from z3 import *

x = BitVec('x', 64)
y = BitVec('y', 64)
output = BitVec('output', 64)
s = Solver()
s.add(x^y==output)
s.add(((y & x)*0xFFFFFFFE) + (y + x)!=output)
print s.check()
\end{lstlisting}

Nope, now it says "sat", meaning, Z3 found at least one counterexample.
Oops, it's because I forgot to extend -2 number to 64-bit value:

\begin{lstlisting}
#!/usr/bin/python
from z3 import *

x = BitVec('x', 64)
y = BitVec('y', 64)
output = BitVec('output', 64)
s = Solver()
s.add(x^y==output)
s.add(((y & x)*0xFFFFFFFFFFFFFFFE) + (y + x)!=output)
print s.check()
\end{lstlisting}

Now it says "unsat", so the formula given by \textit{Aha!} works for 64-bit code as well.

\subsubsection{In SMT-LIB form}

Now we can rephrase our equation to more suitable form: $(x + y - ((x \& y)<<1))$.
It also works well is Z3:

\begin{lstlisting}
#!/usr/bin/python
from z3 import *

x = BitVec('x', 64)
y = BitVec('y', 64)
output = BitVec('output', 64)
s = Solver()
s.add(x^y==output)
s.add((x + y - ((x & y)<<1)) != output)
print s.check()
\end{lstlisting}

Here is how to define it in SMT-LIB way:

\begin{lstlisting}
(declare-const x (_ BitVec 64))
(declare-const y (_ BitVec 64))
(assert 
	(not
		(=
			(bvsub
				(bvadd x y)
				(bvshl (bvand x y) (_ bv1 64)))
			(bvxor x y)
		)
	)
)
(check-sat)
\end{lstlisting}

\subsubsection{Using universal quantifier}

Z3 has universal quantifier \TT{exists}, which defines a constraint, which is true if at least one set of variables satistfied underlying condition:

\begin{lstlisting}
(declare-const x (_ BitVec 64))
(declare-const y (_ BitVec 64))
(assert 
	(exists ((x (_ BitVec 64)) (y (_ BitVec 64)))
		(not (=
			(bvsub 
				(bvadd x y)
				(bvshl (bvand x y) (_ bv1 64))
			)
			(bvxor x y)
		))
	)
)
(check-sat)
\end{lstlisting}

It returns ``unsat'', meaning, Z3 couldn't find any counterexample of the equation, i.e., it's not exist.\\
\\
This is also known as $\exists$ in mathematical logic lingo.\\
\\
Z3 has also universal quantifier \TT{forall}, which defines a constraint, where all possible values for the equation must be true.
So we can rewrite our \ac{SMT} example as:

\begin{lstlisting}
(declare-const x (_ BitVec 64))
(declare-const y (_ BitVec 64))
(assert 
	(forall ((x (_ BitVec 64)) (y (_ BitVec 64)))
		(=
			(bvsub 
				(bvadd x y)
				(bvshl (bvand x y) (_ bv1 64))
			)
			(bvxor x y)
		)
	)
)
(check-sat)
\end{lstlisting}

It returns \TT{sat}, meaning, the equation is correct for all possible 64-bit \TT{x} and \TT{y} values, like them all were checked.

Mathematically speaking: $\forall n\!\in\!\mathbb{N}\; (x \oplus y = (x + y - ((x \& y)<<1)))$
\footnote{
$\forall$ means \textit{equation must be true for all possible values}, which are choosen from natural numbers ($\mathbb{N}$).}

\subsubsection{How the equation works}

First of all, binary addition can be viewed as binary XORing with carrying (\ref{adder}).
Here is an example: let's add 2 (10b) and 2 (10b).
XORing these two values resulting 0, but there is a carry generated during addition of two second bits.
That carry bit is propagated further and settles at the place of the 3rd bit: 100b.
4 (100b) is hence a final result of addition.

If the carry bits are not generated during addition, the addition operation is merely XORing.
For example, let's add 1 (1b) and 2 (10b). $1 + 2$ equals to 3, but $1 \oplus 2$ is also 3.

If the addition is XORing plus carry generation and application, we should eliminate effect of carrying somehow here.
The first part of the equation ($x + y$) is addition, the second ($(x \& y)<<1$) is just calculation of every carry bit which was used during addition.
Subtracting removes carry bits from the result of addition, so the only XOR effect is left then.

It's hard to say how Z3 proves this: maybe it just simplifies the equation down to single XOR using simple boolean algebra rewriting rules?

   % \\
\input{SMT/Dietz.tex} % //
% FIXME \ac{}
\subsection{Cracking LCG with Z3 SMT solver}

% FIXME first appeared in blog...

There are well-known weaknesses of LCG (
\href{http://en.wikipedia.org/wiki/Linear_congruential_generator#Advantages_and_disadvantages_of_LCGs}{1},
\href{http://www.reteam.org/papers/e59.pdf}{2}, % FIXME \cite
\href{http://stackoverflow.com/questions/8569113/why-1103515245-is-used-in-rand/8574774#8574774}{3}
), but let's see, if it would be possible to crack it straightforwardly, without any special knowledge.
We would define all relations between LCG states in term of Z3 SMT solver.
(I first made attempt to do it using \href{https://reference.wolfram.com/language/ref/FindInstance.html}{FindInstance} in Wolfram Mathematica, but failed, perhaps, made a mistake somewhere).
Here is a test progam:

\begin{lstlisting}
#include <stdlib.h>
#include <stdio.h>
#include <time.h>

int main()
{
	int i;

	srand(time(NULL));

	for (i=0; i<10; i++)
		printf ("%d\n", rand()%100);
};
\end{lstlisting}

It is intended to print 10 pseudorandom numbers in 0..99 range.
So it does:

\begin{lstlisting}
37
29
74
95
98
40
23
58
61
17
\end{lstlisting}

Let's say we are observing only 8 of these numbers (from 29 to 61) and we need to predict next one (17) and/or previous one (37).

The program is compiled using MSVC 2013 (I choose it because its LCG is simpler than that in Glib):

\begin{lstlisting}
.text:0040112E rand            proc near
.text:0040112E                 call    __getptd
.text:00401133                 imul    ecx, [eax+0x14], 214013
.text:0040113A                 add     ecx, 2531011
.text:00401140                 mov     [eax+14h], ecx
.text:00401143                 shr     ecx, 16
.text:00401146                 and     ecx, 7FFFh
.text:0040114C                 mov     eax, ecx
.text:0040114E                 retn
.text:0040114E rand            endp
\end{lstlisting}

This is very simple LCG, but the result is not clipped state, but it's rather shifted by 16 bits.
Let's define LCG in Z3:

\begin{lstlisting}
#!/usr/bin/python
from z3 import *

output_prev = BitVec('output_prev', 32)
state1 = BitVec('state1', 32)
state2 = BitVec('state2', 32)
state3 = BitVec('state3', 32)
state4 = BitVec('state4', 32)
state5 = BitVec('state5', 32)
state6 = BitVec('state6', 32)
state7 = BitVec('state7', 32)
state8 = BitVec('state8', 32)
state9 = BitVec('state9', 32)
state10 = BitVec('state10', 32)
output_next = BitVec('output_next', 32)

s = Solver()

s.add(state2 == state1*214013+2531011)
s.add(state3 == state2*214013+2531011)
s.add(state4 == state3*214013+2531011)
s.add(state5 == state4*214013+2531011)
s.add(state6 == state5*214013+2531011)
s.add(state7 == state6*214013+2531011)
s.add(state8 == state7*214013+2531011)
s.add(state9 == state8*214013+2531011)
s.add(state10 == state9*214013+2531011)

s.add(output_prev==URem((state1>>16)&0x7FFF,100))
s.add(URem((state2>>16)&0x7FFF,100)==29)
s.add(URem((state3>>16)&0x7FFF,100)==74)
s.add(URem((state4>>16)&0x7FFF,100)==95)
s.add(URem((state5>>16)&0x7FFF,100)==98)
s.add(URem((state6>>16)&0x7FFF,100)==40)
s.add(URem((state7>>16)&0x7FFF,100)==23)
s.add(URem((state8>>16)&0x7FFF,100)==58)
s.add(URem((state9>>16)&0x7FFF,100)==61)
s.add(output_next==URem((state10>>16)&0x7FFF,100))

print(s.check())
print(s.model())
\end{lstlisting}

URem states for \textit{unsigned remainder}.
It works for some time and gave us correct result!

\begin{lstlisting}
sat
[state3 = 2276903645,
 state4 = 1467740716,
 state5 = 3163191359,
 state7 = 4108542129,
 state8 = 2839445680,
 state2 = 998088354,
 state6 = 4214551046,
 state1 = 1791599627,
 state9 = 548002995,
 output_next = 17,
 output_prev = 37,
 state10 = 1390515370]
\end{lstlisting}

% FIXME tilde
I added ~10 states to be sure result will be correct. It may be not if you supply lesser amount of PRNG numbers.

That is the reason why LCG is not suitable for any security-related task.
This is why \href{https://en.wikipedia.org/wiki/Cryptographically_secure_pseudorandom_number_generator}{cryptographically secure pseudorandom number generators} are exist: they are designed to be protected against such simple attack.
Well, at least if \href{https://en.wikipedia.org/wiki/Dual_EC_DRBG}{NSA is not involved}.

As far, as I can understand, \href{http://en.wikipedia.org/wiki/Security_token}{security tokens} like \href{http://en.wikipedia.org/wiki/RSA_SecurID}{RSA SecurID} can be viewed just as \ac{CPRNG} with a secret seed.
It shows new pseudorandom number each minute, and the server can predict it, because it knows the seed.
Imagine if such token would implement LCG -- it would be much easier to break!


\section{KLEE}

% subsections:
\subsection{School-level equation}

Let's revisit school-level system of equations from (\ref{eq2_SMT}).

We will force KLEE to find a path, where all the constraints are satisfied:

\lstinputlisting{KLEE/klee_eq1.c}

% FIXME:
\begin{lstlisting}
\$ clang -emit-llvm -c -g klee_eq.c
...

\$ klee klee_eq.bc
KLEE: output directory is "/home/klee/klee-out-93"
KLEE: WARNING: undefined reference to function: klee_assert
KLEE: WARNING ONCE: calling external: klee_assert(0)
KLEE: ERROR: /home/klee/klee_eq.c:18: failed external call: klee_assert
KLEE: NOTE: now ignoring this error at this location

KLEE: done: total instructions = 32
KLEE: done: completed paths = 1
KLEE: done: generated tests = 1
\end{lstlisting}

Let's find out, where \TT{klee\_assert()} has been triggered:

% FIXME:
\begin{lstlisting}
\$ ls klee-last | grep err
test000001.external.err

\$ ktest-tool --write-ints klee-last/test000001.ktest
ktest file : 'klee-last/test000001.ktest'
args       : ['klee_eq.bc']
num objects: 3
object    0: name: b'circle'
object    0: size: 4
object    0: data: 5
object    1: name: b'square'
object    1: size: 4
object    1: data: 2
object    2: name: b'triangle'
object    2: size: 4
object    2: data: 1
\end{lstlisting}

This is indeed correct solution to the system of equations.

KLEE has intrinsic \TT{klee\_assume()} which tells KLEE to cut path if some constraint is not true.
So we can rewrite our example in such cleaner way:

\lstinputlisting{KLEE/klee_eq2.c}



\subsection{Zebra puzzle (\ac{AKA} Einstein puzzle)}
\label{zebra_SMT}

Zebra puzzle is a popular puzzle, defined as follows:

% FIXME remove paragraph at first line
\begin{framed}
\begin{quotation}
1.There are five houses.\\
2.The Englishman lives in the red house.\\
3.The Spaniard owns the dog.\\
4.Coffee is drunk in the green house.\\
5.The Ukrainian drinks tea.\\
6.The green house is immediately to the right of the ivory house.\\
7.The Old Gold smoker owns snails.\\
8.Kools are smoked in the yellow house.\\
9.Milk is drunk in the middle house.\\
10.The Norwegian lives in the first house.\\
11.The man who smokes Chesterfields lives in the house next to the man with the fox.\\
12.Kools are smoked in the house next to the house where the horse is kept.\\
13.The Lucky Strike smoker drinks orange juice.\\
14.The Japanese smokes Parliaments.\\
15.The Norwegian lives next to the blue house.\\
\\
Now, who drinks water? Who owns the zebra?\\
\\
In the interest of clarity, it must be added that each of the five houses is painted a different color, and their inhabitants are of different national extractions, own different pets, drink different beverages and smoke different brands of American cigarets [sic]. One other thing: in statement 6, right means your right.
\end{quotation}
\end{framed}
( \url{https://en.wikipedia.org/wiki/Zebra_Puzzle} ) \\
\\
It's a very good example of constraint satisfaction problem (CSP). % FIXME \ac

We would encode each entity as integer variable, representing number of house.

Then, to define that Englishman lives in red house, we will define this constraint: \TT{Englishman == Red}, meaning that number of a house where Englishmen resides and where tea is drunk is the same.

To define that Norwegian lives next to the blue house, we don't realy know, if it is at left side of blue house or at right side, but we know that house numbers are different by just 1.
So we will define this constraint: \TT{Norwegian==Blue-1 OR Norwegian==Blue+1}.

We will also need to limit all house numbers, so they will be in range of 1..5.

We will also use \TT{Distinct} to show that all various entities of the same type are all has different house numbers.

\lstinputlisting{SMT/zebra.py}

When we run it, we got correct result:

\begin{lstlisting}
sat
[Snails = 3,
 Blue = 2,
 Ivory = 4,
 OrangeJuice = 4,
 Parliament = 5,
 Yellow = 1,
 Fox = 1,
 Zebra = 5,
 Horse = 2,
 Dog = 4,
 Tea = 2,
 Water = 1,
 Chesterfield = 2,
 Red = 3,
 Japanese = 5,
 LuckyStrike = 4,
 Norwegian = 1,
 Milk = 3,
 Kools = 1,
 OldGold = 3,
 Ukrainian = 2,
 Coffee = 5,
 Green = 5,
 Spaniard = 4,
 Englishman = 3]
 \end{lstlisting}


\subsection{Sudoku}

I've also rewritten Sudoku example (\ref{sudoku_SMT}) for KLEE:

\lstinputlisting[numbers=left]{KLEE/klee_sudoku_or1.c}

Let's run it:

% FIXME:
\begin{lstlisting}
\$ clang -emit-llvm -c -g klee_sudoku_or1.c
...

\$ time klee klee_sudoku_or1.bc
KLEE: output directory is "/home/klee/klee-out-98"
KLEE: WARNING: undefined reference to function: klee_assert
KLEE: WARNING ONCE: calling external: klee_assert(0)
KLEE: ERROR: /home/klee/klee_sudoku_or1.c:93: failed external call: klee_assert
KLEE: NOTE: now ignoring this error at this location

KLEE: done: total instructions = 7512
KLEE: done: completed paths = 161
KLEE: done: generated tests = 161

real    3m44.111s
user    3m43.319s
sys     0m0.951s
\end{lstlisting}

Now this is really slower (on my Intel Core i3-3110M 2.4GHz notebook) in comparison to Z3Py solution (\ref{sudoku_SMT}).

But the answer is correct:

% FIXME:
\begin{lstlisting}
\$ ls klee-last | grep err
test000161.external.err

\$ ktest-tool --write-ints klee-last/test000161.ktest
ktest file : 'klee-last/test000161.ktest'
args       : ['klee_sudoku_or1.bc']
num objects: 1
object    0: name: b'cells'
object    0: size: 81
object    0: data: b'\x01\x04\x05\x03\x02\x07\x06\t\x08\x08\x03\t\x06\x05\x04\x01\x02\x07\x06\x07\x02\t\x01\x08\x05\x04\x03\x04\t\x06\x01\x08\x05\x03\x07\x02\x02\x01\x08\x04\x07\x03\t\x05\x06\x07\x05\x03\x02\t\x06\x04\x08\x01\x03\x06\x07\x05\x04\x02\x08\x01\t\t\x08\x04\x07\x06\x01\x02\x03\x05\x05\x02\x01\x08\x03\t\x07\x06\x04'
\end{lstlisting}

% FIXME backslash
\TT{\\t} is character with ASCII code of 9 in C/C++, and KLEE attempts to treat byte array as C/C++ string, so it shows some values in such way.
We can just remember that there is 9 at the each place where we see \TT{\\t}.
The solution, while not properly formatted, correct indeed. \\
\\
By the way, at lines 42, 43 you may see how we tell to KLEE that all array elements must be within some limits.
If we comment these lines out, we've got this:

% FIXME:
\begin{lstlisting}
\$ time klee klee_sudoku_or1.bc
KLEE: output directory is "/home/klee/klee-out-100"
KLEE: WARNING: undefined reference to function: klee_assert
KLEE: ERROR: /home/klee/klee_sudoku_or1.c:51: overshift error
KLEE: NOTE: now ignoring this error at this location
KLEE: ERROR: /home/klee/klee_sudoku_or1.c:51: overshift error
KLEE: NOTE: now ignoring this error at this location
KLEE: ERROR: /home/klee/klee_sudoku_or1.c:51: overshift error
KLEE: NOTE: now ignoring this error at this location
...
\end{lstlisting}

KLEE warns us that shift value at line 51 is too big.
Indeed, KLEE may try all byte values up to 255 (0xFF), which are pointless to use there, and may indicate error or bug, so KLEE warns about it.\\
\\
Now let's use \TT{klee\_assume()} again:

\lstinputlisting{KLEE/klee_sudoku_or2.c}

% FIXME:
\begin{lstlisting}
\$ time klee klee_sudoku_or2.bc
KLEE: output directory is "/home/klee/klee-out-99"
KLEE: WARNING: undefined reference to function: klee_assert
KLEE: WARNING ONCE: calling external: klee_assert(0)
KLEE: ERROR: /home/klee/klee_sudoku_or2.c:93: failed external call: klee_assert
KLEE: NOTE: now ignoring this error at this location

KLEE: done: total instructions = 7119
KLEE: done: completed paths = 1
KLEE: done: generated tests = 1

real    0m35.312s
user    0m34.945s
sys     0m0.318s
\end{lstlisting}

That works much faster: perhaps KLEE indeed handle this intrinsic in a special way.
And, as we see, the only one path is generated (one we actually interesting in it) instead of 161.

It's still much slower than Z3Py solution, though.


\input{KLEE/UNIXdatetime.tex}
\input{KLEE/base64.tex}
\subsection{CRC} % FIXME full name

\subsubsection{Buffer alteration case \#1}

Sometimes, you need to alter a piece of data which is \textit{protected} by some kind of checksum or \ac{CRC}, and you can't change checksum or CRC value, but can alter piece of data so that checksum will remain the same.

Let's pretend, we've got a piece of data with ``Hello, world!'' string at the beginning and ``and goodbye'' string at the end.
We can alter 14 characters at the middle, but for some reason, they must be in \textit{a..z} limits, but we can put any characters there.
CRC64 of the whole block must be \TT{0x12345678abcdef12}.

Let's see\footnote{There are several slightly different CRC64 implementations, the one I use here can also be different from popular ones.}:

\lstinputlisting{KLEE/klee_CRC64.c}

Since our code uses memcmp() standard C/C++ function, we need to add \TT{--libc=uclibc} switch, so KLEE will use its own uClibc % FIXME check spelling
implementation. % \ref{} -> closed programs

% FIXME:
\begin{lstlisting}
\$ clang -emit-llvm -c -g klee_CRC64.c

\$ time klee --libc=uclibc klee_CRC64.bc
\end{lstlisting}

It takes about 1 minute (on my XXX) and we getting this:

% FIXME:
\begin{lstlisting}
...
real    0m52.643s
user    0m51.232s
sys     0m0.239s
...
\$ ls klee-last | grep err
test000001.user.err
test000002.user.err
test000003.user.err
test000004.external.err

\$ ktest-tool --write-ints klee-last/test000004.ktest
ktest file : 'klee-last/test000004.ktest'
args       : ['klee_CRC64.bc']
num objects: 1
object    0: name: b'buf'
object    0: size: 46
object    0: data: b'Hello, world!.. qqlicayzceamyw ... and goodbye'
\end{lstlisting}

Maybe it's slow, but definitely faster than bruteforce.
Indeed, $log_2{26^{14}} \approx 65.8$
which is close to 64 bits.
In other words, one need $\approx 14$ latin characters to encode 64 bits.
And KLEE + \ac{SMT} solver needs 64 bits at some place it can alter to make final CRC64 value equal to what we defined.

I tried to reduce length of the \textit{middle block} to 13 characters: no luck for KLEE then, it has no space enough.

\subsubsection{Buffer alteration case \#2}

I went sadistic: what if the buffer must contain the CRC64 value which, after calculation of CRC64, will result in the same value?
Fascinately, % FIXME check spelling
KLEE can solve this.
The buffer will have the following format:

% FIXME:
\begin{lstlisting}
Hello, world! <8-bytes (64-bit value)> and goodbye <6 more bytes>
\end{lstlisting}

% FIXME:
\begin{lstlisting}
int main()
{
#define HEAD_STR "Hello, world!.. "
#define HEAD_SIZE strlen(HEAD_STR)
#define TAIL_STR " ... and goodbye"
#define TAIL_SIZE strlen(TAIL_STR)
// 8 bytes for 64-bit value:
#define MID_SIZE 8
#define BUF_SIZE HEAD_SIZE+TAIL_SIZE+MID_SIZE+6

	char buf[BUF_SIZE];
  
	klee_make_symbolic(buf, sizeof buf, "buf");

	klee_assume (memcmp (buf, HEAD_STR, HEAD_SIZE)==0);

	klee_assume (memcmp (buf+HEAD_SIZE+MID_SIZE, TAIL_STR, TAIL_SIZE)==0);
	
	uint64_t mid_value=*(uint64_t*)(buf+HEAD_SIZE);
	klee_assume (crc64 (0, buf, BUF_SIZE)==mid_value);

	klee_assert(0);

	return 0;
}
\end{lstlisting}

It works:

% FIXME:
\begin{lstlisting}
\$ time klee --libc=uclibc klee_CRC64.bc
...
real    5m17.081s
user    5m17.014s
sys     0m0.319s

\$ ls klee-last | grep err
test000001.user.err
test000002.user.err
test000003.external.err

\$ ktest-tool --write-ints klee-last/test000003.ktest
ktest file : 'klee-last/test000003.ktest'
args       : ['klee_CRC64.bc']
num objects: 1
object    0: name: b'buf'
object    0: size: 46
object    0: data: b'Hello, world!.. T+]\xb9A\x08\x0fq ... and goodbye\xb6\x8f\x9c\xd8\xc5\x00'
\end{lstlisting}

8 bytes between two strings is 64-bit value which equals to CRC64 of this whole block.
Again, it's faster than brute-force way to find it.
If to decrease last spare 6-byte buffer to 4 bytes or less, KLEE works so long so I've stopped it.

\subsubsection{Recovering input data for given CRC32 value of it}

I've always wanted to do so, but everyone knows this is impossible for input buffers larger than 4 bytes.
As my experiments show, it's still possible for tiny input buffers of data, constrained in some way.

The CRC32 value of 6-byte ``SILVER'' string is known: \TT{0xDFA3DFDD}.
KLEE can find this 6-byte string, if it knows that each byte of input buffer is in \textit{A..Z} limits:

\lstinputlisting[numbers=left]{KLEE/klee_SILVER.c}

% FIXME:
\begin{lstlisting}
\$ clang -emit-llvm -c -g klee_SILVER.c
...

\$ klee klee_SILVER.bc
...

\$ ls klee-last | grep err
test000013.external.err

\$ ktest-tool --write-ints klee-last/test000013.ktest
ktest file : 'klee-last/test000013.ktest'
args       : ['klee_SILVER.bc']
num objects: 1
object    0: name: b'str'
object    0: size: 6
object    0: data: b'SILVER'
\end{lstlisting}

Still, it's no magic: if to remove condition at lines 23..25 (i.e., if to relax constraints),
KLEE will produce some other string, which will be still correct for the CRC32 value given.

It works, because 6 Latin characters in \textit{A..Z} limits contain $\approx 28.2$ bits:
$log_2{26^6} \approx 28.2$, which is even smaller value than 32.
In other words, the final CRC32 value holds enough bits to recover $\approx 28.2$ bits of input.

The input buffer can be even bigger, if each byte of it will be in even tighter % FIXME spelling
constraints (decimal digits, binary digits, etc).

\subsubsection{In comparison with other hashing algorithms}

Things are that easy for some other hashing algorithms like \textit{fletcher checksum}, % FIXME URL
but not for cryptographically secure ones (like MD5, SHA1, etc), they are protected from such simple cryptoanalysis. % FIXME \ref{} -> am.crypto


\input{KLEE/LZSS.tex}

\subsection{Exercise}

Here is my crackme/keygenme, which may be tricky, but easy to solve using KLEE:
\url{http://challenges.re/74/}.



\input{SMT/rockey.tex}


\section{Program synthesis}

Program synthesis is a process of automatic program generation, in accordance with some specific goals.

% subsections:
\input{pgm_synth/mult}
\input{pgm_synth/rockey}


\EN{\section{Toy decompiler}
\label{toy_decompiler}

\subsection{Introduction}

A modern-day compiler is a product of hundreds of developer/year.
At the same time, toy compiler can be an exercise for a student for a week (or even weekend).

Likewise, commercial decompiler like Hex-Rays can be extremely complex,
while toy decompiler like this one, can be easy to understand and remake.

The following decompiler written in Python, supports only short basic blocks, with no jumps.
Memory is also not supported.

\subsection{Data structure}

Our toy decompiler will use just one single data structure, representing expression tree.

Many programming textbooks has an example of conversion from Fahrenheit temperature to Celsius, using the following formula:

\begin{center}
{\large $celsius = (fahrenheit - 32) \cdot \frac{5}{9}$}
\end{center}

This expression can be represented as a tree:

% reworked from http://www.texample.net/tikz/examples/decision-tree/
\tikzset{
  treenode/.style = {shape=rectangle, rounded corners,
                     draw, align=center,
                     top color=white, bottom color=blue!20},
  env/.style      = {treenode, font=\ttfamily\normalsize},
}

\begin{center}
\begin{tikzpicture}
[
	grow                    = down,
	sibling distance        = 4em,
	level distance          = 5em,
	edge from parent/.style = {draw, -latex},
	every node/.style       = {font=\footnotesize},
	sloped
]
\node [env] {/}
	child
	{
		node [env] {*}
		child { 
			node [env] {-}
			child { node [env] {INPUT} }
			child { node [env] {32} }
			}
		child { node [env] {5} }
	}
	child { node [env] {9} }
	;

\end{tikzpicture}
\end{center}


How to store it in memory?
We see here 3 types of nodes: 1) numbers (or values); 2) arithmetical operations; 3) symbols (like ``INPUT'').

Many developers with \ac{OOP} in their mind will create some kind of class.
Other developer maybe will use ``variant type''.

I'll use simplest possible way of representing this structure: a Python tuple.
First element of tuple can be a string:
either ``EXPR\_OP'' for operation, ``EXPR\_SYMBOL'' for symbol or ``EXPR\_VALUE'' for value.
In case of symbol or value, it follows the string.
In case of operation, the string followed by another tuples.

Node type and operation type are stored as plain strings---to make debugging output easier to read.

There are \textit{constructors} in our code, in \ac{OOP} sense:

\begin{lstlisting}
def create_val_expr (val):
    return ("EXPR_VALUE", val)

def create_symbol_expr (val):
    return ("EXPR_SYMBOL", val)

def create_binary_expr (op, op1, op2):
    return ("EXPR_OP", op, op1, op2)
\end{lstlisting}

There are also \textit{accessors}:

\begin{lstlisting}
def get_expr_type(e):
    return e[0]

def get_symbol (e):
    assert get_expr_type(e)=="EXPR_SYMBOL"
    return e[1]

def get_val (e):
    assert get_expr_type(e)=="EXPR_VALUE"
    return e[1]

def is_expr_op(e):
    return get_expr_type(e)=="EXPR_OP"

def get_op (e):
    assert is_expr_op(e)
    return e[1]

def get_op1 (e):
    assert is_expr_op(e)
    return e[2]

def get_op2 (e):
    assert is_expr_op(e)
    return e[3]
\end{lstlisting}

The temperature conversion expression we just saw will be represented as:

\begin{center}
\begin{tikzpicture}
[
	grow                    = down,
	sibling distance        = 8em,
	level distance          = 5em,
	edge from parent/.style = {draw, -latex},
	every node/.style       = {font=\footnotesize},
	sloped
]
\node [env] {"EXPR\_OP"\\"/"}
	child
	{
		node [env] {"EXPR\_OP"\\"*"}
		child { 
			node [env] {"EXPR\_OP"\\"-"}
			child { node [env] {"EXPR\_SYMBOL"\\"arg1"} }
			child { node [env] {"EXPR\_VALUE"\\32} }
			}
		child { node [env] {"EXPR\_VALUE"\\5} }
	}
	child { node [env] {"EXPR\_VALUE"\\9} }
	;

\end{tikzpicture}
\end{center}


\dots or as Python expression:

\begin{lstlisting}
('EXPR_OP', '/', 
	('EXPR_OP', '*',
	('EXPR_OP', '-', ('EXPR_SYMBOL', 'arg1'), ('EXPR_VALUE', 32)), 
	('EXPR_VALUE', 5)), 
('EXPR_VALUE', 9))
\end{lstlisting}

In fact, this is \ac{AST} in its simplest form.
\ac{AST}s are used heavily in compilers.

\subsection{Simple examples}

Let's start with simplest example:

\begin{lstlisting}
        mov     rax, rdi
        imul    rax, rsi
\end{lstlisting}

At start, these symbols are assigned to registers:
RAX=initial\_RAX,
RBX=initial\_RBX,
RDI=arg1,
RSI=arg2,
RDX=arg3,
RCX=arg4.

When we handle MOV instruction, we just copy expression from RDI to RAX.
When we handle IMUL instruction, we create a new expression, adding together expressions from RAX and RSI and putting
result into RAX again.

I can feed this to decompiler and we will see how register's state is changed through processing:

\begin{lstlisting}
python td.py --show-registers --python-expr tests/mul.s

...

line=[mov       rax, rdi]
rcx=('EXPR_SYMBOL', 'arg4')
rsi=('EXPR_SYMBOL', 'arg2')
rbx=('EXPR_SYMBOL', 'initial_RBX')
rdx=('EXPR_SYMBOL', 'arg3')
rdi=('EXPR_SYMBOL', 'arg1')
rax=('EXPR_SYMBOL', 'arg1')

line=[imul      rax, rsi]
rcx=('EXPR_SYMBOL', 'arg4')
rsi=('EXPR_SYMBOL', 'arg2')
rbx=('EXPR_SYMBOL', 'initial_RBX')
rdx=('EXPR_SYMBOL', 'arg3')
rdi=('EXPR_SYMBOL', 'arg1')
rax=('EXPR_OP', '*', ('EXPR_SYMBOL', 'arg1'), ('EXPR_SYMBOL', 'arg2'))

...

result=('EXPR_OP', '*', ('EXPR_SYMBOL', 'arg1'), ('EXPR_SYMBOL', 'arg2'))
\end{lstlisting}

IMUL instruction is mapped to ``*'' string, and then new expression is constructed in 
\TT{handle\_binary\_op()}, which puts result into RAX.

In this output, the data structures are dumped using Python \TT{str()} function, which does mostly the same, as \TT{print()}.

Output is bulky, and we can turn off Python expressions output, and see how this internal data structure can be rendered neatly
using our internal \TT{expr\_to\_string()} function:

\begin{lstlisting}
python td.py --show-registers tests/mul.s

...

line=[mov       rax, rdi]
rcx=arg4
rsi=arg2
rbx=initial_RBX
rdx=arg3
rdi=arg1
rax=arg1

line=[imul      rax, rsi]
rcx=arg4
rsi=arg2
rbx=initial_RBX
rdx=arg3
rdi=arg1
rax=(arg1 * arg2)

...

result=(arg1 * arg2)
\end{lstlisting}

Slightly advanced example:

\begin{lstlisting}
        imul    rdi, rsi
        lea     rax, [rdi+rdx]
\end{lstlisting}

LEA instruction is treated just as ADD.

\begin{lstlisting}
python td.py --show-registers --python-expr tests/mul_add.s

...

line=[imul      rdi, rsi]
rcx=('EXPR_SYMBOL', 'arg4')
rsi=('EXPR_SYMBOL', 'arg2')
rbx=('EXPR_SYMBOL', 'initial_RBX')
rdx=('EXPR_SYMBOL', 'arg3')
rdi=('EXPR_OP', '*', ('EXPR_SYMBOL', 'arg1'), ('EXPR_SYMBOL', 'arg2'))
rax=('EXPR_SYMBOL', 'initial_RAX')

line=[lea       rax, [rdi+rdx]]
rcx=('EXPR_SYMBOL', 'arg4')
rsi=('EXPR_SYMBOL', 'arg2')
rbx=('EXPR_SYMBOL', 'initial_RBX')
rdx=('EXPR_SYMBOL', 'arg3')
rdi=('EXPR_OP', '*', ('EXPR_SYMBOL', 'arg1'), ('EXPR_SYMBOL', 'arg2'))
rax=('EXPR_OP', '+', ('EXPR_OP', '*', ('EXPR_SYMBOL', 'arg1'), ('EXPR_SYMBOL', 'arg2')), ('EXPR_SYMBOL', 'arg3'))

...

result=('EXPR_OP', '+', ('EXPR_OP', '*', ('EXPR_SYMBOL', 'arg1'), ('EXPR_SYMBOL', 'arg2')), ('EXPR_SYMBOL', 'arg3'))
\end{lstlisting}

And again, let's see this expression dumped neatly:

\begin{lstlisting}
python td.py --show-registers tests/mul_add.s

...

result=((arg1 * arg2) + arg3)
\end{lstlisting}

Now another example, where we use 2 input arguments:

\begin{lstlisting}
        imul    rdi, rdi, 1234
        imul    rsi, rsi, 5678
        lea     rax, [rdi+rsi]
\end{lstlisting}

\begin{lstlisting}
python td.py --show-registers --python-expr tests/mul_add3.s

...

line=[imul      rdi, rdi, 1234]
rcx=('EXPR_SYMBOL', 'arg4')
rsi=('EXPR_SYMBOL', 'arg2')
rbx=('EXPR_SYMBOL', 'initial_RBX')
rdx=('EXPR_SYMBOL', 'arg3')
rdi=('EXPR_OP', '*', ('EXPR_SYMBOL', 'arg1'), ('EXPR_VALUE', 1234))
rax=('EXPR_SYMBOL', 'initial_RAX')

line=[imul      rsi, rsi, 5678]
rcx=('EXPR_SYMBOL', 'arg4')
rsi=('EXPR_OP', '*', ('EXPR_SYMBOL', 'arg2'), ('EXPR_VALUE', 5678))
rbx=('EXPR_SYMBOL', 'initial_RBX')
rdx=('EXPR_SYMBOL', 'arg3')
rdi=('EXPR_OP', '*', ('EXPR_SYMBOL', 'arg1'), ('EXPR_VALUE', 1234))
rax=('EXPR_SYMBOL', 'initial_RAX')

line=[lea       rax, [rdi+rsi]]
rcx=('EXPR_SYMBOL', 'arg4')
rsi=('EXPR_OP', '*', ('EXPR_SYMBOL', 'arg2'), ('EXPR_VALUE', 5678))
rbx=('EXPR_SYMBOL', 'initial_RBX')
rdx=('EXPR_SYMBOL', 'arg3')
rdi=('EXPR_OP', '*', ('EXPR_SYMBOL', 'arg1'), ('EXPR_VALUE', 1234))
rax=('EXPR_OP', '+', ('EXPR_OP', '*', ('EXPR_SYMBOL', 'arg1'), ('EXPR_VALUE', 1234)), ('EXPR_OP', '*', ('EXPR_SYMBOL', 'arg2'), ('EXPR_VALUE', 5678)))

...

result=('EXPR_OP', '+', ('EXPR_OP', '*', ('EXPR_SYMBOL', 'arg1'), ('EXPR_VALUE', 1234)), ('EXPR_OP', '*', ('EXPR_SYMBOL', 'arg2'), ('EXPR_VALUE', 5678)))
\end{lstlisting}

\dots and now neat output:

\begin{lstlisting}
python td.py --show-registers tests/mul_add3.s

...

result=((arg1 * 1234) + (arg2 * 5678))
\end{lstlisting}

Now conversion program:

\begin{lstlisting}
        mov     rax, rdi
        sub     rax, 32
        imul    rax, 5
        mov     rbx, 9
        idiv    rbx
\end{lstlisting}

You can see, how register's state is changed over execution (or parsing).

Raw:

\lstinputlisting{toy_decompiler/fahr_raw.txt}

Neat:

\lstinputlisting{toy_decompiler/fahr_neat.txt}

It is interesting to note that IDIV instruction also calculates reminder of division, and it is placed into RDX register.
It's not used, but is available for use.

This is how quotient and remainder are stored in registers:

\begin{lstlisting}
def handle_unary_DIV_IDIV (registers, op1):
    op1_expr=register_or_number_in_string_to_expr (registers, op1)
    current_RAX=registers["rax"]
    registers["rax"]=create_binary_expr ("/", current_RAX, op1_expr)
    registers["rdx"]=create_binary_expr ("%", current_RAX, op1_expr)
\end{lstlisting}

Now this is \TT{align2grain()} function\footnote{Taken from \url{https://docs.oracle.com/javase/specs/jvms/se6/html/Compiling.doc.html}}:

\begin{lstlisting}
        ; uint64_t align2grain (uint64_t i, uint64_t grain)
        ;    return ((i + grain-1) & ~(grain-1));

        ; rdi=i
        ; rsi=grain

        sub     rsi, 1
        add     rdi, rsi
        not     rsi
        and     rdi, rsi
        mov     rax, rdi
\end{lstlisting}

\lstinputlisting{toy_decompiler/align2grain.txt}

\subsection{Dealing with compiler optimizations}

The following piece of code \dots

\begin{lstlisting}
        mov     rax, rdi
        add     rax, rax
\end{lstlisting}

\dots will be transormed into \textit{(arg1 + arg1)} expression.
It can be reduced to \textit{(arg1 * 2)}.
Our toy decompiler can identify patterns like such and rewrite them.

\begin{lstlisting}
# X+X -> X*2
def reduce_ADD1 (expr):
    if is_expr_op(expr) and get_op (expr)=="+" and get_op1 (expr)==get_op2 (expr):
        return dbg_print_reduced_expr ("reduce_ADD1", expr, create_binary_expr ("*", get_op1 (expr), create_val_expr (2)))

    return expr # no match
\end{lstlisting}

This function will just test, if the current node has \textit{EXPR\_OP} type,
operation is ``+'' and both children are equal to each other.
By the way, since our data structure is just tuple of tuples, Python can compare them using plain ``=='' operation.
If the testing is finished successfully, current node is then replaced with a new expression:
we take one of children, we construct a node of \textit{EXPR\_VALUE} type with ``2'' number in it,
and then we construct a node of \textit{EXPR\_OP} type with ``*''.

\TT{dbg\_print\_reduced\_expr()} serving solely debugging purposes---it just prints the old and the new (reduced) expressions.

Decompiler is then traverse expression tree recursively in \textit{deep-first search} fashion.

\begin{lstlisting}
def reduce_step (e):
    if is_expr_op (e)==False:
        return e # expr isn't EXPR_OP, nothing to reduce (we don't reduce EXPR_SYMBOL and EXPR_VAL)

    if is_unary_op(get_op(e)):
        # recreate expr with reduced operand:
        return reducers(create_unary_expr (get_op(e), reduce_step (get_op1 (e))))
    else:
        # recreate expr with both reduced operands:
        return reducers(create_binary_expr (get_op(e), reduce_step (get_op1 (e)), reduce_step (get_op2 (e))))

...


# same as "return ...(reduce_MUL1 (reduce_ADD1 (reduce_ADD2 (... expr))))"
reducers=compose([
	...
    reduce_ADD1, ...
    ...])

def reduce (e):
    print "going to reduce " + expr_to_string (e)
    new_expr=reduce_step(e)
    if new_expr==e:
        return new_expr # we are done here, expression can't be reduced further
    else:
        return reduce(new_expr) # reduced expr has been changed, so try to reduce it again
\end{lstlisting}

Reduction functions called again and again, as long, as expression changes.

Now we run it:

\begin{lstlisting}
python td.py tests/add1.s

...

going to reduce (arg1 + arg1)
reduction in reduce_ADD1() (arg1 + arg1) -> (arg1 * 2)
going to reduce (arg1 * 2)
result=(arg1 * 2)
\end{lstlisting}

So far so good, now what if we would try this piece of code?

\begin{lstlisting}
        mov     rax, rdi
        add     rax, rax
        add     rax, rax
        add     rax, rax
\end{lstlisting}

\begin{lstlisting}
python td.py tests/add2.s

...

working out tests/add2.s
going to reduce (((arg1 + arg1) + (arg1 + arg1)) + ((arg1 + arg1) + (arg1 + arg1)))
reduction in reduce_ADD1() (arg1 + arg1) -> (arg1 * 2)
reduction in reduce_ADD1() (arg1 + arg1) -> (arg1 * 2)
reduction in reduce_ADD1() ((arg1 * 2) + (arg1 * 2)) -> ((arg1 * 2) * 2)
reduction in reduce_ADD1() (arg1 + arg1) -> (arg1 * 2)
reduction in reduce_ADD1() (arg1 + arg1) -> (arg1 * 2)
reduction in reduce_ADD1() ((arg1 * 2) + (arg1 * 2)) -> ((arg1 * 2) * 2)
reduction in reduce_ADD1() (((arg1 * 2) * 2) + ((arg1 * 2) * 2)) -> (((arg1 * 2) * 2) * 2)
going to reduce (((arg1 * 2) * 2) * 2)
result=(((arg1 * 2) * 2) * 2)
\end{lstlisting}

This is correct, but too verbose.

We would like to rewrite \textit{(X*n)*m} expression to \textit{X*(n*m)}, where $n$ and $m$ are numbers.
We can do this by adding another function like \TT{reduce\_ADD1()}, but there is much better option:
we can make matcher for tree.
You can think about it as regular expression matcher, but over trees.

\begin{lstlisting}
def bind_expr (key):
    return ("EXPR_WILDCARD", key)

def bind_value (key):
    return ("EXPR_WILDCARD_VALUE", key)

def match_EXPR_WILDCARD (expr, pattern):
    return {pattern[1] : expr} # return {key : expr}

def match_EXPR_WILDCARD_VALUE (expr, pattern):
    if get_expr_type (expr)!="EXPR_VALUE":
        return None
    return {pattern[1] : get_val(expr)} # return {key : expr}

def is_commutative (op):
    return op in ["+", "*", "&", "|", "^"]

def match_two_ops (op1_expr, op1_pattern, op2_expr, op2_pattern):
    m1=match (op1_expr, op1_pattern)
    m2=match (op2_expr, op2_pattern)
    if m1==None or m2==None:
        return None # one of match for operands returned False, so we do the same
    # join two dicts from both operands:
    rt={}
    rt.update(m1)
    rt.update(m2)
    return rt

def match_EXPR_OP (expr, pattern):
    if get_expr_type(expr)!=get_expr_type(pattern): # be sure, both EXPR_OP.
        return None
    if get_op (expr)!=get_op (pattern): # be sure, ops type are the same.
        return None

    if (is_unary_op(get_op(expr))):
        # match unary expression.
        return match (get_op1 (expr), get_op1 (pattern))
    else:     
        # match binary expression.     

        # first try match operands as is.
        m=match_two_ops (get_op1 (expr), get_op1 (pattern), get_op2 (expr), get_op2 (pattern))
        if m!=None:
            return m
        # if matching unsuccessful, AND operation is commutative, try also swapped operands.
        if is_commutative (get_op (expr))==False:
            return None
        return match_two_ops (get_op1 (expr), get_op2 (pattern), get_op2 (expr), get_op1 (pattern))

# returns dict in case of success, or None
def match (expr, pattern):
    t=get_expr_type(pattern)
    if t=="EXPR_WILDCARD":
        return match_EXPR_WILDCARD (expr, pattern)
    elif t=="EXPR_WILDCARD_VALUE":
        return match_EXPR_WILDCARD_VALUE (expr, pattern)
    elif t=="EXPR_SYMBOL":
        if expr==pattern:
            return {}
        else:
            return None
    elif t=="EXPR_VALUE":
        if expr==pattern:
            return {}
        else:
            return None
    elif t=="EXPR_OP":
        return match_EXPR_OP (expr, pattern)
    else:
        raise AssertionError
\end{lstlisting}

Now how we will use it:

\begin{lstlisting}
# (X*A)*B -> X*(A*B)
def reduce_MUL1 (expr):
    m=match (expr, create_binary_expr ("*", (create_binary_expr ("*", bind_expr ("X"), bind_value ("A"))), bind_value ("B")))
    if m==None:
        return expr # no match

    return dbg_print_reduced_expr ("reduce_MUL1", expr, create_binary_expr ("*", 
        m["X"], # new op1
        create_val_expr (m["A"] * m["B"]))) # new op2
\end{lstlisting}

We take input expression, and we also construct pattern to be matched.
Matcher works recursively over both expressions synchronously.
Pattern is also expression, but can use two additional node types: \textit{EXPR\_WILDCARD} and
\textit{EXPR\_WILDCARD\_VALUE}. These nodes are supplied with keys (stored as strings).
When matcher encounters \textit{EXPR\_WILDCARD} in pattern, it just stashes current expression and will return it.
If matcher encounters \textit{EXPR\_WILDCARD\_VALUE}, it does the same, but only in case the current node
has \textit{EXPR\_VALUE} type.

\TT{bind\_expr()} and \TT{bind\_value()} are functions which create nodes with the types we have seen.

All this means, \TT{reduce\_MUL1()} function will search for the expression in form \textit{(X*A)*B}, where $A$ and $B$
are numbers. In other cases, matcher will return input expression untouched, so these reducing function can be chained.

Now when \TT{reduce\_MUL1()} encounters (sub)expression we are interesting in,
it will return dictionary with keys and expressions.
Let's add \TT{print m} call somewhere before return and rerun:

\begin{lstlisting}
python td.py tests/add2.s

...

going to reduce (((arg1 + arg1) + (arg1 + arg1)) + ((arg1 + arg1) + (arg1 + arg1)))
reduction in reduce_ADD1() (arg1 + arg1) -> (arg1 * 2)
reduction in reduce_ADD1() (arg1 + arg1) -> (arg1 * 2)
reduction in reduce_ADD1() ((arg1 * 2) + (arg1 * 2)) -> ((arg1 * 2) * 2)
{'A': 2, 'X': ('EXPR_SYMBOL', 'arg1'), 'B': 2}
reduction in reduce_MUL1() ((arg1 * 2) * 2) -> (arg1 * 4)
reduction in reduce_ADD1() (arg1 + arg1) -> (arg1 * 2)
reduction in reduce_ADD1() (arg1 + arg1) -> (arg1 * 2)
reduction in reduce_ADD1() ((arg1 * 2) + (arg1 * 2)) -> ((arg1 * 2) * 2)
{'A': 2, 'X': ('EXPR_SYMBOL', 'arg1'), 'B': 2}
reduction in reduce_MUL1() ((arg1 * 2) * 2) -> (arg1 * 4)
reduction in reduce_ADD1() ((arg1 * 4) + (arg1 * 4)) -> ((arg1 * 4) * 2)
{'A': 4, 'X': ('EXPR_SYMBOL', 'arg1'), 'B': 2}
reduction in reduce_MUL1() ((arg1 * 4) * 2) -> (arg1 * 8)
going to reduce (arg1 * 8)
...
result=(arg1 * 8)
\end{lstlisting}

The dictionary has keys we supplied plus expressions matcher found.
We then can use them to create new expression and return it.
Numbers are just summed while forming second operand to ``*'' opeartion.

Now a real-world optimization technique---optimizing GCC replaced multiplication by 31 by shifting and subtraction operations:

\begin{lstlisting}
        mov     rax, rdi
        sal     rax, 5
        sub     rax, rdi
\end{lstlisting}

Without reduction functions, our decompiler will translate this into \textit{((arg1 << 5) - arg1)}.
We can replace shifting left by multiplication:

\begin{lstlisting}
# X<<n -> X*(2^n)
def reduce_SHL1 (expr):
    m=match (expr, create_binary_expr ("<<", bind_expr ("X"), bind_value ("Y")))
    if m==None:
        return expr # no match
    
    return dbg_print_reduced_expr ("reduce_SHL1", expr, create_binary_expr ("*", m["X"], create_val_expr (1<<m["Y"])))
\end{lstlisting}

Now we getting \textit{((arg1 * 32) - arg1)}.
We can add another reduction function:

\begin{lstlisting}
# (X*n)-X -> X*(n-1)
def reduce_SUB3 (expr):
    m=match (expr, create_binary_expr ("-",
        create_binary_expr ("*", bind_expr("X1"), bind_value ("N")),
        bind_expr("X2")))
    
    if m!=None and match (m["X1"], m["X2"])!=None:
        return dbg_print_reduced_expr ("reduce_SUB3", expr, create_binary_expr ("*", m["X1"], create_val_expr (m["N"]-1)))
    else:
        return expr # no match
\end{lstlisting}

Matcher will return two X's, and we must be assured that they are equal.
In fact, in previous versions of this toy decompiler, I did comparison with plain ``=='', and it worked.
But we can reuse \TT{match()} function for the same purpose, because it will process commutative operations better.
For example, if X1 is ``Q+1'' and X2 is ``1+Q'', expressions are equal, but plain ``=='' will not work.
On the other side, \TT{match()} function, when encounter ``+'' operation (or another commutative operation),
and it fails with comparison, it will also try swapped operand and will try to compare again.

However, to understand it easier, for a moment, you can imagine there is ``=='' instead of the second \TT{match()}.

Anyway, here is what we've got:

\begin{lstlisting}
working out tests/mul31_GCC.s
going to reduce ((arg1 << 5) - arg1)
reduction in reduce_SHL1() (arg1 << 5) -> (arg1 * 32)
reduction in reduce_SUB3() ((arg1 * 32) - arg1) -> (arg1 * 31)
going to reduce (arg1 * 31)
...
result=(arg1 * 31)
\end{lstlisting}

Another optimization technique is often seen in ARM thumb code: AND-ing a value with a value like 0xFFFFFFF0,
is implemented using shifts:

\begin{lstlisting}
        mov rax, rdi
        shr rax, 4
        shl rax, 4
\end{lstlisting}

This code is quite common in ARM thumb code, because it's a headache to encode 32-bit constants using couple of 16-bit
thumb instructions, while single 16-bit instruction can shift by 4 bits left or right.

Also, the expression \textit{(x>>4)<<4} can be jokingly called as ``twitching operator'':
I've heard the ``-{}-i++'' expression was called like this in Russian-speaking social networks, it was some kind of meme
(``operator podergivaniya'').

Anyway, these reduction functions will be used:

\begin{lstlisting}
# X>>n -> X / (2^n)
...
def reduce_SHR2 (expr):
    m=match(expr, create_binary_expr(">>", bind_expr("X"), bind_value("Y")))
    if m==None or m["Y"]>=64:
        return expr # no match

    return dbg_print_reduced_expr ("reduce_SHR2", expr, create_binary_expr ("/", m["X"],
        create_val_expr (1<<m["Y"])))

...

# X<<n -> X*(2^n)
def reduce_SHL1 (expr):
    m=match (expr, create_binary_expr ("<<", bind_expr ("X"), bind_value ("Y")))
    if m==None:
        return expr # no match
    
    return dbg_print_reduced_expr ("reduce_SHL1", expr, create_binary_expr ("*", m["X"], create_val_expr (1<<m["Y"])))

...

# FIXME: slow
# returns True if n=2^x or popcnt(n)=1
def is_2n(n):
    return bin(n).count("1")==1


# AND operation using DIV/MUL or SHL/SHR
# (X / (2^n)) * (2^n) -> X&(~((2^n)-1))
def reduce_MUL2 (expr):
    m=match(expr, create_binary_expr ("*", create_binary_expr ("/", bind_expr("X"), bind_value("N1")), bind_value("N2")))
    if m==None or m["N1"]!=m["N2"] or is_2n(m["N1"])==False: # short-circuit expression
        return expr # no match

    return dbg_print_reduced_expr("reduce_MUL2", expr, create_binary_expr ("&", m["X"],
        create_val_expr(~(m["N1"]-1)&0xffffffffffffffff)))
\end{lstlisting}

Now the result:

\begin{lstlisting}
working out tests/AND_by_shifts2.s
going to reduce ((arg1 >> 4) << 4)
reduction in reduce_SHR2() (arg1 >> 4) -> (arg1 / 16)
reduction in reduce_SHL1() ((arg1 / 16) << 4) -> ((arg1 / 16) * 16)
reduction in reduce_MUL2() ((arg1 / 16) * 16) -> (arg1 & 0xfffffffffffffff0)
going to reduce (arg1 & 0xfffffffffffffff0)
...
result=(arg1 & 0xfffffffffffffff0)
\end{lstlisting}

\subsubsection{Division using multiplication}

Division is often replaced by multiplication for performance reasons.

From school-level arithmetics, we can remember that division by 3 can be replaced by multiplication by $\frac{1}{3}$.
In fact, sometimes compilers do so for floating-point arithmetics, for example, FDIV instruction in x86 code
can be replaced by FMUL.
At least MSVC 6.0 will replace division by 3 by multiplication by $\frac{1}{3}$ and sometimes it's hard to be sure,
what operation was in original source code.

But when we operate over integer values and CPU registers, we can't use fractions.
However, we can rework fraction:

\begin{center}
{\large $result = \frac{x}{3} = x \cdot \frac{1}{3} = x \cdot \frac{1 \cdot MagicNumber}{3 \cdot MagicNumber}$}
\end{center}

Given the fact that division by $2^n$ is very fast, we now should find that $MagicNumber$, for which the following
equation will be true: $2^n = 3 \cdot MagicNumber$.

This code performing division by 10:

\begin{lstlisting}
        mov     rax, rdi
        movabs  rdx, 0cccccccccccccccdh
        mul     rdx
        shr     rdx, 3
        mov     rax, rdx
\end{lstlisting}

Division by $2^{64}$ is somewhat hidden: lower 64-bit of product in RAX is not used (dropped), only higher 64-bit of
product (in RDX) is used and then shifted by additional 3 bits.

RDX register is set during processing of MUL/IMUL like this:

\begin{lstlisting}
def handle_unary_MUL_IMUL (registers, op1):
    op1_expr=register_or_number_in_string_to_expr (registers, op1)
    result=create_binary_expr ("*", registers["rax"], op1_expr)
    registers["rax"]=result
    registers["rdx"]=create_binary_expr (">>", result, create_val_expr(64))
\end{lstlisting}

In other words, the assembly code we have just seen multiplicates by {\Large $\frac{0cccccccccccccccdh}{2^{64+3}}$},
or divides by {\Large $\frac{2^{64+3}}{0cccccccccccccccdh}$}.
To find divisor we just have to divide numerator by denominator.

\begin{lstlisting}
# n = magic number
# m = shifting coefficient
# return = 1 / (n / 2^m) = 2^m / n
def get_divisor (n, m):
    return (2**float(m))/float(n)

# (X*n)>>m, where m>=64 -> X/...
def reduce_div_by_MUL (expr):
    m=match (expr, create_binary_expr(">>", create_binary_expr ("*", bind_expr("X"), bind_value("N")), bind_value("M")))
    if m==None:
        return expr # no match
    
    divisor=get_divisor(m["N"], m["M"])
    return dbg_print_reduced_expr ("reduce_div_by_MUL", expr, create_binary_expr ("/", m["X"], create_val_expr (int(divisor))))
\end{lstlisting}

This works, but we have a problem: this rule takes \textit{(arg1 * 0xcccccccccccccccd) >> 64} expression first and
finds divisor to be equal to $1.25$.
This is correct: result is shifted by 3 bits after (or divided by 8), and $1.25 \cdot 8 = 10$.
But our toy decompiler doesn't support real numbers.

We can solve this problem in the following way: if divisor has fractional part, we postpone reducing, with a hope,
that two subsequent right shift operations will be reduced into single one:

\begin{lstlisting}
# (X*n)>>m, where m>=64 -> X/...
def reduce_div_by_MUL (expr):
    m=match (expr, create_binary_expr(">>", create_binary_expr ("*", bind_expr("X"), bind_value("N")), bind_value("M")))
    if m==None:
        return expr # no match
    
    divisor=get_divisor(m["N"], m["M"])
    if math.floor(divisor)==divisor:
        return dbg_print_reduced_expr ("reduce_div_by_MUL", expr, create_binary_expr ("/", m["X"], create_val_expr (int(divisor))))
    else:
        print "reduce_div_by_MUL(): postponing reduction, because divisor=", divisor
        return expr
\end{lstlisting}

That works:

\begin{lstlisting}
working out tests/div_by_mult10_unsigned.s
going to reduce (((arg1 * 0xcccccccccccccccd) >> 64) >> 3)
reduce_div_by_MUL(): postponing reduction, because divisor= 1.25
reduction in reduce_SHR1() (((arg1 * 0xcccccccccccccccd) >> 64) >> 3) -> ((arg1 * 0xcccccccccccccccd) >> 67)
going to reduce ((arg1 * 0xcccccccccccccccd) >> 67)
reduction in reduce_div_by_MUL() ((arg1 * 0xcccccccccccccccd) >> 67) -> (arg1 / 10)
going to reduce (arg1 / 10)
result=(arg1 / 10)
\end{lstlisting}

I don't know if this is best solution. In early version of this decompiler, it processed input expression in two passes:
first pass for everything except division by multiplication, and the second pass for the latter.
I don't know which way is better.
Or maybe we could support real numbers in expressions?

Couple of words about better understanding division by multiplication.
Many people miss ``hidden'' division by $2^{32}$ or $2^{64}$,
when lower 32-bit part (or 64-bit part) of product is not used (or just dropped).
Also, there is misconception that modulo inverse is used here. This is close, but not the same thing.
Extended Euclidean algorithm is usually used to find \textit{magic coefficient}, but in fact,
this algorithm is rather used to solve the equation. You can solve it using any other method.
Also, needless to mention, the equation is unsolvable for some divisors, because this is diophantine equation
(i.e., equation allowing result to be only integer), since we work on integer CPU registers, after all.

\subsection{Obfuscation/deobfuscation}

Despite simplicity of our decompiler, we can see how to deobfuscate (or optimize) using several simple tricks.

For example, this piece of code does nothing:

\begin{lstlisting}
        mov rax, rdi
        xor rax, 12345678h
        xor rax, 0deadbeefh
        xor rax, 12345678h
        xor rax, 0deadbeefh
\end{lstlisting}

We would need these rules to tame it:

\begin{lstlisting}
# (X^n)^m -> X^(n^m)
def reduce_XOR4 (expr):
    m=match(expr, 
        create_binary_expr("^",
            create_binary_expr ("^", bind_expr("X"), bind_value("N")),
                bind_value("M")))
    
    if m!=None:
        return dbg_print_reduced_expr ("reduce_XOR4", expr, create_binary_expr ("^", m["X"], 
            create_val_expr (m["N"]^m["M"])))
    else:
        return expr # no match

...

# X op 0 -> X, where op is ADD, OR, XOR, SUB
def reduce_op_0 (expr):
    # try each:
    for op in ["+", "|", "^", "-"]:
        m=match(expr, create_binary_expr(op, bind_expr("X"), create_val_expr (0)))
        if m!=None:
            return dbg_print_reduced_expr ("reduce_op_0", expr, m["X"])

    # default:
    return expr # no match
\end{lstlisting}

\begin{lstlisting}
working out tests/t9_obf.s
going to reduce ((((arg1 ^ 0x12345678) ^ 0xdeadbeef) ^ 0x12345678) ^ 0xdeadbeef)
reduction in reduce_XOR4() ((arg1 ^ 0x12345678) ^ 0xdeadbeef) -> (arg1 ^ 0xcc99e897)
reduction in reduce_XOR4() ((arg1 ^ 0xcc99e897) ^ 0x12345678) -> (arg1 ^ 0xdeadbeef)
reduction in reduce_XOR4() ((arg1 ^ 0xdeadbeef) ^ 0xdeadbeef) -> (arg1 ^ 0x0)
going to reduce (arg1 ^ 0x0)
reduction in reduce_op_0() (arg1 ^ 0x0) -> arg1
going to reduce arg1
result=arg1
\end{lstlisting}

This piece of code can be deobfuscated (or optimized) as well:

\begin{lstlisting}
; toggle last bit

        mov rax, rdi
        mov rbx, rax
        mov rcx, rbx
        mov rsi, rcx
        xor rsi, 12345678h
        xor rsi, 12345679h
        mov rax, rsi
\end{lstlisting}

\begin{lstlisting}
working out tests/t7_obf.s
going to reduce ((arg1 ^ 0x12345678) ^ 0x12345679)
reduction in reduce_XOR4() ((arg1 ^ 0x12345678) ^ 0x12345679) -> (arg1 ^ 1)
going to reduce (arg1 ^ 1)
result=(arg1 ^ 1)
\end{lstlisting}

I also used \textit{aha!}\footnote{\url{http://www.hackersdelight.org/aha/aha.pdf}} superoptimizer to find weird piece of code which does nothing.

\textit{Aha!} is so called superoptimizer, it tries various piece of codes in brute-force manner, in attempt
to find shortest possible alternative for some mathematical operation.
While sane compiler developers use superoptimizers for this task, I tried it in opposite way, to find oddest
pieces of code for some simple operations, including \ac{NOP} operation.
In past, I've used it to find weird alternative to XOR operation (\ref{weird_XOR}).

So here is what \textit{aha!} can find for \ac{NOP}:

\begin{lstlisting}
; do nothing (as found by aha)

        mov rax, rdi
        and rax, rax
        or rax, rax
\end{lstlisting}

\begin{lstlisting}
# X & X -> X
def reduce_AND3 (expr):
    m=match (expr, create_binary_expr ("&", bind_expr ("X1"), bind_expr ("X2")))
    if m!=None and match (m["X1"], m["X2"])!=None:
        return dbg_print_reduced_expr("reduce_AND3", expr, m["X1"])
    else:
        return expr # no match

...

# X | X -> X
def reduce_OR1 (expr):
    m=match (expr, create_binary_expr ("|", bind_expr ("X1"), bind_expr ("X2")))
    if m!=None and match (m["X1"], m["X2"])!=None:
        return dbg_print_reduced_expr("reduce_OR1", expr, m["X1"])
    else:
        return expr # no match
\end{lstlisting}

\begin{lstlisting}
working out tests/t11_obf.s
going to reduce ((arg1 & arg1) | (arg1 & arg1))
reduction in reduce_AND3() (arg1 & arg1) -> arg1
reduction in reduce_AND3() (arg1 & arg1) -> arg1
reduction in reduce_OR1() (arg1 | arg1) -> arg1
going to reduce arg1
result=arg1
\end{lstlisting}

This is weirder:

\begin{lstlisting}
; do nothing (as found by aha)

;Found a 5-operation program:
;   neg   r1,rx
;   neg   r2,rx
;   neg   r3,r1
;   or    r4,rx,2
;   and   r5,r4,r3
;   Expr: ((x | 2) & -(-(x)))

        mov rax, rdi
        neg rax
        neg rax
        or rdi, 2
        and rax, rdi
\end{lstlisting}

Rules added (I used ``NEG'' string to represent sign change and to be different from subtraction operation,
which is just minus (``-'')):

\label{AND2}
\begin{lstlisting}
# (op(op X)) -> X, where both ops are NEG or NOT
def reduce_double_NEG_or_NOT (expr):
    # try each:
    for op in ["NEG", "~"]:
        m=match (expr, create_unary_expr (op, create_unary_expr (op, bind_expr("X"))))
        if m!=None:
            return dbg_print_reduced_expr ("reduce_double_NEG_or_NOT", expr, m["X"])

    # default:
    return expr # no match

...

# X & (X | ...) -> X
def reduce_AND2 (expr):
    m=match (expr, create_binary_expr ("&", create_binary_expr ("|", bind_expr ("X1"), bind_expr ("REST")), bind_expr ("X2")))
    if m!=None and match (m["X1"], m["X2"])!=None:
        return dbg_print_reduced_expr("reduce_AND2", expr, m["X1"])
    else:
        return expr # no match
\end{lstlisting}

\begin{lstlisting}
going to reduce ((-(-arg1)) & (arg1 | 2))
reduction in reduce_double_NEG_or_NOT() (-(-arg1)) -> arg1
reduction in reduce_AND2() (arg1 & (arg1 | 2)) -> arg1
going to reduce arg1
result=arg1
\end{lstlisting}

I also forced \textit{aha!} to find piece of code which adds 2 with no addition/subtraction operations allowed:

\begin{lstlisting}
; arg1+2, without add/sub allowed, as found by aha:

;Found a 4-operation program:
;   not   r1,rx
;   neg   r2,r1
;   not   r3,r2
;   neg   r4,r3
;   Expr: -(~(-(~(x))))

        mov     rax, rdi
        not     rax
        neg     rax
        not     rax
        neg     rax
\end{lstlisting}

Rule:

\begin{lstlisting}
# (- (~X)) -> X+1
def reduce_NEG_NOT (expr):
    m=match (expr, create_unary_expr ("NEG", create_unary_expr ("~", bind_expr("X"))))
    if m==None:
        return expr # no match
    
    return dbg_print_reduced_expr ("reduce_NEG_NOT", expr, create_binary_expr ("+", m["X"],create_val_expr(1)))
\end{lstlisting}

\begin{lstlisting}
working out tests/add_by_not_neg.s
going to reduce (-(~(-(~arg1))))
reduction in reduce_NEG_NOT() (-(~arg1)) -> (arg1 + 1)
reduction in reduce_NEG_NOT() (-(~(arg1 + 1))) -> ((arg1 + 1) + 1)
reduction in reduce_ADD3() ((arg1 + 1) + 1) -> (arg1 + 2)
going to reduce (arg1 + 2)
result=(arg1 + 2)
\end{lstlisting}

This is artifact of two's complement system of signed numbers representation.
Same can be done for subtraction (just swap NEG and NOT operations).

Now let's add some fake luggage to Fahrenheit-to-Celsius example:

\begin{lstlisting}
        ; celsius = 5 * (fahr-32) / 9
        ; fake luggage:
        mov     rbx, 12345h
        mov     rax, rdi
        sub     rax, 32
        ; fake luggage:
        add     rbx, rax
        imul    rax, 5
        mov     rbx, 9
        idiv    rbx
        ; fake luggage:
        sub     rdx, rax
\end{lstlisting}

It's not a problem for our decompiler, because the noise is left in RDX register, and not used at all:

\lstinputlisting{toy_decompiler/fahr_to_celsius_obf1.txt}

We can try to pretend we affect the result with the noise:

\begin{lstlisting}
        ; celsius = 5 * (fahr-32) / 9
        ; fake luggage:
        mov     rbx, 12345h
        mov     rax, rdi
        sub     rax, 32
        ; fake luggage:
        add     rbx, rax
        imul    rax, 5
        mov     rbx, 9
        idiv    rbx
        ; fake luggage:
        sub     rdx, rax
        mov     rcx, rax
        ; OR result with garbage (result of fake luggage):
        or      rcx, rdx
        ; the following instruction shouldn't affect result:
        and     rax, rcx
\end{lstlisting}

\dots but in fact, it's all reduced by \TT{reduce\_AND2()} function we already saw (\ref{AND2}):

\begin{lstlisting}
working out tests/fahr_to_celsius_obf2.s
going to reduce ((((arg1 - 32) * 5) / 9) & ((((arg1 - 32) * 5) / 9) | ((((arg1 - 32) * 5) % 9) - (((arg1 - 32) * 5) / 9))))
reduction in reduce_AND2() ((((arg1 - 32) * 5) / 9) & ((((arg1 - 32) * 5) / 9) | ((((arg1 - 32) * 5) % 9) - (((arg1 - 32) * 5)
/ 9)))) -> (((arg1 - 32) * 5) / 9)
going to reduce (((arg1 - 32) * 5) / 9)
result=(((arg1 - 32) * 5) / 9)
\end{lstlisting}

We can see that deobfuscation is in fact the same thing as optimization used in compilers.
We can try this function in GCC:

\begin{lstlisting}
int f(int a)
{
	return -(~a);
};
\end{lstlisting}

Optimizing GCC 5.4 (x86) generates this:

\begin{lstlisting}
f:
        mov     eax, DWORD PTR [esp+4]
        add     eax, 1
        ret
\end{lstlisting}

GCC has its own rewriting rules, some of which are, probably, close to what we use here.

\subsection{Tests}

Despite simplicity of the decompiler, it's still error-prone.
We need to be sure that original expression and reduced one are equivalent to each other.

\subsubsection{Evaluating expressions}

First of all, we would just evaluate (or \textit{run}, or \textit{execute})
expression with random values as arguments, and then compare results.

Evaluator do arithmetical operations when possible, recursively.
When any symbol is encountered, its value (randomly generated before) is taken from a table.

\begin{lstlisting}
un_ops={"NEG":operator.neg,
        "~":operator.invert}

bin_ops={">>":operator.rshift,
        "<<":(lambda x, c: x<<(c&0x3f)), # operator.lshift should be here, but it doesn't handle too big counts
        "&":operator.and_,
        "|":operator.or_,
        "^":operator.xor,
        "+":operator.add,
        "-":operator.sub,
        "*":operator.mul,
        "/":operator.div,
        "%":operator.mod}

def eval_expr(e, symbols):
    t=get_expr_type (e)
    if t=="EXPR_SYMBOL":
        return symbols[get_symbol(e)]
    elif t=="EXPR_VALUE":
        return get_val (e)
    elif t=="EXPR_OP":
        if is_unary_op (get_op (e)):
            return un_ops[get_op(e)](eval_expr(get_op1(e), symbols))
        else:
            return bin_ops[get_op(e)](eval_expr(get_op1(e), symbols), eval_expr(get_op2(e), symbols))
    else:
        raise AssertionError

def do_selftest(old, new):
    for n in range(100):
        symbols={"arg1":random.getrandbits(64), 
                "arg2":random.getrandbits(64), 
                "arg3":random.getrandbits(64), 
                "arg4":random.getrandbits(64)}
        old_result=eval_expr (old, symbols)&0xffffffffffffffff # signed->unsigned
        new_result=eval_expr (new, symbols)&0xffffffffffffffff # signed->unsigned
        if old_result!=new_result:
            print "self-test failed"
            print "initial expression: "+expr_to_string(old)
            print "reduced expression: "+expr_to_string(new)
            print "initial expression result: ", old_result
            print "reduced expression result: ", new_result
            exit(0)
\end{lstlisting}

In fact, this is very close to what LISP \textit{EVAL} function does, or even LISP interpreter.
However, not all symbols are set.
If the expression is using initial values from RAX or RBX
(to which symbols ``initial\_RAX'' and ``initial\_RBX'' are assigned,
decompiler will stop with exception, because no random values assigned to these registers,
and these symbols are absent in \textit{symbols} dictionary.

Using this test, I've suddenly found a bug here (despite simplicity of all these reduction rules).
Well, no-one protected from eye strain.
Nevertheless, the test has a serious problem: some bugs can be revealed only if one of arguments is $0$, or $1$, or $-1$.
Maybe there are even more special cases exists.

Mentioned above \textit{aha!} superoptimizer tries at least these values as arguments while testing:
1, 0, -1, 0x80000000, 0x7FFFFFFF, 0x80000001, 0x7FFFFFFE, 0x01234567, 0x89ABCDEF, -2, 2, -3, 3,
-64, 64, -5, -31415.

Still, you cannot be sure.

\subsubsection{Using Z3 SMT-solver for testing}

So here we will try Z3 SMT-solver.
SMT-solver can \textit{prove} that two expressions are equivalent to each other.

For example, with the help of \textit{aha!}, I've found another weird piece of code, which does nothing:

\begin{lstlisting}
; do nothing (obfuscation)

;Found a 5-operation program:
;   neg   r1,rx
;   neg   r2,r1
;   sub   r3,r1,3
;   sub   r4,r3,r1
;   sub   r5,r4,r3
;   Expr: (((-(x) - 3) - -(x)) - (-(x) - 3))

        mov rax, rdi
        neg rax
        mov rbx, rax
        ; rbx=-x
        mov rcx, rbx
        sub rcx, 3
        ; rcx=-x-3
        mov rax, rcx
        sub rax, rbx
        ; rax=(-(x) - 3) - -(x)
        sub rax, rcx
\end{lstlisting}

Using toy decompiler, I've found that this piece is reduced to \textit{arg1} expression:

\begin{lstlisting}
working out tests/t5_obf.s
going to reduce ((((-arg1) - 3) - (-arg1)) - ((-arg1) - 3))
reduction in reduce_SUB2() ((-arg1) - 3) -> (-(arg1 + 3))
reduction in reduce_SUB5() ((-(arg1 + 3)) - (-arg1)) -> ((-(arg1 + 3)) + arg1)
reduction in reduce_SUB2() ((-arg1) - 3) -> (-(arg1 + 3))
reduction in reduce_ADD_SUB() (((-(arg1 + 3)) + arg1) - (-(arg1 + 3))) -> arg1
going to reduce arg1
result=arg1
\end{lstlisting}

But is it correct?
I've added a function which can output expression(s) to SMT-LIB-format, it's as simple as a function which converts
expression to string.

And this is SMT-LIB-file for Z3:

\begin{lstlisting}
(assert
    (forall ((arg1 (_ BitVec 64)) (arg2 (_ BitVec 64)) (arg3 (_ BitVec 64)) (arg4 (_ BitVec 64)))
        (=
            (bvsub (bvsub (bvsub (bvneg arg1) #x0000000000000003) (bvneg arg1)) (bvsub (bvneg arg1) #x0000000000000003))
            arg1
        )
    )
)
(check-sat)
\end{lstlisting}

In plain English terms, what we asking it to be sure, that \textit{forall} four 64-bit arguments, two expressions
are equivalent (second is just \textit{arg1}).

The syntax maybe hard to understand, but in fact, this is very close to LISP, and arithmetical operations
are named ``bvsub'', ``bvadd'', etc, because ``bv'' stands for \textit{bit vector}.

While running, Z3 shows ``sat'', meaning ``satisfiable''.
In other words, Z3 couldn't find counterexample for this expression.

In fact, I can rewrite this expression in the following form: \textit{expr1 != expr2}, and we would ask
Z3 to find at least one set of input arguments, for which expressions are not equal to each other:

\begin{lstlisting}
(declare-const arg1 (_ BitVec 64))
(declare-const arg2 (_ BitVec 64))
(declare-const arg3 (_ BitVec 64))
(declare-const arg4 (_ BitVec 64))

(assert
    (not
        (=
            (bvsub (bvsub (bvsub (bvneg arg1) #x0000000000000003) (bvneg arg1)) (bvsub (bvneg arg1) #x0000000000000003))
            arg1
        )
    )
)
(check-sat)
\end{lstlisting}

Z3 says ``unsat'', meaning, it couldn't find any such counterexample.
In other words, for all possible input arguments, results of these two expressions are always equal to each other.

Nevertheless, Z3 is not omnipotent.
It fails to prove equivalence of the code which performs division by multiplication.
First of all, I extended it so boths results will have size of 128 bit instead of 64:

\begin{lstlisting}
(declare-const x (_ BitVec 64))
    (assert
        (forall ((x (_ BitVec 64)))
            (=
                ((_ zero_extend 64) (bvudiv x (_ bv17 64)))
                (bvlshr (bvmul ((_ zero_extend 64) x) #x0000000000000000f0f0f0f0f0f0f0f1) (_ bv68 128))
            )
        )
    )
(check-sat)
(get-model)
\end{lstlisting}

(\textit{bv17} is just 64-bit number 17, etc. ``bv'' stands for ``bit vector'', as opposed to integer value.)

Z3 works too long without any answer, and I had to interrupt it.

As Z3 developers mentioned, such expressions are hard for Z3 so far:
\url{https://github.com/Z3Prover/z3/issues/514}.

Still, division by multiplication can be tested using previously described brute-force check.

\subsection{My other implementations of toy decompiler}

When I made attempt to write it in C++, of course, node in expression was represented using class.
There is also implementation in pure C\footnote{\url{https://github.com/DennisYurichev/SAT_SMT_article/tree/master/toy_decompiler/files/C}}, node is represented using structure.

Matchers in both C++ and C versions doesn't return any dictionary, but instead, \TT{bind\_value()}
functions takes pointer to a variable which will contain value after successful matching.
\TT{bind\_expr()} takes pointer to a pointer, which will points to the part of expression, again, in case of success.
I took this idea from LLVM.

Here are two pieces of code from LLVM source code with couple of reducing rules:

\begin{lstlisting}
// (X >> A) << A -> X
  Value *X;
  if (match(Op0, m_Exact(m_Shr(m_Value(X), m_Specific(Op1)))))
    return X;
\end{lstlisting}

( \href{http://llvm.org/docs/doxygen/html/InstructionSimplify_8cpp_source.html}{lib/Analysis/InstructionSimplify.cpp} )

\begin{lstlisting}
// (A | B) | C  and  A | (B | C)                  -> bswap if possible.
  // (A >> B) | (C << D)  and  (A << B) | (B >> C)  -> bswap if possible.
  if (match(Op0, m_Or(m_Value(), m_Value())) ||
      match(Op1, m_Or(m_Value(), m_Value())) ||
      (match(Op0, m_LogicalShift(m_Value(), m_Value())) &&
       match(Op1, m_LogicalShift(m_Value(), m_Value())))) {
    if (Instruction *BSwap = MatchBSwap(I))
      return BSwap;
\end{lstlisting}
( \href{https://github.com/numba/llvm-mirror/blob/master/lib/Transforms/InstCombine/InstCombineAndOrXor.cpp}{lib/Transforms/InstCombine/InstCombineAndOrXor.cpp} )

As you can see, my matcher tries to mimic LLVM.
What I call \textit{reduction} is called \textit{folding} in LLVM.
Both terms are popular.

I have also a blog post about LLVM obfuscator, in which LLVM matcher is mentioned: \url{https://yurichev.com/blog/llvm/}.

Python version of toy decompiler uses strings in place where enumerate data type is used in C version
(like \textit{OP\_AND}, \textit{OP\_MUL}, etc) and
symbols used in Racket version\footnote{Racket is Scheme (which is, in turn, LISP dialect) dialect.
\url{https://github.com/DennisYurichev/SAT_SMT_article/tree/master/toy_decompiler/files/Racket}} (like \textit{'OP\_DIV}, etc).
This may be seen as inefficient, nevertheless, thanks to strings interning, only address of strings are compared in
Python version, not strings as a whole. So strings in Python can be seen as possible replacement for LISP symbols.

\subsubsection{Even simpler toy decompiler}

Knowledge of LISP makes you understand all these things naturally, without significant effort.
But when I had no knowledge of it, but still tried to make a simple toy decompiler, I made it using usual text strings
which holded expressions for each registers (and even memory).

So when MOV instruction copies value from one register to another, we just copy string.
When arithmetical instruction occurred, we do string concatenation:

\begin{lstlisting}
std::string registers[TOTAL];

...

// all 3 arguments are strings
switch (ins, op1, op2)
{
    ...
    case ADD:    registers[op1]="(" + registers[op1] + " + " + registers[op2] + ")";
                 break;
    ...
    case MUL:    registers[op1]="(" + registers[op1] + " / " + registers[op2] + ")";
                 break;
    ...
}
\end{lstlisting}

Now you'll have long expressions for each register, represented as strings.
For reducing them, you can use plain simple regular expression matcher.

For example, for the rule \TT{(X*n)+(X*m) -> X*(n+m)}, you can match (sub)string using the following
regular expression: \\
\TT{((.*)*(.*))+((.*)*(.*))}
\footnote{This regular expression string hasn't been properly escaped,
for the reason of easier readability and understanding.}.
If the string is matched, you're getting 4 groups (or substrings).
You then just compare 1st and 3rd using string comparison function, then you check if
the 2nd and 4th are numbers, you convert them to numbers, sum them and you make new string, consisting
of 1st group and sum, like this: \TT{(" + X + "*" + (int(n) + int(m)) + ")}.

It was naïve, clumsy, it was source of great embarrassment, but it worked correctly.

\subsection{Difference between toy decompiler and commercial-grade one}

Perhaps, someone, who currently reading this text, may rush into extending my source code.
As an exercise, I would say, that the first step could be support of partial registers: i.e., AL, AX, EAX.
This is tricky, but doable.

Another task may be support of \ac{FPU} x86 instructions (\ac{FPU} stack modeling isn't a big deal).

The gap between toy decompiler and a commercial decompiler like Hex-Rays is still enormous.
Several tricky problems must be solved, at least these:

\begin{itemize}
\item C data types: arrays, structures, pointers, etc.
This problem is virtually non-existent for \ac{JVM} (Java, etc) and .NET decompilers, because type information
is present in binary files.

\item Basic blocks, C/C++ statements. Mike Van Emmerik in his thesis
\footnote{\url{https://yurichev.com/mirrors/vanEmmerik_ssa.pdf}} shows how this can be tackled using \ac{SSA} forms
(which are also used heavily in compilers).

\item Memory support, including local stack. Keep in mind pointer aliasing problem.
Again, decompilers of \ac{JVM} and .NET files are simpler here.
\end{itemize}

\subsection{Further reading}

There are several interesting open-source attempts to build decompiler.
Both source code and theses are interesting study.

\begin{itemize}
	\item \textit{decomp} by Jim Reuter\footnote{
			\url{http://www.program-transformation.org/Transform/DecompReadMe},
			\url{http://www.program-transformation.org/Transform/DecompDecompiler}}.

	\item \textit{DCC} by Cristina Cifuentes\footnote{
			\url{http://www.program-transformation.org/Transform/DccDecompiler},
			thesis: \url{https://yurichev.com/mirrors/DCC_decompilation_thesis.pdf}}.

		It is interesting that this decompiler supports only one type (\textit{int}).
		Maybe this is a reason why DCC decompiler produces source code with \textit{.B} extension?
		Read more about B typeless language (C predecessor): \url{https://yurichev.com/blog/typeless/}.

	\item \textit{Boomerang} by Mike Van Emmerik, Trent Waddington et al\footnote{
			\url{http://boomerang.sourceforge.net/},
			\url{http://www.program-transformation.org/Transform/MikeVanEmmerik},
			thesis: \url{https://yurichev.com/mirrors/vanEmmerik_ssa.pdf}}.
\end{itemize}

As I've said, LISP knowledge can help to understand this all much easier.
Here is well-known micro-interpreter of LISP by Peter Norvig, also written in Python:
\url{https://web.archive.org/web/20161116133448/http://www.norvig.com/lispy.html},
\url{https://web.archive.org/web/20160305172301/http://norvig.com/lispy2.html}.

\subsection{The files}

Python version and tests: \url{https://github.com/DennisYurichev/SAT_SMT_article/tree/master/toy_decompiler/files}.

There are also C and Racket versions, but outdated.

Keep in mind---this decompiler is still at toy level, and it was tested only on tiny test files supplied.

}
\RU{\section{Игрушечный декомпилятор}
\label{toy_decompiler}

\subsection{Введение}

Современный компилятор это результат работы сотен разработчиков/лет.
В то же время, игрушечный компилятор может быть упражнением для студента на неделю (или даже на выходные).

Точно также, коммерческий декомпилятор как Hex-Rays может быть невероятно сложным,
но игрушечный декомпилятор, как этот, может быть легко понят и сделан заново.

Нижеследующий декомпилятор написан на Питоне, поддерживает только короткие бейсик-блоки, без переходов.
Память также не поддерживается.

\subsection{Структура данных}

Наш игрушечный декомпилятор будет использовать только одну единственную структуру данных, которая представляет дерево выражений.

Многие учебники по программированию имеют пример конвертирования температуры из шкалы Фаренгейта в шкалу Цельсия, используя
такую формулу:

\begin{center}
{\large $celsius = (fahrenheit - 32) \cdot \frac{5}{9}$}
\end{center}

Это выражение может быть представлено как дерево:

% reworked from http://www.texample.net/tikz/examples/decision-tree/
\tikzset{
  treenode/.style = {shape=rectangle, rounded corners,
                     draw, align=center,
                     top color=white, bottom color=blue!20},
  env/.style      = {treenode, font=\ttfamily\normalsize},
}

\begin{center}
\begin{tikzpicture}
[
	grow                    = down,
	sibling distance        = 4em,
	level distance          = 5em,
	edge from parent/.style = {draw, -latex},
	every node/.style       = {font=\footnotesize},
	sloped
]
\node [env] {/}
	child
	{
		node [env] {*}
		child { 
			node [env] {-}
			child { node [env] {INPUT} }
			child { node [env] {32} }
			}
		child { node [env] {5} }
	}
	child { node [env] {9} }
	;

\end{tikzpicture}
\end{center}


Как хранить его в памяти?
Мы видим здесь 3 типа узлов: 1) числа (или значения); 2) арифметические операции; 3) символы (как ``INPUT'').

Многие разработчики, любящие использовать \ac{OOP}, создадут что-то вроде класса.
Другие разработчики, может быть, будут использовать ``variant type''.

Я буду использовать простейший способ представления этой структуры: Питоновский кортеж (tuple).
Первый элемент кортежа будет строкой:
``EXPR\_OP'' для операции, ``EXPR\_SYMBOL'' для символа, или ``EXPR\_VALUE'' для значения.
В случае с символом или значением, оно следует за строкой.
В случае операции, за строкой следуют другие кортежи.

Тип узла и тип операции хранятся как простые строки --- для облегчения отладки.

Вот \textit{конструкторы} в нашем коде, в \ac{OOP}-шном смысле.

\begin{lstlisting}
def create_val_expr (val):
    return ("EXPR_VALUE", val)

def create_symbol_expr (val):
    return ("EXPR_SYMBOL", val)

def create_binary_expr (op, op1, op2):
    return ("EXPR_OP", op, op1, op2)
\end{lstlisting}

Это также \textit{аксессоры}:

\begin{lstlisting}
def get_expr_type(e):
    return e[0]

def get_symbol (e):
    assert get_expr_type(e)=="EXPR_SYMBOL"
    return e[1]

def get_val (e):
    assert get_expr_type(e)=="EXPR_VALUE"
    return e[1]

def is_expr_op(e):
    return get_expr_type(e)=="EXPR_OP"

def get_op (e):
    assert is_expr_op(e)
    return e[1]

def get_op1 (e):
    assert is_expr_op(e)
    return e[2]

def get_op2 (e):
    assert is_expr_op(e)
    return e[3]
\end{lstlisting}

Выражение конвертирования температуры, которое мы только что видели, будет представлено как:

\begin{center}
\begin{tikzpicture}
[
	grow                    = down,
	sibling distance        = 8em,
	level distance          = 5em,
	edge from parent/.style = {draw, -latex},
	every node/.style       = {font=\footnotesize},
	sloped
]
\node [env] {"EXPR\_OP"\\"/"}
	child
	{
		node [env] {"EXPR\_OP"\\"*"}
		child { 
			node [env] {"EXPR\_OP"\\"-"}
			child { node [env] {"EXPR\_SYMBOL"\\"arg1"} }
			child { node [env] {"EXPR\_VALUE"\\32} }
			}
		child { node [env] {"EXPR\_VALUE"\\5} }
	}
	child { node [env] {"EXPR\_VALUE"\\9} }
	;

\end{tikzpicture}
\end{center}


\dots или как Питоновское выражение:

\begin{lstlisting}
('EXPR_OP', '/', 
	('EXPR_OP', '*',
	('EXPR_OP', '-', ('EXPR_SYMBOL', 'arg1'), ('EXPR_VALUE', 32)), 
	('EXPR_VALUE', 5)), 
('EXPR_VALUE', 9))
\end{lstlisting}

На самом деле, это \ac{AST} в его простейшем виде.
\ac{AST} активно используются в компиляторах.

\subsection{Простые примеры}

Начнем с простейшего примера:

\begin{lstlisting}
        mov     rax, rdi
        imul    rax, rsi
\end{lstlisting}

В самом начале, такие символы присваиваются регистрами:
RAX=initial\_RAX,
RBX=initial\_RBX,
RDI=arg1,
RSI=arg2,
RDX=arg3,
RCX=arg4.

Когда мы обрабатываем инструкцию MOV, мы просто компируем выражение из RDI в RAX.
Когда мы обрабатываем инструкцию IMUL, мы просто создаем новое выражение, соеденяя вместе выражения из RAX и RSI
и сохраняя результат снова в RAX.

Я могу подать это на вход декомпилятору, и мы увидим, как состояние регистров меняется во время обработки:

\begin{lstlisting}
python td.py --show-registers --python-expr tests/mul.s

...

line=[mov       rax, rdi]
rcx=('EXPR_SYMBOL', 'arg4')
rsi=('EXPR_SYMBOL', 'arg2')
rbx=('EXPR_SYMBOL', 'initial_RBX')
rdx=('EXPR_SYMBOL', 'arg3')
rdi=('EXPR_SYMBOL', 'arg1')
rax=('EXPR_SYMBOL', 'arg1')

line=[imul      rax, rsi]
rcx=('EXPR_SYMBOL', 'arg4')
rsi=('EXPR_SYMBOL', 'arg2')
rbx=('EXPR_SYMBOL', 'initial_RBX')
rdx=('EXPR_SYMBOL', 'arg3')
rdi=('EXPR_SYMBOL', 'arg1')
rax=('EXPR_OP', '*', ('EXPR_SYMBOL', 'arg1'), ('EXPR_SYMBOL', 'arg2'))

...

result=('EXPR_OP', '*', ('EXPR_SYMBOL', 'arg1'), ('EXPR_SYMBOL', 'arg2'))
\end{lstlisting}

Инструкция IMUL связана со строкой ``*'', и затем новое выражение конструируется в
\TT{handle\_binary\_op()}, которая сохраняет результат в RAX.

В этом выводе, структуры данных выводятся используя Питоновскую ф-цию \TT{str()},
которая делает то же самое, что и \TT{print()}.

Вывод громоздкий, и мы можем выключить вывод Питоновских выражений и увидеть, как внутренняя структура данных аккуратно
выводится используя нашу внутреннюю ф-цию \TT{expr\_to\_string()}:

\begin{lstlisting}
python td.py --show-registers tests/mul.s

...

line=[mov       rax, rdi]
rcx=arg4
rsi=arg2
rbx=initial_RBX
rdx=arg3
rdi=arg1
rax=arg1

line=[imul      rax, rsi]
rcx=arg4
rsi=arg2
rbx=initial_RBX
rdx=arg3
rdi=arg1
rax=(arg1 * arg2)

...

result=(arg1 * arg2)
\end{lstlisting}

Более продвинутый пример:

\begin{lstlisting}
        imul    rdi, rsi
        lea     rax, [rdi+rdx]
\end{lstlisting}

Инструкция LEA обрабатывается просто как ADD.

\begin{lstlisting}
python td.py --show-registers --python-expr tests/mul_add.s

...

line=[imul      rdi, rsi]
rcx=('EXPR_SYMBOL', 'arg4')
rsi=('EXPR_SYMBOL', 'arg2')
rbx=('EXPR_SYMBOL', 'initial_RBX')
rdx=('EXPR_SYMBOL', 'arg3')
rdi=('EXPR_OP', '*', ('EXPR_SYMBOL', 'arg1'), ('EXPR_SYMBOL', 'arg2'))
rax=('EXPR_SYMBOL', 'initial_RAX')

line=[lea       rax, [rdi+rdx]]
rcx=('EXPR_SYMBOL', 'arg4')
rsi=('EXPR_SYMBOL', 'arg2')
rbx=('EXPR_SYMBOL', 'initial_RBX')
rdx=('EXPR_SYMBOL', 'arg3')
rdi=('EXPR_OP', '*', ('EXPR_SYMBOL', 'arg1'), ('EXPR_SYMBOL', 'arg2'))
rax=('EXPR_OP', '+', ('EXPR_OP', '*', ('EXPR_SYMBOL', 'arg1'), ('EXPR_SYMBOL', 'arg2')), ('EXPR_SYMBOL', 'arg3'))

...

result=('EXPR_OP', '+', ('EXPR_OP', '*', ('EXPR_SYMBOL', 'arg1'), ('EXPR_SYMBOL', 'arg2')), ('EXPR_SYMBOL', 'arg3'))
\end{lstlisting}

И снова, посмотрим, как это выражение может быть выведено более аккуратно:

\begin{lstlisting}
python td.py --show-registers tests/mul_add.s

...

result=((arg1 * arg2) + arg3)
\end{lstlisting}

Еще один пример, где используются 2 входных аргумента:

\begin{lstlisting}
        imul    rdi, rdi, 1234
        imul    rsi, rsi, 5678
        lea     rax, [rdi+rsi]
\end{lstlisting}

\begin{lstlisting}
python td.py --show-registers --python-expr tests/mul_add3.s

...

line=[imul      rdi, rdi, 1234]
rcx=('EXPR_SYMBOL', 'arg4')
rsi=('EXPR_SYMBOL', 'arg2')
rbx=('EXPR_SYMBOL', 'initial_RBX')
rdx=('EXPR_SYMBOL', 'arg3')
rdi=('EXPR_OP', '*', ('EXPR_SYMBOL', 'arg1'), ('EXPR_VALUE', 1234))
rax=('EXPR_SYMBOL', 'initial_RAX')

line=[imul      rsi, rsi, 5678]
rcx=('EXPR_SYMBOL', 'arg4')
rsi=('EXPR_OP', '*', ('EXPR_SYMBOL', 'arg2'), ('EXPR_VALUE', 5678))
rbx=('EXPR_SYMBOL', 'initial_RBX')
rdx=('EXPR_SYMBOL', 'arg3')
rdi=('EXPR_OP', '*', ('EXPR_SYMBOL', 'arg1'), ('EXPR_VALUE', 1234))
rax=('EXPR_SYMBOL', 'initial_RAX')

line=[lea       rax, [rdi+rsi]]
rcx=('EXPR_SYMBOL', 'arg4')
rsi=('EXPR_OP', '*', ('EXPR_SYMBOL', 'arg2'), ('EXPR_VALUE', 5678))
rbx=('EXPR_SYMBOL', 'initial_RBX')
rdx=('EXPR_SYMBOL', 'arg3')
rdi=('EXPR_OP', '*', ('EXPR_SYMBOL', 'arg1'), ('EXPR_VALUE', 1234))
rax=('EXPR_OP', '+', ('EXPR_OP', '*', ('EXPR_SYMBOL', 'arg1'), ('EXPR_VALUE', 1234)), ('EXPR_OP', '*', ('EXPR_SYMBOL', 'arg2'), ('EXPR_VALUE', 5678)))

...

result=('EXPR_OP', '+', ('EXPR_OP', '*', ('EXPR_SYMBOL', 'arg1'), ('EXPR_VALUE', 1234)), ('EXPR_OP', '*', ('EXPR_SYMBOL', 'arg2'), ('EXPR_VALUE', 5678)))
\end{lstlisting}

\dots и теперь аккуратный вывод:

\begin{lstlisting}
python td.py --show-registers tests/mul_add3.s

...

result=((arg1 * 1234) + (arg2 * 5678))
\end{lstlisting}

Now conversion program:

\begin{lstlisting}
        mov     rax, rdi
        sub     rax, 32
        imul    rax, 5
        mov     rbx, 9
        idiv    rbx
\end{lstlisting}

Вы можете увидеть, как состояние регистров меняется во время исполнения (или парсинга).

Сырое:

\lstinputlisting{toy_decompiler/fahr_raw.txt}

Аккуратное:

\lstinputlisting{toy_decompiler/fahr_neat.txt}

Интересно отметить, что инструкция IDIV вычисляет также остаток от деления, и он сохраняется в регистре RDX.
Он не используется, но доступен для использования.

Вот как частное и остаток сохраняются в регистрах:

\begin{lstlisting}
def handle_unary_DIV_IDIV (registers, op1):
    op1_expr=register_or_number_in_string_to_expr (registers, op1)
    current_RAX=registers["rax"]
    registers["rax"]=create_binary_expr ("/", current_RAX, op1_expr)
    registers["rdx"]=create_binary_expr ("%", current_RAX, op1_expr)
\end{lstlisting}

Теперь ф-ция \TT{align2grain()}\footnote{Взята отсюда: \url{https://docs.oracle.com/javase/specs/jvms/se6/html/Compiling.doc.html}}:

\begin{lstlisting}
        ; uint64_t align2grain (uint64_t i, uint64_t grain)
        ;    return ((i + grain-1) & ~(grain-1));

        ; rdi=i
        ; rsi=grain

        sub     rsi, 1
        add     rdi, rsi
        not     rsi
        and     rdi, rsi
        mov     rax, rdi
\end{lstlisting}

\lstinputlisting{toy_decompiler/align2grain.txt}

\subsection{Работа с оптимизациями компилятора}

Этот фрагмент кода \dots

\begin{lstlisting}
        mov     rax, rdi
        add     rax, rax
\end{lstlisting}

\dots будет трансформирован в выражение \textit{(arg1 + arg1)}.
Он может быть сокращен до \textit{(arg1 * 2)}.
Наш игрушечный декомпилятор может распознавать такие шаблонные образцы и переписывать их.

\begin{lstlisting}
# X+X -> X*2
def reduce_ADD1 (expr):
    if is_expr_op(expr) and get_op (expr)=="+" and get_op1 (expr)==get_op2 (expr):
        return dbg_print_reduced_expr ("reduce_ADD1", expr, create_binary_expr ("*", get_op1 (expr), create_val_expr (2)))

    return expr # no match
\end{lstlisting}

Эта ф-ция будет проверять, является имеет ли текущий узел тип \textit{EXPR\_OP},
операция ``+'' и оба потомка равны друг другу.
Кстати, так как наша структура данных это просто кортеж кортежей, Питон может сравнивать их используя обычную
операцию ``==''.
Если проверка закончена успешна, текущий узел затем заменяется новым выражением:
мы берем одного из потомков, мы конструируем узел типа \textit{EXPR\_VALUE} с числом ``2'' в нем,
и затем мы конструируем узел типа \textit{EXPR\_OP} со строкой ``*''.

\TT{dbg\_print\_reduced\_expr()} служит сугубо для отладочных целей --- он просто выводит старое и новое (сокращенное) выражения.

Декомпилятор затем проходит по дереву выражений в духе 
\textit{deep-first search (поиск в глубину)}.

\begin{lstlisting}
def reduce_step (e):
    if is_expr_op (e)==False:
        return e # expr isn't EXPR_OP, nothing to reduce (we don't reduce EXPR_SYMBOL and EXPR_VAL)

    if is_unary_op(get_op(e)):
        # recreate expr with reduced operand:
        return reducers(create_unary_expr (get_op(e), reduce_step (get_op1 (e))))
    else:
        # recreate expr with both reduced operands:
        return reducers(create_binary_expr (get_op(e), reduce_step (get_op1 (e)), reduce_step (get_op2 (e))))

...


# same as "return ...(reduce_MUL1 (reduce_ADD1 (reduce_ADD2 (... expr))))"
reducers=compose([
	...
    reduce_ADD1, ...
    ...])

def reduce (e):
    print "going to reduce " + expr_to_string (e)
    new_expr=reduce_step(e)
    if new_expr==e:
        return new_expr # we are done here, expression can't be reduced further
    else:
        return reduce(new_expr) # reduced expr has been changed, so try to reduce it again
\end{lstlisting}

Ф-ции сокращения (или редукции) вызываются снова и снова, пока выражение не перестанет меняться.

Запускаем:

\begin{lstlisting}
python td.py tests/add1.s

...

going to reduce (arg1 + arg1)
reduction in reduce_ADD1() (arg1 + arg1) -> (arg1 * 2)
going to reduce (arg1 * 2)
result=(arg1 * 2)
\end{lstlisting}

Пока всё хорошо, но что если мы попробуем этот фрагмент кода?

\begin{lstlisting}
        mov     rax, rdi
        add     rax, rax
        add     rax, rax
        add     rax, rax
\end{lstlisting}

\begin{lstlisting}
python td.py tests/add2.s

...

working out tests/add2.s
going to reduce (((arg1 + arg1) + (arg1 + arg1)) + ((arg1 + arg1) + (arg1 + arg1)))
reduction in reduce_ADD1() (arg1 + arg1) -> (arg1 * 2)
reduction in reduce_ADD1() (arg1 + arg1) -> (arg1 * 2)
reduction in reduce_ADD1() ((arg1 * 2) + (arg1 * 2)) -> ((arg1 * 2) * 2)
reduction in reduce_ADD1() (arg1 + arg1) -> (arg1 * 2)
reduction in reduce_ADD1() (arg1 + arg1) -> (arg1 * 2)
reduction in reduce_ADD1() ((arg1 * 2) + (arg1 * 2)) -> ((arg1 * 2) * 2)
reduction in reduce_ADD1() (((arg1 * 2) * 2) + ((arg1 * 2) * 2)) -> (((arg1 * 2) * 2) * 2)
going to reduce (((arg1 * 2) * 2) * 2)
result=(((arg1 * 2) * 2) * 2)
\end{lstlisting}

Это корректно, но слишком многословно.

Нам нужно переписать выражение \textit{(X*n)*m} в \textit{X*(n*m)}, где $n$ и $m$ это числа.
Мы можем сделать это добавлением еще одной ф-ции, как \TT{reduce\_ADD1()}, но есть способ лучше:
мы можем использовать матчер для дерева.
Вы можете думать о нем как о матчере регулярных выражений, но работающем на деревьях.

\begin{lstlisting}
def bind_expr (key):
    return ("EXPR_WILDCARD", key)

def bind_value (key):
    return ("EXPR_WILDCARD_VALUE", key)

def match_EXPR_WILDCARD (expr, pattern):
    return {pattern[1] : expr} # return {key : expr}

def match_EXPR_WILDCARD_VALUE (expr, pattern):
    if get_expr_type (expr)!="EXPR_VALUE":
        return None
    return {pattern[1] : get_val(expr)} # return {key : expr}

def is_commutative (op):
    return op in ["+", "*", "&", "|", "^"]

def match_two_ops (op1_expr, op1_pattern, op2_expr, op2_pattern):
    m1=match (op1_expr, op1_pattern)
    m2=match (op2_expr, op2_pattern)
    if m1==None or m2==None:
        return None # one of match for operands returned False, so we do the same
    # join two dicts from both operands:
    rt={}
    rt.update(m1)
    rt.update(m2)
    return rt

def match_EXPR_OP (expr, pattern):
    if get_expr_type(expr)!=get_expr_type(pattern): # be sure, both EXPR_OP.
        return None
    if get_op (expr)!=get_op (pattern): # be sure, ops type are the same.
        return None

    if (is_unary_op(get_op(expr))):
        # match unary expression.
        return match (get_op1 (expr), get_op1 (pattern))
    else:     
        # match binary expression.     

        # first try match operands as is.
        m=match_two_ops (get_op1 (expr), get_op1 (pattern), get_op2 (expr), get_op2 (pattern))
        if m!=None:
            return m
        # if matching unsuccessful, AND operation is commutative, try also swapped operands.
        if is_commutative (get_op (expr))==False:
            return None
        return match_two_ops (get_op1 (expr), get_op2 (pattern), get_op2 (expr), get_op1 (pattern))

# returns dict in case of success, or None
def match (expr, pattern):
    t=get_expr_type(pattern)
    if t=="EXPR_WILDCARD":
        return match_EXPR_WILDCARD (expr, pattern)
    elif t=="EXPR_WILDCARD_VALUE":
        return match_EXPR_WILDCARD_VALUE (expr, pattern)
    elif t=="EXPR_SYMBOL":
        if expr==pattern:
            return {}
        else:
            return None
    elif t=="EXPR_VALUE":
        if expr==pattern:
            return {}
        else:
            return None
    elif t=="EXPR_OP":
        return match_EXPR_OP (expr, pattern)
    else:
        raise AssertionError
\end{lstlisting}

Теперь будем использовать его:

\begin{lstlisting}
# (X*A)*B -> X*(A*B)
def reduce_MUL1 (expr):
    m=match (expr, create_binary_expr ("*", (create_binary_expr ("*", bind_expr ("X"), bind_value ("A"))), bind_value ("B")))
    if m==None:
        return expr # no match

    return dbg_print_reduced_expr ("reduce_MUL1", expr, create_binary_expr ("*", 
        m["X"], # new op1
        create_val_expr (m["A"] * m["B"]))) # new op2
\end{lstlisting}

Мы берем входное выражение, и мы также конструируем образец, с которым будет происходить сличение.
Матчер работает рекурсивно над обоими выражениеми, синхронно.
Образец это тоже выражение, но мы можем использовать два дополнительных типа узла: \textit{EXPR\_WILDCARD} и
\textit{EXPR\_WILDCARD\_VALUE}. Эти узлы используются с ключами (хранящимися как строки).
Если матчер встречает \textit{EXPR\_WILDCARD} в образце, он просто сохраняет текущее выражение и вернет его.\\
Если матчер встречает \textit{EXPR\_WILDCARD\_VALUE}, он делает то же самое, но только если текущий узел имеет тип
\textit{EXPR\_VALUE}.

Ф-ции \TT{bind\_expr()} и \TT{bind\_value()} создают узлы с только что описанными типами.

Всё это значит, что ф-ция \TT{reduce\_MUL1()} будет искать выражение вида \textit{(X*A)*B}, где $A$ и $B$
это числа. В остальных случаях, матчер вернет входное выражение неизменным, так что эти ф-ции редукции можно использовать
сцепленными друг с другом.

Теперь, когда \TT{reduce\_MUL1()} встречает интересное нам (под)выражение,
она вернет словарь с ключами и выражениями.
Добавим вызов \TT{print m} где-нибудь перед возвратом, и запустим снова:

\begin{lstlisting}
python td.py tests/add2.s

...

going to reduce (((arg1 + arg1) + (arg1 + arg1)) + ((arg1 + arg1) + (arg1 + arg1)))
reduction in reduce_ADD1() (arg1 + arg1) -> (arg1 * 2)
reduction in reduce_ADD1() (arg1 + arg1) -> (arg1 * 2)
reduction in reduce_ADD1() ((arg1 * 2) + (arg1 * 2)) -> ((arg1 * 2) * 2)
{'A': 2, 'X': ('EXPR_SYMBOL', 'arg1'), 'B': 2}
reduction in reduce_MUL1() ((arg1 * 2) * 2) -> (arg1 * 4)
reduction in reduce_ADD1() (arg1 + arg1) -> (arg1 * 2)
reduction in reduce_ADD1() (arg1 + arg1) -> (arg1 * 2)
reduction in reduce_ADD1() ((arg1 * 2) + (arg1 * 2)) -> ((arg1 * 2) * 2)
{'A': 2, 'X': ('EXPR_SYMBOL', 'arg1'), 'B': 2}
reduction in reduce_MUL1() ((arg1 * 2) * 2) -> (arg1 * 4)
reduction in reduce_ADD1() ((arg1 * 4) + (arg1 * 4)) -> ((arg1 * 4) * 2)
{'A': 4, 'X': ('EXPR_SYMBOL', 'arg1'), 'B': 2}
reduction in reduce_MUL1() ((arg1 * 4) * 2) -> (arg1 * 8)
going to reduce (arg1 * 8)
...
result=(arg1 * 8)
\end{lstlisting}

Словарь имеет ключи, которые мы передали, плюс найденные выражения.
Мы теперь можем их использовать для создания нового выражения и возврата его.
Числа просто суммируются во время формирования второго операнда операции ``*''.

Теперь пример оптимизации из реальности --- оптимизирующий GCC заменяет умножение на 31 используя операции сдвига
и вычитания:

\begin{lstlisting}
        mov     rax, rdi
        sal     rax, 5
        sub     rax, rdi
\end{lstlisting}

Без ф-ций редукции, наш декомпилятор преобразует это в \textit{((arg1 << 5) - arg1)}.
Мы можем заменить сдвиг влево на умножение:

\begin{lstlisting}
# X<<n -> X*(2^n)
def reduce_SHL1 (expr):
    m=match (expr, create_binary_expr ("<<", bind_expr ("X"), bind_value ("Y")))
    if m==None:
        return expr # no match
    
    return dbg_print_reduced_expr ("reduce_SHL1", expr, create_binary_expr ("*", m["X"], create_val_expr (1<<m["Y"])))
\end{lstlisting}

Получаем \textit{((arg1 * 32) - arg1)}.
Мы можем добавить еще одну ф-цию редукции:

\begin{lstlisting}
# (X*n)-X -> X*(n-1)
def reduce_SUB3 (expr):
    m=match (expr, create_binary_expr ("-",
        create_binary_expr ("*", bind_expr("X1"), bind_value ("N")),
        bind_expr("X2")))
    
    if m!=None and match (m["X1"], m["X2"])!=None:
        return dbg_print_reduced_expr ("reduce_SUB3", expr, create_binary_expr ("*", m["X1"], create_val_expr (m["N"]-1)))
    else:
        return expr # no match
\end{lstlisting}

Матчер вернет два X-а, и мы должны быть уверены, что они равны.
На самом деле, в предыдущих версиях этого игрушечного декомпилятора, я сравнивал при помощи простой ``=='', и это работало.
Но мы можем и здесь использовать ф-цию \TT{match()} для тех же целей, потому что она лучше обрабатывает коммутативные операции.
Например, если X1 это ``Q+1'' и X2 это ``1+Q'', выражения равны, но простой ``=='' не сработает.
С другой стороны, ф-ция \TT{match()}, если встретит операцию ``+'' (или другую коммутативную операцию),
и сравнение не сработает, она попробует также поменять операнды местами и сравнит снова.

Хотя, чтобы понимать всё это проще, на время, вы можете представить, что здесь вместо второй \TT{match()} просто ``==''.

Так или иначе, вот что получаем:

\begin{lstlisting}
working out tests/mul31_GCC.s
going to reduce ((arg1 << 5) - arg1)
reduction in reduce_SHL1() (arg1 << 5) -> (arg1 * 32)
reduction in reduce_SUB3() ((arg1 * 32) - arg1) -> (arg1 * 31)
going to reduce (arg1 * 31)
...
result=(arg1 * 31)
\end{lstlisting}

Еще одна техника оптимизации, которую часто может увидеть в ARM thumb-коде: применение И к значению со значением вроде
0xFFFFFFF0, реализуется при помощи сдвигов:

\begin{lstlisting}
        mov rax, rdi
        shr rax, 4
        shl rax, 4
\end{lstlisting}

Такой код часто встречается в ARM thumb-коде, потому что кодировать 32-битные константы используя пару 16-битных инструкций
это тяжело, тогда как при помощи одной 16-битной инструкции можно сдвигать на 4 бита влево или вправо.

Также, выражение \textit{(x>>4)<<4} можно в шутку назвать ``оператор подергивания'':
я слышал как так называют выражение ``-{}-i++'' в русскоязычных соц.сетях, это было что-то вроде мема.

Так или иначе, вот эти ф-ции для сокращения будут использоваться:

\begin{lstlisting}
# X>>n -> X / (2^n)
...
def reduce_SHR2 (expr):
    m=match(expr, create_binary_expr(">>", bind_expr("X"), bind_value("Y")))
    if m==None or m["Y"]>=64:
        return expr # no match

    return dbg_print_reduced_expr ("reduce_SHR2", expr, create_binary_expr ("/", m["X"],
        create_val_expr (1<<m["Y"])))

...

# X<<n -> X*(2^n)
def reduce_SHL1 (expr):
    m=match (expr, create_binary_expr ("<<", bind_expr ("X"), bind_value ("Y")))
    if m==None:
        return expr # no match
    
    return dbg_print_reduced_expr ("reduce_SHL1", expr, create_binary_expr ("*", m["X"], create_val_expr (1<<m["Y"])))

...

# FIXME: slow
# returns True if n=2^x or popcnt(n)=1
def is_2n(n):
    return bin(n).count("1")==1


# AND operation using DIV/MUL or SHL/SHR
# (X / (2^n)) * (2^n) -> X&(~((2^n)-1))
def reduce_MUL2 (expr):
    m=match(expr, create_binary_expr ("*", create_binary_expr ("/", bind_expr("X"), bind_value("N1")), bind_value("N2")))
    if m==None or m["N1"]!=m["N2"] or is_2n(m["N1"])==False: # short-circuit expression
        return expr # no match

    return dbg_print_reduced_expr("reduce_MUL2", expr, create_binary_expr ("&", m["X"],
        create_val_expr(~(m["N1"]-1)&0xffffffffffffffff)))
\end{lstlisting}

И вот результат:

\begin{lstlisting}
working out tests/AND_by_shifts2.s
going to reduce ((arg1 >> 4) << 4)
reduction in reduce_SHR2() (arg1 >> 4) -> (arg1 / 16)
reduction in reduce_SHL1() ((arg1 / 16) << 4) -> ((arg1 / 16) * 16)
reduction in reduce_MUL2() ((arg1 / 16) * 16) -> (arg1 & 0xfffffffffffffff0)
going to reduce (arg1 & 0xfffffffffffffff0)
...
result=(arg1 & 0xfffffffffffffff0)
\end{lstlisting}

\subsubsection{Деление используя умножение}

Деление часто заменяется умножением, ради лучшей производительности.

Из школьной арифметики, мы можем вспомнить, что деление на 9 может быть заменено на умножение на $\frac{1}{9}$.
На самом деле, для чисел с плавающей точкой, иногда компиляторы так и делают,
например, инструкциея FDIV в x86-коде может быть заменена на FMUL.
По крайней мере MSVC 6.0 заменяет деление на 9 на умножение на $0.111111...$ и иногда нельзя быть уверенным в том,
какая операция была в оригинальном исходном коде.

Но когда мы работаем с целочисленными значениями и целочисленными регистрами CPU, мы не можем использовать дроби.
Но мы можем переписать дробь так:

\begin{center}
{\large $result = \frac{x}{3} = x \cdot \frac{1}{3} = x \cdot \frac{1 \cdot MagicNumber}{3 \cdot MagicNumber}$}
\end{center}

Учитывая тот факт, что деление на $2^n$ очень быстро (при помощи сдвигов), теперь нам нужно найти такой $MagicNumber$,
для которого следующее уравнение будет справедливо: $2^n = 9 \cdot MagicNumber$.

Этот код делит на 10:

\begin{lstlisting}
        mov     rax, rdi
        movabs  rdx, 0cccccccccccccccdh
        mul     rdx
        shr     rdx, 3
        mov     rax, rdx
\end{lstlisting}

Деление на $2^{64}$ в каком-то смысле скрыто: младшие 64 бита произведение в RAX не используются (выкидываются),
только старшие 64 бита произведения (в RDX) используются и затем сдвигаются еще на 3 бита.

Регистр RDX выставляется в течении обработки MUL/IMUL вот так:

\begin{lstlisting}
def handle_unary_MUL_IMUL (registers, op1):
    op1_expr=register_or_number_in_string_to_expr (registers, op1)
    result=create_binary_expr ("*", registers["rax"], op1_expr)
    registers["rax"]=result
    registers["rdx"]=create_binary_expr (">>", result, create_val_expr(64))
\end{lstlisting}

Другими словами, только что увиденный код на ассемблере умножает на {\Large $\frac{0cccccccccccccccdh}{2^{64+3}}$},
или делит на {\Large $\frac{2^{64+3}}{0cccccccccccccccdh}$}.
Чтобы найти делитель, нужно просто разделить числитель на знаменатель.

\begin{lstlisting}
# n = magic number
# m = shifting coefficient
# return = 1 / (n / 2^m) = 2^m / n
def get_divisor (n, m):
    return (2**float(m))/float(n)

# (X*n)>>m, where m>=64 -> X/...
def reduce_div_by_MUL (expr):
    m=match (expr, create_binary_expr(">>", create_binary_expr ("*", bind_expr("X"), bind_value("N")), bind_value("M")))
    if m==None:
        return expr # no match
    
    divisor=get_divisor(m["N"], m["M"])
    return dbg_print_reduced_expr ("reduce_div_by_MUL", expr, create_binary_expr ("/", m["X"], create_val_expr (int(divisor))))
\end{lstlisting}

Это работает, но у нас проблема: это правило берет в начале выражение \textit{(arg1 * 0xcccccccccccccccd) >> 64} и находит,
что делитель равен $1.25$.
Это верно: результат сдвигается на 3 бита позже (или делится на 8), и $1.25 \cdot 8 = 10$.
Но наш игрушечный декомпилятор не поддерживает вещественные числа.

Мы можем решить эту проблему так: если делитель имеет дробную часть, мы откладываем сокращение, в надежде,
что две последовательные операции сдвига вправо будут в начале сокращены в одну:

\begin{lstlisting}
# (X*n)>>m, where m>=64 -> X/...
def reduce_div_by_MUL (expr):
    m=match (expr, create_binary_expr(">>", create_binary_expr ("*", bind_expr("X"), bind_value("N")), bind_value("M")))
    if m==None:
        return expr # no match
    
    divisor=get_divisor(m["N"], m["M"])
    if math.floor(divisor)==divisor:
        return dbg_print_reduced_expr ("reduce_div_by_MUL", expr, create_binary_expr ("/", m["X"], create_val_expr (int(divisor))))
    else:
        print "reduce_div_by_MUL(): postponing reduction, because divisor=", divisor
        return expr
\end{lstlisting}

Это работает:

\begin{lstlisting}
working out tests/div_by_mult10_unsigned.s
going to reduce (((arg1 * 0xcccccccccccccccd) >> 64) >> 3)
reduce_div_by_MUL(): postponing reduction, because divisor= 1.25
reduction in reduce_SHR1() (((arg1 * 0xcccccccccccccccd) >> 64) >> 3) -> ((arg1 * 0xcccccccccccccccd) >> 67)
going to reduce ((arg1 * 0xcccccccccccccccd) >> 67)
reduction in reduce_div_by_MUL() ((arg1 * 0xcccccccccccccccd) >> 67) -> (arg1 / 10)
going to reduce (arg1 / 10)
result=(arg1 / 10)
\end{lstlisting}

Не знаю, наилучшее ли это решение. В ранней версии этого декомпилятора, он делал два прохода:
первый проход для всего кроме деления через умножение, и второй проход вместе с последним сокращением.
Не знаю, как лучше.
Может быть, мы могли бы добавить поддержку вещественных чисел в выражениях?

Еще кое что для лучшего понимания.
Многие люди не замечают ``скрытое'' деление на $2^{32}$ или $2^{64}$,
когда младшая 32-битная часть произведения (или 64-битная) не используется.
Также, имеется недоразумение, что здесь используется обратное число по модулю.
Это близко, но не то же самое.
Для поиска \textit{магического коэффициента}, часто используется расширенный алгоритм Эвклида, но на самом деле,
этот алгоритм используется для решения уравнения.
Вы можете решать его его любым другим методом.
Так или иначе, расширенный алгоритм Эвклида наверное самый эффективный метод решения.
Также, нужно упомянуть, что уравнение не решаемо для некоторых делителей, потому что это диофантово уравнение
(т.е., уравнение в котором результат может быть только целым числом), так как, все же, мы работаем с целочисленными
регистрами CPU.

\subsection{Обфускация/деобфускация}

Не смотря на простоту нашего декомпилятора, мы можем увидеть, как деобфусцировать (или оптимизировать) используя
несколько простых трюков.

Например, этот фрагмент кода ничего не делает:

\begin{lstlisting}
        mov rax, rdi
        xor rax, 12345678h
        xor rax, 0deadbeefh
        xor rax, 12345678h
        xor rax, 0deadbeefh
\end{lstlisting}

Чтобы разобраться с этим, нам нужны такие правила:

\begin{lstlisting}
# (X^n)^m -> X^(n^m)
def reduce_XOR4 (expr):
    m=match(expr, 
        create_binary_expr("^",
            create_binary_expr ("^", bind_expr("X"), bind_value("N")),
                bind_value("M")))
    
    if m!=None:
        return dbg_print_reduced_expr ("reduce_XOR4", expr, create_binary_expr ("^", m["X"], 
            create_val_expr (m["N"]^m["M"])))
    else:
        return expr # no match

...

# X op 0 -> X, where op is ADD, OR, XOR, SUB
def reduce_op_0 (expr):
    # try each:
    for op in ["+", "|", "^", "-"]:
        m=match(expr, create_binary_expr(op, bind_expr("X"), create_val_expr (0)))
        if m!=None:
            return dbg_print_reduced_expr ("reduce_op_0", expr, m["X"])

    # default:
    return expr # no match
\end{lstlisting}

\begin{lstlisting}
working out tests/t9_obf.s
going to reduce ((((arg1 ^ 0x12345678) ^ 0xdeadbeef) ^ 0x12345678) ^ 0xdeadbeef)
reduction in reduce_XOR4() ((arg1 ^ 0x12345678) ^ 0xdeadbeef) -> (arg1 ^ 0xcc99e897)
reduction in reduce_XOR4() ((arg1 ^ 0xcc99e897) ^ 0x12345678) -> (arg1 ^ 0xdeadbeef)
reduction in reduce_XOR4() ((arg1 ^ 0xdeadbeef) ^ 0xdeadbeef) -> (arg1 ^ 0x0)
going to reduce (arg1 ^ 0x0)
reduction in reduce_op_0() (arg1 ^ 0x0) -> arg1
going to reduce arg1
result=arg1
\end{lstlisting}

Такой фрагмент кода может быть также деобфусцирован (или оптимизирован):

\begin{lstlisting}
; toggle last bit

        mov rax, rdi
        mov rbx, rax
        mov rcx, rbx
        mov rsi, rcx
        xor rsi, 12345678h
        xor rsi, 12345679h
        mov rax, rsi
\end{lstlisting}

\begin{lstlisting}
working out tests/t7_obf.s
going to reduce ((arg1 ^ 0x12345678) ^ 0x12345679)
reduction in reduce_XOR4() ((arg1 ^ 0x12345678) ^ 0x12345679) -> (arg1 ^ 1)
going to reduce (arg1 ^ 1)
result=(arg1 ^ 1)
\end{lstlisting}

Я также использовал супероптимизатор \textit{aha!}\footnote{\url{http://www.hackersdelight.org/aha/aha.pdf}}
для поиска очень странного кода, который ничего не делает.

\textit{Aha!} это так называемый супероптимизатор, перебирает брутфорсом разные фрагменты кода, в попытке найти
самую короткую альтернативу некоторой математической операции.
В то время как разумные разработчики компиляторов используют для этой задачи супероптимизаторы, я пробовал его
использовал для обратного, находить самые странные фрагменты кода для простых операций, включая \ac{NOP}.
В прошлом, я использовал его для нахождения труднопонимаемой альтернативы операции исключающего ИЛИ (\ref{weird_XOR}).

И вот что \textit{aha!} нашел для \ac{NOP}:

\begin{lstlisting}
; do nothing (as found by aha)

        mov rax, rdi
        and rax, rax
        or rax, rax
\end{lstlisting}

\begin{lstlisting}
# X & X -> X
def reduce_AND3 (expr):
    m=match (expr, create_binary_expr ("&", bind_expr ("X1"), bind_expr ("X2")))
    if m!=None and match (m["X1"], m["X2"])!=None:
        return dbg_print_reduced_expr("reduce_AND3", expr, m["X1"])
    else:
        return expr # no match

...

# X | X -> X
def reduce_OR1 (expr):
    m=match (expr, create_binary_expr ("|", bind_expr ("X1"), bind_expr ("X2")))
    if m!=None and match (m["X1"], m["X2"])!=None:
        return dbg_print_reduced_expr("reduce_OR1", expr, m["X1"])
    else:
        return expr # no match
\end{lstlisting}

\begin{lstlisting}
working out tests/t11_obf.s
going to reduce ((arg1 & arg1) | (arg1 & arg1))
reduction in reduce_AND3() (arg1 & arg1) -> arg1
reduction in reduce_AND3() (arg1 & arg1) -> arg1
reduction in reduce_OR1() (arg1 | arg1) -> arg1
going to reduce arg1
result=arg1
\end{lstlisting}

Это менее понятно:

\begin{lstlisting}
; do nothing (as found by aha)

;Found a 5-operation program:
;   neg   r1,rx
;   neg   r2,rx
;   neg   r3,r1
;   or    r4,rx,2
;   and   r5,r4,r3
;   Expr: ((x | 2) & -(-(x)))

        mov rax, rdi
        neg rax
        neg rax
        or rdi, 2
        and rax, rdi
\end{lstlisting}

Добавленные правила (я использовал строку ``NEG'' для обозначения смены знака, чтобы было отличие от операции вычитания,
которая просто минус (``-'')):

\label{AND2}
\begin{lstlisting}
# (op(op X)) -> X, where both ops are NEG or NOT
def reduce_double_NEG_or_NOT (expr):
    # try each:
    for op in ["NEG", "~"]:
        m=match (expr, create_unary_expr (op, create_unary_expr (op, bind_expr("X"))))
        if m!=None:
            return dbg_print_reduced_expr ("reduce_double_NEG_or_NOT", expr, m["X"])

    # default:
    return expr # no match

...

# X & (X | ...) -> X
def reduce_AND2 (expr):
    m=match (expr, create_binary_expr ("&", create_binary_expr ("|", bind_expr ("X1"), bind_expr ("REST")), bind_expr ("X2")))
    if m!=None and match (m["X1"], m["X2"])!=None:
        return dbg_print_reduced_expr("reduce_AND2", expr, m["X1"])
    else:
        return expr # no match
\end{lstlisting}

\begin{lstlisting}
going to reduce ((-(-arg1)) & (arg1 | 2))
reduction in reduce_double_NEG_or_NOT() (-(-arg1)) -> arg1
reduction in reduce_AND2() (arg1 & (arg1 | 2)) -> arg1
going to reduce arg1
result=arg1
\end{lstlisting}

Я также заставил \textit{aha!} найти код, который прибавляет 2 без использования операций сложения/вычитания:

\begin{lstlisting}
; arg1+2, without add/sub allowed, as found by aha:

;Found a 4-operation program:
;   not   r1,rx
;   neg   r2,r1
;   not   r3,r2
;   neg   r4,r3
;   Expr: -(~(-(~(x))))

        mov     rax, rdi
        not     rax
        neg     rax
        not     rax
        neg     rax
\end{lstlisting}

Правило:

\begin{lstlisting}
# (- (~X)) -> X+1
def reduce_NEG_NOT (expr):
    m=match (expr, create_unary_expr ("NEG", create_unary_expr ("~", bind_expr("X"))))
    if m==None:
        return expr # no match
    
    return dbg_print_reduced_expr ("reduce_NEG_NOT", expr, create_binary_expr ("+", m["X"],create_val_expr(1)))
\end{lstlisting}

\begin{lstlisting}
working out tests/add_by_not_neg.s
going to reduce (-(~(-(~arg1))))
reduction in reduce_NEG_NOT() (-(~arg1)) -> (arg1 + 1)
reduction in reduce_NEG_NOT() (-(~(arg1 + 1))) -> ((arg1 + 1) + 1)
reduction in reduce_ADD3() ((arg1 + 1) + 1) -> (arg1 + 2)
going to reduce (arg1 + 2)
result=(arg1 + 2)
\end{lstlisting}

Это артефакт системы представления знаковых чисел (дополнительный код).
То же самое можно сделать и для вычитания (просто поменяйте местами операции NEG и NOT).

Теперь добавим фальшивый багаж в пример конвертирования из шкалы Фаренгейта в шкалу Цельсия:

\begin{lstlisting}
        ; celsius = 5 * (fahr-32) / 9
        ; fake luggage:
        mov     rbx, 12345h
        mov     rax, rdi
        sub     rax, 32
        ; fake luggage:
        add     rbx, rax
        imul    rax, 5
        mov     rbx, 9
        idiv    rbx
        ; fake luggage:
        sub     rdx, rax
\end{lstlisting}

Это не проблема для нашего декомпилятора, потому что шум оставшийся в регистре RDX не используется вообще:

\lstinputlisting{toy_decompiler/fahr_to_celsius_obf1.txt}

Можем сделать вид, что на результат влияет этот шум:

\begin{lstlisting}
        ; celsius = 5 * (fahr-32) / 9
        ; fake luggage:
        mov     rbx, 12345h
        mov     rax, rdi
        sub     rax, 32
        ; fake luggage:
        add     rbx, rax
        imul    rax, 5
        mov     rbx, 9
        idiv    rbx
        ; fake luggage:
        sub     rdx, rax
        mov     rcx, rax
        ; OR result with garbage (result of fake luggage):
        or      rcx, rdx
        ; the following instruction shouldn't affect result:
        and     rax, rcx
\end{lstlisting}

\dots но на самом деле, всё это сокращается ф-цией \TT{reduce\_AND2()}, которую мы уже видели (\ref{AND2}):

\begin{lstlisting}
working out tests/fahr_to_celsius_obf2.s
going to reduce ((((arg1 - 32) * 5) / 9) & ((((arg1 - 32) * 5) / 9) | ((((arg1 - 32) * 5) % 9) - (((arg1 - 32) * 5) / 9))))
reduction in reduce_AND2() ((((arg1 - 32) * 5) / 9) & ((((arg1 - 32) * 5) / 9) | ((((arg1 - 32) * 5) % 9) - (((arg1 - 32) * 5)
/ 9)))) -> (((arg1 - 32) * 5) / 9)
going to reduce (((arg1 - 32) * 5) / 9)
result=(((arg1 - 32) * 5) / 9)
\end{lstlisting}

Мы можем увидеть, что деобфускация на самом деле это то же что и оптимизация, использующаяся в компиляторах.
Попробуем эту ф-цию в GCC:

\begin{lstlisting}
int f(int a)
{
	return -(~a);
};
\end{lstlisting}

Оптимизирующий GCC 5.4 (x86) генерирует это:

\begin{lstlisting}
f:
        mov     eax, DWORD PTR [esp+4]
        add     eax, 1
        ret
\end{lstlisting}

GCC имеет свои правила для переписывания, некоторые из которых, вероятно, близки к тем, что мы здесь используем.

\subsection{Тесты}

Не смотря на простоту декомпилятора, ошибок избежать трудно.
Нужно удостоверится, что изначальное выражение и сокращенное эквивалентны друг другу.

\subsubsection{Вычисление выражения}

В начале мы вычислим (или \textit{запустим}, или \textit{исполним}) выражение со случайными значениями в аргументах,
и затем сравним результаты.

Калькулятор (\textit{evaluator}) применяет арифметические операции, когда это возможно, рекурсивно.
Когда встречается символ, его значение (которое перед этим было сгенерировано случайно) берется из таблицы.

\begin{lstlisting}
un_ops={"NEG":operator.neg,
        "~":operator.invert}

bin_ops={">>":operator.rshift,
        "<<":(lambda x, c: x<<(c&0x3f)), # operator.lshift should be here, but it doesn't handle too big counts
        "&":operator.and_,
        "|":operator.or_,
        "^":operator.xor,
        "+":operator.add,
        "-":operator.sub,
        "*":operator.mul,
        "/":operator.div,
        "%":operator.mod}

def eval_expr(e, symbols):
    t=get_expr_type (e)
    if t=="EXPR_SYMBOL":
        return symbols[get_symbol(e)]
    elif t=="EXPR_VALUE":
        return get_val (e)
    elif t=="EXPR_OP":
        if is_unary_op (get_op (e)):
            return un_ops[get_op(e)](eval_expr(get_op1(e), symbols))
        else:
            return bin_ops[get_op(e)](eval_expr(get_op1(e), symbols), eval_expr(get_op2(e), symbols))
    else:
        raise AssertionError

def do_selftest(old, new):
    for n in range(100):
        symbols={"arg1":random.getrandbits(64), 
                "arg2":random.getrandbits(64), 
                "arg3":random.getrandbits(64), 
                "arg4":random.getrandbits(64)}
        old_result=eval_expr (old, symbols)&0xffffffffffffffff # signed->unsigned
        new_result=eval_expr (new, symbols)&0xffffffffffffffff # signed->unsigned
        if old_result!=new_result:
            print "self-test failed"
            print "initial expression: "+expr_to_string(old)
            print "reduced expression: "+expr_to_string(new)
            print "initial expression result: ", old_result
            print "reduced expression result: ", new_result
            exit(0)
\end{lstlisting}

На самом деле, это очень близко к тому, что делает ф-ция \textit{EVAL} в LISP-е, или даже к интерпретатору LISP.
Впрочем, не все символы выставлены.
Если выражение использует изначальные значения из RAX или RBX
(которым присвоены символы ``initial\_RAX'' и ``initial\_RBX'',
декомпилятор остановится с исключением, потому что этим регистрам не присвоены никакие случайные значения,
и эти символы отсутствуют в словаре \textit{symbols}.

Используя этот тест, я внезапно нашел тут ошибку (не смотря на простоту всех этих правил для сокращения).
Ну, никто не защищен от усталости глазных мышц.
Тем не мнее, у теста есть серьезная проблема: некоторые ошибки могут обнаружится только если один из аргументов это
$0$, или $1$, или $-1$.
Может быть существуют даже еще более специальные случаи.

Вышеуказанный супероптимизатор \textit{aha!} пробует по крайней мере эти значения в аргументах во время тестирования:
1, 0, -1, 0x80000000, 0x7FFFFFFF, 0x80000001, 0x7FFFFFFE, 0x01234567, 0x89ABCDEF, -2, 2, -3, 3,
-64, 64, -5, -31415.

Но всё-таки, до конца уверенным быть нельзя.

\subsubsection{Использование Z3 \ac{SMT}-солвера для тестирования}

Так что здесь можем попробовать Z3 \ac{SMT}-солвер.
SMT-солвер может \textit{доказать} что два выражения эквивалентны друг другу.

Например, при помощи \textit{aha!}, я нашел еще один престранный фрагмент кода, который ничего не делает:

\begin{lstlisting}
; do nothing (obfuscation)

;Found a 5-operation program:
;   neg   r1,rx
;   neg   r2,r1
;   sub   r3,r1,3
;   sub   r4,r3,r1
;   sub   r5,r4,r3
;   Expr: (((-(x) - 3) - -(x)) - (-(x) - 3))

        mov rax, rdi
        neg rax
        mov rbx, rax
        ; rbx=-x
        mov rcx, rbx
        sub rcx, 3
        ; rcx=-x-3
        mov rax, rcx
        sub rax, rbx
        ; rax=(-(x) - 3) - -(x)
        sub rax, rcx
\end{lstlisting}

Используя игрушечный декомпилятор, я обнаружил что этот фрагмент может быть сокращен до выражения \textit{arg1}:

\begin{lstlisting}
working out tests/t5_obf.s
going to reduce ((((-arg1) - 3) - (-arg1)) - ((-arg1) - 3))
reduction in reduce_SUB2() ((-arg1) - 3) -> (-(arg1 + 3))
reduction in reduce_SUB5() ((-(arg1 + 3)) - (-arg1)) -> ((-(arg1 + 3)) + arg1)
reduction in reduce_SUB2() ((-arg1) - 3) -> (-(arg1 + 3))
reduction in reduce_ADD_SUB() (((-(arg1 + 3)) + arg1) - (-(arg1 + 3))) -> arg1
going to reduce arg1
result=arg1
\end{lstlisting}

Но корректно ли это?
Я добавил ф-цию для вывода выражения в формат SMT-LIB, она такая же простая, как и та ф-ция, что конвертирует
выражение в строку.

И это файл SMT-LIB для Z3:

\begin{lstlisting}
(assert
    (forall ((arg1 (_ BitVec 64)) (arg2 (_ BitVec 64)) (arg3 (_ BitVec 64)) (arg4 (_ BitVec 64)))
        (=
            (bvsub (bvsub (bvsub (bvneg arg1) #x0000000000000003) (bvneg arg1)) (bvsub (bvneg arg1) #x0000000000000003))
            arg1
        )
    )
)
(check-sat)
\end{lstlisting}

В терминах бытового русского языка, мы спрашиваем, можно ли быть уверенным, что \textit{для всех (forall)}
64-битных аргументов, два выражения эквивалентны (второе это просто \textit{arg1}).

Синтакс, может быть, трудно понять, но на самом деле, он очень близок к LISP, и арифметические операции
называются ``bvsub'', ``bvadd'', итд, потому что ``bv'' означает \textit{bit vector}.

Z3 при запуске показывает ``sat'', что означает ``satisfiable''.
Другими словами, Z3 не смог найти контрпримера для этого выражения.

На самом деле, я могу переписать это выражение в такой форме: \textit{expr1 != expr2}, и мы спросим у Z3,
сможет ли Z3 найти хоть один набор выходных аргументов, для которых выражения не равны друг другу:

\begin{lstlisting}
(declare-const arg1 (_ BitVec 64))
(declare-const arg2 (_ BitVec 64))
(declare-const arg3 (_ BitVec 64))
(declare-const arg4 (_ BitVec 64))

(assert
    (not
        (=
            (bvsub (bvsub (bvsub (bvneg arg1) #x0000000000000003) (bvneg arg1)) (bvsub (bvneg arg1) #x0000000000000003))
            arg1
        )
    )
)
(check-sat)
\end{lstlisting}

Z3 овтечает ``unsat'', означая, что он не смог найти такой контрпример.
Другими словами, для всех возможных входных аргументов, результаты этих двух выражений всегда равны друг другу.

Тем не менее, Z3 не всесилен.
Он не может доказать эквивалентность кода, который производит деление через умножение.
Прежде всего, я расширил его, так что оба результата имеют длину 128 бит вместо 64:

\begin{lstlisting}
(declare-const x (_ BitVec 64))
    (assert
        (forall ((x (_ BitVec 64)))
            (=
                ((_ zero_extend 64) (bvudiv x (_ bv17 64)))
                (bvlshr (bvmul ((_ zero_extend 64) x) #x0000000000000000f0f0f0f0f0f0f0f1) (_ bv68 128))
            )
        )
    )
(check-sat)
(get-model)
\end{lstlisting}

(\textit{bv17} это просто 64-битное число 17, итд. ``bv'' означает ``bit vector'', что противопоставляется целочисленному
значению.)

Z3 работал слишком долго без ответа, пришлось прервать.

Как указывают разработчики Z3, такие выражения пока тяжеловаты для Z3:
\url{https://github.com/Z3Prover/z3/issues/514}.

Но все же, деление через умножение можно протестировать при помощи раннее описанного теста-брутфорса.

\subsection{Мои другие реализации игрушечного декомпилятора}

Когда я сделал попытку написать его на Си++, конечно, узел в выражении представлялся в виде класса.
Есть также реализация на чистом Си\footnote{\url{https://github.com/dennis714/SAT_SMT_article/tree/master/toy_decompiler/files/C}}, узел представлен как структура.

Матчеры в версиях на Си++ и Си не возвращают никакой словарь, но вместо этого, ф-ция \TT{bind\_value()}
берет указатель на переменную, которая будет содержать значение после успешного сравнения.
\TT{bind\_expr()} берет указатель на указатель, который затем будет указывать на часть выражения, снова, в случае успеха.
Я взял эту идею из LLVM.

Вот два фрагмента кода из исходников LLVM с парой правил сокращения:

\begin{lstlisting}
// (X >> A) << A -> X
  Value *X;
  if (match(Op0, m_Exact(m_Shr(m_Value(X), m_Specific(Op1)))))
    return X;
\end{lstlisting}

( \href{http://llvm.org/docs/doxygen/html/InstructionSimplify_8cpp_source.html}{lib/Analysis/InstructionSimplify.cpp} )

\begin{lstlisting}
// (A | B) | C  and  A | (B | C)                  -> bswap if possible.
  // (A >> B) | (C << D)  and  (A << B) | (B >> C)  -> bswap if possible.
  if (match(Op0, m_Or(m_Value(), m_Value())) ||
      match(Op1, m_Or(m_Value(), m_Value())) ||
      (match(Op0, m_LogicalShift(m_Value(), m_Value())) &&
       match(Op1, m_LogicalShift(m_Value(), m_Value())))) {
    if (Instruction *BSwap = MatchBSwap(I))
      return BSwap;
\end{lstlisting}
( \href{https://github.com/numba/llvm-mirror/blob/master/lib/Transforms/InstCombine/InstCombineAndOrXor.cpp}{lib/Transforms/InstCombine/InstCombineAndOrXor.cpp} )

Как видно, мой матчер пытается имитировать LLVM.
То, что у меня называется \textit{сокращение (reduction)}, это называется \textit{свертывание (folding)} в LLVM.
Оба термина популярны.

Вот еще у меня есть пост в блоге об обфускаторе на LLVM, в котором упоминается матчер из LLVM:
\url{https://yurichev.com/blog/llvm/}.

Питоновская версия игрушечного декомпилятора использует строки в тех местах, где в Сишной версии перечисляемые типы
(как \textit{OP\_AND}, \textit{OP\_MUL}, итд),
а в Racket-версии используются символы\footnote{Racket это диалект Scheme (который, в свою очередь, диалект LISP-а).
\url{https://github.com/dennis714/SAT_SMT_article/tree/master/toy_decompiler/files/Racket}} (как \textit{'OP\_DIV}, итд).
Это выглядит неэффективно, тем не менее, благодаря пулу строк (string interning), в Питоновской версии сравниваются
только адреса строк, а не сами строки.
Так что строки в Питоне можно рассматривать как возможную замену LISP-овским символам.

\subsubsection{Даже еще более простой игрушечный декомпилятор}

Знание LISP-а помогает все эти вещи понять без особого труда.
Но когда у меня не было этого знания, но я всё еще хотел сделать простой декомпилятор, я сделал его используя обычные
текстовые строки, которые хранили выражения для всех регистров (и даже для памяти).

Так что когда инструкция MOV копирует значение из одного регистра в другой, мы просто копируем строки.
Когда случается арифметическая инструкция, мы склеиваем строки:

\begin{lstlisting}
std::string registers[TOTAL];

...

// all 3 arguments are strings
switch (ins, op1, op2)
{
    ...
    case ADD:    registers[op1]="(" + registers[op1] + " + " + registers[op2] + ")";
                 break;
    ...
    case MUL:    registers[op1]="(" + registers[op1] + " / " + registers[op2] + ")";
                 break;
    ...
}
\end{lstlisting}

Теперь у вас длинные выражения для каждого регистра, представленные как строки.
Для сокращения оных, вы можете использовать самые обычные регулярные выражения.

Например, для правила \TT{(X*n)+(X*m) -> X*(n+m)}, вы можете находить его в (под)строке используя такое регулярное
выражение: \\
\TT{((.*)*(.*))+((.*)*(.*))}
\footnote{В этом регулярном выражении нет правильных escape-в, ради л\'{у}чшего прочтения и понимания.}.
Если строка совпала, вы получаете 4 группы (или подстроки).
Вы затем просто сравниваете 1-ую и 3-ю используя обычную ф-цию сравнения строк, затем проверяете,
являются ли 2-я и 4-я подстрока числами, затем конвертируете их в числа, суммируете, и создаете новую строку, состоящую
из первой группы и суммы, вот как: \TT{(" + X + "*" + (int(n) + int(m)) + ")}.

Это было наивно, топорно, от этого было очень стыдно, но это корректно работало.

\subsection{Разница между игрушечным декомпилятором и коммерческим}

Вероятно, кто-то, кто сейчас читает этот текст, может тут же начать расширять мой исходник.
В качестве упражнения, я бы сказал, первым шагом может быть поддержка частичных регистров, т.е., AL, AX, EAX.
Это на первый взгляд трудно, но возможно.

Еще одна задача может быть: поддержка \ac{FPU}-инструкций в x86 (моделирование \ac{FPU}-стека это не очень сложно).

Разница между игрушечным декомпилятором и коммерческим, как Hex-Rays, всё же будет огромной.
Множество трудных проблем нужно решить, по крайней мере эти:

\begin{itemize}
\item Типы данных в Си: массивы, структуры, указатели, итд.
Этой проблемы почти нет в \ac{JVM} (Java, итд), и в .NET-декомпиляторах, потому что информация о типах присутствует
в бинарных файлах.

\item Бейсик-блоки, Си/Си++ выражения. Mike Van Emmerik в своей диссертации
\footnote{\url{https://yurichev.com/mirrors/vanEmmerik_ssa.pdf}} показывает, как всё это можно решить при помощи
\ac{SSA}-форм (которые также активно используются в компиляторах).

\item Поддержка памяти, включая локальный стек. Не забывайте о проблеме \textit{pointer aliasing}.
И снова, декомпиляторы с \ac{JVM} и .NET здесь проще.
\end{itemize}

\subsection{Дальнейшее чтение}

Есть несколько интересных опен-сорсных попыток сделать декомпилятор.
И исходники и диссертации интересно изучить.

\begin{itemize}
	\item \textit{decomp} сделанный Jim Reuter\footnote{
			\url{http://www.program-transformation.org/Transform/DecompReadMe},
			\url{http://www.program-transformation.org/Transform/DecompDecompiler}}.

	\item \textit{DCC} сделанный Cristina Cifuentes\footnote{
			\url{http://www.program-transformation.org/Transform/DccDecompiler},
			диссертация: \url{https://yurichev.com/mirrors/DCC_decompilation_thesis.pdf}}.

		Интересно что этот декомпилятор поддерживает только один тип (\textit{int}).
		Может быть это причина, почему декомпилятор DCC выдает исходники с расширением \textit{.B}?
		Читайте больше о бестиповом языке B (предтеча Си): \url{https://yurichev.com/blog/typeless/}.

	\item \textit{Boomerang} написанный Mike Van Emmerik, Trent Waddington и прочими\footnote{
			\url{http://boomerang.sourceforge.net/},
			\url{http://www.program-transformation.org/Transform/MikeVanEmmerik},
			диссертация: \url{https://yurichev.com/mirrors/vanEmmerik_ssa.pdf}}.
\end{itemize}

Как я уже сказал, знание LISP-а может сильно упростить понимание.
Есть хорошо известный микро-интерпретатор LISP-а написанный Питером Норвигом на Питоне:
\url{https://web.archive.org/web/20161116133448/http://www.norvig.com/lispy.html},
\url{https://web.archive.org/web/20160305172301/http://norvig.com/lispy2.html}.

Русский перевод: \url{https://habrahabr.ru/post/115206/}.

\subsection{Файлы}

Питоновская версия и тесты: \url{https://github.com/dennis714/SAT_SMT_article/tree/master/toy_decompiler/files}.

Есть также версии на Си и Racket, но немного устаревшие.

Не забывайте --- этот декомпилятор всё еще игрушечный, и он тестировался только на тех крохотных тестовых файлах,
что идут в комплекте.

}


\EN{\section{Symbolic execution}

% subsections:
\input{symbolic/computation_EN}
\subsection{Symbolic execution}
\label{symbolic_exec}

\subsubsection{Swapping two values using XOR}

There is a well-known (but counterintuitive) algorithm for swapping two values in two variables
using XOR operation without use of any additional memory/register:

\begin{lstlisting}
X=X^Y
Y=Y^X
X=X^Y
\end{lstlisting}

How it works?
It would be better to construct an expression at each step of execution.

\lstinputlisting[style=custompy]{symbolic/1_XOR/xor_swap.py}

It works, because Python is dynamicaly typed \ac{PL}, so the function doesn't care what to operate on,
numerical values, or on objects of Expr() class.

Here is result:

\begin{lstlisting}
new_X ((X^Y)^(Y^(X^Y)))
new_Y (Y^(X^Y))
\end{lstlisting}

You can remove double variables in your mind (since XORing by a value twice will result in nothing).
At new\_X we can drop two X-es and two Y-es, and single Y will left.
At new\_Y we can drop two Y-es, and single X will left.

\subsubsection{Change endianness}

What does this code do?

\begin{lstlisting}
mov     eax, ecx
mov     edx, ecx
shl     edx, 16
and     eax, 0000ff00H
or      eax, edx
mov     edx, ecx
and     edx, 00ff0000H
shr     ecx, 16
or      edx, ecx
shl     eax, 8
shr     edx, 8
or      eax, edx
\end{lstlisting}

In fact, many reverse engineers play shell game a lot, keeping track of what is stored where, at each point of time.

\begin{figure}[H]
\centering
\includegraphics[scale=2.5]{symbolic/2_assembly/718px-Conjurer_Bosch.jpg}
\caption{Hieronymus Bosch -- The Conjurer}
\end{figure}

Again, we can build equivalent function which can take both numerical variables and Expr() objects.
We also extend Expr() class to support many arithmetical and boolean operations.
Also, Expr() methods would take both Expr() objects on input and integer values.

\lstinputlisting[style=custompy]{symbolic/2_assembly/1.py}

I run it:

\begin{lstlisting}
((((initial_ECX&65280)|(initial_ECX<<16))<<8)|(((initial_ECX&16711680)|(initial_ECX>>16))>>8))
\end{lstlisting}

Now this is something more readable, however, a bit LISPy at first sight.
In fact, this is a function which change endianness in 32-bit word.

By the way, my Toy Decompiler can do this job as well, but operates on \ac{AST} instead
of plain strings: \ref{toy_decompiler}.

\subsubsection{Fast Fourier transform}

I've found one of the smallest possible FFT implementations on \href{https://www.reddit.com/r/Python/comments/1la4jp/understanding_the_fft_algorithm_with_python/}{reddit}:

\lstinputlisting[style=custompy]{symbolic/3_FFT/FFT.py}

Just interesting, what value has each element on output?

\lstinputlisting[style=custompy]{symbolic/3_FFT/FFT_symb.py}

FFT() function left almost intact, the only thing I added: complex value is converted into string and then
Expr() object is constructed.

\lstinputlisting{symbolic/3_FFT/res1.txt}

We can see subexpressions in form like $x^0$ and $x^1$.
We can eliminate them, since $x^0=1$ and $x^1=x$.
Also, we can reduce subexpressions like $x \cdot 1$ to just $x$.

\begin{lstlisting}
    def __mul__(self, other):
        op1=self.s
        op2=self.convert_to_Expr_if_int(other).s

        if op1=="1":
            return Expr(op2)
        if op2=="1":
            return Expr(op1)

        return Expr("(" + op1 + "*" + op2 + ")")

    def __pow__(self, other):
        op2=self.convert_to_Expr_if_int(other).s
        if op2=="0":
            return Expr("1")
        if op2=="1":
            return Expr(self.s)

        return Expr("(" + self.s + "**" + op2 + ")")
\end{lstlisting}

\lstinputlisting{symbolic/3_FFT/res2.txt}

\subsubsection{Cyclic redundancy check}

I've always been wondering, which input bit affects which bit in the final CRC32 value.

From the \ac{CRC} theory (good and concise introduction:
\url{http://web.archive.org/web/20161220015646/http://www.hackersdelight.org/crc.pdf}
) we know that \ac{CRC} is shifting register with taps.

We will track each bit rather than byte or word, which is highly inefficient, but serves our purpose better:

\lstinputlisting[style=custompy]{symbolic/4_CRC/1.py}

Here are expressions for each CRC32 bit for 1-byte buffer:

\lstinputlisting{symbolic/4_CRC/1byte.txt}

For larger buffer, expressions gets increasing exponentially.
This is 0th bit of the final state for 4-byte buffer:

\begin{lstlisting}
state 0=((((((((((((((in_0_0^1)^(in_0_1^1))^(in_0_2^1))^(in_0_4^1))^(in_0_5^1))^(in_0_7^(1^(in_0_1^1))))^
(in_1_0^(1^(in_0_2^1))))^(in_1_2^(((1^(in_0_0^1))^(in_0_1^1))^(in_0_4^1))))^(in_1_3^(((1^(in_0_1^1))^
(in_0_2^1))^(in_0_5^1))))^(in_1_4^(((1^(in_0_2^1))^(in_0_3^1))^(in_0_6^(1^(in_0_0^1))))))^(in_2_0^((((1^
(in_0_0^1))^(in_0_6^(1^(in_0_0^1))))^(in_0_7^(1^(in_0_1^1))))^(in_1_2^(((1^(in_0_0^1))^(in_0_1^1))^(in_0_4^
1))))))^(in_2_6^(((((((1^(in_0_0^1))^(in_0_1^1))^(in_0_2^1))^(in_0_6^(1^(in_0_0^1))))^(in_1_4^(((1^(in_0_2^1))^
(in_0_3^1))^(in_0_6^(1^(in_0_0^1))))))^(in_1_5^(((1^(in_0_3^1))^(in_0_4^1))^(in_0_7^(1^(in_0_1^1))))))^
(in_2_0^((((1^(in_0_0^1))^(in_0_6^(1^(in_0_0^1))))^(in_0_7^(1^(in_0_1^1))))^(in_1_2^(((1^(in_0_0^1))^(in_0_1^1))^
(in_0_4^1))))))))^(in_2_7^(((((((1^(in_0_1^1))^(in_0_2^1))^(in_0_3^1))^(in_0_7^(1^(in_0_1^1))))^(in_1_5^(((1^
(in_0_3^1))^(in_0_4^1))^(in_0_7^(1^(in_0_1^1))))))^(in_1_6^(((1^(in_0_4^1))^(in_0_5^1))^(in_1_0^(1^(in_0_2^
1))))))^(in_2_1^((((1^(in_0_1^1))^(in_0_7^(1^(in_0_1^1))))^(in_1_0^(1^(in_0_2^1))))^(in_1_3^(((1^(in_0_1^1))^
(in_0_2^1))^(in_0_5^1))))))))^(in_3_2^(((((((((1^(in_0_1^1))^(in_0_2^1))^(in_0_4^1))^(in_0_5^1))^(in_0_6^(1^
(in_0_0^1))))^(in_1_2^(((1^(in_0_0^1))^(in_0_1^1))^(in_0_4^1))))^(in_2_0^((((1^(in_0_0^1))^(in_0_6^(1^(in_0_0^
1))))^(in_0_7^(1^(in_0_1^1))))^(in_1_2^(((1^(in_0_0^1))^(in_0_1^1))^(in_0_4^1))))))^(in_2_1^((((1^(in_0_1^1))^
(in_0_7^(1^(in_0_1^1))))^(in_1_0^(1^(in_0_2^1))))^(in_1_3^(((1^(in_0_1^1))^(in_0_2^1))^(in_0_5^1))))))^(in_2_4^
(((((1^(in_0_0^1))^(in_0_4^1))^(in_1_2^(((1^(in_0_0^1))^(in_0_1^1))^(in_0_4^1))))^(in_1_3^(((1^(in_0_1^1))^
(in_0_2^1))^(in_0_5^1))))^(in_1_6^(((1^(in_0_4^1))^(in_0_5^1))^(in_1_0^(1^(in_0_2^1))))))))))
\end{lstlisting}

Expression for the 0th bit of the final state for 8-byte buffer has length of $\approx 350KiB$,
which is, of course, can be reduced
significantly (because this expression is basically XOR tree), but you can feel the weight of it.

Now we can process this expressions somehow to get a smaller picture on what is affecting what.
Let's say, if we can find ``in\_2\_3'' substring in expression, this means that 3rd bit of 2nd byte of input
affects this expression.
But even more than that: since this is XOR tree (i.e., expression consisting only of XOR operations),
if some input variable is occurring twice, it's \textit{annihilated}, since $x \oplus x=0$.
More than that: if a vairable occurred even number of times (2, 4, 8, etc), it's annihilated, but left if it's occurred
odd number of times (1, 3, 5, etc).

\begin{lstlisting}
    for i in range(32):
        #print "state %d=%s" % (i, state[31-i])
        sys.stdout.write ("state %02d: " % i)
        for byte in range(BYTES):
            for bit in range(8):
                s="in_%d_%d" % (byte, bit)
                if str(state[31-i]).count(s) & 1:
                    sys.stdout.write ("*")
                else:
                    sys.stdout.write (" ")
        sys.stdout.write ("\n")
\end{lstlisting}

( \url{https://github.com/DennisYurichev/SAT_SMT_by_example/blob/master/symbolic/4_CRC/2.py} )

Now this how each bit of 1-byte input buffer affects each bit of the final CRC32 state:

\lstinputlisting{symbolic/4_CRC/1byte_tbl.txt}

This is 8*8=64 bits of 8-byte input buffer:

\lstinputlisting{symbolic/4_CRC/8byte_tbl.txt}

\subsubsection{Linear congruential generator}

This is popular \ac{PRNG} from OpenWatcom \ac{CRT} library: \url{https://github.com/open-watcom/open-watcom-v2/blob/d468b609ba6ca61eeddad80dd2485e3256fc5261/bld/clib/math/c/rand.c}.

What expression it generates on each step?

\lstinputlisting[style=custompy]{symbolic/5_LCG/LCG.py}

\lstinputlisting{symbolic/5_LCG/1.txt}

Now if we once got several values from this PRNG, like 4583, 16304, 14440, 32315, 28670, 12568..., how would we
recover the initial seed?
The problem in fact is solving a system of equations:

\begin{lstlisting}
((((initial_seed*1103515245)+12345)>>16)&32767)==4583
((((((initial_seed*1103515245)+12345)*1103515245)+12345)>>16)&32767)==16304
((((((((initial_seed*1103515245)+12345)*1103515245)+12345)*1103515245)+12345)>>16)&32767)==14440
((((((((((initial_seed*1103515245)+12345)*1103515245)+12345)*1103515245)+12345)*1103515245)+12345)>>16)&32767)==32315
\end{lstlisting}

As it turns out, Z3 can solve this system correctly using only two equations:

\lstinputlisting[style=custompy]{symbolic/5_LCG/Z3_solve.py}

\begin{lstlisting}
[x = 11223344]
\end{lstlisting}

(Though, it takes $\approx 20$ seconds on my ancient Intel Atom netbook.)

\subsubsection{Path constraint}

How to get weekday from UNIX timestamp?

\begin{lstlisting}
#!/usr/bin/env python

input=...
SECS_DAY=24*60*60
dayno = input / SECS_DAY
wday = (dayno + 4) % 7
if wday==5:
    print "Thanks God, it's Friday!"
\end{lstlisting}

Let's say, we should find a way to run the block with print() call in it.
What input value should be?

First, let's build expression of $wday$ variable:

\lstinputlisting[style=custompy]{symbolic/6_TGIF/TGIF.py}

\begin{lstlisting}
(((input/86400)+4)%7)
\end{lstlisting}

In order to execute the block, we should solve this equation: $((\frac{input}{86400}+4) \equiv 5 \mod 7$.

So far, this is easy task for Z3:

\lstinputlisting[style=custompy]{symbolic/6_TGIF/Z3_solve.py}

\begin{lstlisting}
[x = 86438]
\end{lstlisting}

This is indeed correct UNIX timestamp for Friday:

\begin{lstlisting}
% date --date='@86438'
Fri Jan  2 03:00:38 MSK 1970
\end{lstlisting}

Though the date back in year 1970, but it's still correct!

This is also called ``path constraint'', i.e., what constraint must be satisified to execute specific block?
Several tools has ``path'' in their names, like
``pathgrind'', 
\href{http://babelfish.arc.nasa.gov/trac/jpf/wiki/projects/jpf-symbc}{Symbolic PathFinder}, CodeSurfer Path Inspector, etc.

Like the shell game, this task is also often encounters in practice.
You can see that something dangerous can be executed inside some basic block and you're trying to deduce,
what input values can cause execution of it.
It may be buffer overflow, etc.
Such input values are sometimes also called ``inputs of death''.

Many crackmes are solved in this way, all you need is find a path into block which prints ``key is correct''
or something like that.

We can extend this tiny example:

\begin{lstlisting}
input=...
SECS_DAY=24*60*60
dayno = input / SECS_DAY
wday = (dayno + 4) % 7
print wday
if wday==5:
    print "Thanks God, it's Friday!"
else:
    print "Got to wait a little"
\end{lstlisting}

Now we have two blocks: for the first we should solve this equation: $((\frac{input}{86400}+4) \equiv 5 \mod 7$.
But for the second we should solve inverted equation: $((\frac{input}{86400}+4) \not\equiv 5 \mod 7$.
By solving these equations, we will find two paths into both blocks.

KLEE (or similar tool) tries to find path to each [basic] block and produces ``ideal'' unit test.
Hence, KLEE can find a path into the block which crashes everything, or reporting about correctness of the input
key/license, etc.
Surprisingly, KLEE can find backdoors in the very same manner.

KLEE is also called ``KLEE Symbolic Virtual Machine'' -- by that its creators mean that the KLEE is \ac{VM} which executes a code symbolically rather than numerically (like usual \ac{CPU}).

Let's extend our tiny example again.
We would like to find Friday 13th. To make things simpler, we can limit ourselves to year 1970.
Let's get all 12 13th days of year 1970:

\lstinputlisting{symbolic/6_TGIF/13th.txt}

The script checking if the current date is Friday 13th:

\begin{lstlisting}
input=...
SECS_DAY=24*60*60
dayno = input / SECS_DAY
wday = (dayno + 4) % 7
print wday
if wday==5:
    print "Thanks God, it's Friday!"
 
    if dayno in [13,44,72,103,133,164,194,225,256,286,317,347]:
        print "Friday 13th"
\end{lstlisting}

To get the second "print" executed, we must satisfy two constraints:

\lstinputlisting[style=custompy]{symbolic/6_TGIF/Z3_solve2.py}

Easy task for Z3 as well:

\begin{lstlisting}
 % python Z3_solve2.py
[dayno = 316, x = 27302400]
 % date --date='@27302400'
Fri Nov 13 03:00:00 MSK 1970
\end{lstlisting}

This is an UNIX date for which both constructs are satisfied: 13th November 1970, Friday.

\subsubsection{Division by zero}

If division by zero is unwrapped by sanitizing check, and exception isn't caught, it can crash process.

Let's calculate simple expression $\frac{x}{2y + 4z - 12}$.
We can add a warning into \TT{\_\_div\_\_} method:

\lstinputlisting[style=custompy]{symbolic/7_div/1.py}

\dots so it will report about dangerous states and conditions:

\begin{lstlisting}
warning: division by zero if (((y*2)+(z*4))-12)==0
(x/(((y*2)+(z*4))-12))
\end{lstlisting}

This equation is easy to solve, let's try Wolfram Mathematica this time:

\begin{lstlisting}
In[]:= FindInstance[{(y*2 + z*4) - 12 == 0}, {y, z}, Integers]
Out[]= {{y -> 0, z -> 3}}
\end{lstlisting}

These values for $y$ and $z$ can also be called ``inputs of death''.

\subsubsection{Merge sort}

How merge sort works?
I have copypasted Python code from rosettacode.com almost intact:

\lstinputlisting[style=custompy]{symbolic/8_sorting/1.py}

But here is a function which compares elements.
Obviously, it wouldn't work correctly without it.

So we can track both expression for each element and numerical value.
Both will be printed finally.
But whenever values are to be compared, only numerical parts will be used.

Result:

\lstinputlisting{symbolic/8_sorting/result.txt}

\subsubsection{Extending Expr class}

This is somewhat senseless, nevertheless, it's easy task to extend my Expr class to support \ac{AST} instead of
plain strings.
It's also possible to add folding steps (like I demonstrated in Toy Decompiler: \ref{toy_decompiler}).
Maybe someone will want to do this as an exercise.
By the way, the toy decompiler can be used as simple symbolic engine as well,
just feed all the instructions to it and it will track contents of each register.

\subsubsection{Conclusion}

For the sake of demonstration, I made things as simple as possible.
But reality is always harsh and inconvenient, so all this shouldn't be taken as a silver bullet.

The files used in this part: \url{https://github.com/DennisYurichev/SAT_SMT_by_example/tree/master/symbolic}.



\subsection{Further reading}

James C. King --- Symbolic Execution and Program Testing
\footnote{\url{https://yurichev.com/mirrors/king76symbolicexecution.pdf}}

}
\RU{\section{Символьное исполнение}

% subsections:
\subsection{Символьные вычисления}

Начнем с символьных вычислений\footnote{\url{https://en.wikipedia.org/wiki/Symbolic_computation}}.

Некоторые числа могут быть представлены в двоичной системе только в некотором приближении, как $\frac{1}{3}$ и $\pi$.
Если вычислять $\frac{1}{3} \cdot 3$ пошагово, мы можем получить потерю значимости.
Мы также знаем, что $sin(\frac{\pi}{2}) = 1$, но вычисляя это выражение обычным образом, мы также получим шум в результате.
Арифметика произвольной точности (arbitrary-precision arithmetic)\footnote{\url{https://en.wikipedia.org/wiki/Arbitrary-precision_arithmetic}} это не решение, потому что эти числа не могут быть представлены в памяти 
как двоичное число конечной длины.

Как можно решить эту проблему?
Люди сокращают подобные выражения используя бумагу и карандаш без всяких вычислений.
Мы можем имитировать человеческое поведение программно если мы будем сохранять выражение как дерево,
а символы вроде $\pi$ будут конвертироваться в числа на самом последнем шаге.

Это то что делает Wolfram Mathematica\footnote{Другие хорошо известные системы символьной математики это 
\href{https://en.wikipedia.org/wiki/Maxima_\%28software\%29}{Maxima} и 
\href{https://en.wikipedia.org/wiki/SymPy}{SymPy}}.
Запустим и попробуем это:

\begin{lstlisting}
In[]:= x + 2*8
Out[]= 16 + x
\end{lstlisting}

Так как Mathematica не имеет понятия что такое $x$, оно оставляется \textit{как есть}, но $2 \cdot 8$ можно легко
сократить, это может сделать и Mathematica и человек, вот это и произошло.
В какой-то момент в будущем, пользователь Mathematica может присвоить какое-то число переменной
$x$ и затем Mathematica сократит это выражение дальше.

Mathematica делает это потому что она парсит выражение и находит некоторые известные ей шаблонные правила.
Это также называется \textit{переписывание термов (term rewriting)}\footnote{\url{https://en.wikipedia.org/wiki/Rewriting}}.
В обычном русском языке это может звучать так:
``если там где-то есть оператор $+$ между двумя известными числами,
замени это подвыражение на вычисленное число, которое является
суммой этих двух чисел, если это возможно''.
Точно также, как это делают люди.

Mathematica также имеет правила вроде ``замени $sin(\pi)$ на 0'' и ``замени $sin(\frac{\pi}{2})$ на 1'', но как вы видите,
$\pi$ должно быть сохранено как что-то вроде символа, вместо числа.

% TODO example
Так что Mathematica оставила $x$ как неизвестное значение.
Кстати, это распространенная ошибка пользователей Mathematica: мелкая опечатка во входном выражении может привести к
огромному несократимому выражению, в котором остается эта же опечатка.

Другой пример: Mathematica сознательно оставляет это во время вычисления двоичного логарифма:

\begin{lstlisting}
In[]:= Log[2, 36]
Out[]= Log[36]/Log[2]
\end{lstlisting}

Потому что она имеет надежду что в какой-то момент времени, в будущем, это выражение станет подвыражением другого
выражения, и оно будет красиво сокращено в самом конце.
Но если нам действительно нужен целочисленный ответ, мы можем заставить Mathematica вычислить его:

\begin{lstlisting}
In[]:= Log[2, 36] // N
Out[]= 5.16993
\end{lstlisting}

Иногда значения, как невычисленные символы, желательны:

\begin{lstlisting}
In[]:= Union[{a, b, a, c}, {d, a, e, b}, {c, a}]
Out[]= {a, b, c, d, e}
\end{lstlisting}

Символы в выражении это просто невычисленные символы\footnote{\textit{Символ} как в LISP} без привязки
к числам или другим выражениям, так что Mathematica оставила их \textit{как есть}.

Другой пример из реального мира это символьная интеграция\footnote{\url{https://en.wikipedia.org/wiki/Symbolic_integration}}, 
т.е., нахожение формулы интеграла путем переписывания изначального выражения используя некоторые предопределенные правила.
Mathematica тоже делает это:

\begin{lstlisting}
In[]:= Integrate[1/(x^5), x]
Out[]= -(1/(4 x^4))
\end{lstlisting}

Преимущества символьных вычислений очевидны: нет проблем с \textit{loss of significance}\footnote{\url{https://en.wikipedia.org/wiki/Loss_of_significance}} и ошибок округления\footnote{\url{https://en.wikipedia.org/wiki/Round-off_error}}, 
но недостатки тоже очевидны: вам нужно хранить где-то дерево выражения (возможно, огромное), и обрабатывать его много раз.
Переписывание термов может быть медленным.
Все эти вещи очень неуклюжи в сравнении с быстрым \ac{FPU}.

``Символьные вычисления'' противопоставляются ``численным вычислениям'', последнее это просто обработка чисел шаг за шагом,
используя калькулятор, \ac{CPU} или \ac{FPU}.\\
\\
Некоторые задачи лучше решать первым методом, некоторые другие --- вторым.

\subsubsection{Дробный (rational) тип данных}

Некоторые реализации LISP-а могут хранить число как дробь
\footnote{\url{https://en.wikipedia.org/wiki/Rational_data_type}}, т.е., храня два числа в ячейке (которая, в данном случае, называется \textit{атомом} на сленге LISP-а).
Например, если вы делите 1 на 3, и интерпретатор, понимая что $\frac{1}{3}$ это
несократимая дробь\footnote{\url{http://bit.ly/2pXhzpy}}, создает ячейку с числами 1 и 3.
В какой-то момент позже, вы можете умножить эту ячейку на 6, и умножающая ф-ция внутри интерпретатора LISP-а может вернуть
намного лучший результат (2 без \textit{шума}).

Печатающая ф-ция в интерпретаторе может вывести что-то вроде \TT{1 / 3} вместо числа с плавающей точкой.

Иногда это называется ``fractional arithmetic'' [см. Donald E. Knuth, \textit{The Art of Computing Programming}, 3-е изд., (1997), 4.5.1, стр.330].

Это не символьные вычисления ни в каком смысле, но это немного лучше чем хранить дроби как обычные числа с плавающей точкой.

Недостатки очевидны: вам нужно больше памяти чтобы хранить дробь вместо числа;
и все арифметические ф-ции более сложные и медленные, потому что они должны поддерживать и числа и дроби.

Вероятно, из-за недостатков, некоторые языки программирования предлагают отдельный тип данных (\textit{rational}),
как опцию языка, или как поддержку библиотекой
\footnote{Более полный список: \url{https://en.wikipedia.org/wiki/Rational_data_type}}:
Haskell, OCaml, Perl, Ruby, Python (\textit{fractions}), Smalltalk, Java, Clojure, C/C++
\footnote{При помощи GNU Multiple Precision Arithmetic Library}.


\subsection{Символьное исполнение (symbolic execution)}
\label{symbolic_exec}

\subsubsection{Обмен значений используя исключающее ИЛИ}

Есть хорошо известный (но контринтуитивный) алгоритм для обмена двух значений в двух переменных используя операцию
исключающего ИЛИ, без использования дополнительного регистра/ячейки памяти.

\begin{lstlisting}
X=X^Y
Y=Y^X
X=X^Y
\end{lstlisting}

Как он работает?
Было бы лучше сконструировать выражение на каждом шаге исполнения.

\lstinputlisting{symbolic/1_XOR/xor_swap.py}

Это работает, потому что Питон это \ac{PL} с динамической типизацией, так что ф-ции не важно, над чем работать,
над числами, или над объектами класса Expr().

И вот результат:

\begin{lstlisting}
new_X ((X^Y)^(Y^(X^Y)))
new_Y (Y^(X^Y))
\end{lstlisting}

Можете убрать двойные переменные в уме (т.е., применение исключающего ИЛИ к переменной дважды в итоге ничего не дает).
С new\_X мы можем выбросить два X-а и два Y-а, и остается один Y.
С new\_Y мы можем выбросить два Y-а, и останется один X.

\subsubsection{Смена порядка байт (endianness)}

Что делает этот код?

\begin{lstlisting}
mov     eax, ecx
mov     edx, ecx
shl     edx, 16
and     eax, 0000ff00H
or      eax, edx
mov     edx, ecx
and     edx, 00ff0000H
shr     ecx, 16
or      edx, ecx
shl     eax, 8
shr     edx, 8
or      eax, edx
\end{lstlisting}

На самом деле, многие реверс инженеры играют в игру в наперстки, много, запоминая что где лежит, в каждый момент времени.

\begin{figure}[H]
\centering
\includegraphics[scale=2.5]{symbolic/2_assembly/718px-Conjurer_Bosch.jpg}
\caption{Иероним Босх -- Фокусник}
\end{figure}

И снова, мы можем сделать эквивалентную ф-цию, которая может брать и переменные с числами и объекты Expr().
Мы также расширяем класс Expr() чтобы он поддерживал многие арифметические и булевы операции.
Также, методы Expr() будут брать на вход и объекты Expr() и целочисленные значения.

\lstinputlisting{symbolic/2_assembly/1.py}

Запускаю:

\begin{lstlisting}
((((initial_ECX&65280)|(initial_ECX<<16))<<8)|(((initial_ECX&16711680)|(initial_ECX>>16))>>8))
\end{lstlisting}

Теперь понятнее, хотя и немного напоминает LISP, с первого взгляда.
На самом деле эта ф-ция меняет порядок байт (endianness) в 32-битном слове.

Кстати, мой игрушечный декомпилятор тоже может это делать, но он оперирует \ac{AST} вместо обычных строк:
\ref{toy_decompiler}.

\subsubsection{Быстрое преобразование Фурье}

Я нашел одну из самых маленьких реализаций FFT на \href{https://www.reddit.com/r/Python/comments/1la4jp/understanding_the_fft_algorithm_with_python/}{реддите}:

\lstinputlisting{symbolic/3_FFT/FFT.py}

Просто интересно, какое значение имеет каждый элемент на выходе?

\lstinputlisting{symbolic/3_FFT/FFT_symb.py}

Ф-ция FFT() оставлена почти без изменений, единственное что я добавил, это то что комплексное значение в начале конвертируется
в строку, и затем конструируется объект Expr().

\lstinputlisting{symbolic/3_FFT/res1.txt}

Мы видим подвыражения вида $x^0$ и $x^1$.
Мы можем их сократить, т.к., $x^0=1$ и $x^1=x$.
Также, мы можем сократить подвыражения вроде $x \cdot 1$ до $x$.

\begin{lstlisting}
    def __mul__(self, other):
        op1=self.s
        op2=self.convert_to_Expr_if_int(other).s

        if op1=="1":
            return Expr(op2)
        if op2=="1":
            return Expr(op1)

        return Expr("(" + op1 + "*" + op2 + ")")

    def __pow__(self, other):
        op2=self.convert_to_Expr_if_int(other).s
        if op2=="0":
            return Expr("1")
        if op2=="1":
            return Expr(self.s)

        return Expr("(" + self.s + "**" + op2 + ")")
\end{lstlisting}

\lstinputlisting{symbolic/3_FFT/res2.txt}

\subsubsection{Циклический избыточный код (\ac{CRC})}

Мне всегда было интересно, какой входной бит влияет на какой бит конечного значения CRC32.

Из теории \ac{CRC} (хорошее и короткое введение:
\url{http://web.archive.org/web/20161220015646/http://www.hackersdelight.org/crc.pdf}
) мы знаем что \ac{CRC} это регистр сдвига с отводами.

Мы будем следить за каждым битом а не байтом или словом, что в свою очередь очень неэффективно, но лучше служит нашим целям.

\lstinputlisting{symbolic/4_CRC/1.py}

Вот выражения для каждого бита CRC32 для 1-байтного буфера:

\lstinputlisting{symbolic/4_CRC/1byte.txt}

Для б\'{о}льших буферов, выражения увеличиваются экспоненциально.
Это 0-й бит конечного состояния для 4-байтного буфера:

\begin{lstlisting}
state 0=((((((((((((((in_0_0^1)^(in_0_1^1))^(in_0_2^1))^(in_0_4^1))^(in_0_5^1))^(in_0_7^(1^(in_0_1^1))))^
(in_1_0^(1^(in_0_2^1))))^(in_1_2^(((1^(in_0_0^1))^(in_0_1^1))^(in_0_4^1))))^(in_1_3^(((1^(in_0_1^1))^
(in_0_2^1))^(in_0_5^1))))^(in_1_4^(((1^(in_0_2^1))^(in_0_3^1))^(in_0_6^(1^(in_0_0^1))))))^(in_2_0^((((1^
(in_0_0^1))^(in_0_6^(1^(in_0_0^1))))^(in_0_7^(1^(in_0_1^1))))^(in_1_2^(((1^(in_0_0^1))^(in_0_1^1))^(in_0_4^
1))))))^(in_2_6^(((((((1^(in_0_0^1))^(in_0_1^1))^(in_0_2^1))^(in_0_6^(1^(in_0_0^1))))^(in_1_4^(((1^(in_0_2^1))^
(in_0_3^1))^(in_0_6^(1^(in_0_0^1))))))^(in_1_5^(((1^(in_0_3^1))^(in_0_4^1))^(in_0_7^(1^(in_0_1^1))))))^
(in_2_0^((((1^(in_0_0^1))^(in_0_6^(1^(in_0_0^1))))^(in_0_7^(1^(in_0_1^1))))^(in_1_2^(((1^(in_0_0^1))^(in_0_1^1))^
(in_0_4^1))))))))^(in_2_7^(((((((1^(in_0_1^1))^(in_0_2^1))^(in_0_3^1))^(in_0_7^(1^(in_0_1^1))))^(in_1_5^(((1^
(in_0_3^1))^(in_0_4^1))^(in_0_7^(1^(in_0_1^1))))))^(in_1_6^(((1^(in_0_4^1))^(in_0_5^1))^(in_1_0^(1^(in_0_2^
1))))))^(in_2_1^((((1^(in_0_1^1))^(in_0_7^(1^(in_0_1^1))))^(in_1_0^(1^(in_0_2^1))))^(in_1_3^(((1^(in_0_1^1))^
(in_0_2^1))^(in_0_5^1))))))))^(in_3_2^(((((((((1^(in_0_1^1))^(in_0_2^1))^(in_0_4^1))^(in_0_5^1))^(in_0_6^(1^
(in_0_0^1))))^(in_1_2^(((1^(in_0_0^1))^(in_0_1^1))^(in_0_4^1))))^(in_2_0^((((1^(in_0_0^1))^(in_0_6^(1^(in_0_0^
1))))^(in_0_7^(1^(in_0_1^1))))^(in_1_2^(((1^(in_0_0^1))^(in_0_1^1))^(in_0_4^1))))))^(in_2_1^((((1^(in_0_1^1))^
(in_0_7^(1^(in_0_1^1))))^(in_1_0^(1^(in_0_2^1))))^(in_1_3^(((1^(in_0_1^1))^(in_0_2^1))^(in_0_5^1))))))^(in_2_4^
(((((1^(in_0_0^1))^(in_0_4^1))^(in_1_2^(((1^(in_0_0^1))^(in_0_1^1))^(in_0_4^1))))^(in_1_3^(((1^(in_0_1^1))^
(in_0_2^1))^(in_0_5^1))))^(in_1_6^(((1^(in_0_4^1))^(in_0_5^1))^(in_1_0^(1^(in_0_2^1))))))))))
\end{lstlisting}

Выражение для 0-го бита конечного состояния 8-байтного буфера имеет длину
$\approx 350KiB$, что, конечно, можно существенно сократить (потому что выражение это дерево из операций исключающего ИЛИ),
но вы можете ощутить его вес.

Теперь мы можем как-нибудь обработать эти выражения, чтобы получить меньшую картину того, что на что влияет.
Скажем так, если мы находим подстроку ``in\_2\_3'' в выражении, это означает что 3-й бит 2-го байта входа влияет на 
это выражение.
И дажее более того, т.к. это дерево из операций исключающего ИЛИ (т.е., выражение состоящее только из этих операций),
если входная переменная встречается дважды, она \textit{аннигилируется}, т.к. $x \oplus x=0$.
Более того: если переменная встречается четное количество раз (2, 4, 8, итд), она аннигилируется,
но остается, если она встречается нечетное кол-во раз (1, 3, 5, итд).

\begin{lstlisting}
    for i in range(32):
        #print "state %d=%s" % (i, state[31-i])
        sys.stdout.write ("state %02d: " % i)
        for byte in range(BYTES):
            for bit in range(8):
                s="in_%d_%d" % (byte, bit)
                if str(state[31-i]).count(s) & 1:
                    sys.stdout.write ("*")
                else:
                    sys.stdout.write (" ")
        sys.stdout.write ("\n")
\end{lstlisting}

( \url{https://github.com/DennisYurichev/SAT_SMT_article/blob/master/symbolic/4_CRC/2.py} )

Вот как каждый бит входного 1-байтного буфера влияет на каждый бит конечного состояния CRC32:

\lstinputlisting{symbolic/4_CRC/1byte_tbl.txt}

Это 8*8=64 бита 8-байтного входного буфера:

\lstinputlisting{symbolic/4_CRC/8byte_tbl.txt}

\subsubsection{Линейный конгруэнтный генератор}

Это популярный \ac{PRNG} из \ac{CRT}-библиотеки OpenWatcom: \url{https://github.com/open-watcom/open-watcom-v2/blob/d468b609ba6ca61eeddad80dd2485e3256fc5261/bld/clib/math/c/rand.c}.

Какое выражение он генерирует на каждом шаге?

\lstinputlisting{symbolic/5_LCG/LCG.py}

\lstinputlisting{symbolic/5_LCG/1.txt}

Теперь, если однажды мы получили несколько значений из этого PRNG, например 4583, 16304, 14440, 32315, 28670, 12568...,
можем ли мы восстановить изначальное состояние генератора?
Эта проблема на самом деле это решение системы уравнений:

\begin{lstlisting}
((((initial_seed*1103515245)+12345)>>16)&32767)==4583
((((((initial_seed*1103515245)+12345)*1103515245)+12345)>>16)&32767)==16304
((((((((initial_seed*1103515245)+12345)*1103515245)+12345)*1103515245)+12345)>>16)&32767)==14440
((((((((((initial_seed*1103515245)+12345)*1103515245)+12345)*1103515245)+12345)*1103515245)+12345)>>16)&32767)==32315
\end{lstlisting}

Как выясняется, Z3 может решить эту систему корректно используя только 2 уравнения:

\lstinputlisting{symbolic/5_LCG/Z3_solve.py}

\begin{lstlisting}
[x = 11223344]
\end{lstlisting}

(Хотя, это требует $\approx 20$ секунд на моем древнем нетбуке на Intel Atom.)

\subsubsection{Констрайнт пути (path constraint)}

Как получить день недели из UNIX-даты?

\begin{lstlisting}
#!/usr/bin/env python

input=...
SECS_DAY=24*60*60
dayno = input / SECS_DAY
wday = (dayno + 4) % 7
if wday==5:
    print "Thanks God, it's Friday!"
\end{lstlisting}

Скажем, нам нужно найти способ исполнить блок с вызовом print().
Какой должен быть вход?

В начале построим выражение для переменной $wday$:

\lstinputlisting{symbolic/6_TGIF/TGIF.py}

\begin{lstlisting}
(((input/86400)+4)%7)
\end{lstlisting}

Чтобы исполнить этот блок, мы должны решить это уравнение: $((\frac{input}{86400}+4) \equiv 5 \mod 7$.

Пока что, это простая задача для Z3:

\lstinputlisting{symbolic/6_TGIF/Z3_solve.py}

\begin{lstlisting}
[x = 86438]
\end{lstlisting}

Это действительно корректная UNIX-дата для пятницы:

\begin{lstlisting}
% date --date='@86438'
Fri Jan  2 03:00:38 MSK 1970
\end{lstlisting}

Хотя и дата имеет 1970-й год, но она верна!

Это также называется ``констрайнт пути'' (``path constraint''), т.е., какой констрайнт должен быть удовлетворен,
чтобы исполнить определнный блок?
Некоторые инструменты имеют слово ``path'' в своих названиях, как
``pathgrind'', 
\href{http://babelfish.arc.nasa.gov/trac/jpf/wiki/projects/jpf-symbc}{Symbolic PathFinder}, CodeSurfer Path Inspector, итд.

Как и игра в наперстки, эта задача очень часто встречается на практике.
Вы можете видеть, что что-то опасное может быть исполнено внутри некоторого бейсик-блока и вы пытаетесь понять,
какие входные значения приведут к его исполнению.
Это может быть переполнение буфера, итд.
Такие входные значения иногда называют ``входные значения смерти'' (``inputs of death'').

Многие кракми решаются таким же образом, всё что вам нужно это найти путь в блок, который выводит ``ключ верен''
или что-то в этом роде.

Мы можем расширить наш крохотный пример:

\begin{lstlisting}
input=...
SECS_DAY=24*60*60
dayno = input / SECS_DAY
wday = (dayno + 4) % 7
print wday
if wday==5:
    print "Thanks God, it's Friday!"
else:
    print "Got to wait a little"
\end{lstlisting}

Теперь у нас два блока: для первого мы должны решить это уравнение: $((\frac{input}{86400}+4) \equiv 5 \mod 7$.
Для второго мы должны решить инвертированное уравнение: $((\frac{input}{86400}+4) \not\equiv 5 \mod 7$.
Решая эти уравнения, мы найдем два пути в оба блока.

KLEE (или похожий инструмент) пытается найти путь в каждый [бейсик] блок и производит ``идеальный'' unit-тест.
Так, KLEE может найти путь в блок, который сваливает процесс, или сообщает о корректности введенного ключа/лицензии, итд.
Удивительно, но KLEE точно так же может находить и бэкдоры.

KLEE также называется ``KLEE Symbolic Virtual Machine'' -- этим создатели хотят подчеркнуть, что KLEE это \ac{VM},
которая исполняет код символьно, а не численно (как обычный \ac{CPU}).

Снова расширим наш крохотный пример.
Мы хотим найти пятницу 13-го. Чтобы всё упростить, ограничим себя 1970-м годом.
Найдем все 12 13-х дней 70-го года:

\lstinputlisting{symbolic/6_TGIF/13th.txt}

Скрипт, проверяющий, является ли текущая дата пятницей 13-го:

\begin{lstlisting}
input=...
SECS_DAY=24*60*60
dayno = input / SECS_DAY
wday = (dayno + 4) % 7
print wday
if wday==5:
    print "Thanks God, it's Friday!"
 
    if dayno in [13,44,72,103,133,164,194,225,256,286,317,347]:
        print "Friday 13th"
\end{lstlisting}

Чтобы добиться исполнения второго "print"-а, мы должны удовлетворить два констрайнта:

\lstinputlisting{symbolic/6_TGIF/Z3_solve2.py}

И это тоже легкая задача для Z3:

\begin{lstlisting}
 % python Z3_solve2.py
[dayno = 316, x = 27302400]
 % date --date='@27302400'
Fri Nov 13 03:00:00 MSK 1970
\end{lstlisting}

Это UNIX-дата, для которой оба констрайнта удовлетворены: 13-ое ноября 1970, пятница.

\subsubsection{Деление на ноль}

Если деление на ноль не \textit{обернуто} проверкой, и исключение не было обработано, это может свалить процесс.

Будем вычислять простое выражение $\frac{x}{2y + 4z - 12}$.
Можем добавить предупреждение в метод \TT{\_\_div\_\_}:

\lstinputlisting{symbolic/7_div/1.py}

\dots так что он будет сообщать об опасных состояниях и условиях:

\begin{lstlisting}
warning: division by zero if (((y*2)+(z*4))-12)==0
(x/(((y*2)+(z*4))-12))
\end{lstlisting}

Уравнение легко решить, попробуем на этот раз Wolfram Mathematica:

\begin{lstlisting}
In[]:= FindInstance[{(y*2 + z*4) - 12 == 0}, {y, z}, Integers]
Out[]= {{y -> 0, z -> 3}}
\end{lstlisting}

Эти значения для $y$ и $z$ также можно назвать ``inputs of death''.

% FIXME translation
\subsubsection{Сортировка слиянием (merge sort)}

Как работает сортировка слиянием (merge sort)?
Я скопипастил Питоновский исходник из rosettacode.com, почти без изменений:

\lstinputlisting{symbolic/8_sorting/1.py}

Но здесь есть ф-ция которая сравнивает элементы.
Очевидно, без нее ничего работать не будет.

Так что мы будем хранить и выражения для каждого элемента, и численные значения.
В итоге выведутся оба.
Но когда значения будут сравниваться, будут использоваться только численные части.

Результат:

\lstinputlisting{symbolic/8_sorting/result.txt}

\subsubsection{Расширение класса Expr}

Это наверное бессмысленно, но тем не менее, легко расширить класс Expr для поддержки \ac{AST} вместо обычных строк.
Можно также добавить шаги для сокращения выражения (как я показал в игрушечном декомпиляторе: \ref{toy_decompiler}).
Может быть, кто-то захочет сделать это как упражнение.
Кстати, мой игрушечный декомпилятор может использоваться как простейший symbolic engine --- просто подавайте ему на вход
все инструкции, и он будет следить за состоянием каждого регистра.

\subsubsection{Вывод}

Ради демонстрации, я всё сделал настолько просто, насколько это возможно.
Но реальность всегда намного жестче и неудобнее, так что всё это не нужно воспринимать как серебряную пулю.

Файлы использованные в этой части: \url{https://github.com/DennisYurichev/SAT_SMT_article/tree/master/symbolic}.



\subsection{Дальнейшее чтение}

% TODO year, URL
Robert W. Floyd --- Assigning meaning to programs.

James C. King --- Symbolic Execution and Program Testing
\footnote{\url{https://yurichev.com/mirrors/king76symbolicexecution.pdf}}

}


\section{KLEE}

% subsections:
\subsection{School-level equation}

Let's revisit school-level system of equations from (\ref{eq2_SMT}).

We will force KLEE to find a path, where all the constraints are satisfied:

\lstinputlisting{KLEE/klee_eq1.c}

% FIXME:
\begin{lstlisting}
\$ clang -emit-llvm -c -g klee_eq.c
...

\$ klee klee_eq.bc
KLEE: output directory is "/home/klee/klee-out-93"
KLEE: WARNING: undefined reference to function: klee_assert
KLEE: WARNING ONCE: calling external: klee_assert(0)
KLEE: ERROR: /home/klee/klee_eq.c:18: failed external call: klee_assert
KLEE: NOTE: now ignoring this error at this location

KLEE: done: total instructions = 32
KLEE: done: completed paths = 1
KLEE: done: generated tests = 1
\end{lstlisting}

Let's find out, where \TT{klee\_assert()} has been triggered:

% FIXME:
\begin{lstlisting}
\$ ls klee-last | grep err
test000001.external.err

\$ ktest-tool --write-ints klee-last/test000001.ktest
ktest file : 'klee-last/test000001.ktest'
args       : ['klee_eq.bc']
num objects: 3
object    0: name: b'circle'
object    0: size: 4
object    0: data: 5
object    1: name: b'square'
object    1: size: 4
object    1: data: 2
object    2: name: b'triangle'
object    2: size: 4
object    2: data: 1
\end{lstlisting}

This is indeed correct solution to the system of equations.

KLEE has intrinsic \TT{klee\_assume()} which tells KLEE to cut path if some constraint is not true.
So we can rewrite our example in such cleaner way:

\lstinputlisting{KLEE/klee_eq2.c}



\subsection{Zebra puzzle (\ac{AKA} Einstein puzzle)}
\label{zebra_SMT}

Zebra puzzle is a popular puzzle, defined as follows:

% FIXME remove paragraph at first line
\begin{framed}
\begin{quotation}
1.There are five houses.\\
2.The Englishman lives in the red house.\\
3.The Spaniard owns the dog.\\
4.Coffee is drunk in the green house.\\
5.The Ukrainian drinks tea.\\
6.The green house is immediately to the right of the ivory house.\\
7.The Old Gold smoker owns snails.\\
8.Kools are smoked in the yellow house.\\
9.Milk is drunk in the middle house.\\
10.The Norwegian lives in the first house.\\
11.The man who smokes Chesterfields lives in the house next to the man with the fox.\\
12.Kools are smoked in the house next to the house where the horse is kept.\\
13.The Lucky Strike smoker drinks orange juice.\\
14.The Japanese smokes Parliaments.\\
15.The Norwegian lives next to the blue house.\\
\\
Now, who drinks water? Who owns the zebra?\\
\\
In the interest of clarity, it must be added that each of the five houses is painted a different color, and their inhabitants are of different national extractions, own different pets, drink different beverages and smoke different brands of American cigarets [sic]. One other thing: in statement 6, right means your right.
\end{quotation}
\end{framed}
( \url{https://en.wikipedia.org/wiki/Zebra_Puzzle} ) \\
\\
It's a very good example of constraint satisfaction problem (CSP). % FIXME \ac

We would encode each entity as integer variable, representing number of house.

Then, to define that Englishman lives in red house, we will define this constraint: \TT{Englishman == Red}, meaning that number of a house where Englishmen resides and where tea is drunk is the same.

To define that Norwegian lives next to the blue house, we don't realy know, if it is at left side of blue house or at right side, but we know that house numbers are different by just 1.
So we will define this constraint: \TT{Norwegian==Blue-1 OR Norwegian==Blue+1}.

We will also need to limit all house numbers, so they will be in range of 1..5.

We will also use \TT{Distinct} to show that all various entities of the same type are all has different house numbers.

\lstinputlisting{SMT/zebra.py}

When we run it, we got correct result:

\begin{lstlisting}
sat
[Snails = 3,
 Blue = 2,
 Ivory = 4,
 OrangeJuice = 4,
 Parliament = 5,
 Yellow = 1,
 Fox = 1,
 Zebra = 5,
 Horse = 2,
 Dog = 4,
 Tea = 2,
 Water = 1,
 Chesterfield = 2,
 Red = 3,
 Japanese = 5,
 LuckyStrike = 4,
 Norwegian = 1,
 Milk = 3,
 Kools = 1,
 OldGold = 3,
 Ukrainian = 2,
 Coffee = 5,
 Green = 5,
 Spaniard = 4,
 Englishman = 3]
 \end{lstlisting}


\subsection{Sudoku}

I've also rewritten Sudoku example (\ref{sudoku_SMT}) for KLEE:

\lstinputlisting[numbers=left]{KLEE/klee_sudoku_or1.c}

Let's run it:

% FIXME:
\begin{lstlisting}
\$ clang -emit-llvm -c -g klee_sudoku_or1.c
...

\$ time klee klee_sudoku_or1.bc
KLEE: output directory is "/home/klee/klee-out-98"
KLEE: WARNING: undefined reference to function: klee_assert
KLEE: WARNING ONCE: calling external: klee_assert(0)
KLEE: ERROR: /home/klee/klee_sudoku_or1.c:93: failed external call: klee_assert
KLEE: NOTE: now ignoring this error at this location

KLEE: done: total instructions = 7512
KLEE: done: completed paths = 161
KLEE: done: generated tests = 161

real    3m44.111s
user    3m43.319s
sys     0m0.951s
\end{lstlisting}

Now this is really slower (on my Intel Core i3-3110M 2.4GHz notebook) in comparison to Z3Py solution (\ref{sudoku_SMT}).

But the answer is correct:

% FIXME:
\begin{lstlisting}
\$ ls klee-last | grep err
test000161.external.err

\$ ktest-tool --write-ints klee-last/test000161.ktest
ktest file : 'klee-last/test000161.ktest'
args       : ['klee_sudoku_or1.bc']
num objects: 1
object    0: name: b'cells'
object    0: size: 81
object    0: data: b'\x01\x04\x05\x03\x02\x07\x06\t\x08\x08\x03\t\x06\x05\x04\x01\x02\x07\x06\x07\x02\t\x01\x08\x05\x04\x03\x04\t\x06\x01\x08\x05\x03\x07\x02\x02\x01\x08\x04\x07\x03\t\x05\x06\x07\x05\x03\x02\t\x06\x04\x08\x01\x03\x06\x07\x05\x04\x02\x08\x01\t\t\x08\x04\x07\x06\x01\x02\x03\x05\x05\x02\x01\x08\x03\t\x07\x06\x04'
\end{lstlisting}

% FIXME backslash
\TT{\\t} is character with ASCII code of 9 in C/C++, and KLEE attempts to treat byte array as C/C++ string, so it shows some values in such way.
We can just remember that there is 9 at the each place where we see \TT{\\t}.
The solution, while not properly formatted, correct indeed. \\
\\
By the way, at lines 42, 43 you may see how we tell to KLEE that all array elements must be within some limits.
If we comment these lines out, we've got this:

% FIXME:
\begin{lstlisting}
\$ time klee klee_sudoku_or1.bc
KLEE: output directory is "/home/klee/klee-out-100"
KLEE: WARNING: undefined reference to function: klee_assert
KLEE: ERROR: /home/klee/klee_sudoku_or1.c:51: overshift error
KLEE: NOTE: now ignoring this error at this location
KLEE: ERROR: /home/klee/klee_sudoku_or1.c:51: overshift error
KLEE: NOTE: now ignoring this error at this location
KLEE: ERROR: /home/klee/klee_sudoku_or1.c:51: overshift error
KLEE: NOTE: now ignoring this error at this location
...
\end{lstlisting}

KLEE warns us that shift value at line 51 is too big.
Indeed, KLEE may try all byte values up to 255 (0xFF), which are pointless to use there, and may indicate error or bug, so KLEE warns about it.\\
\\
Now let's use \TT{klee\_assume()} again:

\lstinputlisting{KLEE/klee_sudoku_or2.c}

% FIXME:
\begin{lstlisting}
\$ time klee klee_sudoku_or2.bc
KLEE: output directory is "/home/klee/klee-out-99"
KLEE: WARNING: undefined reference to function: klee_assert
KLEE: WARNING ONCE: calling external: klee_assert(0)
KLEE: ERROR: /home/klee/klee_sudoku_or2.c:93: failed external call: klee_assert
KLEE: NOTE: now ignoring this error at this location

KLEE: done: total instructions = 7119
KLEE: done: completed paths = 1
KLEE: done: generated tests = 1

real    0m35.312s
user    0m34.945s
sys     0m0.318s
\end{lstlisting}

That works much faster: perhaps KLEE indeed handle this intrinsic in a special way.
And, as we see, the only one path is generated (one we actually interesting in it) instead of 161.

It's still much slower than Z3Py solution, though.


\input{KLEE/UNIXdatetime.tex}
\input{KLEE/base64.tex}
\subsection{CRC} % FIXME full name

\subsubsection{Buffer alteration case \#1}

Sometimes, you need to alter a piece of data which is \textit{protected} by some kind of checksum or \ac{CRC}, and you can't change checksum or CRC value, but can alter piece of data so that checksum will remain the same.

Let's pretend, we've got a piece of data with ``Hello, world!'' string at the beginning and ``and goodbye'' string at the end.
We can alter 14 characters at the middle, but for some reason, they must be in \textit{a..z} limits, but we can put any characters there.
CRC64 of the whole block must be \TT{0x12345678abcdef12}.

Let's see\footnote{There are several slightly different CRC64 implementations, the one I use here can also be different from popular ones.}:

\lstinputlisting{KLEE/klee_CRC64.c}

Since our code uses memcmp() standard C/C++ function, we need to add \TT{--libc=uclibc} switch, so KLEE will use its own uClibc % FIXME check spelling
implementation. % \ref{} -> closed programs

% FIXME:
\begin{lstlisting}
\$ clang -emit-llvm -c -g klee_CRC64.c

\$ time klee --libc=uclibc klee_CRC64.bc
\end{lstlisting}

It takes about 1 minute (on my XXX) and we getting this:

% FIXME:
\begin{lstlisting}
...
real    0m52.643s
user    0m51.232s
sys     0m0.239s
...
\$ ls klee-last | grep err
test000001.user.err
test000002.user.err
test000003.user.err
test000004.external.err

\$ ktest-tool --write-ints klee-last/test000004.ktest
ktest file : 'klee-last/test000004.ktest'
args       : ['klee_CRC64.bc']
num objects: 1
object    0: name: b'buf'
object    0: size: 46
object    0: data: b'Hello, world!.. qqlicayzceamyw ... and goodbye'
\end{lstlisting}

Maybe it's slow, but definitely faster than bruteforce.
Indeed, $log_2{26^{14}} \approx 65.8$
which is close to 64 bits.
In other words, one need $\approx 14$ latin characters to encode 64 bits.
And KLEE + \ac{SMT} solver needs 64 bits at some place it can alter to make final CRC64 value equal to what we defined.

I tried to reduce length of the \textit{middle block} to 13 characters: no luck for KLEE then, it has no space enough.

\subsubsection{Buffer alteration case \#2}

I went sadistic: what if the buffer must contain the CRC64 value which, after calculation of CRC64, will result in the same value?
Fascinately, % FIXME check spelling
KLEE can solve this.
The buffer will have the following format:

% FIXME:
\begin{lstlisting}
Hello, world! <8-bytes (64-bit value)> and goodbye <6 more bytes>
\end{lstlisting}

% FIXME:
\begin{lstlisting}
int main()
{
#define HEAD_STR "Hello, world!.. "
#define HEAD_SIZE strlen(HEAD_STR)
#define TAIL_STR " ... and goodbye"
#define TAIL_SIZE strlen(TAIL_STR)
// 8 bytes for 64-bit value:
#define MID_SIZE 8
#define BUF_SIZE HEAD_SIZE+TAIL_SIZE+MID_SIZE+6

	char buf[BUF_SIZE];
  
	klee_make_symbolic(buf, sizeof buf, "buf");

	klee_assume (memcmp (buf, HEAD_STR, HEAD_SIZE)==0);

	klee_assume (memcmp (buf+HEAD_SIZE+MID_SIZE, TAIL_STR, TAIL_SIZE)==0);
	
	uint64_t mid_value=*(uint64_t*)(buf+HEAD_SIZE);
	klee_assume (crc64 (0, buf, BUF_SIZE)==mid_value);

	klee_assert(0);

	return 0;
}
\end{lstlisting}

It works:

% FIXME:
\begin{lstlisting}
\$ time klee --libc=uclibc klee_CRC64.bc
...
real    5m17.081s
user    5m17.014s
sys     0m0.319s

\$ ls klee-last | grep err
test000001.user.err
test000002.user.err
test000003.external.err

\$ ktest-tool --write-ints klee-last/test000003.ktest
ktest file : 'klee-last/test000003.ktest'
args       : ['klee_CRC64.bc']
num objects: 1
object    0: name: b'buf'
object    0: size: 46
object    0: data: b'Hello, world!.. T+]\xb9A\x08\x0fq ... and goodbye\xb6\x8f\x9c\xd8\xc5\x00'
\end{lstlisting}

8 bytes between two strings is 64-bit value which equals to CRC64 of this whole block.
Again, it's faster than brute-force way to find it.
If to decrease last spare 6-byte buffer to 4 bytes or less, KLEE works so long so I've stopped it.

\subsubsection{Recovering input data for given CRC32 value of it}

I've always wanted to do so, but everyone knows this is impossible for input buffers larger than 4 bytes.
As my experiments show, it's still possible for tiny input buffers of data, constrained in some way.

The CRC32 value of 6-byte ``SILVER'' string is known: \TT{0xDFA3DFDD}.
KLEE can find this 6-byte string, if it knows that each byte of input buffer is in \textit{A..Z} limits:

\lstinputlisting[numbers=left]{KLEE/klee_SILVER.c}

% FIXME:
\begin{lstlisting}
\$ clang -emit-llvm -c -g klee_SILVER.c
...

\$ klee klee_SILVER.bc
...

\$ ls klee-last | grep err
test000013.external.err

\$ ktest-tool --write-ints klee-last/test000013.ktest
ktest file : 'klee-last/test000013.ktest'
args       : ['klee_SILVER.bc']
num objects: 1
object    0: name: b'str'
object    0: size: 6
object    0: data: b'SILVER'
\end{lstlisting}

Still, it's no magic: if to remove condition at lines 23..25 (i.e., if to relax constraints),
KLEE will produce some other string, which will be still correct for the CRC32 value given.

It works, because 6 Latin characters in \textit{A..Z} limits contain $\approx 28.2$ bits:
$log_2{26^6} \approx 28.2$, which is even smaller value than 32.
In other words, the final CRC32 value holds enough bits to recover $\approx 28.2$ bits of input.

The input buffer can be even bigger, if each byte of it will be in even tighter % FIXME spelling
constraints (decimal digits, binary digits, etc).

\subsubsection{In comparison with other hashing algorithms}

Things are that easy for some other hashing algorithms like \textit{fletcher checksum}, % FIXME URL
but not for cryptographically secure ones (like MD5, SHA1, etc), they are protected from such simple cryptoanalysis. % FIXME \ref{} -> am.crypto


\input{KLEE/LZSS.tex}

\subsection{Exercise}

Here is my crackme/keygenme, which may be tricky, but easy to solve using KLEE:
\url{http://challenges.re/74/}.



\EN{\section{(Amateur) cryptography}
\label{crypto}

\subsection{\textit{Serious} cryptography}

Let's back to the method we previously used (\ref{symbolic_exec}) to construct expressions using running Python function.

We can try to build expression for the output of XXTEA encryption algorithm:

\lstinputlisting{crypto/xxtea.py}

A key is choosen according to input data, and, obviously, we can't know it during symbolic execution, so we leave expression like \TT{k[...]}.

Now results for just one round, for each of 4 outputs:

\lstinputlisting{crypto/1round.txt}

Somehow, size of expression for each subsequent output is bigger. I hope I haven't been mistaken?
And this is just for 1 round.
For 2 rounds, size of all 4 expression is $\approx 970KB$.
For 3 rounds, this is $\approx 115MB$.
For 4 rounds, I have not enough RAM on my computer.
Expressions \textit{exploding} exponentially.
And there are 19 rounds.
You can weigh it.

Perhaps, you can simplify these expressions: there are a lot of excessive parenthesis,
but I'm highly pessimistic, cryptoalgorithms constructed in such a way to not have any spare operations.

In order to crack it, you can use these expressions as system of equation and try to solve it using SMT-solver.
This is called ``algebraic attack''.

In other words, theoretically, you can build system of equation like this: $MD5(x)=12341234...$,
but expressions are so huge so it's impossible to solve them.
Yes, cryptographers are fully aware of this and one of the goals of the successful cipher is
to make expressions as big as possible, using resonable time and size of algorithm.

Nevertheless, you can find numerous papers about breaking these cryptosystems with reduced number of rounds:
when expression isn't \textit{exploded} yet, sometimes it's possible.
This cannot be applied in practice, but such experience has some interesting theoretical results.

\subsubsection{Attempts to break ``serious'' crypto}

CryptoMiniSat itself exist to support XOR operation, which is ubiquitous in cryptography.

\begin{itemize}
\item Bitcoin mining with SAT solver: \url{http://jheusser.github.io/2013/02/03/satcoin.html}, \url{https://github.com/msoos/sha256-sat-bitcoin}.

\item \href{http://2015.phdays.ru/program/dev/40400/}{Alexander Semenov, attempts to break A5/1, etc. (Russian presentation)}

\item \href{https://yurichev.com/mirrors/SAT_SMT_crypto/thesis-output.pdf}{Vegard Nossum - SAT-based preimage attacks on SHA-1}

\item \href{https://yurichev.com/mirrors/SAT_SMT_crypto/166.pdf}{Algebraic Attacks on the Crypto-1 Stream Cipher in MiFare Classic and Oyster Cards}

\item \href{https://yurichev.com/mirrors/SAT_SMT_crypto/Attacking-Bivium-Using-SAT-Solvers.pdf}{Attacking Bivium Using SAT Solvers}

\item \href{https://yurichev.com/mirrors/SAT_SMT_crypto/Extending_SAT_2009.pdf}{Extending SAT Solvers to Cryptographic Problems}

\item \href{https://yurichev.com/mirrors/SAT_SMT_crypto/sat-hash.pdf}{Applications of SAT Solvers to Cryptanalysis of Hash Functions}

\item \href{https://yurichev.com/mirrors/SAT_SMT_crypto/slidesC2DES.pdf}{Algebraic-Differential Cryptanalysis of DES}

\end{itemize}

\subsection{Amateur cryptography}

This is what you can find in serial numbers, license keys, executable file packers, \ac{CTF}, malware, etc.
Sometimes even ransomware (but rarely nowadays, in 2017).

Amateur cryptography is often can be broken using SMT solver, or even KLEE.

Amateur cryptography is usually based not on theory, but on visual complexity:
if its creator getting results which are seems chaotic enough, often, one stops to improve it further.
This is security not even on obscurity, but on chaotic mess.
This is also sometimes called ``The Fallacy of Complex Manipulation''
(see \href{https://tools.ietf.org/html/rfc4086}{RFC4086}).

Devising your own cryptoalgorithm is a very tricky thing to do.
This can be compared to devising your own \ac{PRNG}.
Even famous Donald Knuth in 1959 constructed one, and it was visually very complex,
but, as it turns out in practice, it has very short cycle of length 3178.
[See also: The Art of Computer Programming vol.II page 4, (1997).]

The very first problem is that making an algorithm which can generate very long expressions is tricky thing itself.
Common error is to use operations like XOR and rotations/permutations, which can't help much.
Even worse: some people think that XORing a value several times can be better, like: $(x \oplus 1234) \oplus 5678$.
Obviously, these two XOR operations (or more precisely, any number of it) can be reduced to a single one.
Same story about applied operations like addition and subtraction---they all also can be reduced to single one.

Real cryptoalgorithms, like IDEA, can use several operations from different groups, like XOR, addition and multiplication.
Applying them all in specific order will make resulting expression irreducible.

When I prepared this part, I tried to make an example of such amateur hash function:

\lstinputlisting{crypto/1.c}

KLEE can break it with little effort.
Functions of such complexity is common in shareware, which checks license keys, etc.

Here is how we can make its work harder by making rotations dependent of inputs,
and this makes number of possible inputs much, much bigger:

\lstinputlisting{crypto/2.c}

Addition (or \href{https://yurichev.com/blog/modulo/}{modular addition}, as cryptographers say) can make thing even harder:

\lstinputlisting{crypto/3.c}

As en exercise, you can try to make a block cipher which KLEE wouldn't break.
This is quite sobering experience.
But even if you can, this is not a panacea, there is an array of other cryptoanalytical methods to break it.

Summary: if you deal with amateur cryptography, you can always give KLEE and SMT solver a try.
Even more: sometimes you have only decryption function, and if algorithm is simple enough,
KLEE or SMT solver can reverse things back.

One fun thing to mention: if you try to implement amateur cryptoalgorithm in Verilog/VHDL language to run it on \ac{FPGA},
maybe in brute-force way,
you can find that \ac{EDA} tools can optimize many things during synthesis
(this is the word they use for ``compilation'') and can leave cryptoalgorithm much smaller/simpler than it was.
Even if you try to implement DES algorithm \textit{in bare metal} with a fixed key,
Altera Quartus can optimize first round of it and make it smaller than others.

\subsubsection{Bugs}

Another prominent feature of amateur cryptography is bugs.
Bugs here often left uncaught because output of encrypting function visually looked ``good enough'' or ``obfuscated enough'',
so a developer stopped to work on it.

This is especially feature of hash functions, because when you work on block cipher, you have to do two functions
(encryption/decryption), while hashing function is single.

Weirdest ever amateur encryption algorithm I once saw, encrypted only odd bytes of input block, while even bytes
left untouched, so the input plain text has been partially seen in the resulting encrypted block.
It was encryption routine used in license key validation.
Hard to believe someone did this on purpose.
Most likely, it was just an unnoticed bug.

\subsubsection{XOR ciphers}

Simplest possible amateur cryptography is just application of XOR operation using some kind of table.
Maybe even simpler. This is a real algorithm I once saw:

\begin{lstlisting}
for (i=0; i<size; i++)
    buf[i]=buf[i]^(31*(i+1));
\end{lstlisting}

This is not even encryption, rather concealing or hiding.

\subsubsection{Other features}

\textbf{Tables} There are often table(s) with pseudorandom data, which is/are used chaotically.

\textbf{Checksumming} End-users can have proclivity to changing license codes, serial numbers, etc., with a hope
this could affect behaviour of software.
So there is often some kind of checksum: starting at simple summing and \ac{CRC}.
This is close to \ac{MAC} in real cryptography.

\subsubsection{Examples}

\begin{itemize}

\item A popular FLEXlm license manager was based on a simple amateur cryptoalgorithm
(before they switched to \ac{ECC}), which can be cracked easily.

\item Pegasus Mail Password Decoder: \url{http://phrack.org/issues/52/3.html} -
a very typical example.

\item You can find a lot of blog posts about breaking \ac{CTF}-level crypto using Z3, etc.
Here is one of them: \url{http://doar-e.github.io/blog/2015/08/18/keygenning-with-klee/}.

\item Another: \href{http://blog.cr4.sh/2015/03/automated-algebraic-cryptanalysis-with.html}{Automated algebraic cryptanalysis with OpenREIL and Z3}.
By the way, this solution tracks state of each register at each EIP/RIP,
this is almost the same as \ac{SSA}, which is heavily used in compiers and worth learning.

\item Many examples of amateur cryptography I've taken from an old Fravia website:
\url{https://yurichev.com/mirrors/amateur_crypto_examples_from_Fravia/}.

\end{itemize}

% subsection
\subsection{Case study: simple hash function}

(This piece of text was initially added to my ``Reverse Engineering for Beginners'' book (\url{beginners.re}) at March 2014)
\footnote{This example was also used by Murphy Berzish in his lecture about \ac{SAT} and \ac{SMT}:
\url{http://mirror.csclub.uwaterloo.ca/csclub/mtrberzi-sat-smt-slides.pdf},
\url{http://mirror.csclub.uwaterloo.ca/csclub/mtrberzi-sat-smt.mp4}}.

Here is one-way hash function, that converted a 64-bit value to another and we need to try to reverse its flow back.

\subsubsection{Manual decompiling}

Here its assembly language listing in IDA:

\lstinputlisting{crypto/hash/algo_1.asm}

The example was compiled by GCC, so the first argument is passed in ECX.

If you don't have Hex-Rays, or if you distrust to it, you can try to reverse this code manually.
One method is to represent the CPU registers as local C variables and replace each instruction by
a one-line equivalent expression, like:

\lstinputlisting{crypto/hash/algo_2.c}

If you are careful enough, this code can be compiled and will even work in the same way as the original.

Then, we are going to rewrite it gradually, keeping in mind all registers usage.
Attention and focus is very important here---any tiny typo may ruin all your work!

Here is the first step:

\lstinputlisting{crypto/hash/algo_3.c}

Next step:

\lstinputlisting{crypto/hash/algo_4.c}

We can spot the division using multiplication.
Indeed, let's calculate the divider in Wolfram Mathematica:

\begin{lstlisting}[caption=Wolfram Mathematica]
In[1]:=N[2^(64 + 5)/16^^8888888888888889]
Out[1]:=60.
\end{lstlisting}

We get this:

\lstinputlisting{crypto/hash/algo_5.c}

One more step:

\lstinputlisting{crypto/hash/algo_6.c}

By simple reducing, we finally see that it's calculating the remainder, not the quotient:

\lstinputlisting{crypto/hash/algo_7.c}

We end up with this fancy formatted source-code:

\lstinputlisting{crypto/hash/algo_src.c}

Since we are not cryptoanalysts we can't find an easy way to generate the input value for some specific output value.
The rotate instruction's coefficients look frightening---it's a warranty that the function is not bijective,
it has collisions, or, speaking more simply, many inputs may be possible for one output.

Brute-force is not solution because values are 64-bit ones, that's beyond reality.

\subsubsection{Now let's use the Z3}

Still, without any special cryptographic knowledge, we may try to break this algorithm using Z3.

Here is the Python source code:

\lstinputlisting[numbers=left]{crypto/hash/1.py}

This is going to be our first solver.

We see the variable definitions on line 7.
These are just 64-bit variables.
\TT{i1..i6} are intermediate variables, representing the values in the registers between instruction executions.

Then we add the so-called constraints on lines 10..15.
The last constraint at 17 is the most important one: 
we are going to try to find an input value for which our algorithm will produce 10816636949158156260.

\textit{RotateRight, RotateLeft, URem}---are functions from the Z3 Python API, not related to Python language.

Then we run it:

\begin{lstlisting}
...>python.exe 1.py
sat
[i1 = 3959740824832824396,
 i3 = 8957124831728646493,
 i5 = 10816636949158156260,
 inp = 1364123924608584563,
 outp = 10816636949158156260,
 i4 = 14065440378185297801,
 i2 = 4954926323707358301]
 inp=0x12EE577B63E80B73
outp=0x961C69FF0AEFD7E4
\end{lstlisting}

``sat'' mean ``satisfiable'', i.e., the solver was able to find at least one solution.
The solution is printed in the square brackets.
The last two lines are the input/output pair in hexadecimal form.
Yes, indeed, if we run our function with \TT{0x12EE577B63E80B73} as input,
the algorithm will produce the value we were looking for.

But, as we noticed before, the function we work with is not bijective, so there may be other correct input values.
The Z3 is not capable of producing more than one result, but let's hack our example slightly, 
by adding line 19, which implies ``look for any other results than this'':

\lstinputlisting[numbers=left]{crypto/hash/2.py}

Indeed, it finds another correct result:

\begin{lstlisting}
...>python.exe 2.py
sat
[i1 = 3959740824832824396,
 i3 = 8957124831728646493,
 i5 = 10816636949158156260,
 inp = 10587495961463360371,
 outp = 10816636949158156260,
 i4 = 14065440378185297801,
 i2 = 4954926323707358301]
 inp=0x92EE577B63E80B73
outp=0x961C69FF0AEFD7E4
\end{lstlisting}

This can be automated.
Each found result can be added as a constraint and then the next result will be searched for.
Here is a slightly more sophisticated example:

\lstinputlisting[numbers=left]{crypto/hash/3.py}

We got:

\begin{lstlisting}
1364123924608584563
1234567890
9223372038089343698
4611686019661955794
13835058056516731602
3096040143925676201
12319412180780452009
7707726162353064105
16931098199207839913
1906652839273745429
11130024876128521237
15741710894555909141
6518338857701133333
5975809943035972467
15199181979890748275
10587495961463360371
results total= 16
\end{lstlisting}

So there are 16 correct input values for \TT{0x92EE577B63E80B73} as a result.

The second is 1234567890---it is indeed the value which was used by me originally while preparing this example.

Let's also try to research our algorithm a bit more.
Acting on a sadistic whim, let's find if there are any possible input/output pairs in 
which the lower 32-bit parts are equal to each other?

Let's remove the \textit{outp} constraint and add another, at line 17:

\lstinputlisting[numbers=left]{crypto/hash/4.py}

It is indeed so:

\begin{lstlisting}
sat
[i1 = 14869545517796235860,
 i3 = 8388171335828825253,
 i5 = 6918262285561543945,
 inp = 1370377541658871093,
 outp = 14543180351754208565,
 i4 = 10167065714588685486,
 i2 = 5541032613289652645]
 inp=0x13048F1D12C00535
outp=0xC9D3C17A12C00535
\end{lstlisting}

Let's be more sadistic and add another constraint: last 16 bits must be \TT{0x1234}:

\lstinputlisting[numbers=left]{crypto/hash/5.py}

Oh yes, this possible as well:

\begin{lstlisting}
sat
[i1 = 2834222860503985872,
 i3 = 2294680776671411152,
 i5 = 17492621421353821227,
 inp = 461881484695179828,
 outp = 419247225543463476,
 i4 = 2294680776671411152,
 i2 = 2834222860503985872]
 inp=0x668EEC35F961234
outp=0x5D177215F961234
\end{lstlisting}

Z3 works very fast and it implies that the algorithm is weak, it is not cryptographic at all
(like the most of the amateur cryptography).



}
\RU{\section{(Любительская) криптография}
\label{crypto}

\subsection{\textit{Серьезная} криптография}

Вернемся к раннее использованнму методу (\ref{symbolic_exec}), чтобы сконструировать выражения используя запущенную
Питоновскую ф-цию.

Можно найти выражения для всех четырех выходов алгоритма шифрования XXTEA:

\lstinputlisting{crypto/xxtea.py}

Ключ выбирается в зависимости от входных данных, и, очевидно, мы не знаем его во время символьного исполнения,
так что мы оставляем выражение вроде \TT{k[...]}.

Теперь результаты для одного раунда, для каждого из 4-х выходов:

\lstinputlisting{crypto/1round.txt}

Каким-то образом, выражение для каждого последующего выхода больше. Надеюсь, я нигде не ошибся?
И это просто для одного раунда.
Для двух раундов, размер всех 4-х выражений $\approx 970KB$.
Для трех, это $\approx 115MB$.
Для четырех, у меня не хватило памяти на моем компьютере.
Выражения \textit{взрываются} экспоненциально.
А здесь 19 раундов.
Можете ощутить вес.

Вероятно, вы можете упростить эти выражения: здесь очень много лишних скобок,
но я очень пессимистичен, криптоалгоритмы и создаются таким образом, чтобы не иметь лишних операций.

Чтобы взломать его, вы можете использовать эти выражения как систему уравнений и попытаться решить её при помощи SMT-солвера.
Это называется ``алгебраическая атака''.

Другими словами, теоретически, вы можете построить систему уравнений вроде: $MD5(x)=12341234...$,
но эти выражения настолько огромные, что решить это нельзя.
Да, криптографы прекрасно осведомлены об этом и одна из задач успешного шифра в том, чтобы сделать выражения
настолько большими, насколько это возможно, используя разумное время и размер алгоритма.

Тем не менее, вы можете найти много статей о взломе этих криптосистем учитывая сокращенное количество раундов:
пока вырежение еще не \textit{взорвалось}, иногда это возможно.
Это не применимо затем на практике, но подобный опыт имеет некоторые интересные теоретические результаты.

\subsubsection{Попытки взлома ``серьезных'' шифров}

CryptoMiniSat сам по себе существует для поддержки операции исключающего ИЛИ, которая активно используется
в криптографии.

\begin{itemize}
\item Bitcoin mining with SAT solver: \url{http://jheusser.github.io/2013/02/03/satcoin.html}, \url{https://github.com/msoos/sha256-sat-bitcoin}.

\item \href{http://2015.phdays.ru/program/dev/40400/}{Александр Семенов, попытки взлома A5/1, итд. (на русском)}

\item \href{https://yurichev.com/mirrors/SAT_SMT_crypto/thesis-output.pdf}{Vegard Nossum - SAT-based preimage attacks on SHA-1}

\item \href{https://yurichev.com/mirrors/SAT_SMT_crypto/166.pdf}{Algebraic Attacks on the Crypto-1 Stream Cipher in MiFare Classic and Oyster Cards}

\item \href{https://yurichev.com/mirrors/SAT_SMT_crypto/Attacking-Bivium-Using-SAT-Solvers.pdf}{Attacking Bivium Using SAT Solvers}

\item \href{https://yurichev.com/mirrors/SAT_SMT_crypto/Extending_SAT_2009.pdf}{Extending SAT Solvers to Cryptographic Problems}

\item \href{https://yurichev.com/mirrors/SAT_SMT_crypto/sat-hash.pdf}{Applications of SAT Solvers to Cryptanalysis of Hash Functions}

\item \href{https://yurichev.com/mirrors/SAT_SMT_crypto/slidesC2DES.pdf}{Algebraic-Differential Cryptanalysis of DES}

\end{itemize}

\subsection{Любительская криптография}

Это то, что вы можете найти в серийных номерах, ключах с лицензиями, запаковщиками исполняемых файлов, \ac{CTF},
малварь (зловреды), итд.
Иногда даже в ransomware (но в наше время (2017) редко).

Любительскую криптографию очень часто можно взломать используя SMT-солвер, или даже KLEE.

Любительская криптография обычно основывается не на теории, а на визуальной сложности:
если её создатель получает результаты, которые выглядят достаточно хаотичными, часто, он прекращает разрабатывать его далее.
Это даже не безопасность через запутанность (\textit{security through obscurity}),
а даже безопасность через хаотическую путанницу.
Иногда это называется ``The Fallacy of Complex Manipulation''
(см.также \href{https://tools.ietf.org/html/rfc4086}{RFC4086}).

Разработка своего собственного криптоалгоритма это не такая уж и простая вещь.
Это можно сравнить с разработкой своего собственного \ac{PRNG}.
Даже знаменитый Дональд Кнут в создал свой в 1959, и визуально он был очень сложным,
но как потом выяснилось на практике, у него был очень короткий цикл длиной 3178.
[См.также: The Art of Computer Programming том.II стр.4, (1997).]

Самая первая проблема это создание алгоритма, который может генерировать очень длинные выражения.
Частая ошибка это использование операций вроде исключающего ИЛИ и сдвигов/перестановок, что не очень сильно помогает.
Даже хуже: некоторые люди думают, что применение операции исключающего ИЛИ несколько раз может сделать лучше,
например: $(x \oplus 1234) \oplus 5678$.
Очевидно, эти две операции (точнее, любое их количество) можно сократить до одной.
Та же история с применением операций вроде сложения и вычитания --- они все могут быть сокращены до одной операции.

Настоящие криптоалгоритмы вроде IDEA могут использовать несколько операций из разных групп, как исключающее ИЛИ,
сложение и умножение.
Применение их всех в определенном порядке сделает итоговое выражение несократимым.

Когда я готовил эту часть, я попробовал сделать пример любительской хэш-функции:

\lstinputlisting{crypto/1.c}

KLEE может сломать её без особого труда.
Фцнкции такой сложности часто присутствуют в shareware, которые проверяют лицензионные ключи, итд.

А вот как мы можем сделать работу KLEE труднее используя сдвиги, зависимые от входов,
и это делает количество возможных входных значений намного больше:

\lstinputlisting{crypto/2.c}

Сложение (или, как говорят криптографы, \href{https://yurichev.com/blog/modulo/}{модульное сложение}) может всё усложнить
еще сильнее:

\lstinputlisting{crypto/3.c}

В качестве упражнения, можете попробовать сделать блочный шифр, который KLEE не сможет сломать.
Это очень отрезвляющий опыт.
Но даже если вы и сможете, это не панацея, у криптографов есть еще масса криптоаналитических методов.

Итог: если вы имеете дело с любительской криптографией, вы всегда можете попробовать KLEE и SMT-солвер.
И даже более того: иногда у вас есть только ф-ция для дешифрования, и если алгоритм достаточно прост,
при помощи KLEE или SMT-солвера, можно вернуть всё назад.

Еще одна смешная вещь: если вы пытаетесь реализовать любительский криптоалгоритм на языке Verilog/VHDL чтобы запустить
его на \ac{FPGA}, может быть, с целями брутфорса, вы можете обнаружить, что инструменты \ac{EDA} могут оптимизировать
его во время синтеза (это слово, которое они используют вместо ``компиляция''), и может оставить этот криптоалгоритм
в намного меньшем размере, работающим быстрее, чем это было в начале.
Даже если вы попытаетесь реализовать \textit{в железе} алгоритм DES с фиксированным ключом,
Altera Quartus может оптимизировать его первый раунд, и он будет немного меньше остальных.

\subsubsection{Ошибки}

Еще одна выдающаяся особенность любительской криптографии это ошибки.
Ошибки часто остаются невыявленными, потому что выход ф-ции шифрования визуально выглядит ``достаточно хорошим''
или ``достаточно запутанным'', так что создатель бросает работу над ним.

Это особенно справедливо для хэширующих ф-ций, потому что когда вы работаете над блочным шифром, вам нужны
две ф-ции (шифрование/дешифрование), в то время как хэширующая ф-ция одна.

Самый странный любительский криптоалгоритм виденный мною, шифровал только нечетные байты входного блока,
оставляя четные байты нетронутыми, так что входной текст частично присутствовал в итоговом зашифрованном блоке.
Это была ф-ция шифрования использованная в проверке лицензионного ключа.
Трудно поверить в то, что кто-то сделал это намеренно.
Скорее всего, это была просто незамеченная ошибка.

\subsubsection{XOR-шифры}

Самый простой любительский криптоалгоритм просто применяет исключающее ИЛИ используя некоторую таблицу.
В русском языке также применятся термин ``гаммирование''.
Может быть даже еще проще. Вот реальный алгоритм, который я однажды видел:

\begin{lstlisting}
for (i=0; i<size; i++)
    buf[i]=buf[i]^(31*(i+1));
\end{lstlisting}

Это даже и не шифрование, скорее, сокрытие или упрятывание.

\subsubsection{Другие особенности}

\textbf{Таблицы} Часто приствует таблица/таблицы с псевдослучайными данными, которая/которые хаотично используются.

\textbf{Контрольная сумма} Конечные пользователи имеют склонность изменять коды лицензий, серийные номера, итд,
в надежде, что это как-то повлияет на работу программы.
Так что часто присутствует некоторая контрольная сумма: начиная с простого суммирования и \ac{CRC}.
Это близко к \ac{MAC} в настоящей криптографии.

\subsubsection{Примеры}

\begin{itemize}

\item Популярный менеджер лицензий FLEXlm использовал простой любительский криптоалгоритм
(перед тем, как они переключились на \ac{ECC}), который легко можно было сломать.

\item Pegasus Mail Password Decoder: \url{http://phrack.org/issues/52/3.html} -
очень типичный пример.

\item Вы можете найти массу постов в блогах о взломе криптографии уровня \ac{CTF} используя Z3, итд.
Вот один из них: \url{http://doar-e.github.io/blog/2015/08/18/keygenning-with-klee/}.

\item Еще: \href{http://blog.cr4.sh/2015/03/automated-algebraic-cryptanalysis-with.html}{Automated algebraic cryptanalysis with OpenREIL and Z3}.
Кстати, это решение следит за состоянием каждого регистра на каждом EIP/RIP, а это почти то же самое, что и \ac{SSA},
которая активно применяется в компиляторах, и стоит изучения.

\item Массу примеров любительской криптографии я взял со старого сайта Fravia:
\url{https://yurichev.com/mirrors/amateur_crypto_examples_from_Fravia/}.

\end{itemize}

% subsection
\subsection{Пример: простая хэш-функция}

(Этот текст был впервые добавлен в мою книгу ``Reverse Engineering для начинающих'' (\url{beginners.re}) в марте 2014
\footnote{Этот пример также использовался Murphy Berzish в его лекции о \ac{SAT} и \ac{SMT}:
\url{http://mirror.csclub.uwaterloo.ca/csclub/mtrberzi-sat-smt-slides.pdf},
\url{http://mirror.csclub.uwaterloo.ca/csclub/mtrberzi-sat-smt.mp4}}.)

Вот необратимая хэш-функция, которая конвертирует одно 64-битное значение в другое,
и нам нужно попытаться развернуть её работу назад.

\subsubsection{Ручная декомпиляция}

Вот листинг на ассемблере в IDA:

\lstinputlisting{crypto/hash/algo_1.asm}

Пример был скомпилирован в GCC, so the first argument is passed in ECX.

Если вы не имеете Hex-Rays, либо вы не доверяете его результатам, мы можем попробовать
переписать всё это на Си вручную.
Один из методов, это представить регистры \ac{CPU} в виде локальных переменных Си и заменить каждую инструкцию
эквивалентным выражением, например:

\lstinputlisting{crypto/hash/algo_2.c}

Если быть достаточно аккуратным, этот код можно скомпилировать, и он даже будет работать, 
точно так же, как оригинальный.

Затем, будем переписывать его постепенно, не забывая об использовании регистров.
Внимание и фокусирование здесь крайне важно --- любая самая мелкая опечатка может испортить всю работу!

Первый шаг:

\lstinputlisting{crypto/hash/algo_3.c}

Следующий шаг:

\lstinputlisting{crypto/hash/algo_4.c}


Мы находим деление через умножение.
Действительно, найдем делитель в Wolfram Mathematica:

\begin{lstlisting}[caption=Wolfram Mathematica]
In[1]:=N[2^(64 + 5)/16^^8888888888888889]
Out[1]:=60.
\end{lstlisting}

Получаем:

\lstinputlisting{crypto/hash/algo_5.c}

Еще один шаг:

\lstinputlisting{crypto/hash/algo_6.c}

Простым сокращением, мы видим, что вычислялось вовсе не частное, а остаток от деления:

\lstinputlisting{crypto/hash/algo_7.c}

Заканчиваем на приятно отформатированном исходном коде:

\lstinputlisting{crypto/hash/algo_src.c}

Так как мы не криптоаналитики, мы не можем найти простой способ найти входное значение
для определенного выходного значения.
Коэффициенты инструкций сдвигов выглядят очень пугающе --- это гарантия что функция не биективная,
она скорее сюръективная,
она имеет коллизии, или, говоря проще, возможны несколько значений на входе для одного на выходе.

Брут-форс это тоже не решение, т.к., значения 64-битные, и это совершенно нереально.

\subsubsection{Попробуем Z3}

Но все же, без всяких специальных знаний из криптографии, мы можем попытаться взломать алгоритм при помощи Z3.

Вот исходный код на Питоне:

\lstinputlisting[numbers=left]{crypto/hash/1.py}

Это будет наш первый солвер.

На строке 7 мы видим объявление переменных.
Это просто 64-битные переменные.
\TT{i1..i6} это промежуточные переменные, отражающие значения в регистрах между исполнениями инструкций.

Потом добавляем т.н. констрайнты, в строках 10..15.
Самый последний констрайнт в строке 17 это наиболее важный: мы будем искать входное значение для
нашего алгоритма, при котором он выдаст на выходе 10816636949158156260.

\textit{RotateRight, RotateLeft, URem} --- это функции из Питоновского Z3 \ac{API} для описания выражений, 
они не связаны с ЯП Python.

Запускаем:

\begin{lstlisting}
...>python.exe 1.py
sat
[i1 = 3959740824832824396,
 i3 = 8957124831728646493,
 i5 = 10816636949158156260,
 inp = 1364123924608584563,
 outp = 10816636949158156260,
 i4 = 14065440378185297801,
 i2 = 4954926323707358301]
 inp=0x12EE577B63E80B73
outp=0x961C69FF0AEFD7E4
\end{lstlisting}

``sat'' означает ``satisfiable'', т.е. солвер нашел по крайней мере одно решение.
Решение выведено внутри квадратных скобок.
Две последние строки это пара входного/выходного значения в шестнадцатеричном виде.
Да, действительно, если мы запустим нашу функцию с
\TT{0x12EE577B63E80B73} на входе, алгоритм выдаст искомое значение.

Но, как мы заметили ранее, функция не биективная, так что тут могут быть и другие корректные входные значения.
Z3 SMT-солвер не выдает результаты больше одного, но мы можем хакнуть наш пример немного, 
добавив констрайнт в строке 19, означая, что мы ищем какие угодно другие результаты кроме этого:

\lstinputlisting[numbers=left]{crypto/hash/2.py}

Действительно, получаем еще один верный результат:

\begin{lstlisting}
...>python.exe 2.py
sat
[i1 = 3959740824832824396,
 i3 = 8957124831728646493,
 i5 = 10816636949158156260,
 inp = 10587495961463360371,
 outp = 10816636949158156260,
 i4 = 14065440378185297801,
 i2 = 4954926323707358301]
 inp=0x92EE577B63E80B73
outp=0x961C69FF0AEFD7E4
\end{lstlisting}

Это можно автоматизировать.
Каждый найденный результат можно добавлять в качестве констрайнта и искать следующий.
Пример немного сложнее:

\lstinputlisting[numbers=left]{crypto/hash/3.py}

Получаем:

\begin{lstlisting}
1364123924608584563
1234567890
9223372038089343698
4611686019661955794
13835058056516731602
3096040143925676201
12319412180780452009
7707726162353064105
16931098199207839913
1906652839273745429
11130024876128521237
15741710894555909141
6518338857701133333
5975809943035972467
15199181979890748275
10587495961463360371
results total= 16
\end{lstlisting}

Так что имеется 16 верных входных значений для \TT{0x92EE577B63E80B73} на выходе.

Второй это 1234567890 --- действительно, это значение было использовано изначально,
при подготовке этого примера.

Попробуем изучить алгоритм немного больше.
В порыве садистских желаний, попробуем найти, есть ли здесь какая-нибудь возможная пара входов/выходов,
в которых младшие 32-битные части равны друг другу?

Уберем констрайнт \textit{outp} и добавим другой, в строке 17:

\lstinputlisting[numbers=left]{crypto/hash/4.py}

И действительно:

\begin{lstlisting}
sat
[i1 = 14869545517796235860,
 i3 = 8388171335828825253,
 i5 = 6918262285561543945,
 inp = 1370377541658871093,
 outp = 14543180351754208565,
 i4 = 10167065714588685486,
 i2 = 5541032613289652645]
 inp=0x13048F1D12C00535
outp=0xC9D3C17A12C00535
\end{lstlisting}

Можем упражняться в садизме и далее: пусть последние 16-бит всегда будут \TT{0x1234}:

\lstinputlisting[numbers=left]{crypto/hash/5.py}

Это так же возможно:

\begin{lstlisting}
sat
[i1 = 2834222860503985872,
 i3 = 2294680776671411152,
 i5 = 17492621421353821227,
 inp = 461881484695179828,
 outp = 419247225543463476,
 i4 = 2294680776671411152,
 i2 = 2834222860503985872]
 inp=0x668EEC35F961234
outp=0x5D177215F961234
\end{lstlisting}

Z3 работает крайне быстро и это означает что алгоритм слаб, и вообще не относится к криптографическим 
(как и почти вся любительская криптография).



}


\EN{\section{\ac{SAT}-solvers}

SMT vs. SAT is like high level \ac{PL} vs. assembly language.
The latter can be much more efficient, but it's hard to program in it.

\subsection{CNF form}

\ac{CNF}\footnote{\url{https://en.wikipedia.org/wiki/Conjunctive_normal_form}} is a \textit{normal form}.

% TODO recheck
% TODO write abt it!
%\textit{normal form} is somewhat similar to polynomials in algebra. 
%What is polynomial?
%It is a standard way to express unsystematic equations like $2x \cdot x$ as $3x$ polynomial, 
%and so you will be able to apply some operations to polynomials like summing, etc.

Any boolean expression can be converted to \textit{normal form} and \ac{CNF} is one of them.
The \ac{CNF} expression is a bunch of clauses (sub-expressions) constisting of terms (variables), ORs and NOTs, 
all of which are then glueled together with AND into a full expression.
There is a way to memorize it: \ac{CNF} is ``AND of ORs'' (or ``product of sums'') and \ac{DNF} is ``OR of ANDs'' (or ``sum of products'').

Example is: $(\neg A \vee B) \wedge (C \vee \neg D)$.

$\vee$ stands for OR (logical disjunction\footnote{\url{https://en.wikipedia.org/wiki/Logical_disjunction}}), 
``+'' sign is also sometimes used for OR.

$\wedge$ stands for AND (logical conjunction\footnote{\url{https://en.wikipedia.org/wiki/Logical_conjunction}}).
It is easy to memorize: $\wedge$ looks like ``A'' letter.
``$\cdot$'' is also sometimes used for AND.

$\neg$ is negation (NOT).

% TODO A/B is the first clause, C/D is second

\subsection{Example: 2-bit adder}
\label{adder}

\ac{SAT}-solver is merely a solver of huge boolean equations in CNF form.
It just gives the answer, if there is a set of input values which can satisfy CNF expression, and what input values must be.

Here is a 2-bit adder for example:

\begin{figure}[ht!]
\centering
\includegraphics[scale=0.75]{SAT/adder_logisim.png}
\caption{2-bit adder circuit}
\end{figure}

The adder in its simplest form: it has no carry-in and carry-out, and it has 3 XOR gates and one AND gate.
Let's try to figure out, which sets of input values will force adder to set both two output bits?
By doing quick memory calculation, we can see that there are 4 ways to do so: $0+3=3$, $1+2=3$, $2+1=3$, $3+0=3$.
Here is also truth table, with these rows highlighted:

\newcommand{\HLcell}{\cellcolor{blue!25}}

\begin{center}
\begin{doublespace}
\noindent\(\begin{array}{l|llllll}
  & \text{aH} & \text{aL} & \text{bH} & \text{bL} & \text{qH} & \text{qL} \\
\hline
 \text{3+3 = 6 $\equiv $ 2 (mod 4)} & 1 & 1 & 1 & 1 & 1 & 0 \\
 \text{3+2 = 5 $\equiv $ 1 (mod 4)} & 1 & 1 & 1 & 0 & 0 & 1 \\
 \text{3+1 = 4 $\equiv $ 0 (mod 4)} & 1 & 1 & 0 & 1 & 0 & 0 \\
 \text{\HLcell{}3+0 = 3 $\equiv $ 3 (mod 4)} & \HLcell{}1 & \HLcell{}1 & \HLcell{}0 & \HLcell{}0 & \HLcell{}1 & \HLcell{}1 \\
 \text{2+3 = 5 $\equiv $ 1 (mod 4)} & 1 & 0 & 1 & 1 & 0 & 1 \\
 \text{2+2 = 4 $\equiv $ 0 (mod 4)} & 1 & 0 & 1 & 0 & 0 & 0 \\
 \text{\HLcell{}2+1 = 3 $\equiv $ 3 (mod 4)} & \HLcell{}1 & \HLcell{}0 & \HLcell{}0 & \HLcell{}1 & \HLcell{}1 & \HLcell{}1 \\
 \text{2+0 = 2 $\equiv $ 2 (mod 4)} & 1 & 0 & 0 & 0 & 1 & 0 \\
 \text{1+3 = 4 $\equiv $ 0 (mod 4)} & 0 & 1 & 1 & 1 & 0 & 0 \\
 \text{\HLcell{}1+2 = 3 $\equiv $ 3 (mod 4)} & \HLcell{}0 & \HLcell{}1 & \HLcell{}1 & \HLcell{}0 & \HLcell{}1 & \HLcell{}1 \\
 \text{1+1 = 2 $\equiv $ 2 (mod 4)} & 0 & 1 & 0 & 1 & 1 & 0 \\
 \text{1+0 = 1 $\equiv $ 1 (mod 4)} & 0 & 1 & 0 & 0 & 0 & 1 \\
 \text{\HLcell{}0+3 = 3 $\equiv $ 3 (mod 4)} & \HLcell{}0 & \HLcell{}0 & \HLcell{}1 & \HLcell{}1 & \HLcell{}1 & \HLcell{}1 \\
 \text{0+2 = 2 $\equiv $ 2 (mod 4)} & 0 & 0 & 1 & 0 & 1 & 0 \\
 \text{0+1 = 1 $\equiv $ 1 (mod 4)} & 0 & 0 & 0 & 1 & 0 & 1 \\
 \text{0+0 = 0 $\equiv $ 0 (mod 4)} & 0 & 0 & 0 & 0 & 0 & 0 \\
\end{array}\)
\end{doublespace}
\end{center}


Let's find, what \ac{SAT}-solver can say about it?

First, we should represent our 2-bit adder as \ac{CNF} expression.

Using Wolfram Mathematica, we can express 1-bit expression for both adder outputs:\\
\\
\textbf{\texttt{In[]:=AdderQ0[aL$\_$,bL$\_$]=Xor[aL,bL]}} \\
\textbf{\texttt{Out[]:=aL $\veebar$ bL}} \\
\\
\textbf{\texttt{In[]:=AdderQ1[aL$\_$,aH$\_$,bL$\_$,bH$\_$]=Xor[And[aL,bL],Xor[aH,bH]]}} \\
\textbf{\texttt{Out[]:=aH $\veebar$ bH $\veebar$ (aL \&\& bL)}} \\
\\
We need such expression, where both parts will generate 1's.
Let's use Wolfram Mathematica find all instances of such expression (I glueled both parts with And): \\
\\
\textbf{\texttt{In[]:=Boole[SatisfiabilityInstances[And[AdderQ0[aL,bL],AdderQ1[aL,aH,bL,bH]],\{aL,aH,bL,bH\},4]]}} \\
\textbf{\texttt{Out[]:=\{1,1,0,0\},\{1,0,0,1\},\{0,1,1,0\},\{0,0,1,1\}}} \\
\\
Yes, indeed, Mathematica says, there are 4 inputs which will lead to the result we need.
So, Mathematica can also be used as \ac{SAT} solver.

Nevertheless, let's proceed to \ac{CNF} form. Using Mathematica again, let's convert our expression to \ac{CNF} form:\\
\\
\textbf{\texttt{In[]:=cnf=BooleanConvert[And[AdderQ0[aL,bL],AdderQ1[aL,aH,bL,bH]],``CNF'']}} \\
\textbf{\texttt{Out[]:=(!aH $\|$ !bH) \&\& (aH $\|$ bH) \&\& (!aL $\|$ !bL) \&\& (aL $\|$ bL)}} \\
\\
Looks more complex. The reason of such verbosity is that \ac{CNF} form doesn't allow XOR operations.
% FIXME: TeX form of the expression!

\subsubsection{MiniSat}

For the starters, we can try MiniSat\footnote{\url{http://minisat.se/MiniSat.html}}.
The standard way to encode \ac{CNF} expression for MiniSat is to enumerate all OR parts at each line.
Also, MiniSat doesn't support variable names, just numbers.
Let's enumerate our variables: 1 will be aH, 2 -- aL, 3 -- bH, 4 -- bL.

Here is what I've got when I converted Mathematica expression to the MiniSat input file:

\begin{lstlisting}
p cnf 4 4
-1 -3 0
1 3 0
-2 -4 0
2 4 0
\end{lstlisting}

Two 4's at the first lines are number of variables and number of clauses respectively.
There are 4 lines then, each for each OR clause.
Minus before variable number meaning that the variable is negated.
Absence of minus -- not negated.
Zero at the end is just terminating zero, meaning end of the clause.

In other words, each line is OR-clause with optional negations,
and the task of MiniSat is to find such set of input, which can satisfy all lines in the input file.

That file I named as \textit{adder.cnf} and now let's try MiniSat:

\begin{lstlisting}
% minisat -verb=0 adder.cnf results.txt
SATISFIABLE
\end{lstlisting}

The results are in \textit{results.txt} file:

\begin{lstlisting}
SAT
-1 -2 3 4 0
\end{lstlisting}

This means, if the first two variables (aH and aL) will be \textit{false}, and the last two variables (bH and bL) will be set to \textit{true},
the whole \ac{CNF} expression is satisfiable.
Seems to be true: if bH and bL are the only inputs set to \textit{true}, both resulting bits are also has \textit{true} states.

Now how to get other instances? \ac{SAT}-solvers, like \ac{SMT} solvers, produce only one solution (or \textit{instance}).

MiniSat uses \ac{PRNG} and its initial seed can be set explicitely. I tried different values, but result is still the same.
Nevertheless, CryptoMiniSat in this case was able to show all possible 4 instances, in chaotic order, though.
So this is not very robust way.

Perhaps, the only known way is to negate solution clause and add it to the input expression.
We've got \TT{-1 -2 3 4}, 
now we can negate all values in it (just toggle minuses: \TT{1 2 -3 -4}) and add it to the end of the input file:

\begin{lstlisting}
p cnf 4 5
-1 -3 0
1 3 0
-2 -4 0
2 4 0
1 2 -3 -4
\end{lstlisting}

Now we've got another result:

\begin{lstlisting}
SAT
1 2 -3 -4 0
\end{lstlisting}

This means, aH and aL must be both \textit{true} and bH and bL must be \textit{false}, to satisfy the input expression.
Let's negate this clause and add it again:

\begin{lstlisting}
p cnf 4 6
-1 -3 0
1 3 0
-2 -4 0
2 4 0
1 2 -3 -4
-1 -2 3 4 0
\end{lstlisting}

The result is:

\begin{lstlisting}
SAT
-1 2 3 -4 0
\end{lstlisting}

aH=false, aL=true, bH=true, bL=false. This is also correct, according to our truth table.

Let's add it again:

\begin{lstlisting}
p cnf 4 7
-1 -3 0
1 3 0
-2 -4 0
2 4 0
1 2 -3 -4
-1 -2 3 4 0
1 -2 -3 4 0
\end{lstlisting}

\begin{lstlisting}
SAT
1 -2 -3 4 0
\end{lstlisting}

\textit{aH=true, aL=false, bH=false, bL=true.} This is also correct.

This is fourth result. There are shouldn't be more. What if to add it?

\begin{lstlisting}
p cnf 4 8
-1 -3 0
1 3 0
-2 -4 0
2 4 0
1 2 -3 -4
-1 -2 3 4 0
1 -2 -3 4 0
-1 2 3 -4 0
\end{lstlisting}

Now MiniSat just says ``UNSATISFIABLE'' without any additional information in the resulting file.

Our example is tiny, but MiniSat can work with huge \ac{CNF} expressions.

\subsubsection{CryptoMiniSat}

XOR operation is absent in \ac{CNF} form, but crucial in cryptographical algorithms.
Simplest possible way to represent single XOR operation in \ac{CNF} form is:
$(\neg x \vee \neg y) \wedge (x \vee y)$ -- not that small expression, 
though, many XOR operations in single expression can be optimized better.

One significant difference between MiniSat and CryptoMiniSat is that
the latter supports clauses with XOR operations instead of ORs,
because CryptoMiniSat has aim to analyze crypto algorithms\footnote{\url{http://www.msoos.org/xor-clauses/}}.
XOR clauses are handled by CryptoMiniSat in a special way without translating to OR clauses.

You need just to prepend a clause with ``x'' in \ac{CNF} file and OR clause is then treated as XOR clause by CryptoMiniSat.
As of 2-bit adder, this smallest possible XOR-CNF expression can be used to find all inputs where both output adder bits are set:

$(aH \oplus bH) \wedge (aL \oplus bL)$

This is \TT{.cnf} file for CryptoMiniSat:

\begin{lstlisting}
p cnf 4 2
x1 3 0
x2 4 0
\end{lstlisting}

Now I run CryptoMiniSat with various random values to initialize its \ac{PRNG} \dots

\begin{lstlisting}
% cryptominisat4 --verb 0 --random 0 XOR_adder.cnf
s SATISFIABLE
v 1 2 -3 -4 0
% cryptominisat4 --verb 0 --random 1 XOR_adder.cnf
s SATISFIABLE
v -1 -2 3 4 0
% cryptominisat4 --verb 0 --random 2 XOR_adder.cnf
s SATISFIABLE
v 1 -2 -3 4 0
% cryptominisat4 --verb 0 --random 3 XOR_adder.cnf
s SATISFIABLE
v 1 2 -3 -4 0
% cryptominisat4 --verb 0 --random 4 XOR_adder.cnf
s SATISFIABLE
v -1 2 3 -4 0
% cryptominisat4 --verb 0 --random 5 XOR_adder.cnf
s SATISFIABLE
v -1 2 3 -4 0
% cryptominisat4 --verb 0 --random 6 XOR_adder.cnf
s SATISFIABLE
v -1 -2 3 4 0
% cryptominisat4 --verb 0 --random 7 XOR_adder.cnf
s SATISFIABLE
v 1 -2 -3 4 0
% cryptominisat4 --verb 0 --random 8 XOR_adder.cnf
s SATISFIABLE
v 1 2 -3 -4 0
% cryptominisat4 --verb 0 --random 9 XOR_adder.cnf
s SATISFIABLE
v 1 2 -3 -4 0
\end{lstlisting}

Nevertheless, all 4 possible solutions are:

\begin{lstlisting}
v -1 -2 3 4 0
v -1 2 3 -4 0
v 1 -2 -3 4 0
v 1 2 -3 -4 0
\end{lstlisting}

\dots the same as reported by MiniSat.

% subsections:
\subsection{Cracking Minesweeper with SAT solver}
\label{minesweeper_SAT}

See also about cracking it using Z3: \ref{minesweeper_SMT}.

SAT solvers are very different in that sense that they are at low-level, and can take only CNF expressions on input.

\subsubsection{Simple population count function}

First of all, somehow we need to count neighbour bombs.
The counting function is very similar to \textit{population count} function.

We can try to make CNF expression in Wolfram Mathematica.
This will be a function, returning True if any of 2 bits of 8 inputs bits are True and others are False.
First, we make truth table of such function:

\begin{lstlisting}
In[]:= tbl2 = 
 Table[PadLeft[IntegerDigits[i, 2], 8] -> 
   If[Equal[DigitCount[i, 2][[1]], 2], 1, 0], {i, 0, 255}]

Out[]= {{0, 0, 0, 0, 0, 0, 0, 0} -> 0, {0, 0, 0, 0, 0, 0, 0, 1} -> 0, 
{0, 0, 0, 0, 0, 0, 1, 0} -> 0, {0, 0, 0, 0, 0, 0, 1, 1} -> 1, 
{0, 0, 0, 0, 0, 1, 0, 0} -> 0, {0, 0, 0, 0, 0, 1, 0, 1} -> 1, 
{0, 0, 0, 0, 0, 1, 1, 0} -> 1, {0, 0, 0, 0, 0, 1, 1, 1} -> 0, 
{0, 0, 0, 0, 1, 0, 0, 0} -> 0, {0, 0, 0, 0, 1, 0, 0, 1} -> 1, 
{0, 0, 0, 0, 1, 0, 1, 0} -> 1, {0, 0, 0, 0, 1, 0, 1, 1} -> 0, 
...
{1, 1, 1, 1, 1, 0, 1, 0} -> 0, {1, 1, 1, 1, 1, 0, 1, 1} -> 0, 
{1, 1, 1, 1, 1, 1, 0, 0} -> 0, {1, 1, 1, 1, 1, 1, 0, 1} -> 0, 
{1, 1, 1, 1, 1, 1, 1, 0} -> 0, {1, 1, 1, 1, 1, 1, 1, 1} -> 0}
\end{lstlisting}

Now we can make CNF expression using this truth table:

\begin{lstlisting}
In[]:= BooleanConvert[
 BooleanFunction[tbl2, {a, b, c, d, e, f, g, h}], "CNF"]

Out[]= (! a || ! b || ! c) && (! a || ! b || ! d) && (! a || ! 
    b || ! e) && (! a || ! b || ! f) && (! a || ! b || ! g) && (! 
    a || ! b || ! h) && (! a || ! c || ! d) && (! a || ! c || ! 
    e) && (! a || ! c || ! f) && (! a || ! c || ! g) && (! a || ! 
    c || ! h) && (! a || ! d || ! e) && (! a || ! d || ! f) && (! 
    a || ! d || ! g) && (! a || ! d || ! h) && (! a || ! e || ! 
    f) && (! a || ! e || ! g) && (! a || ! e || ! h) && (! a || ! 
    f || ! g) && (! a || ! f || ! h) && (! a || ! g || ! h) && (a || 
   b || c || d || e || f || g) && (a || b || c || d || e || f || 
   h) && (a || b || c || d || e || g || h) && (a || b || c || d || f ||
    g || h) && (a || b || c || e || f || g || h) && (a || b || d || 
   e || f || g || h) && (a || c || d || e || f || g || 
   h) && (! b || ! c || ! d) && (! b || ! c || ! e) && (! b || ! 
    c || ! f) && (! b || ! c || ! g) && (! b || ! c || ! h) && (! 
    b || ! d || ! e) && (! b || ! d || ! f) && (! b || ! d || ! 
    g) && (! b || ! d || ! h) && (! b || ! e || ! f) && (! b || ! 
    e || ! g) && (! b || ! e || ! h) && (! b || ! f || ! g) && (! 
    b || ! f || ! h) && (! b || ! g || ! h) && (b || c || d || e || 
   f || g || 
   h) && (! c || ! d || ! e) && (! c || ! d || ! f) && (! c || ! 
    d || ! g) && (! c || ! d || ! h) && (! c || ! e || ! f) && (! 
    c || ! e || ! g) && (! c || ! e || ! h) && (! c || ! f || ! 
    g) && (! c || ! f || ! h) && (! c || ! g || ! h) && (! d || ! 
    e || ! f) && (! d || ! e || ! g) && (! d || ! e || ! h) && (! 
    d || ! f || ! g) && (! d || ! f || ! h) && (! d || ! g || ! 
    h) && (! e || ! f || ! g) && (! e || ! f || ! h) && (! e || ! 
    g || ! h) && (! f || ! g || ! h)
\end{lstlisting}

The syntax is similar to C/C++.
Let's check it.

I wrote a Python function to convert Mathematica's output into CNF file which can be feeded to SAT solver:

\lstinputlisting{SAT/minesweeper/tst.py}

It replaces a/b/c/... variables to the variable names passed (1/2/3...), reworks syntax, etc.
Here is a result:

\lstinputlisting{SAT/minesweeper/tst1.cnf}

I can run it:

\begin{lstlisting}
% minisat -verb=0 tst1.cnf results.txt
WARNING: for repeatability, setting FPU to use double precision
SATISFIABLE

% cat results.txt
SAT
1 -2 -3 -4 -5 -6 -7 8 0
\end{lstlisting}

The variable name in results lacking minus sign is "True".
Variable name with minus sign is "False".
We see there are just two variables are "True": 1 and 8.
This is indeed correct: MiniSat solver found a condition, for which our function returns "True".
Zero at the end is just a terminal symbol which means nothing.

We can ask MiniSat for another solution, by adding current solution to the input CNF file, but with all variables negated:

\begin{lstlisting}
...
-5 -6 -8 0
-5 -7 -8 0
-6 -7 -8 0
-1 2 3 4 5 6 7 -8 0
\end{lstlisting}

In plain English language, this means "give me ANY solution which can satisfy all clauses, but also not equal to the last clause we've just added".

MiniSat, indeed, found another solution, again, with only 2 variables equal to "True":

\begin{lstlisting}
% minisat -verb=0 tst2.cnf results.txt
WARNING: for repeatability, setting FPU to use double precision
SATISFIABLE

% cat results.txt
SAT
1 2 -3 -4 -5 -6 -7 -8 0
\end{lstlisting}

By the way, <i>population count</i> function for 8 neighbours in CNF form is simplest:

\begin{lstlisting}
a&&b&&c&&d&&e&&f&&g&&h
\end{lstlisting}

Indeed: it's true if all 8 input bits are "True".

The function for 0 neighbours is also simple:

\begin{lstlisting}
!a&&!b&&!c&&!d&&!e&&!f&&!g&&!h
\end{lstlisting}

It means, it will return "True", if all input variables are "False".

By the way, function for POPCNT1 is also simple:

\begin{lstlisting}
(!a||!b)&&(!a||!c)&&(!a||!d)&&(!a||!e)&&(!a||!f)&&(!a||!g)&&(!a||!h)&&(a||b||c||d||e||f||g||h)&&
(!b||!c)&&(!b||!d)&&(!b||!e)&&(!b||!f)&&(!b||!g)&&(!b||!h)&&(!c||!d)&&(!c||!e)&&(!c||!f)&&(!c||!g)&&
(!c||!h)&&(!d||!e)&&(!d||!f)&&(!d||!g)&&(!d||!h)&&(!e||!f)&&(!e||!g)&&(!e||!h)&&(!f||!g)&&(!f||!h)&&(!g||!h)
\end{lstlisting}

It just enumerates all possible pairs of 8 variables (a/b, a/c, a/d, etc) and says: no two bits must be present
simultaneously in each possible pair.
And there is another clause: "(a||b||c||d||e||f||g||h)", which says: at least one bit must be present among 8 variables.

And yes, you can ask Mathematica for finding CNF expressions for any other truth table.

\subsubsection{Minesweeper}

Now we can use Mathematica to get all \textit{population count} functions for 0..8 neighbours.

For 9*9 Minesweeper grid including invisible border, there will be 11*11=121 variables, mapped to Minesweeper grid like this:

\begin{lstlisting}
 1    2   3   4   5   6   7   8   9  10  11
12   13  14  15  16  17  18  19  20  21  22
23   24  25  26  27  28  29  30  31  32  33
34   35  36  37  38  39  40  41  42  43  44

...

100 101 102 103 104 105 106 107 108 109 110
111 112 113 114 115 116 117 118 119 120 121
\end{lstlisting}

Then we write a Python script which stacks all \textit{population count} functions: each function for each known number of neighbours (digit on Minesweeper field).
Each POPCNTx() function takes list of variable numbers and outputs list of clauses to be added to the final CNF file.

As of empty cells, we also add them as clauses, but with minus sign, which means, the variable must be False.
Whenever we try to place bomb, we add its variable as clause without minus sign, this means the variable must be True.

Then we execute external minisat process.
The only thing we need from it is exit code.
If an input CNF is UNSAT, it returns 20:

\lstinputlisting{SAT/minesweeper/minesweeper_SAT.py}

The output CNF file can be large, up to ~2000 clauses, or more, here is an example: URL sample.cnf

Anyway, it works just like my previous Z3Py script:

\begin{lstlisting}
row=1, col=3, unsat!
row=6, col=2, unsat!
row=6, col=3, unsat!
row=7, col=4, unsat!
row=7, col=9, unsat!
row=8, col=9, unsat!
\end{lstlisting}

... but it runs way faster, even considering overhead of executing external program.
Perhaps, Z3Py version could be optimized much better?

The files, including Wolfram Mathematica notebook: URL.


\section{KLEE}

% subsections:
\input{KLEE/eq.tex}
\input{KLEE/zebra.tex}
\input{KLEE/sudoku.tex}
\input{KLEE/UNIXdatetime.tex}
\input{KLEE/base64.tex}
\input{KLEE/CRC.tex}
\input{KLEE/LZSS.tex}

\subsection{Exercise}

Here is my crackme/keygenme, which may be tricky, but easy to solve using KLEE:
\url{http://challenges.re/74/}.




}
\RU{\section{\ac{SAT}-солверы}

Сравнивать SMT с SAT, это как сравнивать высокоуровневый ЯП с языком ассемблера.
Последний может быть куда более эффективным, но на нем труднее программировать.

\subsection{CNF форма}

\ac{CNF}\footnote{\url{https://en.wikipedia.org/wiki/Conjunctive_normal_form}} это так называемая \textit{нормальная форма}.

% TODO recheck
% TODO write abt it!
%\textit{normal form} is somewhat similar to polynomials in algebra. 
%What is polynomial?
%It is a standard way to express unsystematic equations like $2x \cdot x$ as $3x$ polynomial, 
%and so you will be able to apply some operations to polynomials like summing, etc.

Любое булево выражение может быть сконвертировано в \textit{нормальную форму}, и \ac{CNF} это одна из них.
\ac{CNF}-выражение это пачка клозов (подвыражений) состоящих их литералов (или термов, переменных), операций ИЛИ и НЕ,
все из которых склеены друг с другом в полное выражение операцией И.
Вот способ запомнить: \ac{CNF} это ``И всех ИЛИ'' (или ``произведение всех сумм'')
и \ac{DNF} это ``ИЛИ всех И'' (или ``сумма всех произведений'').

Пример: $(\neg A \vee B) \wedge (C \vee \neg D)$.

$\vee$ означает ИЛИ (логическая дизьюнкция\footnote{\url{https://en.wikipedia.org/wiki/Logical_disjunction}}), 
знак ``+'' также иногда используется для ИЛИ.

$\wedge$ означает ИЛИ (логическая коньюнкция\footnote{\url{https://en.wikipedia.org/wiki/Logical_conjunction}}).
Легко запомнить: $\wedge$ выглядит как буква ``A''.
Знак ``$\cdot$'' также иногда используется для И.

$\neg$ это отрицание (НЕ).

% TODO A/B is the first clause, C/D is second

\subsection{Пример: двухбитный сумматор}
\label{adder}

В сущности, \ac{SAT}-солвер это солвер огромных булевых уравнений в CNF-форме.
Он просто выдает ответ, есть ли набор входных значений, удовлетворяющий CNF-выражению, и какие это значения должны быть.

Вот для примера двухбитный сумматор:

\begin{figure}[ht!]
\centering
\includegraphics[scale=0.75]{SAT/adder_logisim.png}
\caption{Схема двухбитного сумматора}
\end{figure}

Сумматор здесь в самом простом возможном виде: у него нет входных и выходных переносов, и тут только 3 XOR-гейта
и один AND-гейт.
Попробуем разобраться, какой набор входных переменных заставит сумматор выставить оба выходных бита?
Просто подсчитав в уме, мы можем увидеть, что таких способа 4: $0+3=3$, $1+2=3$, $2+1=3$, $3+0=3$.
Вот также таблица истинности, с подсвеченными соответствующими рядами:

\newcommand{\HLcell}{\cellcolor{blue!25}}

\begin{center}
\begin{doublespace}
\noindent\(\begin{array}{l|llllll}
  & \text{aH} & \text{aL} & \text{bH} & \text{bL} & \text{qH} & \text{qL} \\
\hline
 \text{3+3 = 6 $\equiv $ 2 (mod 4)} & 1 & 1 & 1 & 1 & 1 & 0 \\
 \text{3+2 = 5 $\equiv $ 1 (mod 4)} & 1 & 1 & 1 & 0 & 0 & 1 \\
 \text{3+1 = 4 $\equiv $ 0 (mod 4)} & 1 & 1 & 0 & 1 & 0 & 0 \\
 \text{\HLcell{}3+0 = 3 $\equiv $ 3 (mod 4)} & \HLcell{}1 & \HLcell{}1 & \HLcell{}0 & \HLcell{}0 & \HLcell{}1 & \HLcell{}1 \\
 \text{2+3 = 5 $\equiv $ 1 (mod 4)} & 1 & 0 & 1 & 1 & 0 & 1 \\
 \text{2+2 = 4 $\equiv $ 0 (mod 4)} & 1 & 0 & 1 & 0 & 0 & 0 \\
 \text{\HLcell{}2+1 = 3 $\equiv $ 3 (mod 4)} & \HLcell{}1 & \HLcell{}0 & \HLcell{}0 & \HLcell{}1 & \HLcell{}1 & \HLcell{}1 \\
 \text{2+0 = 2 $\equiv $ 2 (mod 4)} & 1 & 0 & 0 & 0 & 1 & 0 \\
 \text{1+3 = 4 $\equiv $ 0 (mod 4)} & 0 & 1 & 1 & 1 & 0 & 0 \\
 \text{\HLcell{}1+2 = 3 $\equiv $ 3 (mod 4)} & \HLcell{}0 & \HLcell{}1 & \HLcell{}1 & \HLcell{}0 & \HLcell{}1 & \HLcell{}1 \\
 \text{1+1 = 2 $\equiv $ 2 (mod 4)} & 0 & 1 & 0 & 1 & 1 & 0 \\
 \text{1+0 = 1 $\equiv $ 1 (mod 4)} & 0 & 1 & 0 & 0 & 0 & 1 \\
 \text{\HLcell{}0+3 = 3 $\equiv $ 3 (mod 4)} & \HLcell{}0 & \HLcell{}0 & \HLcell{}1 & \HLcell{}1 & \HLcell{}1 & \HLcell{}1 \\
 \text{0+2 = 2 $\equiv $ 2 (mod 4)} & 0 & 0 & 1 & 0 & 1 & 0 \\
 \text{0+1 = 1 $\equiv $ 1 (mod 4)} & 0 & 0 & 0 & 1 & 0 & 1 \\
 \text{0+0 = 0 $\equiv $ 0 (mod 4)} & 0 & 0 & 0 & 0 & 0 & 0 \\
\end{array}\)
\end{doublespace}
\end{center}


Посмотрим, что об этом скажет SAT-солвер?

В начале, нам нужно представить наш двухбитный сумматор как CNF-выражение.

Используя Wolfram Mathematica, можно выразить 1-битное выражение для обоих выходов сумматоров:\\
\\
\textbf{\texttt{In[]:=AdderQ0[aL$\_$,bL$\_$]=Xor[aL,bL]}} \\
\textbf{\texttt{Out[]:=aL $\veebar$ bL}} \\
\\
\textbf{\texttt{In[]:=AdderQ1[aL$\_$,aH$\_$,bL$\_$,bH$\_$]=Xor[And[aL,bL],Xor[aH,bH]]}} \\
\textbf{\texttt{Out[]:=aH $\veebar$ bH $\veebar$ (aL \&\& bL)}} \\
\\
Нам нужно такое выражение, где обе части выдадут единицы.
Используя Wolfram Mathematica, найдем все возможные входы такого выражения (я склеил обе части при помощи And): \\
\\
\textbf{\texttt{In[]:=Boole[SatisfiabilityInstances[And[AdderQ0[aL,bL],AdderQ1[aL,aH,bL,bH]],\{aL,aH,bL,bH\},4]]}} \\
\textbf{\texttt{Out[]:=\{1,1,0,0\},\{1,0,0,1\},\{0,1,1,0\},\{0,0,1,1\}}} \\
\\
Да, действительно, Mathematica говорит, что здесь 4 входа, которые приведут к нужному нам результату.
Так что, Mathematica тоже может использоваться как \ac{SAT}-солвер.

Тем не менее, перейдем к CNF-форме. Используя Mathematica, сконвертируем наше выражение в CNF-форму:\\
\\
\textbf{\texttt{In[]:=cnf=BooleanConvert[And[AdderQ0[aL,bL],AdderQ1[aL,aH,bL,bH]],``CNF'']}} \\
\textbf{\texttt{Out[]:=(!aH $\|$ !bH) \&\& (aH $\|$ bH) \&\& (!aL $\|$ !bL) \&\& (aL $\|$ bL)}} \\
\\
Выглядит посложнее. Причина такой многословности в том, что \ac{CNF}-форма не поддерживает операцию исключающего
ИЛИ.
% FIXME: TeX form of the expression!

\subsubsection{MiniSat}

Для начала, попробуем MiniSat\footnote{\url{http://minisat.se/MiniSat.html}}.
Стандартный способ закодировать \ac{CNF}-выражение для MiniSat это перечислить все части ИЛИ в каждой строке.
Также, MiniSat не поддерживает имена переменных, только числа.
Перечислим наши переменные: 1 будет aH, 2 -- aL, 3 -- bH, 4 -- bL.

Вот что получилось, когда я сконвертировал выражение из Mathematica во входной файл для MiniSat:

\lstinputlisting{SAT/adder.cnf}

Две четверки в первой строке это, соответственно, число переменных и число клозов.
Так что тут 4 строки, каждая для каждого клоза ИЛИ.
Минус перед номером переменной означает что переменная инвертирована.
Отсутствие минуса -- не инвертирована.
Ноль в конце это просто оконечивающий ноль, означающий конец клоза.

Другими словами, каждая строка это ИЛИ-клоз с возможными инвертированиями,
и задача MiniSat в том, чтобы найти такой набор входных переменных, который удовлетворит все строки во входном файле.

Этот файл я назвал \textit{adder.cnf} и теперь попробуем MiniSat:

\begin{lstlisting}
% minisat -verb=0 adder.cnf results.txt
SATISFIABLE
\end{lstlisting}

Результаты в файле \textit{results.txt}:

\begin{lstlisting}
SAT
-1 -2 3 4 0
\end{lstlisting}

Это означает, что если первые две переменных (aH и aL) будут \textit{false},
и две последние переменные (bH и bL) будут \textit{true},
все \ac{CNF}-выражение будет истинно (satisfiable).
Похоже на правду: если bH и bL выставить в \textit{true}, оба бита результата также будут \textit{true}.

Как получить другие решения (instances)?
\ac{SAT}-солверы, как и \ac{SMT}-солверы, выдают только одно решение (или \textit{instance}).

MiniSat использует \ac{PRNG}, и его изначальное состояние (seed) можно задать явно.
Я попробовал разные значения, но результат всё тот же.
Тем не менее, CryptoMiniSat в этом случае может показать все возможные 4 решения, хотя и в хаотичном порядке.
Так что это не очень надежный способ.

Видимо, единственный способ, это инвертировать клоз решения и добавить его во входное выражение.
Мы получили \TT{-1 -2 3 4}, 
теперь мы можем инвертировать все значения в нем (просто поменяйте минусы: \TT{1 2 -3 -4}),
и добавим это в конец входного файла:

\begin{lstlisting}
p cnf 4 5
-1 -3 0
1 3 0
-2 -4 0
2 4 0
1 2 -3 -4
\end{lstlisting}

Получаем другой результат:

\begin{lstlisting}
SAT
1 2 -3 -4 0
\end{lstlisting}

Это означает что обе aH и aL должны быть \textit{true} и bH и bL должны быть \textit{false}, чтобы удовлетворить
входное выражение.
Снова инвертируем это решение и снова добавим:

\begin{lstlisting}
p cnf 4 6
-1 -3 0
1 3 0
-2 -4 0
2 4 0
1 2 -3 -4
-1 -2 3 4 0
\end{lstlisting}

Результат:

\begin{lstlisting}
SAT
-1 2 3 -4 0
\end{lstlisting}

aH=false, aL=true, bH=true, bL=false. Это также корректно, в соответствии с таблицей истинности.

Добавим снова:

\begin{lstlisting}
p cnf 4 7
-1 -3 0
1 3 0
-2 -4 0
2 4 0
1 2 -3 -4
-1 -2 3 4 0
1 -2 -3 4 0
\end{lstlisting}

\begin{lstlisting}
SAT
1 -2 -3 4 0
\end{lstlisting}

\textit{aH=true, aL=false, bH=false, bL=true.} Это тоже верно.

Это четвертый результат. Больше быть не должно. Что если добавим и это?

\begin{lstlisting}
p cnf 4 8
-1 -3 0
1 3 0
-2 -4 0
2 4 0
1 2 -3 -4
-1 -2 3 4 0
1 -2 -3 4 0
-1 2 3 -4 0
\end{lstlisting}

Теперь MiniSat просто говорит ``UNSATISFIABLE'' без всякой дополнительной информации в файле результатов.

Нам пример крохотный, но MiniSat может работать с огромными \ac{CNF}-выражениями.

\subsubsection{CryptoMiniSat}

Операция исключающего ИЛИ (XOR) отсутствует в CNF-форме, но она очень важна в криптографических алгоритмах.
Простейший способ представить одну единственную XOR-операцию в CNF-форме, это:
$(\neg x \vee \neg y) \wedge (x \vee y)$ -- не очень короткое выражение,
хотя, множество XOR-операций в одном выражении могут оптимизироваться лучше.

Одна значительная разница между MiniSat и CryptoMiniSat в том, что последний поддерживает
клозы с операцией XOR вместо ИЛИ,
потому что CryptoMiniSat предназначен больше для анализа криптоалгоритмов\footnote{\url{http://www.msoos.org/xor-clauses/}}.
XOR-клозы поддерживаются в CryptoMiniSat специальным образом, без трансляции в клозы ИЛИ.

Вам нужно просто прибавить ``x'' к клозу в \ac{CNF}-файле и CryptoMiniSat затем считает обычный ИЛИ-клоз как XOR-клоз.
Что до двухбитного сумматора, вот самое короткое из возможных XOR-CNF выражений, которое можно использовать
для поиска всех входных значений, где оба выходных бита выставлены:

$(aH \oplus bH) \wedge (aL \oplus bL)$

Это \TT{.cnf}-файл CryptoMiniSat:

\begin{lstlisting}
p cnf 4 2
x1 3 0
x2 4 0
\end{lstlisting}

Запускаю CryptoMiniSat с разными значениями для инициализации его \ac{PRNG} \dots

\begin{lstlisting}
% cryptominisat4 --verb 0 --random 0 XOR_adder.cnf
s SATISFIABLE
v 1 2 -3 -4 0
% cryptominisat4 --verb 0 --random 1 XOR_adder.cnf
s SATISFIABLE
v -1 -2 3 4 0
% cryptominisat4 --verb 0 --random 2 XOR_adder.cnf
s SATISFIABLE
v 1 -2 -3 4 0
% cryptominisat4 --verb 0 --random 3 XOR_adder.cnf
s SATISFIABLE
v 1 2 -3 -4 0
% cryptominisat4 --verb 0 --random 4 XOR_adder.cnf
s SATISFIABLE
v -1 2 3 -4 0
% cryptominisat4 --verb 0 --random 5 XOR_adder.cnf
s SATISFIABLE
v -1 2 3 -4 0
% cryptominisat4 --verb 0 --random 6 XOR_adder.cnf
s SATISFIABLE
v -1 -2 3 4 0
% cryptominisat4 --verb 0 --random 7 XOR_adder.cnf
s SATISFIABLE
v 1 -2 -3 4 0
% cryptominisat4 --verb 0 --random 8 XOR_adder.cnf
s SATISFIABLE
v 1 2 -3 -4 0
% cryptominisat4 --verb 0 --random 9 XOR_adder.cnf
s SATISFIABLE
v 1 2 -3 -4 0
\end{lstlisting}

Тем не менее, все 4 возможных решения, это:

\begin{lstlisting}
v -1 -2 3 4 0
v -1 2 3 -4 0
v 1 -2 -3 4 0
v 1 2 -3 -4 0
\end{lstlisting}

\dots то же, что и выдал MiniSat.

\subsection{Picosat}

По крайней мере Picosat может перечислить все возможные решения без тех костылей, которые я только что показывал:

\begin{lstlisting}
% picosat --all adder.cnf
s SATISFIABLE
v -1 -2 3 4 0
s SATISFIABLE
v -1 2 3 -4 0
s SATISFIABLE
v 1 2 -3 -4 0
s SATISFIABLE
v 1 -2 -3 4 0
s SOLUTIONS 4
\end{lstlisting}

% subsections:
\subsection{Головоломка о восьми ферзях}
\label{EightQueens}

Восемь ферзей это популярная головоломка, и она часто используется для измерения скорости работы SAT-солверов.
Нужно расставить на шахматной доске 8 ферзей так, чтобы они не атаковали друг друга.
Например:

\begin{lstlisting}
| | | |*| | | | |
| | | | | | |*| |
| | | | |*| | | |
| |*| | | | | | |
| | | | | |*| | |
|*| | | | | | | |
| | |*| | | | | |
| | | | | | | |*|
\end{lstlisting}

Попробуем разобраться, как её решить.

\subsubsection{POPCNT1}
\label{POPCNTOne}

Одна важная ф-ция, которую мы будем (часто) использовать это \TT{POPCNT1}.
Это ф-ция, которая возвращает \textit{Истинно}, если один из входов истинен, остальные ложны.
Она вернет \textit{Ложно} в остальных случаях.

В моих других примерах, я использовал Wolfram Mathematica для генерирования CNF-клозов для этого, например: \ref{minesweeper_SAT}.
Какое выражение сгенерирует Mathematica для ф-ции \TT{POPCNT1} для 8-и входов?

\begin{lstlisting}
(!a||!b)&&(!a||!c)&&(!a||!d)&&(!a||!e)&&(!a||!f)&&(!a||!g)&&(!a||!h)&&(a||b||c||d||e||f||g||h)&&
(!b||!c)&&(!b||!d)&&(!b||!e)&&(!b||!f)&&(!b||!g)&&(!b||!h)&&(!c||!d)&&(!c||!e)&&(!c||!f)&&(!c||!g)&&
(!c||!h)&&(!d||!e)&&(!d||!f)&&(!d||!g)&&(!d||!h)&&(!e||!f)&&(!e||!g)&&(!e||!h)&&(!f||!g)&&(!f||!h)&&(!g||!h)
\end{lstlisting}

Мы можем ясно увидеть что выражение состоит из всех возможных пар переменных (инвертированных) плюс
перечисление всех переменных (не инвертированных).
На обычном русском языке это означает: ``ни одна пара не должна быть равна двум \textit{Истинно} \textit{И}
по крайней мере одно \textit{Истинно} должно
присутствовать среди переменных''.

Вот как это работает: если две переменных будут \textit{Истино}, инвертированными они обе будут \textit{Ложно},
и этот клоз не будет
вычислен как \textit{Истинно}, а это наша конечная цель.
Если одна из переменных будет \textit{Истинно}, инвертированными, одна будет \textit{Истинно},
вторая \textit{Ложно} (хорошо).
Если обе переменных будут \textit{Ложно}, инвертированными, они обе будут \textit{Истинно} (тоже хорошо).

Вот как мы можем сгенерировать клозы для этой ф-ции используя модуль \textit{itertools} из Питона,
который также содержит много важных ф-ций из комбинаторики:

\begin{lstlisting}
    # naive/pairwise encoding   
    def AtMost1(self, lst):
        for pair in itertools.combinations(lst, r=2):
            self.add_clause([self.neg(pair[0]), self.neg(pair[1])])
        
    def POPCNT1(self, lst):
        self.AtMost1(lst)
        self.OR_always(lst)
\end{lstlisting}

Ф-ция \TT{AtMost1()} перечисляет все возможные пары используя ф-цию \textit{combinations()} из
\textit{itertools}.

Ф-ция \TT{POPCNT1()} делает то же самое, только добавляет последний клоз, который заставляет иметь хотя бы одну
переменную, равную Истинно.

Какие клозы будут сгенерированы для 5-и переменных (1..5)?

\lstinputlisting{SAT/8queens/popcnt1.cnf}

Да, это все возможные пары чисел 1..5 + все 5 чисел.

Можем посмотреть все решения используя Picosat:

\begin{lstlisting}
% picosat --all popcnt1.cnf
s SATISFIABLE
v -1 -2 -3 -4 5 0
s SATISFIABLE
v -1 -2 -3 4 -5 0
s SATISFIABLE
v -1 -2 3 -4 -5 0
s SATISFIABLE
v -1 2 -3 -4 -5 0
s SATISFIABLE
v 1 -2 -3 -4 -5 0
s SOLUTIONS 5
\end{lstlisting}

Действительно, 5 возможных решений.

\subsubsection{Восемь ферзей}

Теперь вернемся назад к восьми ферзям.

Мы можем назначить 64 переменных для $8 \cdot 8=64$ клеток.
Клетка на которой есть ферзь будет равна \textit{Истинно}, пустая клетка будет \textit{Ложно}.

Проблема расположения неатакующих (друг друга) ферзей на шахматной доске (любого размера), на обычном русском
языке может быть выражена так:

\begin{itemize}
\item один единственный ферзь должен присутствовать в каждом ряду;

\item один единственный ферзь должен присутствовать в каждом столбце;

\item или один ферзь должен присутствовать на каждой диагонали, или вовсе отсутствовать (пустые диагонали могут быть
и в правильном решении).
\end{itemize}

Эти правила можно перевести так:

\begin{itemize}
\item POPCNT1(каждый ряд)==\textit{Истинно}

\item POPCNT1(каждый столбец)==\textit{Истинно}

\item AtMost1(каждая диагональ)==\textit{Истинно}
\end{itemize}

Теперь мы должны перечислить ряды, столбцы и диагонали, и собрать все клозы:

\lstinputlisting{SAT/8queens/8queens.py}
( \url{https://github.com/dennis714/SAT_SMT_article/blob/master/SAT/8queens/8queens.py} )

Возможно, ф-ция \TT{gen\_diagonal()} выглядит не очень эстетично:
она перечисляет также поддиагонали более длинных диагоналей, которые уже были раннее.
Чтобы не было повторяющихся клозов, глобальная переменная \textit{clauses} это не список, а множество,
которое может содержать в себе только уникальные данные.

Также, я использовал ф-цию \TT{AtMost1} для каждого столбца, это поможет генерировать чуть меньше клозов.
Каждый столбец будет содержать ферзя в любом случае, это следует из первого правила (\TT{POPCNT1} для каждого ряда).

После запуска, получаем CNF-файл с 64-я переменными и 736-я клозами (\url{https://github.com/dennis714/SAT_SMT_article/blob/master/SAT/8queens/8queens.cnf}).
Вот одно из решений:

\begin{lstlisting}
% python 8queens.py
len(clauses)= 736
| | | |*| | | | |
| | | | | | |*| |
| | | | |*| | | |
| |*| | | | | | |
| | | | | |*| | |
|*| | | | | | | |
| | |*| | | | | |
| | | | | | | |*|
\end{lstlisting}

Как много здесь возможных решений?
Picosat говорит что 92, что действительно корректное число решений (\url{https://oeis.org/A000170}).

Скорость Picosat не очень впечатляет, вероятно потому что ему приходится выводить все решения.
Моему древнему нетбуку с Intel Atom 1.66GHz, понадобилось 34 для перечисления всех решений для шахматной доски
$11 \cdot 11$ 
(2680 решения),
что намного медленнее, чем моя прямолинейная программа полного перебора: \url{https://yurichev.com/blog/8queens/}.
Тем не менее, для поиска первого решения, Picosat работает крайне быстро (как и другие SAT-солверы).

Эта SAT-задача также достаточно проста, чтобы её можно было легко решить при помощи моего простейшего
SAT-солвера, работающего на базе поиска с возвратом (\textit{backtracking}):
\ref{SAT_backtrack}.

\subsubsection{Подсчет всех решений}

Мы получаем решение, инвертируем его и добавляем как новый констрайнт.
На обычном русском языке это звучит ``найди решение, котороые также не ровно тому, что мы только что нашли/добавили''.
Мы добавляем их последовательно, и процесс замедляется --- потому что размер проблемы (\textit{instance}) растет 
и SAT-солверу всё труднее находить новое решение.

\subsubsection{Пропуск симметрических решений}

Мы также можем добавлять повернутое и отраженное (горизонтально) решение, чтобы пропускать симметрические решения.
Делая так, мы получаем 12 решений для доски 8*8, 46 для 9*9, итд.
Это \url{https://oeis.org/A002562}.


\subsection{Sudoku в SAT}
\label{Sudoku_SAT}

Кто-то может подумать, что мы можем закодировать каждое число 1..9 в двоичном виде: 5 бит или переменных было бы достаточно.
Но есть даже еще более простой способ: выделить 9 бит, где только один бит будет \textit{Истинен}.
Число 1 может быт закодировано как [1, 0, 0, 0, 0, 0, 0, 0, 0], число 3 как [0, 0, 1, 0, 0, 0, 0, 0, 0], итд.
Выглядит неэкономично? Да, но другие операции будут проще.

Прежде всего, мы будем снова использовать важную ф-цию \TT{POPCNT1}, которую я описывал раннее: \ref{POPCNTOne}.

Вторая нужная нам важная операция, которую нам нужно придумать, это как сделать 9 чисел уникальными.
Если каждое число закодировано как 9-битный вектор, 9 чисел могут сформировать матрицу, вроде:

\begin{lstlisting}
0 0 0 0 0 0 1 0 0 <- §1-е§ число
0 0 0 0 0 1 0 0 0 <- §2-е§ число
0 1 0 0 0 0 0 0 0 <- ...
0 0 1 0 0 0 0 0 0 <- ...
0 0 0 0 0 0 0 0 1 <- ...
0 0 0 0 1 0 0 0 0 <- ...
0 0 0 0 0 0 0 1 0 <- ...
1 0 0 0 0 0 0 0 0 <- ...
0 0 0 1 0 0 0 0 0 <- §9-е§ число
\end{lstlisting}

Теперь будем использовать ф-цию \TT{POPCNT1} чтобы сделать каждый ряд в матрице содержащим только один бит \textit{Истина},
и это будет сохранять корректность нашего способа кодирования, т.к., вектор не может иметь более одного бита \textit{Истина},
либо не иметь битов \textit{Истина} вообще.
Затем мы будем использовать ф-цию \TT{POPCNT1} снова чтобы сделать так, чтобы каждый столбец в матрице имел только
один единственный бит \textit{Истина}.
Это сделает все ряды в матрице уникальными, другими словами, все 9 закодированных чисел всегда будут уникальными.

После применения ф-ции \TT{POPCNT1} 9+9=18 раз, у нас будет 9 уникальных чисел в пределах 1..9.

Используя эту операцию мы можем сделать каждый ряд в головоломке Судоку уникальным, каждый столбец уникальным,
и каждый квадрат $3 \cdot 3=9$ тоже уникальным.

\lstinputlisting{SAT/sudoku/sudoku_SAT.py}
( \url{https://github.com/DennisYurichev/SAT_SMT_article/blob/master/SAT/sudoku/sudoku_SAT.py} )

Ф-ция \TT{make\_distinct\_bits\_in\_vector()} сохраняет корректность кодирования.\\
Ф-ция \TT{make\_distinct\_vectors()} делает 9 чисел уникальными.\\
Ф-ция \TT{cvt\_vector\_to\_number()} декодирует вектор в число.\\
Ф-ция \TT{number\_to\_vector()} кодирует число в вектор.\\
Ф-ция \TT{main()} содержит все необходимые вызовы, чтобы сделать уникальными ряды/столбцы и квадраты $3\cdot 3$.

Работает:

\begin{lstlisting}
% python sudoku_SAT.py
len(clauses)= 12195
1 4 5 3 2 7 6 9 8
8 3 9 6 5 4 1 2 7
6 7 2 9 1 8 5 4 3
4 9 6 1 8 5 3 7 2
2 1 8 4 7 3 9 5 6
7 5 3 2 9 6 4 8 1
3 6 7 5 4 2 8 1 9
9 8 4 7 6 1 2 3 5
5 2 1 8 3 9 7 6 4
\end{lstlisting}

Такое же решение как и раннее: \ref{sudoku_SMT}.

Picosat говорит что эта SAT-проблема имеет только одно решение.
Действительно, как говорят, настоящая головоломка Судоку может иметь только одно решение.

\subsubsection{Избавление от одного вызова POPCNT1}
\label{OR_in_POPCNT1}

Чтобы сделать 9 уникальный чисел 1..9 мы можем использовать ф-цию \TT{POPCNT1}, чтобы сделать уникальным каждый ряд
в матрице, и использовать операцию \textit{ИЛИ} для всех столбцов.
Это будет иметь такой же эффект: все ряды должны быть уникальны, чтобы каждый столбец вычислялся в \textit{Истино}
после применения операции \textit{ИЛИ} ко всем переменным в столбце.
(Я буду делать так в следующем примере: \ref{Zebra_SAT}.)

Это приведет к тому, что будет 3447 клозов вместо 12195, но почему-то, SAT-солверы работают медленнее. Не знаю, почему.


\subsection{Взлом Сапёра при помощи SAT}
\label{minesweeper_SAT}

См.также о взломе оного при помощи Z3: \ref{minesweeper_SMT}.

\subsubsection{Простая ф-ция подсчета бит (\textit{population count})}

Прежде всего, нам нужно как считать количество соседних бомб.
Ф-ция подсчета та же, что и ф-ция подсчета бит (\textit{population count}).

Мы можем создать \ac{CNF}-выражение используя Wolfram Mathematica.
Это будет ф-ция, возвращающая \textit{True} если любые из двух бит 8-битного входа равняются \textit{True},
а остальные --- \textit{False}.
В начале, сделаем таблицу истинности для такой ф-ции:

\begin{lstlisting}
In[]:= tbl2 = 
 Table[PadLeft[IntegerDigits[i, 2], 8] -> 
   If[Equal[DigitCount[i, 2][[1]], 2], 1, 0], {i, 0, 255}]

Out[]= {{0, 0, 0, 0, 0, 0, 0, 0} -> 0, {0, 0, 0, 0, 0, 0, 0, 1} -> 0, 
{0, 0, 0, 0, 0, 0, 1, 0} -> 0, {0, 0, 0, 0, 0, 0, 1, 1} -> 1, 
{0, 0, 0, 0, 0, 1, 0, 0} -> 0, {0, 0, 0, 0, 0, 1, 0, 1} -> 1, 
{0, 0, 0, 0, 0, 1, 1, 0} -> 1, {0, 0, 0, 0, 0, 1, 1, 1} -> 0, 
{0, 0, 0, 0, 1, 0, 0, 0} -> 0, {0, 0, 0, 0, 1, 0, 0, 1} -> 1, 
{0, 0, 0, 0, 1, 0, 1, 0} -> 1, {0, 0, 0, 0, 1, 0, 1, 1} -> 0, 
...
{1, 1, 1, 1, 1, 0, 1, 0} -> 0, {1, 1, 1, 1, 1, 0, 1, 1} -> 0, 
{1, 1, 1, 1, 1, 1, 0, 0} -> 0, {1, 1, 1, 1, 1, 1, 0, 1} -> 0, 
{1, 1, 1, 1, 1, 1, 1, 0} -> 0, {1, 1, 1, 1, 1, 1, 1, 1} -> 0}
\end{lstlisting}

Теперь можем сделать \ac{CNF}-выражение используя эту таблицу истинности:

\begin{lstlisting}
In[]:= BooleanConvert[
 BooleanFunction[tbl2, {a, b, c, d, e, f, g, h}], "CNF"]

Out[]= (! a || ! b || ! c) && (! a || ! b || ! d) && (! a || ! 
    b || ! e) && (! a || ! b || ! f) && (! a || ! b || ! g) && (! 
    a || ! b || ! h) && (! a || ! c || ! d) && (! a || ! c || ! 
    e) && (! a || ! c || ! f) && (! a || ! c || ! g) && (! a || ! 
    c || ! h) && (! a || ! d || ! e) && (! a || ! d || ! f) && (! 
    a || ! d || ! g) && (! a || ! d || ! h) && (! a || ! e || ! 
    f) && (! a || ! e || ! g) && (! a || ! e || ! h) && (! a || ! 
    f || ! g) && (! a || ! f || ! h) && (! a || ! g || ! h) && (a || 
   b || c || d || e || f || g) && (a || b || c || d || e || f || 
   h) && (a || b || c || d || e || g || h) && (a || b || c || d || f ||
    g || h) && (a || b || c || e || f || g || h) && (a || b || d || 
   e || f || g || h) && (a || c || d || e || f || g || 
   h) && (! b || ! c || ! d) && (! b || ! c || ! e) && (! b || ! 
    c || ! f) && (! b || ! c || ! g) && (! b || ! c || ! h) && (! 
    b || ! d || ! e) && (! b || ! d || ! f) && (! b || ! d || ! 
    g) && (! b || ! d || ! h) && (! b || ! e || ! f) && (! b || ! 
    e || ! g) && (! b || ! e || ! h) && (! b || ! f || ! g) && (! 
    b || ! f || ! h) && (! b || ! g || ! h) && (b || c || d || e || 
   f || g || 
   h) && (! c || ! d || ! e) && (! c || ! d || ! f) && (! c || ! 
    d || ! g) && (! c || ! d || ! h) && (! c || ! e || ! f) && (! 
    c || ! e || ! g) && (! c || ! e || ! h) && (! c || ! f || ! 
    g) && (! c || ! f || ! h) && (! c || ! g || ! h) && (! d || ! 
    e || ! f) && (! d || ! e || ! g) && (! d || ! e || ! h) && (! 
    d || ! f || ! g) && (! d || ! f || ! h) && (! d || ! g || ! 
    h) && (! e || ! f || ! g) && (! e || ! f || ! h) && (! e || ! 
    g || ! h) && (! f || ! g || ! h)
\end{lstlisting}

Синтаксис такой же как и в Си/Си++
Проверим.

Я написал Питоновскую ф-цию для конвертирования вывода Mathematica в \ac{CNF}-файл, который можно подать на вход
SAT-солверу:

\lstinputlisting{SAT/minesweeper/tst.py}

Она заменяет переменные a/b/c/... на переданные имена переменных (1/2/3...), перерабатыает синтаксис, итд.
Here is a result:

\lstinputlisting{SAT/minesweeper/tst1.cnf}

Могу запустить:

\begin{lstlisting}
% minisat -verb=0 tst1.cnf results.txt
SATISFIABLE

% cat results.txt
SAT
1 -2 -3 -4 -5 -6 -7 8 0
\end{lstlisting}

Имя переменной в результате без знака минуса, это \textit{True}.
Имя переменной со знаком минус, это \textit{False}.
Мы здесь видим только две переменных \textit{True}: 1 и 8.
Это действительно корректно: солвер MiniSat нашел условие, для которого наша ф-ция возвращает \textit{True}.
Ноль в конце это просто терминирующий символ, который ничего не означает.

Мы можем попросить MiniSat найти еще одно решение, добавив текущее решение во входной CNF-файл,
но где все переменные инвертированы:

\begin{lstlisting}
...
-5 -6 -8 0
-5 -7 -8 0
-6 -7 -8 0
-1 2 3 4 5 6 7 -8 0
\end{lstlisting}

В обычном русском языке, это означает ``дайте ЛЮБОЕ решение, которые удовлетворяет все клозы, но также не равно
последнему клозу, которое мы только что добавили''.

MiniSat, действительно, находит еще одно решение, и снова, только с двумя переменными, равными \textit{True}:

\begin{lstlisting}
% minisat -verb=0 tst2.cnf results.txt
SATISFIABLE

% cat results.txt
SAT
1 2 -3 -4 -5 -6 -7 -8 0
\end{lstlisting}

Кстати, ф-ция \textit{population count} для 8-и соседей (POPCNT8) в CNF-форме, самая простая:

\begin{lstlisting}
a&&b&&c&&d&&e&&f&&g&&h
\end{lstlisting}

Действительно: она истинна, если все 8 входных бит тоже истинны.

Ф-ция для отсутствия соседей (POPCNT0) тоже очень простая:

\begin{lstlisting}
!a&&!b&&!c&&!d&&!e&&!f&&!g&&!h
\end{lstlisting}

Это означает, что она вернет \textit{True}, если все входные переменные \textit{False}.

Кстати, ф-ция POPCNT1 тоже простая:

\begin{lstlisting}
(!a||!b)&&(!a||!c)&&(!a||!d)&&(!a||!e)&&(!a||!f)&&(!a||!g)&&(!a||!h)&&(a||b||c||d||e||f||g||h)&&
(!b||!c)&&(!b||!d)&&(!b||!e)&&(!b||!f)&&(!b||!g)&&(!b||!h)&&(!c||!d)&&(!c||!e)&&(!c||!f)&&(!c||!g)&&
(!c||!h)&&(!d||!e)&&(!d||!f)&&(!d||!g)&&(!d||!h)&&(!e||!f)&&(!e||!g)&&(!e||!h)&&(!f||!g)&&(!f||!h)&&(!g||!h)
\end{lstlisting}

Здесь просто перечисление всех возможных пар 8-и переменных
(a/b, a/c, a/d, итд), что подразумевает: не должно присутствовать одновременно двух бит в каждой возможной паре.
И еще один клоз: ``(a||b||c||d||e||f||g||h)'', что подразумевает: минимум один бит должен присутствовать
среди 8-и переменных.

И да, вы можете использовать Mathematica для поиска \ac{CNF}-выражения для любой другой таблицы истинности.

\subsubsection{Сапёр}

Теперь можем использовать Mathematica для генерации всех ф-ций \textit{population count} для количества соседей 0..8.

Для Сапёра с матрицей $9 \cdot 9$ включая невидимую рамку, здесь будет $11 \cdot 11=121$ переменных,
связанных с матрицей Сапёра вот так:

\begin{lstlisting}
 1    2   3   4   5   6   7   8   9  10  11
12   13  14  15  16  17  18  19  20  21  22
23   24  25  26  27  28  29  30  31  32  33
34   35  36  37  38  39  40  41  42  43  44

...

100 101 102 103 104 105 106 107 108 109 110
111 112 113 114 115 116 117 118 119 120 121
\end{lstlisting}

Потом мы пишем Питоновский скрипт, складывающий все ф-ции \textit{population count}:
каждая ф-ция для каждого известного числа соседей (число на поле Сапёра).
Каждая ф-ция POPCNTx() берет на вход список переменных и выдает список клозов, которые будут добавлены
в итоговый \ac{CNF}-файл.

Что до пустых клеток, мы тоже добавляем их как клозы, но со знаком минус, что означает, что переменная
должна быть \textit{False}.
А когда мы пытаемся поместить бомбу, мы добавляем её переменную как клоз без знака минуса, что означает
что переменная должна быть \textit{True}.

Затем запускаем внешний процесс minisat.
Всё что нам от него нужно, это код возврата.
Если входной \ac{CNF} это \TT{UNSAT}, он возвращает 20:

Мы также используем здесь информацию из предыдущего решения Сапёра: \ref{minesweeper_SMT}.

\lstinputlisting{SAT/minesweeper/minesweeper_SAT.py}

( \url{https://github.com/dennis714/SAT_SMT_article/blob/master/SAT/minesweeper/minesweeper_SAT.py} ) \\
\\
Выходной \ac{CNF}-файл большой, вплоть до $\approx 2000$ клозов, и даже больше, вот, например: \url{https://github.com/dennis714/SAT_SMT_article/blob/master/SAT/minesweeper/sample.cnf}.

Так или иначе, это работает так же, как мой предыдущий скрипт для Z3Py:

\begin{lstlisting}
row=1, col=3, unsat!
row=6, col=2, unsat!
row=6, col=3, unsat!
row=7, col=4, unsat!
row=7, col=9, unsat!
row=8, col=9, unsat!
\end{lstlisting}

\dots но работает намного быстрее, даже учитывая запуск внешней программы.
Вероятно, версию для Z3Py можно было бы оптимизировать получше?

Файлы, включая файл для Wolfram Mathematica: \url{https://github.com/dennis714/SAT_SMT_article/tree/master/SAT/minesweeper}.


% TODO translate src
\subsection{Головоломка Зебры как SAT-проблема}
\label{Zebra_SAT}

Попробуем решить головоломку Зебры (\ref{zebra_SMT}) в SAT.

Я определю каждую переменную как вектор из пяти переменных, как я делал это раннее в солвере Судоку: \ref{Sudoku_SAT}.

Я также использую ф-цию \TT{POPCNT1}, но в отличие от предыдущего примера,
я использовал Wolfram Mathematica для генерирования её в CNF-форме:

\begin{lstlisting}
In[]:= tbl1=Table[PadLeft[IntegerDigits[i,2],5] ->If[Equal[DigitCount[i,2][[1]],1],1,0],{i,0,63}]
Out[]= {{0,0,0,0,0}->0,
{0,0,0,0,1}->1,
{0,0,0,1,0}->1,
{0,0,0,1,1}->0,
{0,0,1,0,0}->1,
{0,0,1,0,1}->0,

...

{1,1,1,1,0}->0,
{1,1,1,1,1}->0}

In[]:= BooleanConvert[BooleanFunction[tbl1,{a,b,c,d,e}],"CNF"]
Out[]= (!a||!b)&&(!a||!c)&&(!a||!d)&&(!a||!e)&&(a||b||c||d||e)&&(!b||!c)&&(!b||!d)&&(!b||!e)&&(!c||!d)&&(!c||!e)&&(!d||!e)
\end{lstlisting}

Также, как я предлагал раньше (\ref{OR_in_POPCNT1}), я использовал операцию \textit{ИЛИ} для второго шага.

\begin{lstlisting}
def mathematica_to_CNF (s, d):
    for k in d.keys():
        s=s.replace(k, d[k])
    s=s.replace("!", "-").replace("||", " ").replace("(", "").replace(")", "")
    s=s.split ("&&")
    return s

def add_popcnt1(v1, v2, v3, v4, v5):
    global clauses
    s="(!a||!b)&&" \
      "(!a||!c)&&" \
      "(!a||!d)&&" \
      "(!a||!e)&&" \
      "(!b||!c)&&" \
      "(!b||!d)&&" \
      "(!b||!e)&&" \
      "(!c||!d)&&" \
      "(!c||!e)&&" \
      "(!d||!e)&&" \
      "(a||b||c||d||e)"

    clauses=clauses+mathematica_to_CNF(s, {"a":v1, "b":v2, "c":v3, "d":v4, "e":v5})

...

# k=tuple: ("high-level" variable name, number of bit (0..4))
# v=variable number in CNF
vars={}
vars_last=1

...

def alloc_distinct_variables(names):
    global vars
    global vars_last
    for name in names:
        for i in range(5):
            vars[(name,i)]=str(vars_last)
            vars_last=vars_last+1

        add_popcnt1(vars[(name,0)], vars[(name,1)], vars[(name,2)], vars[(name,3)], vars[(name,4)])

    # make them distinct:
    for i in range(5):
        clauses.append(vars[(names[0],i)] + " " + vars[(names[1],i)] + " " + vars[(names[2],i)] + " " + vars[(names[3],i)] + " " + vars[(names[4],i)])

...

alloc_distinct_variables(["Yellow", "Blue", "Red", "Ivory", "Green"])
alloc_distinct_variables(["Norwegian", "Ukrainian", "Englishman", "Spaniard", "Japanese"])
alloc_distinct_variables(["Water", "Tea", "Milk", "OrangeJuice", "Coffee"])
alloc_distinct_variables(["Kools", "Chesterfield", "OldGold", "LuckyStrike", "Parliament"])
alloc_distinct_variables(["Fox", "Horse", "Snails", "Dog", "Zebra"])

...

\end{lstlisting}

Теперь у нас пять булевых переменных для каждой \textit{высокоуровневной} переменной,
и каждая группа переменных гарантированно будет иметь разные значения.

Теперь перечитаем условие головоломки: ``2. Англичанин живёт в красном доме.''.
Это легко.
В моих примерах на Z3 и KLEE я просто написал ``Englishman==Red''.
Та же история и здесь: мы просто добавляем клозы, показывающие, что 5 булевых переменных для ``Englishman''
должны равняться пяти переменных для ``Red''.

На самом низком уровне CNF, если мы хотим сказать, что две переменных должны равняться друг другу,
мы добавляем два клоза:

$(var1 \vee \neg var2) \wedge (\neg var1 \vee var2)$

Это означает что значения обоих \textit{var1} и \textit{var2} должны быть или \textit{Ложно} или \textit{Истинно},
но они не могут быть разными.

\begin{lstlisting}
def add_eq_clauses(var1, var2):
    global clauses
    clauses.append(var1 + " -" + var2)
    clauses.append("-"+var1 + " " + var2)

def add_eq (n1, n2):
    for i in range(5):
        add_eq_clauses(vars[(n1,i)], vars[(n2, i)])

...

# 2.The Englishman lives in the red house.
add_eq("Englishman","Red")

# 3.The Spaniard owns the dog.
add_eq("Spaniard","Dog")

# 4.Coffee is drunk in the green house.
add_eq("Coffee","Green")

...

\end{lstlisting}

Теперь следующие условия:
``9. В центральном доме пьют молоко.'' (т.е., в третьем доме), ``10. Норвежец живёт в первом доме.''
Мы можем присвоить булевы значения напрямую:

\begin{lstlisting}
# n=1..5
def add_eq_var_n (name, n):
    global clauses
    global vars
    for i in range(5):
        if i==n-1:
            clauses.append(vars[(name,i)]) # always True
        else:
            clauses.append("-"+vars[(name,i)]) # always False

...

# 9.Milk is drunk in the middle house.
add_eq_var_n("Milk",3) # i.e., 3rd house

# 10.The Norwegian lives in the first house.
add_eq_var_n("Norwegian",1)
\end{lstlisting}

Для ``Milk'' у нас значение ``0 0 1 0 0'', для ``Norwegian'': ``1 0 0 0 0''.

Что делать с этим?
``6. Зелёный дом стоит сразу справа от белого дома.''
Я могу сконструировать такое условие:

\begin{lstlisting}
    Ivory      Green
AND(1 0 0 0 0  0 1 0 0 0)
.. OR ..
AND(0 1 0 0 0  0 0 1 0 0)
.. OR ..
AND(0 0 1 0 0  0 0 0 1 0)
.. OR ..
AND(0 0 0 1 0  0 0 0 0 1)
\end{lstlisting}

Для ``белого/ivory'' тут нет ``0 0 0 0 1'', потому что он не может быть последним.
Теперь я конвертирую эти условия в CNF при помощи Wolfram Mathematica:

\begin{lstlisting}
In[]:= BooleanConvert[(a1&& !b1&&!c1&&!d1&&!e1&&!a2&& b2&&!c2&&!d2&&!e2) ||
(!a1&& b1&&!c1&&!d1&&!e1&&!a2&& !b2&&c2&&!d2&&!e2) ||
(!a1&& !b1&&c1&&!d1&&!e1&&!a2&& !b2&&!c2&&d2&&!e2) ||
(!a1&& !b1&&!c1&&d1&&!e1&&!a2&& !b2&&!c2&&!d2&&e2) ,"CNF"]

Out[]= (!a1||!b1)&&(!a1||!c1)&&(!a1||!d1)&&(a1||b1||c1||d1)&&!a2&&(!b1||!b2)&&(!b1||!c1)&&
(!b1||!d1)&&(b1||b2||c1||d1)&&(!b2||!c1)&&(!b2||!c2)&&(!b2||!d1)&&(!b2||!d2)&&(!b2||!e2)&&
(b2||c1||c2||d1)&&(b2||c2||d1||d2)&&(b2||c2||d2||e2)&&(!c1||!c2)&&(!c1||!d1)&&(!c2||!d1)&&
(!c2||!d2)&&(!c2||!e2)&&(!d1||!d2)&&(!d2||!e2)&&!e1
\end{lstlisting}

И вот фрагмент моего кода на Питоне:

\begin{lstlisting}
def add_right (n1, n2):
    global clauses
    s="(!a1||!b1)&&(!a1||!c1)&&(!a1||!d1)&&(a1||b1||c1||d1)&&!a2&&(!b1||!b2)&&(!b1||!c1)&&(!b1||!d1)&&" \
      "(b1||b2||c1||d1)&&(!b2||!c1)&&(!b2||!c2)&&(!b2||!d1)&&(!b2||!d2)&&(!b2||!e2)&&(b2||c1||c2||d1)&&" \
      "(b2||c2||d1||d2)&&(b2||c2||d2||e2)&&(!c1||!c2)&&(!c1||!d1)&&(!c2||!d1)&&(!c2||!d2)&&(!c2||!e2)&&" \
      "(!d1||!d2)&&(!d2||!e2)&&!e1"

    clauses=clauses+mathematica_to_CNF(s, {
	"a1": vars[(n1,0)], "b1": vars[(n1,1)], "c1": vars[(n1,2)], "d1": vars[(n1,3)], "e1": vars[(n1,4)],
	"a2": vars[(n2,0)], "b2": vars[(n2,1)], "c2": vars[(n2,2)], "d2": vars[(n2,3)], "e2": vars[(n2,4)]})

...

# 6.The green house is immediately to the right of the ivory house.
add_right("Ivory", "Green")
\end{lstlisting}

Что мы будем делать с этим?
``11. Сосед того, кто курит Chesterfield, держит лису.''
``12. В доме по соседству с тем, в котором держат лошадь, курят Kool.''

Мы не знаем с какой стороны, слева или справа, но знаем что они отличаются на единицу.
Вот какие клозы я добавлю:

\begin{lstlisting}
    Chesterfield  Fox
AND(0 0 0 0 1     0 0 0 1 0)
.. OR ..
AND(0 0 0 1 0     0 0 0 0 1)
AND(0 0 0 1 0     0 0 1 0 0)
.. OR ..
AND(0 0 1 0 0     0 1 0 0 0)
AND(0 0 1 0 0     0 0 0 1 0)
.. OR ..
AND(0 1 0 0 0     1 0 0 0 0)
AND(0 1 0 0 0     0 0 1 0 0)
.. OR ..
AND(1 0 0 0 0     0 1 0 0 0)
\end{lstlisting}

И снова могу сконвертировать это всё в CNF при помощи Mathematica:

\begin{lstlisting}
In[]:= BooleanConvert[(a1&& !b1&&!c1&&!d1&&!e1&&!a2&& b2&&!c2&&!d2&&!e2) ||

(!a1&& b1&&!c1&&!d1&&!e1&&a2&& !b2&&!c2&&!d2&&!e2) ||
(!a1&& b1&&!c1&&!d1&&!e1&&!a2&& !b2&&c2&&!d2&&!e2) ||

(!a1&& !b1&&c1&&!d1&&!e1&&!a2&& b2&&!c2&&!d2&&!e2) ||
(!a1&& !b1&&c1&&!d1&&!e1&&!a2&& !b2&&!c2&&d2&&!e2) ||

(!a1&& !b1&&!c1&&d1&&!e1&&!a2&& !b2&&c2&&!d2&&!e2) ||
(!a1&& !b1&&!c1&&d1&&!e1&&!a2&& !b2&&!c2&&!d2&&e2) ||

(!a1&& !b1&&!c1&&!d1&&e1&&!a2&& !b2&&!c2&&d2&&!e2) ,"CNF"]

Out[]= (!a1||!b1)&&(!a1||!c1)&&(!a1||!d1)&&(!a1||!e1)&&(a1||b1||c1||d1||e1)&&(!a2||b1)&&(!a2||!b2)&&
(!a2||!c2)&&(!a2||!d2)&&(!a2||!e2)&&(a2||b2||c1||c2||d1||e1)&&(a2||b2||c2||d1||d2)&&(a2||b2||c2||d2||e2)&&
(!b1||!b2)&&(!b1||!c1)&&(!b1||!d1)&&(!b1||!e1)&&(b1||b2||c1||d1||e1)&&(!b2||!c2)&&(!b2||!d1)&&(!b2||!d2)&&
(!b2||!e1)&&(!b2||!e2)&&(!c1||!c2)&&(!c1||!d1)&&(!c1||!e1)&&(!c2||!d2)&&(!c2||!e1)&&(!c2||!e2)&&
(!d1||!d2)&&(!d1||!e1)&&(!d2||!e2)
\end{lstlisting}

И вот мой код:

\begin{lstlisting}
def add_right_or_left (n1, n2):
    global clauses
    s="(!a1||!b1)&&(!a1||!c1)&&(!a1||!d1)&&(!a1||!e1)&&(a1||b1||c1||d1||e1)&&(!a2||b1)&&" \
      "(!a2||!b2)&&(!a2||!c2)&&(!a2||!d2)&&(!a2||!e2)&&(a2||b2||c1||c2||d1||e1)&&(a2||b2||c2||d1||d2)&&" \
       "(a2||b2||c2||d2||e2)&&(!b1||!b2)&&(!b1||!c1)&&(!b1||!d1)&&(!b1||!e1)&&(b1||b2||c1||d1||e1)&&" \
       "(!b2||!c2)&&(!b2||!d1)&&(!b2||!d2)&&(!b2||!e1)&&(!b2||!e2)&&(!c1||!c2)&&(!c1||!d1)&&(!c1||!e1)&&" \
       "(!c2||!d2)&&(!c2||!e1)&&(!c2||!e2)&&(!d1||!d2)&&(!d1||!e1)&&(!d2||!e2)"
    
    clauses=clauses+mathematica_to_CNF(s, {
	"a1": vars[(n1,0)], "b1": vars[(n1,1)], "c1": vars[(n1,2)], "d1": vars[(n1,3)], "e1": vars[(n1,4)],
	"a2": vars[(n2,0)], "b2": vars[(n2,1)], "c2": vars[(n2,2)], "d2": vars[(n2,3)], "e2": vars[(n2,4)]})

...

# 11.The man who smokes Chesterfields lives in the house next to the man with the fox.
add_right_or_left("Chesterfield","Fox") # left or right

# 12.Kools are smoked in the house next to the house where the horse is kept.
add_right_or_left("Kools","Horse") # left or right
\end{lstlisting}

Вот и всё!
Полный исходный код: \url{https://github.com/DennisYurichev/SAT_SMT_article/blob/master/SAT/zebra/zebra_SAT.py}.

Итоговая CNF-проблема имеет 125 булевых переменных и 511 клозов: \\
\url{https://github.com/DennisYurichev/SAT_SMT_article/blob/master/SAT/zebra/1.cnf}.
Это очень легкая задача для любого SAT-солвера.
Даже мой игрушечный SAT-солвер (\ref{SAT_backtrack}) может решить её за \textasciitilde{}1 секунду на моем древнем
нетбуке с Intel Atom.

И конечно же, тут только одно решение, что и подтверждается при помощи Picosat.

\begin{lstlisting}
% python zebra_SAT.py
Yellow 1
Blue 2
Red 3
Ivory 4
Green 5
Norwegian 1
Ukrainian 2
Englishman 3
Spaniard 4
Japanese 5
Water 1
Tea 2
Milk 3
OrangeJuice 4
Coffee 5
Kools 1
Chesterfield 2
OldGold 3
LuckyStrike 4
Parliament 5
Fox 1
Horse 2
Snails 3
Dog 4
Zebra 5
\end{lstlisting}


\subsection{Простейший SAT-солвер в \textasciitilde{}120 строках}
\label{SAT_backtrack}

Это простейший SAT-солвер работающий на базе поиска с возвратом (\textit{backtracking}) (не \ac{DPLL}), написанный
на Питоне.
Он использует тот же поиск с возвратом, который можно найти в простейших солверах Судоку и задачи о восьми ферзях.
Он работает значительно медленнее, но, из-за предельной простоты, он также может подсчитывать количество решений.
Например, он может подсчитать все решения для задачи о восьми ферзях (\ref{EightQueens}).

Также, имеется 70 решений для ф-ции POPCNT4
\footnote{\url{https://github.com/DennisYurichev/SAT_SMT_article/blob/master/SAT/backtrack/POPCNT4.cnf}}
(ф-ция истинна, если любые из её 4-х входов из 8-и истинны):

\begin{lstlisting}
SAT
-1 -2 -3 -4 5 6 7 8 0
SAT
-1 -2 -3 4 -5 6 7 8 0
SAT
-1 -2 -3 4 5 -6 7 8 0
SAT
-1 -2 -3 4 5 6 -7 8 0
...

SAT
1 2 3 -4 -5 6 -7 -8 0
SAT
1 2 3 -4 5 -6 -7 -8 0
SAT
1 2 3 4 -5 -6 -7 -8 0
UNSAT
solutions= 70
\end{lstlisting}

Солвер также тестировался на моем взломщике Сапёра основанном на SAT (\ref{minesweeper_SAT}),
и заканчивает работу в разумное время (хотя и медленнее чем MiniSat раз в \textasciitilde{}10).

На б\'{о}льших \ac{CNF}-задачах он зависает.

Исходный код:
% TODO: translate to RU:
\lstinputlisting{SAT/backtrack/SAT_backtrack.py}

Как вы видите, всё что он делает, это перечисляет все возможные решения, но отсекает поисковое дерево настолько рано,
насколько это возможно.
Это и есть поиск с возвратом (\textit{backtracking}).

Файлы: \url{https://github.com/DennisYurichev/SAT_SMT_article/tree/master/SAT/backtrack}.

Некоторые комментарии: \url{https://www.reddit.com/r/compsci/comments/6jn3th/simplest_sat_solver_in_120_lines/}.



}



\section{Acronyms used}

\begin{acronym}
\acro{CNF}{Conjunctive normal form}
\acro{DNF}{Disjunctive normal form}
\acro{DSL}{Domain-specific language}
\acro{CPRNG}{Cryptographically Secure Pseudorandom Number Generator}
\acro{SMT}{Satisfiability modulo theories}
\acro{SAT}{Boolean satisfiability problem}
\acro{LCG}{Linear congruential generator}
\acro{PL}{Programming Language}
\acro{OOP}{Object-oriented programming}
\acro{SSA}{Static single assignment form}
\acro{CPU}{Central processing unit}
\acro{FPU}{Floating-point unit}
\acro{PRNG}{Pseudorandom number generator}
\acro{CRT}{C runtime library}
\acro{CRC}{Cyclic redundancy check}
\acro{AST}{Abstract syntax tree}
\acro{AKA}{Also Known As}
\acro{CTF}{Capture the Flag}
\acro{ISA}{Instruction Set Architecture}
\end{acronym}

\end{document}


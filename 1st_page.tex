% To translators: don't bother to translate this... english-only version.

\begin{center}
\LARGE{} This is my own bulletin board \normalsize{}
\end{center}

My dear readers! From time to time, I have questions, I don't know who (or where) to ask.
Or I just lazy...
Can you please help me?

\myhrule{}

I have a set of sets.

Some of them intersects with each other, some are not: (a,b,c) (b,c,d) (x,z) (z,y)

I want to group interesting ones, so the output will be: (a,b,c,d) (x,y,z)

What is the proper formal name for such an operation?

\myhrule{}

A delivery service company uses 10 (decimal) digit tracking numbers.
Numbers are often transmitten by phone, scribbled on scraps of paper, etc.
You can add, say, 2-3 more (decimal) digits to add some redundancy, so the 10-digit code would be restored
if one digit was transmitten incorrectly.
How to do it?

\myhrule{}

Is there a formal name for such a graphs, that are somewhat similar to codeflow graphs?
They are also planar.
And has some kind of "nested" structure.

\begin{figure}[H]
\centering
\includegraphics[width=\textwidth]{1st_page}
\caption{A graph}
\end{figure}

\myhrule{}

If you know something, please help me: dennis@yurichev.com


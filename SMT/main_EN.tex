\section{\ac{SMT}-solvers}

\subsection{School-level system of equations}

I've got this school-level system of equations copypasted from Wikipedia
\footnote{\url{https://en.wikipedia.org/wiki/System_of_linear_equations}}:

\begin{alignat*}{7}
3x &&\; + \;&& 2y             &&\; - \;&& z  &&\; = \;&& 1 & \\
2x &&\; - \;&& 2y             &&\; + \;&& 4z &&\; = \;&& -2 & \\
-x &&\; + \;&& \tfrac{1}{2} y &&\; - \;&& z  &&\; = \;&& 0 &
\end{alignat*}

Will it be possible to solve it using Z3? Here it is:

\begin{lstlisting}
#!/usr/bin/python
from z3 import *

x = Real('x')
y = Real('y')
z = Real('z')
s = Solver()
s.add(3*x + 2*y - z == 1)
s.add(2*x - 2*y + 4*z == -2)
s.add(-x + 0.5*y - z == 0)
print s.check()
print s.model()
\end{lstlisting}

We see this after run:

\begin{lstlisting}
sat
[z = -2, y = -2, x = 1]
\end{lstlisting}

If we change any equation in some way so it will have no solution, s.check() will return ``unsat''.

I've used ``Real'' \textit{sort} (some kind of data type in \ac{SMT}-solvers)
because the last expression equals to $\frac{1}{2}$, which is, of course, a real number.
For the integer system of equations, ``Int'' \textit{sort} would work fine.

Python (and other high-level \ac{PL}s like C\#) interface is highly popular, because it's practical, but in fact, 
there is a standard language for \ac{SMT}-solvers called SMT-LIB
\footnote{\url{http://smtlib.cs.uiowa.edu/papers/smt-lib-reference-v2.5-r2015-06-28.pdf}}.

Our example rewritten to it looks like this:

\begin{lstlisting}
(declare-const x Real)
(declare-const y Real)
(declare-const z Real)
(assert (=(-(+(* 3 x) (* 2 y)) z) 1))
(assert (=(+(-(* 2 x) (* 2 y)) (* 4 z)) -2))
(assert (=(-(+ (- 0 x) (* 0.5 y)) z) 0))
(check-sat)
(get-model)
\end{lstlisting}

This language is very close to LISP, but is somewhat hard to read for untrained eyes.

Now we run it:

\begin{lstlisting}
% z3 -smt2 example.smt
sat
(model
  (define-fun z () Real
    (- 2.0))
  (define-fun y () Real
    (- 2.0))
  (define-fun x () Real
    1.0)
)
\end{lstlisting}

So when you look back to my Python code, you may feel that these 3 expressions could be executed.
This is not true: Z3Py API offers overloaded operators, so expressions are constructed and passed into the guts of Z3 without any execution
\footnote{\url{https://github.com/Z3Prover/z3/blob/6e852762baf568af2aad1e35019fdf41189e4e12/src/api/python/z3.py}}.
I would call it ``embedded \ac{DSL}''.

Same thing for Z3 C++ API, you may find there ``operator+'' declarations and many more
\footnote{\url{https://github.com/Z3Prover/z3/blob/6e852762baf568af2aad1e35019fdf41189e4e12/src/api/c\%2B\%2B/z3\%2B\%2B.h}}.

Z3 \ac{API}s for Java, ML and .NET are also exist
\footnote{\url{https://github.com/Z3Prover/z3/tree/6e852762baf568af2aad1e35019fdf41189e4e12/src/api}}.\\
\\
Z3Py tutorial: \url{https://github.com/ericpony/z3py-tutorial}.

Z3 tutorial which uses SMT-LIB language: \url{http://rise4fun.com/Z3/tutorial/guide}.

\subsection{Another school-level system of equations}
\label{eq2_SMT}

I've found this somewhere at Facebook:

\begin{figure}[H]
\centering
\includegraphics[scale=0.3]{SMT/equation.jpg}
\caption{System of equations}
\end{figure}

It's that easy to solve it in Z3:

\begin{lstlisting}
#!/usr/bin/python
from z3 import *

circle, square, triangle = Ints('circle square triangle')
s = Solver()
s.add(circle+circle==10)
s.add(circle*square+square==12)
s.add(circle*square-triangle*circle==circle)
print s.check()
print s.model()
\end{lstlisting}

\begin{lstlisting}
sat
[triangle = 1, square = 2, circle = 5]
\end{lstlisting}

\subsection{Connection between \ac{SAT} and \ac{SMT} solvers}

\ac{SMT}-solvers are frontends to \ac{SAT} solvers, i.e.,
they translating input SMT expressions into \ac{CNF} and feed SAT-solver with it.
Translation process is sometimes called ``bit blasting''.
Some \ac{SMT}-solvers uses external SAT-solver: STP uses MiniSAT or CryptoMiniSAT as backend.
Some other \ac{SMT}-solvers (like Z3) has their own SAT solver.

% subsections
\section{Yet another explanation of modulo inverse using SMT-solvers}

\MathForProg has a part about modulo arithmetics and modulo inverse.

By which constant we must multiply a random number, so that the result would be as if we divided them by 3?

\begin{lstlisting}
from z3 import *

m=BitVec('m', 32)

s=Solver()

# wouldn't work for 10, etc
divisor=3

# random constant, must be divisible by divisor:
const=(0x1234567*divisor)

s.add(const*m == const/divisor)

print s.check()
print "%x" % s.model()[m].as_long()
\end{lstlisting}

The magic number is:

\begin{lstlisting}
sat
aaaaaaab
\end{lstlisting}

Indeed, this is modulo inverse of 3 modulo $2^{32}$: \url{https://www.wolframalpha.com/input/?i=PowerMod%5B3,-1,2%5E32%5D}.

Let's check using \href{https://github.com/DennisYurichev/progcalc}{my calculator}:

\begin{lstlisting}
[3] 123456*0xaaaaaaab
[3] (unsigned) 353492988371136 0x141800000a0c0 0b1010000011000000000000000000000001010000011000000
[4] 123456/3
[4] (unsigned) 41152 0xa0c0 0b1010000011000000
\end{lstlisting}

The problem is simple enough to be solved using MK85:

\lstinputlisting[style=customsmt]{equations/modinv/modinv.smt}

\lstinputlisting{equations/modinv/modinv.correct}

However, it wouldn't work for 10, because there are no modulo inverse of 10 modulo $2^{32}$, SMT solver would give "unsat".


\section{Yet another explanation of modulo inverse using SMT-solvers}

\MathForProg has a part about modulo arithmetics and modulo inverse.

By which constant we must multiply a random number, so that the result would be as if we divided them by 3?

\begin{lstlisting}
from z3 import *

m=BitVec('m', 32)

s=Solver()

# wouldn't work for 10, etc
divisor=3

# random constant, must be divisible by divisor:
const=(0x1234567*divisor)

s.add(const*m == const/divisor)

print s.check()
print "%x" % s.model()[m].as_long()
\end{lstlisting}

The magic number is:

\begin{lstlisting}
sat
aaaaaaab
\end{lstlisting}

Indeed, this is modulo inverse of 3 modulo $2^{32}$: \url{https://www.wolframalpha.com/input/?i=PowerMod%5B3,-1,2%5E32%5D}.

Let's check using \href{https://github.com/DennisYurichev/progcalc}{my calculator}:

\begin{lstlisting}
[3] 123456*0xaaaaaaab
[3] (unsigned) 353492988371136 0x141800000a0c0 0b1010000011000000000000000000000001010000011000000
[4] 123456/3
[4] (unsigned) 41152 0xa0c0 0b1010000011000000
\end{lstlisting}

The problem is simple enough to be solved using MK85:

\lstinputlisting[style=customsmt]{equations/modinv/modinv.smt}

\lstinputlisting{equations/modinv/modinv.correct}

However, it wouldn't work for 10, because there are no modulo inverse of 10 modulo $2^{32}$, SMT solver would give "unsat".


\subsection{Zebra puzzle}

Let's revisit zebra puzzle from (\ref{zebra_SMT}).

We just define all variables and add constraints:

\lstinputlisting{KLEE/klee_zebra1.c}

I force KLEE to find distinct values for colors, nationalities, cigarettes, etc, in the same way as I did for Sudoku earlier 
(\ref{sudoku_SMT}).

Let's run it:

% FIXME:
\begin{lstlisting}
% clang -emit-llvm -c -g klee_zebra1.c
...

% klee klee_zebra1.bc
KLEE: output directory is "/home/klee/klee-out-97"
KLEE: WARNING: undefined reference to function: klee_assert
KLEE: WARNING ONCE: calling external: klee_assert(0)
KLEE: ERROR: /home/klee/klee_zebra1.c:130: failed external call: klee_assert
KLEE: NOTE: now ignoring this error at this location

KLEE: done: total instructions = 761
KLEE: done: completed paths = 55
KLEE: done: generated tests = 55
\end{lstlisting}

It works for $\approx 7$ seconds on my Intel Core i3-3110M 2.4GHz notebook.
Let's find out path, where \TT{klee\_assert()} has been executed:

% FIXME:
\begin{lstlisting}
% ls klee-last | grep err
test000051.external.err

% ktest-tool --write-ints klee-last/test000051.ktest | less

ktest file : 'klee-last/test000051.ktest'
args       : ['klee_zebra1.bc']
num objects: 25
object    0: name: b'Yellow'
object    0: size: 4
object    0: data: 1
object    1: name: b'Blue'
object    1: size: 4
object    1: data: 2
object    2: name: b'Red'
object    2: size: 4
object    2: data: 3
object    3: name: b'Ivory'
object    3: size: 4
object    3: data: 4

...

object   21: name: b'Horse'
object   21: size: 4
object   21: data: 2
object   22: name: b'Snails'
object   22: size: 4
object   22: data: 3
object   23: name: b'Dog'
object   23: size: 4
object   23: data: 4
object   24: name: b'Zebra'
object   24: size: 4
object   24: data: 5
\end{lstlisting}

This is indeed correct solution.

\TT{klee\_assume()} also can be used this time:

\lstinputlisting{KLEE/klee_zebra2.c}

\dots and this version works slightly faster ($\approx 5$ seconds),
maybe because KLEE is aware of this \textit{intrinsic} and handles it in a special way?


\subsection{Solving Problem Euler 31: ``Coin sums''}

(This text was first published in my blog\footnote{\url{http://dennisyurichev.blogspot.de/2013/05/in-england-currency-is-made-up-of-pound.html}} at 10-May-2013.)

\begin{framed}
\begin{quotation}
In England the currency is made up of pound, £, and pence, p, and there are eight coins in general circulation:

1p, 2p, 5p, 10p, 20p, 50p, £1 (100p) and £2 (200p).
It is possible to make £2 in the following way:

1£1 + 150p + 220p + 15p + 12p + 31p
How many different ways can £2 be made using any number of coins?
\end{quotation}
\end{framed}
( \href{http://projecteuler.net/problem=31}{Problem Euler 31 --- Coin sums} )

\label{SMTEnumerate}
Using Z3 for solving this is overkill, and also slow, but nevertheless, it works, showing all possible solutions as well.
The piece of code for blocking already found solution and search for next, and thus, counting all solutions, was taken from Stack Overflow answer
\footnote{\url{http://stackoverflow.com/questions/11867611/z3py-checking-all-solutions-for-equation}, 
another question: \url{http://stackoverflow.com/questions/13395391/z3-finding-all-satisfying-models}}.
This is also called ``model counting''.
Constraints like ``a>=0'' must be present, because Z3 solver will find solutions with negative numbers.

\begin{lstlisting}
#!/usr/bin/python

from z3 import *

a,b,c,d,e,f,g,h = Ints('a b c d e f g h')
s = Solver()
s.add(1*a + 2*b + 5*c + 10*d + 20*e + 50*f + 100*g + 200*h == 200, 
   a>=0, b>=0, c>=0, d>=0, e>=0, f>=0, g>=0, h>=0)
result=[]

while True:
    if s.check() == sat:
        m = s.model()
        print m
        result.append(m)
        # Create a new constraint the blocks the current model
        block = []
        for d in m:
            # d is a declaration
            if d.arity() > 0:
                raise Z3Exception("uninterpreted functions are not suppported")
            # create a constant from declaration
            c=d()
            #print c, m[d]
            if is_array(c) or c.sort().kind() == Z3_UNINTERPRETED_SORT:
                raise Z3Exception("arrays and uninterpreted sorts are not supported")
            block.append(c != m[d])
        #print "new constraint:",block
        s.add(Or(block))
    else:
        print len(result)
        break
\end{lstlisting}

Works very slow, and this is what it produces:

\begin{lstlisting}
[h = 0, g = 0, f = 0, e = 0, d = 0, c = 0, b = 0, a = 200]
[f = 1, b = 5, a = 0, d = 1, g = 1, h = 0, c = 2, e = 1]
[f = 0, b = 1, a = 153, d = 0, g = 0, h = 0, c = 1, e = 2]
...
[f = 0, b = 31, a = 33, d = 2, g = 0, h = 0, c = 17, e = 0]
[f = 0, b = 30, a = 35, d = 2, g = 0, h = 0, c = 17, e = 0]
[f = 0, b = 5, a = 50, d = 2, g = 0, h = 0, c = 24, e = 0]
\end{lstlisting}

73682 results in total.

\section{Using Z3 theorem prover to prove equivalence of some weird alternative to XOR operation}
\label{weird_XOR}

(The test was first published in my blog at April 2015: \url{http://blog.yurichev.com/node/86}).

There is a ``A Hacker's Assistant'' program\footnote{\url{http://www.hackersdelight.org/}} (\textit{Aha!}) written by Henry Warren,
who is also the author of the great ``Hacker's Delight'' book.

The \textit{Aha!} program is essentially \textit{superoptimizer}\footnote{\url{http://en.wikipedia.org/wiki/Superoptimization}},
which blindly brute-force a list of some generic RISC CPU instructions to achieve shortest possible (and jumpless or branch-free) 
CPU code sequence for desired operation.
For example, \textit{Aha!} can find jumpless version of abs() function easily.

Compiler developers use superoptimization to find shortest possible (and/or jumpless) code,
but I tried to do otherwise---to find longest code for some primitive operation.
I tried \textit{Aha!} to find equivalent of basic XOR operation without usage of the actual XOR instruction,
and the most bizarre example \textit{Aha!} gave is:

\begin{lstlisting}
Found a 4-operation program:
   add   r1,ry,rx
   and   r2,ry,rx
   mul   r3,r2,-2
   add   r4,r3,r1
   Expr: (((y & x)*-2) + (y + x))
\end{lstlisting}

And it's hard to say, why/where we can use it, maybe for obfuscation, I'm not sure.
I would call this \textit{suboptimization} (as opposed to \textit{superoptimization}).
Or maybe \textit{superdeoptimization}.

But my another question was also, is it possible to prove that this is correct formula at all?
The \textit{Aha!} checking some intput/output values against XOR operation, but of course, not all the possible values.
It is 32-bit code, so it may take very long time to try all possible 32-bit inputs to test it.

We can try Z3 theorem prover for the job. It's called \textit{prover}, after all.

So I wrote this:

\begin{lstlisting}
#!/usr/bin/python
from z3 import *

x = BitVec('x', 32)
y = BitVec('y', 32)
output = BitVec('output', 32)
s = Solver()
s.add(x^y==output)
s.add(((y & x)*0xFFFFFFFE) + (y + x)!=output)
print s.check()
\end{lstlisting}

In plain English language, this means
``are there any case for $x$ and $y$ where $x \oplus y$ doesn't equals to $((y \& x)*-2) + (y + x)$?''
\dots and Z3 prints ``unsat'', meaning, it can't find any counterexample to the equation.
So this \textit{Aha!} result is proved to be working just like XOR operation.

Oh, I also tried to extend the formula to 64 bit:

\begin{lstlisting}
#!/usr/bin/python
from z3 import *

x = BitVec('x', 64)
y = BitVec('y', 64)
output = BitVec('output', 64)
s = Solver()
s.add(x^y==output)
s.add(((y & x)*0xFFFFFFFE) + (y + x)!=output)
print s.check()
\end{lstlisting}

Nope, now it says ``sat'', meaning, Z3 found at least one counterexample.
Oops, it's because I forgot to extend -2 number to 64-bit value:

\begin{lstlisting}
#!/usr/bin/python
from z3 import *

x = BitVec('x', 64)
y = BitVec('y', 64)
output = BitVec('output', 64)
s = Solver()
s.add(x^y==output)
s.add(((y & x)*0xFFFFFFFFFFFFFFFE) + (y + x)!=output)
print s.check()
\end{lstlisting}

Now it says ``unsat'', so the formula given by \textit{Aha!} works for 64-bit code as well.

\subsection{In SMT-LIB form}

Now we can rephrase our expression to more suitable form: $(x + y - ((x \& y)<<1))$.
It also works well in Z3Py:

\begin{lstlisting}
#!/usr/bin/python
from z3 import *

x = BitVec('x', 64)
y = BitVec('y', 64)
output = BitVec('output', 64)
s = Solver()
s.add(x^y==output)
s.add((x + y - ((x & y)<<1)) != output)
print s.check()
\end{lstlisting}

Here is how to define it in SMT-LIB way:

\begin{lstlisting}
(declare-const x (_ BitVec 64))
(declare-const y (_ BitVec 64))
(assert 
	(not
		(=
			(bvsub
				(bvadd x y)
				(bvshl (bvand x y) (_ bv1 64)))
			(bvxor x y)
		)
	)
)
(check-sat)
\end{lstlisting}

\subsection{Using universal quantifier}

Z3 supports universal quantifier \TT{exists}, which is true
if at least one set of variables satistfied underlying condition:

\begin{lstlisting}
(declare-const x (_ BitVec 64))
(declare-const y (_ BitVec 64))
(assert 
	(exists ((x (_ BitVec 64)) (y (_ BitVec 64)))
		(not (=
			(bvsub 
				(bvadd x y)
				(bvshl (bvand x y) (_ bv1 64))
			)
			(bvxor x y)
		))
	)
)
(check-sat)
\end{lstlisting}

It returns ``unsat'', meaning, Z3 couldn't find any counterexample of the equation, i.e., it's not exist.\\
\\
This is also known as $\exists$ in mathematical logic lingo.\\
\\
Z3 also supports universal quantifier \TT{forall}, which is true if the equation is true for all
possible values.
So we can rewrite our SMT-LIB example as:

\begin{lstlisting}
(declare-const x (_ BitVec 64))
(declare-const y (_ BitVec 64))
(assert 
	(forall ((x (_ BitVec 64)) (y (_ BitVec 64)))
		(=
			(bvsub 
				(bvadd x y)
				(bvshl (bvand x y) (_ bv1 64))
			)
			(bvxor x y)
		)
	)
)
(check-sat)
\end{lstlisting}

It returns ``sat'', meaning, the equation is correct for all possible 64-bit \TT{x} and \TT{y} values,
like them all were checked.

Mathematically speaking: $\forall n\!\in\!\mathbb{N}\; (x \oplus y = (x + y - ((x \& y)<<1)))$
\footnote{
$\forall$ means \textit{equation must be true for all possible values}, which are choosen from natural numbers ($\mathbb{N}$).}

\subsection{How the expression works}

First of all, binary addition can be viewed as binary XORing with carrying (\ref{adder}).
Here is an example: let's add 2 (10b) and 2 (10b).
XORing these two values resulting 0, but there is a carry generated during addition of two second bits.
That carry bit is propagated further and settles at the place of the 3rd bit: 100b.
4 (100b) is hence a final result of addition.

If the carry bits are not generated during addition, the addition operation is merely XORing.
For example, let's add 1 (1b) and 2 (10b). $1 + 2$ equals to 3, but $1 \oplus 2$ is also 3.

If the addition is XORing plus carry generation and application, we should eliminate effect of carrying somehow here.
The first part of the expression ($x + y$) is addition, the second ($(x \& y)<<1$) is just calculation of every carry bit which was used during addition.
If to subtract carry bits from the result of addition, the only XOR effect is left then.

It's hard to say how Z3 proves this:
maybe it just simplifies the equation down to single XOR using simple boolean algebra rewriting rules?

   % \\
\section{Dietz's formula}
\label{Dietz}

One of the impressive examples of \textit{Aha!} work is finding of Dietz's formula\footnote{\url{http://aggregate.org/MAGIC/\#Average\%20of\%20Integers}},
which is the code of computing average number of two numbers without overflow (which is important if you want to find average number of numbers like 0xFFFFFF00 and so on, using 32-bit registers).

Taking this in input:

\begin{lstlisting}
int userfun(int x, int y) {     // To find Dietz's formula for
                                // the floor-average of two
                                // unsigned integers.
   return ((unsigned long long)x + (unsigned long long)y) >> 1;
}
\end{lstlisting}

\dots the \textit{Aha!} gives this:

\begin{lstlisting}
Found a 4-operation program:
   and   r1,ry,rx
   xor   r2,ry,rx
   shrs  r3,r2,1
   add   r4,r3,r1
   Expr: (((y ^ x) >>s 1) + (y & x))
\end{lstlisting}

And it works correctly\footnote{For those who interesting how it works,
its mechanics is closely related to the weird XOR alternative we just saw.
That's why I placed these two pieces of text one after another.}.
But how to prove it?

We will place Dietz's formula on the left side of equation and $x+y/2$ (or $x+y>>1$) on the right side:

\begin{center}
$\forall n \in 0..2^{64}-1 . (x\&y) + (x \oplus y)>>1 = x+y>>1$
\end{center}

One important thing is that we can't operate on 64-bit values on right side, because result will overflow.
So we will zero extend inputs on right side by 1 bit (in other words, we will just 1 zero bit before each value).
The result of Dietz's formula will also be extended by 1 bit.
Hence, both sides of the equation will have a width of 65 bits:

\begin{lstlisting}
(declare-const x (_ BitVec 64))
(declare-const y (_ BitVec 64))
(assert 
	(forall ((x (_ BitVec 64)) (y (_ BitVec 64)))
		(=
			((_ zero_extend 1)
				(bvadd
					(bvand x y)
					(bvlshr (bvxor x y) (_ bv1 64))
				)
			)
			(bvlshr
				(bvadd ((_ zero_extend 1) x) ((_ zero_extend 1) y))
				(_ bv1 65)
			)
		)
	)
)
(check-sat)
\end{lstlisting}

Z3 says ``sat''.\\
\\
65 bits are enough, because the result of addition of two biggest 64-bit values has width of 65 bits: \\
\TT{0xFF...FF + 0xFF...FF = 0x1FF...FE}.\\
\\
As in previous example about XOR equivalent, \TT{(not (= ... ))} and \TT{exists} can also be used here instead of \TT{forall}.

 % //
\subsection{XOR swapping algorithm}

This is well-known XOR swap algorithm (which don't use additional variable).
How it works?

\lstinputlisting[numbers=left,style=custompy]{proofs/xor_swap_Z3_check.py}

Now we see a final states of X/Y variables:

\begin{lstlisting}
X= init_X ^ init_Y ^ init_Y ^ init_X ^ init_Y
Y= init_Y ^ init_X ^ init_Y
unsat
\end{lstlisting}

Z3 gave "unsat", meaning, it can't find any counterexample to the last equation (line 18).
Hence, the equation is correct and so is the whole algorithm.

\subsubsection{In SMT-LIB form}

\lstinputlisting{proofs/XOR_swap.smt}

\lstinputlisting{proofs/XOR_swap2.smt}


\section{Cracking \ac{LCG} with Z3}

There are well-known weaknesses of \ac{LCG}
\footnote{\url{http://en.wikipedia.org/wiki/Linear_congruential_generator\#Advantages_and_disadvantages_of_LCGs},
\url{http://www.reteam.org/papers/e59.pdf},
\url{http://stackoverflow.com/questions/8569113/why-1103515245-is-used-in-rand/8574774\#8574774}},
but let's see, if it would be possible to crack it straightforwardly, without any special knowledge.
We will define all relations between LCG states in terms of Z3.
Here is a test progam:

\begin{lstlisting}
#include <stdlib.h>
#include <stdio.h>
#include <time.h>

int main()
{
	int i;

	srand(time(NULL));

	for (i=0; i<10; i++)
		printf ("%d\n", rand()%100);
};
\end{lstlisting}

It is printing 10 pseudorandom numbers in 0..99 range:

\begin{lstlisting}
37
29
74
95
98
40
23
58
61
17
\end{lstlisting}

Let's say we are observing only 8 of these numbers (from 29 to 61) and we need to predict next one (17) and/or previous one (37).

The program is compiled using MSVC 2013 (I choose it because its LCG is simpler than that in Glib):

\begin{lstlisting}
.text:0040112E rand            proc near
.text:0040112E                 call    __getptd
.text:00401133                 imul    ecx, [eax+0x14], 214013
.text:0040113A                 add     ecx, 2531011
.text:00401140                 mov     [eax+14h], ecx
.text:00401143                 shr     ecx, 16
.text:00401146                 and     ecx, 7FFFh
.text:0040114C                 mov     eax, ecx
.text:0040114E                 retn
.text:0040114E rand            endp
\end{lstlisting}

Let's define \ac{LCG} in Z3Py:

\begin{lstlisting}
#!/usr/bin/python
from z3 import *

output_prev = BitVec('output_prev', 32)
state1 = BitVec('state1', 32)
state2 = BitVec('state2', 32)
state3 = BitVec('state3', 32)
state4 = BitVec('state4', 32)
state5 = BitVec('state5', 32)
state6 = BitVec('state6', 32)
state7 = BitVec('state7', 32)
state8 = BitVec('state8', 32)
state9 = BitVec('state9', 32)
state10 = BitVec('state10', 32)
output_next = BitVec('output_next', 32)

s = Solver()

s.add(state2 == state1*214013+2531011)
s.add(state3 == state2*214013+2531011)
s.add(state4 == state3*214013+2531011)
s.add(state5 == state4*214013+2531011)
s.add(state6 == state5*214013+2531011)
s.add(state7 == state6*214013+2531011)
s.add(state8 == state7*214013+2531011)
s.add(state9 == state8*214013+2531011)
s.add(state10 == state9*214013+2531011)

s.add(output_prev==URem((state1>>16)&0x7FFF,100))
s.add(URem((state2>>16)&0x7FFF,100)==29)
s.add(URem((state3>>16)&0x7FFF,100)==74)
s.add(URem((state4>>16)&0x7FFF,100)==95)
s.add(URem((state5>>16)&0x7FFF,100)==98)
s.add(URem((state6>>16)&0x7FFF,100)==40)
s.add(URem((state7>>16)&0x7FFF,100)==23)
s.add(URem((state8>>16)&0x7FFF,100)==58)
s.add(URem((state9>>16)&0x7FFF,100)==61)
s.add(output_next==URem((state10>>16)&0x7FFF,100))

print(s.check())
print(s.model())
\end{lstlisting}

\emph{URem} states for \emph{unsigned remainder}.
It works for some time and gave us correct result!

\begin{lstlisting}
sat
[state3 = 2276903645,
 state4 = 1467740716,
 state5 = 3163191359,
 state7 = 4108542129,
 state8 = 2839445680,
 state2 = 998088354,
 state6 = 4214551046,
 state1 = 1791599627,
 state9 = 548002995,
 output_next = 17,
 output_prev = 37,
 state10 = 1390515370]
\end{lstlisting}

I added $\approx 10$ states to be sure result will be correct.
It may be not in case of smaller set of information.

That is the reason why \ac{LCG} is not suitable for any security-related task.
This is why cryptographically secure pseudorandom number generators exist:
they are designed to be protected against such simple attack.
Well, at least if \ac{NSA} don't get involved
\footnote{\url{https://en.wikipedia.org/wiki/Dual_EC_DRBG}}.

Security tokens like ``RSA SecurID'' can be viewed just as \ac{CPRNG} with a secret seed.
It shows new pseudorandom number each minute, and the server can predict it, because it knows the seed.
Imagine if such token would implement \ac{LCG}---it would be much easier to break!


\subsection{Can rand() generate 10 consecutive zeroes?}

\renewcommand{\CURPATH}{equations/LCG}

I've always been wondering, if it's possible or not.
As of simplest linear congruential generator from MSVC's rand(), I could get a state at which rand() will output 8 zeroes modulo 10:

\lstinputlisting[style=custompy]{\CURPATH/LCG10.py}

\begin{lstlisting}
sat
[state3 = 1181667981,
 state4 = 342792988,
 state5 = 4116856175,
 state7 = 1741999969,
 state8 = 3185636512,
 state2 = 1478548498,
 state6 = 4036911734,
 state1 = 286227003,
 state9 = 1700675811]
\end{lstlisting}

This is a case if, in some video game, you'll find a code:

\begin{lstlisting}
for (int i=0; i<8; i++)
    printf ("%d\n", rand() % 10);
\end{lstlisting}

... and at some point, this piece of code can generate 8 zeroes in row, if the state will be 286227003 (decimal).

Just checked this piece of code in MSVC 2015:

\begin{lstlisting}
// MSVC 2015 x86

#include <stdio.h>

int main()
{
	srand(286227003);

	for (int i=0; i<8; i++)
		printf ("%d\n", rand() % 10);
};
\end{lstlisting}

Yes, its output is 8 zeroes!

What about other modulos?

I can get 4 consecutive zeroes modulo 100:

\lstinputlisting[style=custompy]{\CURPATH/LCG100.py}

\begin{lstlisting}
sat
[state3 = 635704497,
 state4 = 1644979376,
 state2 = 1055176198,
 state1 = 3865742399,
 state5 = 1389375667]
\end{lstlisting}

However, 4 consecutive zeroes modulo 100 is impossible (given these constants at least), this gives ``unsat'':
\url{https://github.com/DennisYurichev/SAT_SMT_by_example/blob/master/equations/LCG/LCG100_v1.py}.

... and 3 consecutive zeroes modulo 1000:

\lstinputlisting[style=custompy]{\CURPATH/LCG1000.py}

\begin{lstlisting}
sat
[state3 = 1179663182,
 state2 = 720934183,
 state1 = 4090229556,
 state4 = 786474201]
\end{lstlisting}

What if we could use rand()'s output without division? Which is in 0..0x7fff range (i.e., 15 bits)?
As it can be checked quickly, 2 zeroes at output is possible:

\lstinputlisting[style=custompy]{\CURPATH/LCG.py}

\begin{lstlisting}
sat
[state2 = 20057, state1 = 3385131726, state3 = 22456]
\end{lstlisting}

\subsubsection{UNIX time and srand(time(NULL))}

Given the fact that it's highly popular to initialize LCG PRNG with UNIX time (i.e., \TT{srand(time(NULL))}), you can probably calculate a moment in time so that LCG PRNG will be initialized as you want to.

For example, can we get a moment in time from now (5-Dec-2017) till 12-Dec-2017 (that is one week from now), when, if initialized by UNIX time, rand() will output as many similar numbers (modulo 10), as possible?

\lstinputlisting[style=custompy]{\CURPATH/LCG10_time.py}

Yes:

\begin{lstlisting}
sat
[state3 = 2234253076,
 state4 = 497021319,
 state5 = 4160988718,
 c = 3,
 state2 = 333151205,
 state6 = 46785593,
 state1 = 1512500810,
 state7 = 1158878744]
\end{lstlisting}

If \TT{srand(time(NULL))} will be executed at \TT{Tue Dec  5 21:06:50 EET 2017} (this precise second, UNIX time=1512500810),
a next 6 \TT{rand() \% 10} lines will output six numbers of 3 in a row.
Don't know if it useful or not, but you've got the idea.

\subsubsection{etc:}

The files: \url{https://github.com/DennisYurichev/SAT_SMT_by_example/tree/master/equations/LCG}.

Further work: check glibc's \TT{rand()}, Mersenne Twister, etc. Simple 32-bit LCG as described can be checked using simple brute-force, I think.

\subsubsection{Fun story}

The software checked protection key (dongle) randomly, from time to time.
This code snippet is from a real one:

\begin{lstlisting}[style=customc]
void init_all()
{
	...

	srand(time(NULL));

	...
};

...

void check_protection_thread()
{
	// get in 0..9 range
	int t=(int)((double)rand()/3276);
	if (t== 5)
	{
		check protection
	}
};
\end{lstlisting}

Perhaps, we can find the most optimal UNIX time to start the software, so the protection will not be checked as long as possible...

\subsubsection{Further reading}

Breaking JavaScript's \ac{PRNG} (XorShift128+):
\url{https://blog.securityevaluators.com/hacking-the-javascript-lottery-80cc437e3b7f}.


\input{SMT/pipe_EN}
\section{Yet another explanation of modulo inverse using SMT-solvers}

\MathForProg has a part about modulo arithmetics and modulo inverse.

By which constant we must multiply a random number, so that the result would be as if we divided them by 3?

\begin{lstlisting}
from z3 import *

m=BitVec('m', 32)

s=Solver()

# wouldn't work for 10, etc
divisor=3

# random constant, must be divisible by divisor:
const=(0x1234567*divisor)

s.add(const*m == const/divisor)

print s.check()
print "%x" % s.model()[m].as_long()
\end{lstlisting}

The magic number is:

\begin{lstlisting}
sat
aaaaaaab
\end{lstlisting}

Indeed, this is modulo inverse of 3 modulo $2^{32}$: \url{https://www.wolframalpha.com/input/?i=PowerMod%5B3,-1,2%5E32%5D}.

Let's check using \href{https://github.com/DennisYurichev/progcalc}{my calculator}:

\begin{lstlisting}
[3] 123456*0xaaaaaaab
[3] (unsigned) 353492988371136 0x141800000a0c0 0b1010000011000000000000000000000001010000011000000
[4] 123456/3
[4] (unsigned) 41152 0xa0c0 0b1010000011000000
\end{lstlisting}

The problem is simple enough to be solved using MK85:

\lstinputlisting[style=customsmt]{equations/modinv/modinv.smt}

\lstinputlisting{equations/modinv/modinv.correct}

However, it wouldn't work for 10, because there are no modulo inverse of 10 modulo $2^{32}$, SMT solver would give "unsat".


\subsection{Recalculating micro-spreadsheet using Z3Py}

There is a nice exercise\footnote{Blog post in Russian: \url{http://thesz.livejournal.com/280784.html}}:
write a program to recalculate micro-spreadsheet, like this one:

\lstinputlisting{SMT/spreadsheet/test1}

As it turns out, though overkill, this can be solved using Z3 with little effort:

\lstinputlisting{SMT/spreadsheet/1.py}

( \url{https://github.com/DennisYurichev/yurichev.com/blob/master/blog/spreadsheet/1.py} )

All we do is just creating pack of variables for each cell, named A0, B1, etc, of integer type.
All of them are stored in \textit{cells[]} dictionary.
Key is a string.
Then we parse all the strings from cells, and add to list of constraints \textit{A0=123}
(in case of number in cell) or \textit{A0=B1+C2} (in case of expression in cell).
There is a slight preparation: string like \textit{A0+B2} becomes \textit{cells["A0"]+cells["B2"]}.

Then the string is evaluated using Python \textit{eval()} method,
which is highly dangerous
\footnote{\url{http://stackoverflow.com/questions/1832940/is-using-eval-in-python-a-bad-practice}}:
imagine if end-user could add a string to cell other than expression?
Nevertheless, it serves our purposes well, because this is a simplest way to pass a string with expression into Z3.

Z3 do the job with little effort:

\begin{lstlisting}
 % python 1.py test1
sat
1       0       135     82041
123     10      12      11
667     11      1342    83383
\end{lstlisting}

\subsubsection{Unsat core}

Now the problem: what if there is circular dependency? Like:

\lstinputlisting{SMT/spreadsheet/test_circular}

Two first cells of the last row (C0 and C1) are linked to each other.
Our program will just tells ``unsat'', meaning, it couldn't satisfy all constraints together.
We can't use this as error message reported to end-user, because it's highly unfriendly.

However, we can fetch \textit{unsat core}, i.e., list of variables which Z3 finds conflicting.

\begin{lstlisting}
...
s=Solver()
s.set(unsat_core=True)
...
        # add constraint:
        s.assert_and_track(e, coord_to_name(cur_R, cur_C))
...
if result=="sat":
...
else:
    print s.unsat_core()
\end{lstlisting}

( \url{https://github.com/DennisYurichev/yurichev.com/blob/master/blog/spreadsheet/2.py} )

We should explicitly turn on unsat core support and use \textit{assert\_and\_track()} instead of \textit{add()} method,
because this feature slows down the whole process, and is turned off by default.
That works:

\begin{lstlisting}
 % python 2.py test_circular
unsat
[C0, C1]
\end{lstlisting}

Perhaps, these variables could be removed from the 2D array, marked as \textit{unresolved}
and the whole spreadsheet could be recalculated again.

\subsubsection{Stress test}

How to generate large random spreadsheet?
What we can do.
First, create random \ac{DAG}, like this one:

\begin{figure}[H]
\centering
\includegraphics[width=\textwidth]{SMT/spreadsheet/1.png}
\caption{Random DAG}
\end{figure}

Arrows will represent information flow.
So a vertex (node) which has no incoming arrows to it (indegree=0), can be set to a random number.
Then we use topological sort to find dependencies between vertices.
Then we assign spreadsheet cell names to each vertex.
Then we generate random expression with random operations/numbers/cells to each cell,
with the use of information from topological sorted graph.

Wolfram Mathematica:

\begin{lstlisting}
(* Utility functions *)
In[1]:= findSublistBeforeElementByValue[lst_,element_]:=lst[[ 1;;Position[lst, element][[1]][[1]]-1]]

(* Input in 1..∞ range. 1->A0, 2->A1, etc *)
In[2]:= vertexToName[x_,width_]:=StringJoin[FromCharacterCode[ToCharacterCode["A"][[1]]+Floor[(x-1)/width]],ToString[Mod[(x-1),width]]]

In[3]:= randomNumberAsString[]:=ToString[RandomInteger[{1,1000}]]

In[4]:= interleaveListWithRandomNumbersAsStrings[lst_]:=Riffle[lst,Table[randomNumberAsString[],Length[lst]-1]]

(* We omit division operation because micro-spreadsheet evaluator can't handle division by zero *)
In[5]:= interleaveListWithRandomOperationsAsStrings[lst_]:=Riffle[lst,Table[RandomChoice[{"+","-","*"}],Length[lst]-1]]

In[6]:= randomNonNumberExpression[g_,vertex_]:=StringJoin[interleaveListWithRandomOperationsAsStrings[interleaveListWithRandomNumbersAsStrings[Map[vertexToName[#,WIDTH]&,pickRandomNonDependentVertices[g,vertex]]]]]

In[7]:= pickRandomNonDependentVertices[g_,vertex_]:=DeleteDuplicates[RandomChoice[findSublistBeforeElementByValue[TopologicalSort[g],vertex],RandomInteger[{1,5}]]]

In[8]:= assignNumberOrExpr[g_,vertex_]:=If[VertexInDegree[g,vertex]==0,randomNumberAsString[],randomNonNumberExpression[g,vertex]]

(* Main part *) 
(* Create random graph *)
In[21]:= WIDTH=7;HEIGHT=8;TOTAL=WIDTH*HEIGHT
Out[21]= 56

In[24]:= g=DirectedGraph[RandomGraph[BernoulliGraphDistribution[TOTAL,0.05]],"Acyclic"];

...

(* Generate random expressions and numbers *)
In[26]:= expressions=Map[assignNumberOrExpr[g,#]&,VertexList[g]];

(* Make 2D table of it *)
In[27]:= t=Partition[expressions,WIDTH];

(* Export as tab-separated values *)
In[28]:= Export["/home/dennis/1.txt",t,"TSV"]
Out[28]= /home/dennis/1.txt

In[29]:= Grid[t,Frame->All,Alignment->Left]
\end{lstlisting}

Here is an output from \textit{Grid[]}:

\begin{center}
\begin{tabular}{ | l | l | l | l | l | l | l |}
\hline
846 & 499 & A3*913-H4 & ... & ... & ... & ... \\
\hline
B4*860+D2 & 999 & 59 & ... & ... & ... & ... \\
\hline
G6*379-C3-436-C4-289+H6 & 972 & 804 & ... & ... & ... & ... \\
\hline
F2 & E0 & B6-731-D3+791+B4*92+C1 & ... & ... & ... & ... \\
\hline
519 & G1*402+D1*107*G3-458*A1 & D3 & ... & ... & ... & ... \\
\hline
F5-531+B5-222*E4 & 9 & B5+106*B6+600-B1 & ... & ... & ... & ... \\
\hline
C3-956*A5 & G4*408-D3*290*B6-899*G5+400+F1 & B2-701+H6 & ... & ... & ... & .. \\
\hline
B4-792*H4*407+F6-425-E1 & D2 & D3 & ... & ... & ... & ... \\
\hline
\end{tabular}
\end{center}



Using this script, I can generate random spreadsheet of $26 \cdot 500=13000$ cells,
which seems to be processed in couple of seconds.

\subsubsection{The files}

The files, including Mathematica notebook: \url{https://github.com/DennisYurichev/yurichev.com/tree/master/blog/spreadsheet}.


\subsection{Discrete tomography}

How computed tomography (CT scan) actually works?
A human body is bombarded by X-rays in various angles by X-ray tube in rotating torus.
X-ray detectors are also located in torus, and all the information is recorded.

Here is we can simulate simple tomograph.
An ``i'' character is rotating and will be ``enlighten'' at 4 angles.
Let's imagine, character is bombarded by X-ray tube at left.
All asterisks in each row is then summed and sum is "received" by X-ray detector at the right.

\begin{lstlisting}
WIDTH= 11 HEIGHT= 11
angle=(π/4)*0
    **      2
    **      2
            0
   ***      3
    **      2
    **      2
    **      2
    **      2
    **      2
   ****     4
            0
[2, 2, 0, 3, 2, 2, 2, 2, 2, 4, 0] ,
angle=(π/4)*1
            0
            0
  *         1
 **         2
    *       1
    **      2
     **     2
     ****   4
       *    1
      *     1
            0
[0, 0, 1, 2, 1, 2, 2, 4, 1, 1, 0] ,
angle=(π/4)*2
            0
            0
            0
            0
         *  1
** *******  9
** *******  9
   *     *  2
            0
            0
            0
[0, 0, 0, 0, 1, 9, 9, 2, 0, 0, 0] ,
angle=(π/4)*3
            0
            0
       *    1
       **   2
      ** *  3
     ***    3
    **      2
            0
  **        2
   *        1
            0
[0, 0, 1, 2, 3, 3, 2, 0, 2, 1, 0] ,
\end{lstlisting}

( The source code: \url{https://github.com/dennis714/SAT_SMT_article/blob/master/SMT/tomo/gen.py} )

All we got from our toy-level tomograph is 4 vectors, these are sums of all asterisks in rows for 4 angles:

\begin{lstlisting}
[2, 2, 0, 3, 2, 2, 2, 2, 2, 4, 0] ,
[0, 0, 1, 2, 1, 2, 2, 4, 1, 1, 0] ,
[0, 0, 0, 0, 1, 9, 9, 2, 0, 0, 0] ,
[0, 0, 1, 2, 3, 3, 2, 0, 2, 1, 0] ,
\end{lstlisting}

How do we recover initial image?
We are going to represent 11*11 matrix, where sum of each row must be equal to some value we already know.
Then we rotate matrix, and do this again.

The ``rotate'' function has been taken from the generation program, because, due to Python's dynamic typization nature,
it's not important for the function to what operate on:
strings, characters, or Z3 variable instances, so it works very well for all of them.

\begin{lstlisting}
#-*- coding: utf-8 -*-

import math, sys
from z3 import *

# https://en.wikipedia.org/wiki/Rotation_matrix
def rotate(pic, angle):
    WIDTH=len(pic[0])
    HEIGHT=len(pic)
    #print WIDTH, HEIGHT
    assert WIDTH==HEIGHT
    ofs=WIDTH/2

    out = [[0 for x in range(WIDTH)] for y in range(HEIGHT)]

    for x in range(-ofs,ofs):
        for y in range(-ofs,ofs):
            newX = int(round(math.cos(angle)*x - math.sin(angle)*y,3))+ofs
            newY = int(round(math.sin(angle)*x + math.cos(angle)*y,3))+ofs
            # clip at boundaries, hence min(..., HEIGHT-1)
            out[min(newX,HEIGHT-1)][min(newY,WIDTH-1)]=pic[x+ofs][y+ofs]
    return out

vectors=[
[2, 2, 0, 3, 2, 2, 2, 2, 2, 4, 0] ,
[0, 0, 1, 2, 1, 2, 2, 4, 1, 1, 0] ,
[0, 0, 0, 0, 1, 9, 9, 2, 0, 0, 0] ,
[0, 0, 1, 2, 3, 3, 2, 0, 2, 1, 0]]

WIDTH = HEIGHT = len(vectors[0])

s=Solver()
cells=[[Int('cell_r=%d_c=%d' % (r,c)) for c in range(WIDTH)] for r in range(HEIGHT)]

# monochrome picture, only 0's or 1's:
for c in range(WIDTH):
    for r in range(HEIGHT):
        s.add(Or(cells[r][c]==0, cells[r][c]==1))

def all_zeroes_in_vector(vec):
    for v in vec:
        if v!=0:
            return False
    return True

ANGLES=len(vectors)
for a in range(ANGLES):
    angle=a*(math.pi/ANGLES)
    rows=rotate(cells, angle)
    r=0
    for row in rows:
        # skip empty rows:
        if all_zeroes_in_vector(row)==False:
            # sum of row must be equal to the corresponding element of vector:
            s.add(Sum(*row)==vectors[a][r])
        r=r+1

print s.check()
m=s.model()
for r in range(HEIGHT):
    for c in range(WIDTH):
        if str(m[cells[r][c]])=="1":
            sys.stdout.write("*")
        else:
            sys.stdout.write(" ")
    print ""
\end{lstlisting}

( The source code: \url{https://github.com/dennis714/SAT_SMT_article/blob/master/SMT/tomo/solve.py} )

That works:

\begin{lstlisting}
% python solve.py
sat
    **
    **

   ***
    **
    **
    **
    **
    **
   ****
\end{lstlisting}

In other words, all SMT-solver does here is solving a system of equations.

So, 4 angles are enough.
What if we could use only 3 angles?

\begin{lstlisting}
WIDTH= 11 HEIGHT= 11
angle=(π/3)*0
    **      2
    **      2
            0
   ***      3
    **      2
    **      2
    **      2
    **      2
    **      2
   ****     4
            0
[2, 2, 0, 3, 2, 2, 2, 2, 2, 4, 0] ,
angle=(π/3)*1
            0
            0
            0
 **         2
 **         2
   ***      3
     ****   4
       **   2
       *    1
            0
            0
[0, 0, 0, 2, 2, 3, 4, 2, 1, 0, 0] ,
angle=(π/3)*2
            0
            0
            0
       **   2
       **   2
     *****  5
    **      2
 **         2
  *         1
            0
            0
[0, 0, 0, 2, 2, 5, 2, 2, 1, 0, 0] ,
\end{lstlisting}

No, it's not enough:

\begin{lstlisting}
% time python solve3.py
sat
 *  *
    **

     * **
   **
   *  *
    **
     *   *
*   *
   ****
\end{lstlisting}

However, the result is correct, but only 3 vectors allows too many possible ``initial images'',
and Z3 SMT-solver finds first.

Further reading:
\url{https://en.wikipedia.org/wiki/Discrete_tomography},
\url{https://en.wikipedia.org/wiki/2-satisfiability#Discrete_tomography}.


\section{Simplifying long and messy expressions using Mathematica and Z3}

\dots which can be results of Hex-Rays and/or manual rewriting.

I've added to my RE4B book about Wolfram Mathematica capabilities to minimize expressions
\footnote{\url{https://github.com/DennisYurichev/RE-for-beginners/blob/cd85356051937e87f90967cc272248084808223b/other/hexrays_EN.tex\#L412}, \url{https://beginners.re/}}.

Today I stumbled upon this Hex-Rays output:

\begin{lstlisting}
if ( ( x != 7 || y!=0 ) && (x < 6 || x > 7) )
{
        ...
};
\end{lstlisting}

Both Mathematica and Z3 (using ``simplify'' command) can't make it shorter, but I've got gut feeling,
that there is something redundant.

Let's take a look at the right part of the expression.
If $x$ must be less than 6 \emph{OR} greater than 7, then it can hold any value except 6 \emph{AND} 7, right?
So I can rewrite this manually:

\begin{lstlisting}
if ( ( x != 7 || y!=0 ) && x != 6 && x != 7) )
{
        ...
};
\end{lstlisting}

And this is what Mathematica can simplify:

\begin{lstlisting}
In[]:= BooleanMinimize[(x != 7 || y != 0) && (x != 6 && x != 7)]
Out[]:= x != 6 && x != 7
\end{lstlisting}

$y$ gets reduced.

But am I really right?
And why Mathematica and Z3 didn't simplify this at first place?

I can use Z3 to prove that these expressions are equal to each other:

\begin{lstlisting}
#!/usr/bin/env python

from z3 import *

x=Int('x')
y=Int('y')

s=Solver()

exp1=And(Or(x!=7, y!=0), Or(x<6, x>7))
exp2=And(x!=6, x!=7)

s.add(exp1!=exp2)

print simplify(exp1) # no luck

print s.check()
print s.model()
\end{lstlisting}

Z3 can't find counterexample, so it says ``unsat'', meaning, these expressions are equivalent to each other.
So I've rewritten this expression in my code, tests has been passed, etc.

Yes, using both Mathematica and Z3 is overkill, and this is basic boolean algebra,
but after \textasciitilde{}10 hours of sitting at a computer you can make really dumb mistakes,
and additional proof your piece of code is correct is never unwanted.


\section{Yet another explanation of modulo inverse using SMT-solvers}

\MathForProg has a part about modulo arithmetics and modulo inverse.

By which constant we must multiply a random number, so that the result would be as if we divided them by 3?

\begin{lstlisting}
from z3 import *

m=BitVec('m', 32)

s=Solver()

# wouldn't work for 10, etc
divisor=3

# random constant, must be divisible by divisor:
const=(0x1234567*divisor)

s.add(const*m == const/divisor)

print s.check()
print "%x" % s.model()[m].as_long()
\end{lstlisting}

The magic number is:

\begin{lstlisting}
sat
aaaaaaab
\end{lstlisting}

Indeed, this is modulo inverse of 3 modulo $2^{32}$: \url{https://www.wolframalpha.com/input/?i=PowerMod%5B3,-1,2%5E32%5D}.

Let's check using \href{https://github.com/DennisYurichev/progcalc}{my calculator}:

\begin{lstlisting}
[3] 123456*0xaaaaaaab
[3] (unsigned) 353492988371136 0x141800000a0c0 0b1010000011000000000000000000000001010000011000000
[4] 123456/3
[4] (unsigned) 41152 0xa0c0 0b1010000011000000
\end{lstlisting}

The problem is simple enough to be solved using MK85:

\lstinputlisting[style=customsmt]{equations/modinv/modinv.smt}

\lstinputlisting{equations/modinv/modinv.correct}

However, it wouldn't work for 10, because there are no modulo inverse of 10 modulo $2^{32}$, SMT solver would give "unsat".


\section{Yet another explanation of modulo inverse using SMT-solvers}

\MathForProg has a part about modulo arithmetics and modulo inverse.

By which constant we must multiply a random number, so that the result would be as if we divided them by 3?

\begin{lstlisting}
from z3 import *

m=BitVec('m', 32)

s=Solver()

# wouldn't work for 10, etc
divisor=3

# random constant, must be divisible by divisor:
const=(0x1234567*divisor)

s.add(const*m == const/divisor)

print s.check()
print "%x" % s.model()[m].as_long()
\end{lstlisting}

The magic number is:

\begin{lstlisting}
sat
aaaaaaab
\end{lstlisting}

Indeed, this is modulo inverse of 3 modulo $2^{32}$: \url{https://www.wolframalpha.com/input/?i=PowerMod%5B3,-1,2%5E32%5D}.

Let's check using \href{https://github.com/DennisYurichev/progcalc}{my calculator}:

\begin{lstlisting}
[3] 123456*0xaaaaaaab
[3] (unsigned) 353492988371136 0x141800000a0c0 0b1010000011000000000000000000000001010000011000000
[4] 123456/3
[4] (unsigned) 41152 0xa0c0 0b1010000011000000
\end{lstlisting}

The problem is simple enough to be solved using MK85:

\lstinputlisting[style=customsmt]{equations/modinv/modinv.smt}

\lstinputlisting{equations/modinv/modinv.correct}

However, it wouldn't work for 10, because there are no modulo inverse of 10 modulo $2^{32}$, SMT solver would give "unsat".


\section{Yet another explanation of modulo inverse using SMT-solvers}

\MathForProg has a part about modulo arithmetics and modulo inverse.

By which constant we must multiply a random number, so that the result would be as if we divided them by 3?

\begin{lstlisting}
from z3 import *

m=BitVec('m', 32)

s=Solver()

# wouldn't work for 10, etc
divisor=3

# random constant, must be divisible by divisor:
const=(0x1234567*divisor)

s.add(const*m == const/divisor)

print s.check()
print "%x" % s.model()[m].as_long()
\end{lstlisting}

The magic number is:

\begin{lstlisting}
sat
aaaaaaab
\end{lstlisting}

Indeed, this is modulo inverse of 3 modulo $2^{32}$: \url{https://www.wolframalpha.com/input/?i=PowerMod%5B3,-1,2%5E32%5D}.

Let's check using \href{https://github.com/DennisYurichev/progcalc}{my calculator}:

\begin{lstlisting}
[3] 123456*0xaaaaaaab
[3] (unsigned) 353492988371136 0x141800000a0c0 0b1010000011000000000000000000000001010000011000000
[4] 123456/3
[4] (unsigned) 41152 0xa0c0 0b1010000011000000
\end{lstlisting}

The problem is simple enough to be solved using MK85:

\lstinputlisting[style=customsmt]{equations/modinv/modinv.smt}

\lstinputlisting{equations/modinv/modinv.correct}

However, it wouldn't work for 10, because there are no modulo inverse of 10 modulo $2^{32}$, SMT solver would give "unsat".


\subsection{Integer factorization using Z3 SMT solver}
\label{factor_Z3}

Integer factorization is method of breaking a composite (non-prime number) into prime factors.
Like 12345 = 3*4*823.

Though for small numbers, this task can be accomplished by Z3:

\lstinputlisting{SMT/factor/factor_z3.py}

% TODO FIX URL
( The source code: \url{https://github.com/dennis714/yurichev.com/blob/master/blog/factor/factor_z3.py} )

When factoring 1234567890 recursively:

\begin{lstlisting}
% time python z.py
factoring 1234567890
factors of 1234567890 are 342270 and 3607
factoring 342270
factors of 342270 are 2 and 171135
factoring 2
2 is prime (unsat)
factoring 171135
factors of 171135 are 3803 and 45
factoring 3803
3803 is prime (unsat)
factoring 45
factors of 45 are 3 and 15
factoring 3
3 is prime (unsat)
factoring 15
factors of 15 are 5 and 3
factoring 5
5 is prime (unsat)
factoring 3
3 is prime (unsat)
factoring 3607
3607 is prime (unsat)
[2, 3, 3, 5, 3607, 3803]
python z.py  19.30s user 0.02s system 99% cpu 19.443 total
\end{lstlisting}

So, 1234567890 = 2*3*3*5*3607*3803.

One important note: there is no primality test, no lookup tables, etc.
Prime number is a number for which "x*y=prime" (where x>1 and y>1) diophantine equation (which allows only integers in solution) has no solutions.
It can be solved for real numbers, though.

Z3 is \href{https://github.com/Z3Prover/z3/issues/1264}{not yet good enough for non-linear integer arithmetic}
and sometimes returns "unknown" instead of "unsat", but,
as Leonardo de Moura (one of Z3's author) commented about this:

\begin{lstlisting}
...Z3 will solve the problem as a real problem. If no real solution is found, we know there is no integer solution.
If a solution is found, Z3 will check if the solution is really assigning integer values to integer variables.
If that is not the case, it will return unknown to indicate it failed to solve the problem.
\end{lstlisting}
( \url{https://stackoverflow.com/questions/13898175/how-does-z3-handle-non-linear-integer-arithmetic} )

Probably, this is the case: we getting "unknown" in the case when a number cannot be factored, i.e., it's prime.

It's also very slow. Wolfram Mathematica can factor number around $2^{80}$ in a matter of seconds.
Still, I've written this for demonstration.

The problem of breaking \ac{RSA} is a problem of factorization of very large numbers, up to $2^{4096}$.
It's currently not possible to do this in practice.

See also: integer factorization using SAT solver (\ref{factor_SAT}).


\section{Yet another explanation of modulo inverse using SMT-solvers}

\MathForProg has a part about modulo arithmetics and modulo inverse.

By which constant we must multiply a random number, so that the result would be as if we divided them by 3?

\begin{lstlisting}
from z3 import *

m=BitVec('m', 32)

s=Solver()

# wouldn't work for 10, etc
divisor=3

# random constant, must be divisible by divisor:
const=(0x1234567*divisor)

s.add(const*m == const/divisor)

print s.check()
print "%x" % s.model()[m].as_long()
\end{lstlisting}

The magic number is:

\begin{lstlisting}
sat
aaaaaaab
\end{lstlisting}

Indeed, this is modulo inverse of 3 modulo $2^{32}$: \url{https://www.wolframalpha.com/input/?i=PowerMod%5B3,-1,2%5E32%5D}.

Let's check using \href{https://github.com/DennisYurichev/progcalc}{my calculator}:

\begin{lstlisting}
[3] 123456*0xaaaaaaab
[3] (unsigned) 353492988371136 0x141800000a0c0 0b1010000011000000000000000000000001010000011000000
[4] 123456/3
[4] (unsigned) 41152 0xa0c0 0b1010000011000000
\end{lstlisting}

The problem is simple enough to be solved using MK85:

\lstinputlisting[style=customsmt]{equations/modinv/modinv.smt}

\lstinputlisting{equations/modinv/modinv.correct}

However, it wouldn't work for 10, because there are no modulo inverse of 10 modulo $2^{32}$, SMT solver would give "unsat".


\subsection{Solving pocket Rubik’s cube (2*2*2) using Z3}
\label{PocketCubeSMT}

\begin{figure}[H]
\centering
\includegraphics[scale=0.75]{SMT/rubik2/failed/190px-Pocket_cube_scrambled.jpg}
\caption{Pocket cube}
\end{figure}

( The image has been taken \href{https://en.wikipedia.org/wiki/Pocket_Cube}{from Wikipedia}. )

Solving Rubik's cube is not a problem, finding shortest solution is.

\subsubsection{Intro}

First, a bit of terminology.
There are 6 colors we have: white, green, blue, orange, red, yellow.
We also have 6 sides: front, up, down, left, right, back.

This is how we will name all facelets:

% TODO TikZ
\begin{lstlisting}
        U1 U2
        U3 U4

       -------
L1 L2 | F1 F2 | R1 R2 | B1 B2
L3 L4 | F3 F4 | R3 R4 | B3 B4
       -------

        D1 D2
        D3 D4
\end{lstlisting}

Colors on a solved cube are:

\begin{lstlisting}
    G G
    G G
    ---
R R|W W|O O|Y Y
R R|W W|O O|Y Y
    ---
    B B
    B B
\end{lstlisting}

There are 6 possible turns: front, left, right, back, up, down.
But each turn can be clockwise, counterclockwise and half-turn (equal to two CW or two CCW).
Each CW is equal to 3 CCW and vice versa.
Hence, there are 6*3=18 possible turns.

It is known, that 11 turns (including half-turns) are enough to solve any pocket cube
(\href{https://en.wikipedia.org/wiki/Optimal_solutions_for_Rubik%27s_Cube}{God’s algorithm}).
This means, \href{http://mathworld.wolfram.com/GraphDiameter.html}{graph has a diameter} of 11.
For 3*3*3 cube one need 20 turns (\url{http://www.cube20.org/}).
See also: \url{https://en.wikipedia.org/wiki/Rubik%27s_Cube_group}.

\subsubsection{Z3}

There are 6 sides and 4 facelets on each, hence, 6*4=24 variables we need to define a state.

Then we define how state is transformed after each possible turn:

\begin{lstlisting}
FACE_F, FACE_U, FACE_D, FACE_R, FACE_L, FACE_B = 0,1,2,3,4,5

def rotate_FCW(s):
    return [
        [ s[FACE_F][2], s[FACE_F][0], s[FACE_F][3], s[FACE_F][1] ],   # for F
        [ s[FACE_U][0], s[FACE_U][1], s[FACE_L][3], s[FACE_L][1] ],   # for U
        [ s[FACE_R][2], s[FACE_R][0], s[FACE_D][2], s[FACE_D][3] ],   # for D
        [ s[FACE_U][2], s[FACE_R][1], s[FACE_U][3], s[FACE_R][3] ],   # for R
        [ s[FACE_L][0], s[FACE_D][0], s[FACE_L][2], s[FACE_D][1] ],   # for L
        [ s[FACE_B][0], s[FACE_B][1], s[FACE_B][2], s[FACE_B][3] ] ]  # for B

def rotate_FH(s):
    return [
        [ s[FACE_F][3], s[FACE_F][2], s[FACE_F][1], s[FACE_F][0] ],
        [ s[FACE_U][0], s[FACE_U][1], s[FACE_D][1], s[FACE_D][0] ],
        [ s[FACE_U][3], s[FACE_U][2], s[FACE_D][2], s[FACE_D][3] ],
        [ s[FACE_L][3], s[FACE_R][1], s[FACE_L][1], s[FACE_R][3] ],
        [ s[FACE_L][0], s[FACE_R][2], s[FACE_L][2], s[FACE_R][0] ],
        [ s[FACE_B][0], s[FACE_B][1], s[FACE_B][2], s[FACE_B][3] ] ]

...
\end{lstlisting}

Then we define a function, which takes turn number and transforms a state:

\begin{lstlisting}
# op is turn number
def rotate(turn, state, face, facelet):
    return If(op==0,  rotate_FCW (state)[face][facelet],
           If(op==1,  rotate_FCCW(state)[face][facelet],
           If(op==2,  rotate_UCW (state)[face][facelet],
           If(op==3,  rotate_UCCW(state)[face][facelet],
           If(op==4,  rotate_DCW (state)[face][facelet],

...

           If(op==17, rotate_BH  (state)[face][facelet],
                      0))))))))))))))))))
\end{lstlisting}

Now set "solved" state, initial state and connect everything:

\begin{lstlisting}
move_names=["FCW", "FCCW", "UCW", "UCCW", "DCW", "DCCW", "RCW", "RCCW", "LCW", "LCCW", "BCW", "BCCW", "FH", "UH", "DH", "RH", "LH", "BH"]

def colors_to_array_of_ints(s):
    return [{"W":0, "G":1, "B":2, "O":3, "R":4, "Y":5}[c] for c in s]

def set_current_state (d):
    F=colors_to_array_of_ints(d["FACE_F"])
    U=colors_to_array_of_ints(d["FACE_U"])
    D=colors_to_array_of_ints(d["FACE_D"])
    R=colors_to_array_of_ints(d["FACE_R"])
    L=colors_to_array_of_ints(d["FACE_L"])
    B=colors_to_array_of_ints(d["FACE_B"])
    return F,U,D,R,L,B # return tuple

# 4
init_F, init_U, init_D, init_R, init_L, init_B=set_current_state({"FACE_F":"RYOG", "FACE_U":"YRGO", "FACE_D":"WRBO", "FACE_R":"GYWB", "FACE_L":"BYWG", "FACE_B":"BOWR"})

...

for TURNS in range(1,12): # 1..11
    print "turns=", TURNS

    s=Solver()

    state=[[[Int('state%d_%d_%d' % (n, side, i)) for i in range(FACELETS)] for side in range(FACES)] for n in range(TURNS+1)]
    op=[Int('op%d' % n) for n in range(TURNS+1)]

    for i in range(FACELETS):
        s.add(state[0][FACE_F][i]==init_F[i])
        s.add(state[0][FACE_U][i]==init_U[i])
        s.add(state[0][FACE_D][i]==init_D[i])
        s.add(state[0][FACE_R][i]==init_R[i])
        s.add(state[0][FACE_L][i]==init_L[i])
        s.add(state[0][FACE_B][i]==init_B[i])

    # solved state
    for face in range(FACES):
        for facelet in range(FACELETS):
            s.add(state[TURNS][face][facelet]==face)

    # turns:
    for turn in range(TURNS):
        for face in range(FACES):
            for facelet in range(FACELETS):
                s.add(state[turn+1][face][facelet]==rotate(op[turn], state[turn], face, facelet))

    if s.check()==sat:
        print "sat"
        m=s.model()
        for turn in range(TURNS):
            print move_names[int(str(m[op[turn]]))]
        exit(0)
\end{lstlisting}

% FIXME URL
( The full source code: \url{https://github.com/DennisYurichev/yurichev.com/blob/master/blog/rubik/rubik2_z3.py} )

That works:

\begin{lstlisting}
turns= 1
turns= 2
turns= 3
turns= 4
sat
RCW
UCW
DCW
RCW
\end{lstlisting}

...but very slow. It takes up to 1 hours to find a path of 8 turns, which is not enough, we need 11.

Nevetheless, I decided to include Z3 solver as a demonstration.

See also: solving pocket cube using SAT solver: \ref{PocketCubeSAT}.


\section{Yet another explanation of modulo inverse using SMT-solvers}

\MathForProg has a part about modulo arithmetics and modulo inverse.

By which constant we must multiply a random number, so that the result would be as if we divided them by 3?

\begin{lstlisting}
from z3 import *

m=BitVec('m', 32)

s=Solver()

# wouldn't work for 10, etc
divisor=3

# random constant, must be divisible by divisor:
const=(0x1234567*divisor)

s.add(const*m == const/divisor)

print s.check()
print "%x" % s.model()[m].as_long()
\end{lstlisting}

The magic number is:

\begin{lstlisting}
sat
aaaaaaab
\end{lstlisting}

Indeed, this is modulo inverse of 3 modulo $2^{32}$: \url{https://www.wolframalpha.com/input/?i=PowerMod%5B3,-1,2%5E32%5D}.

Let's check using \href{https://github.com/DennisYurichev/progcalc}{my calculator}:

\begin{lstlisting}
[3] 123456*0xaaaaaaab
[3] (unsigned) 353492988371136 0x141800000a0c0 0b1010000011000000000000000000000001010000011000000
[4] 123456/3
[4] (unsigned) 41152 0xa0c0 0b1010000011000000
\end{lstlisting}

The problem is simple enough to be solved using MK85:

\lstinputlisting[style=customsmt]{equations/modinv/modinv.smt}

\lstinputlisting{equations/modinv/modinv.correct}

However, it wouldn't work for 10, because there are no modulo inverse of 10 modulo $2^{32}$, SMT solver would give "unsat".


\input{SMT/cribbage_EN}
\input{SMT/menage_EN}
\subsection{Proving sorting network correctness}

Sorting networks are highly popular in electronics, GPGPU and even in SAT encodings:
\url{https://en.wikipedia.org/wiki/Sorting_network}.

Especially bitonic sorters, which are also sorting networks:
\url{https://en.wikipedia.org/wiki/Bitonic_sorter}.

Its popularity is probably related to the fact they can be parallelized easily.

They are relatively easy to construct, but, finding a smallest possible is a challenge.

There is a smallest network (only 25 comparators) for 9-channel sorting network:

\begin{figure}[H]
\label{fig:pipe_shuffled}
\centering
\includegraphics[scale=0.75]{SMT/sorting_network/network9.png}
\caption{Smallest}
\end{figure}

This is combinational circuit, each connection is a comparator+swapper, it swaps if one of input values is bigger and passes output to the next level.

I copypasted it from \href{https://arxiv.org/pdf/1405.5754.pdf}{the article}:
Michael Codish, Lu ́ıs Cruz-Filipe, Michael Frank, and Peter Schneider-Kamp --
``Twenty-Five Comparators is Optimal when Sorting Nine Inputs (and Twenty-Nine for Ten)''.

Another article about it: \href{http://larc.unt.edu/ian/pubs/9-input.pdf}{Ian Parberry -- A Computer Assisted Optimal Depth Lower Bound for Nine-Input Sorting Networks}.

I don't know (yet) how they proved it, but it's interesting, that it's extremely easy to prove its correctness using Z3 SMT solver.
We just construct network out of comparators/swappers and asking Z3 to find counterexample, for which the output of the network will not be sorted.
And it can't, meaning, output's state is always sorted, no matter what values are plugged into inputs.

\lstinputlisting{SMT/sorting_network/test9.py}

( The full source code: \url{URL/test9.py}. )

There is also smaller 4-channel network I copypasted from Wikipedia:

\begin{lstlisting}
...

l=line(l, " + +")
l=line(l, "+ + ")
l=line(l, "++++")
l=line(l, " ++ ")

...
\end{lstlisting}

( The full source code: \url{URL/test4.py}. )

It also proved to be correct, but it's interesting, what Z3Py expression we've got at each of 4 outputs:

\begin{lstlisting}
If(If(a < c, a, c) < If(b < d, b, d),
   If(a < c, a, c),
   If(b < d, b, d))

If(If(If(a < c, a, c) > If(b < d, b, d),
      If(a < c, a, c),
      If(b < d, b, d)) <
   If(If(a > c, a, c) < If(b > d, b, d),
      If(a > c, a, c),
      If(b > d, b, d)),
   If(If(a < c, a, c) > If(b < d, b, d),
      If(a < c, a, c),
      If(b < d, b, d)),
   If(If(a > c, a, c) < If(b > d, b, d),
      If(a > c, a, c),
      If(b > d, b, d)))

If(If(If(a < c, a, c) > If(b < d, b, d),
      If(a < c, a, c),
      If(b < d, b, d)) >
   If(If(a > c, a, c) < If(b > d, b, d),
      If(a > c, a, c),
      If(b > d, b, d)),
   If(If(a < c, a, c) > If(b < d, b, d),
      If(a < c, a, c),
      If(b < d, b, d)),
   If(If(a > c, a, c) < If(b > d, b, d),
      If(a > c, a, c),
      If(b > d, b, d)))

If(If(a > c, a, c) > If(b > d, b, d),
   If(a > c, a, c),
   If(b > d, b, d))
\end{lstlisting}

The first and the last are shorter than the 2nd and the 3rd, they are just
$min(min(min(a,b),c),d)$ and 
$max(max(max(a,b),c),d)$.


\section{Yet another explanation of modulo inverse using SMT-solvers}

\MathForProg has a part about modulo arithmetics and modulo inverse.

By which constant we must multiply a random number, so that the result would be as if we divided them by 3?

\begin{lstlisting}
from z3 import *

m=BitVec('m', 32)

s=Solver()

# wouldn't work for 10, etc
divisor=3

# random constant, must be divisible by divisor:
const=(0x1234567*divisor)

s.add(const*m == const/divisor)

print s.check()
print "%x" % s.model()[m].as_long()
\end{lstlisting}

The magic number is:

\begin{lstlisting}
sat
aaaaaaab
\end{lstlisting}

Indeed, this is modulo inverse of 3 modulo $2^{32}$: \url{https://www.wolframalpha.com/input/?i=PowerMod%5B3,-1,2%5E32%5D}.

Let's check using \href{https://github.com/DennisYurichev/progcalc}{my calculator}:

\begin{lstlisting}
[3] 123456*0xaaaaaaab
[3] (unsigned) 353492988371136 0x141800000a0c0 0b1010000011000000000000000000000001010000011000000
[4] 123456/3
[4] (unsigned) 41152 0xa0c0 0b1010000011000000
\end{lstlisting}

The problem is simple enough to be solved using MK85:

\lstinputlisting[style=customsmt]{equations/modinv/modinv.smt}

\lstinputlisting{equations/modinv/modinv.correct}

However, it wouldn't work for 10, because there are no modulo inverse of 10 modulo $2^{32}$, SMT solver would give "unsat".


\subsection{Enumerating all possible inputs for a specific regular expression}

Regular expression if first converted to \ac{FSM} before matching.
Hence, many \ac{RE} libraries has two functions: ``compile'' and ``execute''
(when you match many strings against single RE, no need to recompile it to \ac{FSM} each time).

And I've found this website, which can visualize FSM (finite state machine) for a regular expression.
\url{http://hokein.github.io/Automata.js/}.
This is fun!

This \ac{FSM} (\ac{DFA}) is for the expression \TT{(dark|light)?(red|blue|green)(ish)?}

\begin{figure}[H]
\centering
\includegraphics[scale=0.6]{SMT/regexp/1.png}
\caption{}
\end{figure}

% FSM.png
Another version: URL.

Accepting states are in double circles, these are the states where matching process stops.

How can we generate an input string which regular expression would match?
In other words, which inputs \ac{FSM} would accept?
This task is surprisingly simple for SMT-solver.

We just define a transition function.
For each pair (state, input) it defines new state.

\ac{FSM} has been visualized by the website mentioned above, and I used this information to write ``transition()'' function.

Then we chain transition functions... then we try a chain for all lengths in range of 2..14.

\lstinputlisting{SMT/regexp/re.py}

Results:

\lstinputlisting{SMT/regexp/res.txt}

As simple as this.

% TODO \gls
It can be said, what we did is enumeration of all paths between two vertices of a digraph (representing \ac{FSM}).

Also, the ``transition()'' function itself can act as a RE matcher, with no relevance to SMT solver(s).
Just feed input characters to it and track state.
Whenever you hit one of accepting states, return ``match'', whenever you hit \TT{INVALID\_STATE}, return ``no match''.


\section{Yet another explanation of modulo inverse using SMT-solvers}

\MathForProg has a part about modulo arithmetics and modulo inverse.

By which constant we must multiply a random number, so that the result would be as if we divided them by 3?

\begin{lstlisting}
from z3 import *

m=BitVec('m', 32)

s=Solver()

# wouldn't work for 10, etc
divisor=3

# random constant, must be divisible by divisor:
const=(0x1234567*divisor)

s.add(const*m == const/divisor)

print s.check()
print "%x" % s.model()[m].as_long()
\end{lstlisting}

The magic number is:

\begin{lstlisting}
sat
aaaaaaab
\end{lstlisting}

Indeed, this is modulo inverse of 3 modulo $2^{32}$: \url{https://www.wolframalpha.com/input/?i=PowerMod%5B3,-1,2%5E32%5D}.

Let's check using \href{https://github.com/DennisYurichev/progcalc}{my calculator}:

\begin{lstlisting}
[3] 123456*0xaaaaaaab
[3] (unsigned) 353492988371136 0x141800000a0c0 0b1010000011000000000000000000000001010000011000000
[4] 123456/3
[4] (unsigned) 41152 0xa0c0 0b1010000011000000
\end{lstlisting}

The problem is simple enough to be solved using MK85:

\lstinputlisting[style=customsmt]{equations/modinv/modinv.smt}

\lstinputlisting{equations/modinv/modinv.correct}

However, it wouldn't work for 10, because there are no modulo inverse of 10 modulo $2^{32}$, SMT solver would give "unsat".


\section{Yet another explanation of modulo inverse using SMT-solvers}

\MathForProg has a part about modulo arithmetics and modulo inverse.

By which constant we must multiply a random number, so that the result would be as if we divided them by 3?

\begin{lstlisting}
from z3 import *

m=BitVec('m', 32)

s=Solver()

# wouldn't work for 10, etc
divisor=3

# random constant, must be divisible by divisor:
const=(0x1234567*divisor)

s.add(const*m == const/divisor)

print s.check()
print "%x" % s.model()[m].as_long()
\end{lstlisting}

The magic number is:

\begin{lstlisting}
sat
aaaaaaab
\end{lstlisting}

Indeed, this is modulo inverse of 3 modulo $2^{32}$: \url{https://www.wolframalpha.com/input/?i=PowerMod%5B3,-1,2%5E32%5D}.

Let's check using \href{https://github.com/DennisYurichev/progcalc}{my calculator}:

\begin{lstlisting}
[3] 123456*0xaaaaaaab
[3] (unsigned) 353492988371136 0x141800000a0c0 0b1010000011000000000000000000000001010000011000000
[4] 123456/3
[4] (unsigned) 41152 0xa0c0 0b1010000011000000
\end{lstlisting}

The problem is simple enough to be solved using MK85:

\lstinputlisting[style=customsmt]{equations/modinv/modinv.smt}

\lstinputlisting{equations/modinv/modinv.correct}

However, it wouldn't work for 10, because there are no modulo inverse of 10 modulo $2^{32}$, SMT solver would give "unsat".


\section{Yet another explanation of modulo inverse using SMT-solvers}

\MathForProg has a part about modulo arithmetics and modulo inverse.

By which constant we must multiply a random number, so that the result would be as if we divided them by 3?

\begin{lstlisting}
from z3 import *

m=BitVec('m', 32)

s=Solver()

# wouldn't work for 10, etc
divisor=3

# random constant, must be divisible by divisor:
const=(0x1234567*divisor)

s.add(const*m == const/divisor)

print s.check()
print "%x" % s.model()[m].as_long()
\end{lstlisting}

The magic number is:

\begin{lstlisting}
sat
aaaaaaab
\end{lstlisting}

Indeed, this is modulo inverse of 3 modulo $2^{32}$: \url{https://www.wolframalpha.com/input/?i=PowerMod%5B3,-1,2%5E32%5D}.

Let's check using \href{https://github.com/DennisYurichev/progcalc}{my calculator}:

\begin{lstlisting}
[3] 123456*0xaaaaaaab
[3] (unsigned) 353492988371136 0x141800000a0c0 0b1010000011000000000000000000000001010000011000000
[4] 123456/3
[4] (unsigned) 41152 0xa0c0 0b1010000011000000
\end{lstlisting}

The problem is simple enough to be solved using MK85:

\lstinputlisting[style=customsmt]{equations/modinv/modinv.smt}

\lstinputlisting{equations/modinv/modinv.correct}

However, it wouldn't work for 10, because there are no modulo inverse of 10 modulo $2^{32}$, SMT solver would give "unsat".


\subsection{Exercise 15 from TAOCP ``7.1.3 Bitwise tricks and techniques''}

Page 53 from the fasc1a.ps, or: \url{http://www.cs.utsa.edu/~wagner/knuth/fasc1a.pdf}

\begin{figure}[H]
\label{fig:pipe_shuffled}
\centering
\frame{\includegraphics[scale=0.6]{SMT/TAOCP_7_1_3_exercise_15/page53.png}}
\caption{Page 53}
\end{figure}

Soltuion:

\begin{lstlisting}
from z3 import *

s=Solver()

a, b=BitVecs('a b', 4)
x, y=BitVecs('x y', 4)

s.add(ForAll(x, ForAll(y,  ((x+a)^b)-a == ((x-a)^b)+a  )))

# enumerate all possible solutions:
results=[]
while True:
    if s.check() == sat:
        m = s.model()
        print m

        results.append(m)
        block = []
        for d in m:
            c=d()
            block.append(c != m[d])
        s.add(Or(block))
    else:
        print "results total=", len(results)
        break
\end{lstlisting}

For 4-bit bitvectors:

\begin{lstlisting}

...

[b = 7, a = 0]
[b = 6, a = 8]
[b = 7, a = 8]
[b = 6, a = 12]
[b = 7, a = 12]
[b = 12, a = 0]
[b = 13, a = 0]
[b = 12, a = 8]
[b = 13, a = 8]
[b = 12, a = 4]
[b = 13, a = 4]
[b = 12, a = 12]
[b = 13, a = 12]
[b = 14, a = 0]
[b = 15, a = 0]
[b = 14, a = 4]
[b = 15, a = 4]
[b = 14, a = 8]
[b = 15, a = 8]
[b = 14, a = 12]
[b = 15, a = 12]
results total= 128
\end{lstlisting}


\section{Yet another explanation of modulo inverse using SMT-solvers}

\MathForProg has a part about modulo arithmetics and modulo inverse.

By which constant we must multiply a random number, so that the result would be as if we divided them by 3?

\begin{lstlisting}
from z3 import *

m=BitVec('m', 32)

s=Solver()

# wouldn't work for 10, etc
divisor=3

# random constant, must be divisible by divisor:
const=(0x1234567*divisor)

s.add(const*m == const/divisor)

print s.check()
print "%x" % s.model()[m].as_long()
\end{lstlisting}

The magic number is:

\begin{lstlisting}
sat
aaaaaaab
\end{lstlisting}

Indeed, this is modulo inverse of 3 modulo $2^{32}$: \url{https://www.wolframalpha.com/input/?i=PowerMod%5B3,-1,2%5E32%5D}.

Let's check using \href{https://github.com/DennisYurichev/progcalc}{my calculator}:

\begin{lstlisting}
[3] 123456*0xaaaaaaab
[3] (unsigned) 353492988371136 0x141800000a0c0 0b1010000011000000000000000000000001010000011000000
[4] 123456/3
[4] (unsigned) 41152 0xa0c0 0b1010000011000000
\end{lstlisting}

The problem is simple enough to be solved using MK85:

\lstinputlisting[style=customsmt]{equations/modinv/modinv.smt}

\lstinputlisting{equations/modinv/modinv.correct}

However, it wouldn't work for 10, because there are no modulo inverse of 10 modulo $2^{32}$, SMT solver would give "unsat".


\subsection{Recreational math, calculator's keypad and divisibility}

I've once read about this puzzle.
Imagine calculator's keypad:

\begin{lstlisting}
789
456
123
\end{lstlisting}

If you form any rectangle or square out of keys, like:

\begin{lstlisting}
 7 8 9
+---+
|4 5|6
|1 2|3
+---+
\end{lstlisting}

The number is 4521. Or 2145, or 5214.
All these numbers are divisible by 11, 111 and 111.
One explanation: \url{https://files.eric.ed.gov/fulltext/EJ891796.pdf}.

However, I could try to prove that all these numbers are indeed divisible.

\begin{lstlisting}[style=custompy]

from z3 import *

"""
We will keep track on numbers using row/col representation:

 |0 1 2 <-col
-|- - -
0|7 8 9
1|4 5 6
2|1 2 3
^
|
row

"""

# map coordinates to number on keypad:
def coords_to_num (r, c):
    return If(And(r==0, c==0), 7,
    If(And(r==0, c==1), 8,
    If(And(r==0, c==2), 9,
    If(And(r==1, c==0), 4,
    If(And(r==1, c==1), 5,
    If(And(r==1, c==2), 6,
    If(And(r==2, c==0), 1,
    If(And(r==2, c==1), 2,
    If(And(r==2, c==2), 3, 9999)))))))))

s=Solver()

# coordinates of upper left corner:
from_r, from_c = Ints('from_r from_c')
# coordinates of bottom right corner:
to_r, to_c = Ints('to_r to_c')

# all coordinates are in [0..2]:
s.add(And(from_r>=0, from_r<=2, from_c>=0, from_c<=2))
s.add(And(to_r>=0, to_r<=2, to_c>=0, to_c<=2))

# bottom-right corner is always under left-upper corner, or equal to it, or to the right of it:
s.add(to_r>=from_r)
s.add(to_c>=from_c)

# numbers on keypads for all 4 corners:
LT, RT, BL, BR = Ints('LT RT BL BR')

# ... which are:
s.add(LT==coords_to_num(from_r, from_c))
s.add(RT==coords_to_num(from_r, to_c))
s.add(BL==coords_to_num(to_r, from_c))
s.add(BR==coords_to_num(to_r, to_c))

# 4 possible 4-digit numbers formed by passing by 4 corners:
n1, n2, n3, n4 = Ints('n1 n2 n3 n4')

s.add(n1==LT*1000 + RT*100 + BR*10 + BL)
s.add(n2==RT*1000 + BR*100 + BL*10 + LT)
s.add(n3==BR*1000 + BL*100 + LT*10 + RT)
s.add(n4==BL*1000 + LT*100 + RT*10 + BR)

# what we're going to do?
prove=False
enumerate_rectangles=True

assert prove != enumerate_rectangles

if prove:
    # prove by finding counterexample.
    # find any variable state for which remainder will be non-zero:
    s.add(And((n1%11) != 0), (n1%111) != 0, (n1%1111) != 0)
    s.add(And((n2%11) != 0), (n2%111) != 0, (n2%1111) != 0)
    s.add(And((n3%11) != 0), (n3%111) != 0, (n3%1111) != 0)
    s.add(And((n4%11) != 0), (n4%111) != 0, (n4%1111) != 0)

    # this is impossible, we're getting unsat here, because no counterexample exist:
    print s.check()

# ... or ...

if enumerate_rectangles:
    # enumerate all possible solutions:
    results=[]
    while True:
        if s.check() == sat:
            m = s.model()
            #print_model(m)
            print m
            print m[n1]
            print m[n2]
            print m[n3]
            print m[n4]
            results.append(m)
            block = []
            for d in m:
                c=d()
                block.append(c != m[d])
            s.add(Or(block))
        else:
            print "results total=", len(results)
            break

\end{lstlisting}

Enumeration. only 36 rectangles exist on 3*3 keypad:

\begin{lstlisting}
[n1 = 7821,
 BL = 1,
 n2 = 8217,
 to_r = 2,
 LT = 7,
 RT = 8,
 BR = 2,
 n4 = 1782,
 from_r = 0,
 n3 = 2178,
 from_c = 0,
 to_c = 1]
7821
8217
2178
1782
[n1 = 7931,
 BL = 1,
 n2 = 9317,
 to_r = 2,
 LT = 7,
 RT = 9,
 BR = 3,
 n4 = 1793,
 from_r = 0,
 n3 = 3179,
 from_c = 0,
 to_c = 2]
7931
9317
3179
1793

...

[n1 = 5522,
 BL = 2,
 n2 = 5225,
 to_r = 2,
 LT = 5,
 RT = 5,
 BR = 2,
 n4 = 2552,
 from_r = 1,
 n3 = 2255,
 from_c = 1,
 to_c = 1]
5522
5225
2255
2552
results total= 36
\end{lstlisting}


\subsection{Knight's tour}

\lstinputlisting[style=custompy]{SMT/knight_tour/knight_tour_Z3.py}

Can find a closed knight's tour on 8*8 chess board for 150s on Intel Quad-Core Xeon E3-1220 3.10GHz:

\begin{lstlisting}
 0 57 44 41  2 39 12 29
43 46  1 58 11 30 23 38
56 63 42 45 40  3 28 13
47  8 59 10 31 24 37 22
60 55 62 51  4 27 14 25
 7 48  9 32 17 34 21 36
54 61 50  5 52 19 26 15
49  6 53 18 33 16 35 20
\end{lstlisting}

However, this is WAY slower than C implementation on Rosetta Code: \url{https://rosettacode.org/wiki/Knight%27s_tour#C}
... which uses Warnsdorf's rule: \url{https://en.wikipedia.org/wiki/Knight%27s_tour#Warnsdorff.27s_algorithm}.

Another program for Z3 for finding Hamiltonian cycle: \url{https://github.com/Z3Prover/z3/blob/master/examples/python/hamiltonian/hamiltonian.py}.
(Clever trick of using remainder.)


\subsection{Hilbert's 10th problem, Fermat’s last theorem and SMT solvers}

Hilbert's 10th problem states that you cannot devise an algorithm which can solve any diophantine equation over integers.
However, it's important to understand, that this is possible over fixed-size bitvectors.

Fermat's last theorem states that there are no integer solution(s) for $a^n + b^n = c^n$, for $n>=3$.

Let's prove it for n=3 and for a in 0..255 range:

\lstinputlisting[style=custompy]{SMT/Hilbert_10/fermat.py}

Z3 gives "unsat", meaning, it couldn't find any a/b/c.
However, this is possible to check even using brute-force search.

If to replace "BitVecs" by "Ints", Z3 would give "unknown":

\lstinputlisting[style=custompy]{SMT/Hilbert_10/fermat2.py}

In short: anything is decidable (you can build an algorithm which can solve equation or not) under fixed-size bitvectors.
Given enough computational power, you can solve such equations for big bit-vectors.
But this is not possible for integers or bit-vectors of any size.

Another interesting reading about this by Leonardo de Moura:
\url{https://stackoverflow.com/questions/13898175/how-does-z3-handle-non-linear-integer-arithmetic}.



\subsubsection{List of SAT-solvers}

% TODO authors, URLs

\begin{itemize}

\item MiniSat\footnote{\url{http://minisat.se/}}, serving as a base for some others

\item PicoSat, PrecoSat, Lingeling. Created by Armin Biere. Plingeling supports multithreading.

\item CryptoMiniSat. Created by Mate Soos for cryptographical problems exploration.
Supports XOR clauses, multithreading.
Has Python API.

\end{itemize}

MaxSAT solvers:

\begin{itemize}

\item Open-WBO\footnote{\url{http://sat.inesc-id.pt/open-wbo/}}, by Ruben Martins, Vasco Manquinho, Inês Lynce.

\end{itemize}




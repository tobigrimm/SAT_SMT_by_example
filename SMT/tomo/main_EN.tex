\subsection{Discrete tomography}

How computed tomography (CT scan) actually works?
A human body is bombarded by X-rays in various angles by X-ray tube in rotating torus.
X-ray detectors are also located in torus, and all the information is recorded.

Here is we can simulate simple tomograph.
An ``i'' character is rotating and will be ``enlighten'' at 4 angles.
Let's imagine, character is bombarded by X-ray tube at left.
All asterisks in each row is then summed and sum is "received" by X-ray detector at the right.

\begin{lstlisting}
WIDTH= 11 HEIGHT= 11
angle=(π/4)*0
    **      2
    **      2
            0
   ***      3
    **      2
    **      2
    **      2
    **      2
    **      2
   ****     4
            0
[2, 2, 0, 3, 2, 2, 2, 2, 2, 4, 0] ,
angle=(π/4)*1
            0
            0
  *         1
 **         2
    *       1
    **      2
     **     2
     ****   4
       *    1
      *     1
            0
[0, 0, 1, 2, 1, 2, 2, 4, 1, 1, 0] ,
angle=(π/4)*2
            0
            0
            0
            0
         *  1
** *******  9
** *******  9
   *     *  2
            0
            0
            0
[0, 0, 0, 0, 1, 9, 9, 2, 0, 0, 0] ,
angle=(π/4)*3
            0
            0
       *    1
       **   2
      ** *  3
     ***    3
    **      2
            0
  **        2
   *        1
            0
[0, 0, 1, 2, 3, 3, 2, 0, 2, 1, 0] ,
\end{lstlisting}

( The source code: \url{https://github.com/dennis714/SAT_SMT_article/blob/master/SMT/tomo/gen.py} )

All we got from our toy-level tomograph is 4 vectors, these are sums of all asterisks in rows for 4 angles:

\begin{lstlisting}
[2, 2, 0, 3, 2, 2, 2, 2, 2, 4, 0] ,
[0, 0, 1, 2, 1, 2, 2, 4, 1, 1, 0] ,
[0, 0, 0, 0, 1, 9, 9, 2, 0, 0, 0] ,
[0, 0, 1, 2, 3, 3, 2, 0, 2, 1, 0] ,
\end{lstlisting}

How do we recover initial image?
We are going to represent 11*11 matrix, where sum of each row must be equal to some value we already know.
Then we rotate matrix, and do this again.

The ``rotate'' function has been taken from the generation program, because, due to Python's dynamic typization nature,
it's not important for the function to what operate on:
strings, characters, or Z3 variable instances, so it works very well for all of them.

\begin{lstlisting}
#-*- coding: utf-8 -*-

import math, sys
from z3 import *

# https://en.wikipedia.org/wiki/Rotation_matrix
def rotate(pic, angle):
    WIDTH=len(pic[0])
    HEIGHT=len(pic)
    #print WIDTH, HEIGHT
    assert WIDTH==HEIGHT
    ofs=WIDTH/2

    out = [[0 for x in range(WIDTH)] for y in range(HEIGHT)]

    for x in range(-ofs,ofs):
        for y in range(-ofs,ofs):
            newX = int(round(math.cos(angle)*x - math.sin(angle)*y,3))+ofs
            newY = int(round(math.sin(angle)*x + math.cos(angle)*y,3))+ofs
            # clip at boundaries, hence min(..., HEIGHT-1)
            out[min(newX,HEIGHT-1)][min(newY,WIDTH-1)]=pic[x+ofs][y+ofs]
    return out

vectors=[
[2, 2, 0, 3, 2, 2, 2, 2, 2, 4, 0] ,
[0, 0, 1, 2, 1, 2, 2, 4, 1, 1, 0] ,
[0, 0, 0, 0, 1, 9, 9, 2, 0, 0, 0] ,
[0, 0, 1, 2, 3, 3, 2, 0, 2, 1, 0]]

WIDTH = HEIGHT = len(vectors[0])

s=Solver()
cells=[[Int('cell_r=%d_c=%d' % (r,c)) for c in range(WIDTH)] for r in range(HEIGHT)]

# monochrome picture, only 0's or 1's:
for c in range(WIDTH):
    for r in range(HEIGHT):
        s.add(Or(cells[r][c]==0, cells[r][c]==1))

def all_zeroes_in_vector(vec):
    for v in vec:
        if v!=0:
            return False
    return True

ANGLES=len(vectors)
for a in range(ANGLES):
    angle=a*(math.pi/ANGLES)
    rows=rotate(cells, angle)
    r=0
    for row in rows:
        # skip empty rows:
        if all_zeroes_in_vector(row)==False:
            # sum of row must be equal to the corresponding element of vector:
            s.add(Sum(*row)==vectors[a][r])
        r=r+1

print s.check()
m=s.model()
for r in range(HEIGHT):
    for c in range(WIDTH):
        if str(m[cells[r][c]])=="1":
            sys.stdout.write("*")
        else:
            sys.stdout.write(" ")
    print ""
\end{lstlisting}

( The source code: \url{https://github.com/dennis714/SAT_SMT_article/blob/master/SMT/tomo/solve.py} )

That works:

\begin{lstlisting}
% python solve.py
sat
    **
    **

   ***
    **
    **
    **
    **
    **
   ****
\end{lstlisting}

In other words, all SMT-solver does here is solving a system of equations.

So, 4 angles are enough.
What if we could use only 3 angles?

\begin{lstlisting}
WIDTH= 11 HEIGHT= 11
angle=(π/3)*0
    **      2
    **      2
            0
   ***      3
    **      2
    **      2
    **      2
    **      2
    **      2
   ****     4
            0
[2, 2, 0, 3, 2, 2, 2, 2, 2, 4, 0] ,
angle=(π/3)*1
            0
            0
            0
 **         2
 **         2
   ***      3
     ****   4
       **   2
       *    1
            0
            0
[0, 0, 0, 2, 2, 3, 4, 2, 1, 0, 0] ,
angle=(π/3)*2
            0
            0
            0
       **   2
       **   2
     *****  5
    **      2
 **         2
  *         1
            0
            0
[0, 0, 0, 2, 2, 5, 2, 2, 1, 0, 0] ,
\end{lstlisting}

No, it's not enough:

\begin{lstlisting}
% time python solve3.py
sat
 *  *
    **

     * **
   **
   *  *
    **
     *   *
*   *
   ****
\end{lstlisting}

However, the result is correct, but only 3 vectors allows too many possible ``initial images'',
and Z3 SMT-solver finds first.

Further reading:
\url{https://en.wikipedia.org/wiki/Discrete_tomography},
\url{https://en.wikipedia.org/wiki/2-satisfiability#Discrete_tomography}.


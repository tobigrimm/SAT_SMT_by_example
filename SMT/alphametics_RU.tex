\section{Альфаметика}

Согласно Дональду Кнуту, термин ``Альфаметика'' был придуман Дж. Эйч. Аш. Хантером.
Это головоломка: какие десятичные цифры в пределах 0..9 нужно присвоить каждой букве, чтобы это уравнение было справедливо?

\begin{lstlisting}
  SEND
+ MORE
 -----
 MONEY
\end{lstlisting}

Для Z3 это легко:

\lstinputlisting{SMT/alpha.py}

Вывод:

\begin{lstlisting}
sat
[E, = 5,
 S, = 9,
 M, = 1,
 N, = 6,
 D, = 7,
 R, = 8,
 O, = 0,
 Y = 2]
\end{lstlisting}

Вот еще одна, из \ac{TAOCP} том IV (\url{http://www-cs-faculty.stanford.edu/~uno/fasc2b.ps.gz}):

\lstinputlisting{SMT/alpha2.py}

\begin{lstlisting}
sat
[L, = 6,
 S, = 7,
 N, = 2,
 T, = 1,
 I, = 5,
 V = 3,
 A, = 8,
 R, = 9,
 O, = 4,
 TRIO = 1954,
 SONATA, = 742818,
 VIOLA, = 35468,
 VIOLIN, = 354652]
\end{lstlisting}

Эту головоломку я нашел в примерах pySMT:

\lstinputlisting{SMT/alpha3.py}

\begin{lstlisting}
sat
[D = 5, R = 4, O = 3, E = 8, L = 6, W = 7, H = 2]
\end{lstlisting}


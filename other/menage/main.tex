\subsection{Ménage problem}

\begin{lstlisting}
In combinatorial mathematics, the ménage problem or problème des ménages[1] asks for the number of different ways in which it is possible to seat a set of male-female couples at a dining table so that men and women alternate and nobody sits next to his or her partner. This problem was formulated in 1891 by Édouard Lucas and independently, a few years earlier, by Peter Guthrie Tait in connection with knot theory.[2] For a number of couples equal to 3, 4, 5, ...  the number of seating arrangements is

    12, 96, 3120, 115200, 5836320, 382072320, 31488549120, ... (sequence A059375 in the OEIS). 
\end{lstlisting}

( \href{https://en.wikipedia.org/wiki/M%C3%A9nage_problem}{Wikipedia}. )

We can count it using Z3, but also get actual men/women allocations:

\lstinputlisting[style=custompy]{other/menage/menage.py}

( \url{https://github.com/DennisYurichev/SAT_SMT_by_example/blob/master/other/menage/menage.py} )

For 3 couples:

\begin{lstlisting}
  men 0 2 1
women  1 0 2

  men 1 2 0
women  0 1 2

  men 0 1 2
women  2 0 1

  men 2 1 0
women  0 2 1

  men 2 0 1
women  1 2 0

  men 1 0 2
women  2 1 0

results total= 6
however, according to https://oeis.org/A059375 : 12
\end{lstlisting}

We are getting ``half'' of results because men and women can be then swapped (their sex swapped (or reassigned))
and you've got another 6 results.
6+6=12 in total.
This is kind of symmetry.

For 4 couples:

\begin{lstlisting}

...

  men 3 0 2 1
women  1 3 0 2

  men 3 0 1 2
women  2 3 0 1

  men 1 0 2 3
women  3 1 0 2

  men 2 0 1 3
women  3 2 0 1

results total= 48
however, according to https://oeis.org/A059375 : 96
\end{lstlisting}

For 5 couples:

\begin{lstlisting}

...

  men 0 4 1 2 3
women  1 3 0 4 2

  men 0 3 1 2 4
women  1 4 0 3 2

  men 0 3 1 2 4
women  1 0 4 3 2

  men 4 3 1 0 2
women  0 2 4 1 3

results total= 1560
however, according to https://oeis.org/A059375 : 3120
\end{lstlisting}


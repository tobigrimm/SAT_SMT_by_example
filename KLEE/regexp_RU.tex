\subsection{Регулярные выражения}

Я всегда хотел как-нибудь сгенерировать возможные строки для определенного регулярного выражения.
Это не очень трудно, если окунуться в теорию стояющую за матчером регулярных выражений, но можно ли заставить
этот матчер сделать это?

Я нашел самый легковесный RE-движок здесь: \url{https://github.com/cesanta/slre}, и написал вот это:

\begin{lstlisting}
int main(void)
{
	char s[6];
	klee_make_symbolic(s, sizeof s, "s");
	s[5]=0;
	if (slre_match("^\\d[a-c]+(x|y|z)", s, 5, NULL, 0, 0)==5)
		klee_assert(0);
}
\end{lstlisting}

Так что я хотел строку состоящую из цифры, буквы ``a'' или ``b'' или ``c'' (как минимум один символ) и ``x'' или ``y'' или ``z'' (oдин символ).
Вся строка должна иметь размер в 5 символов.

\begin{lstlisting}
% klee --libc=uclibc slre.bc
...
KLEE: ERROR: /home/klee/slre.c:445: failed external call: klee_assert
KLEE: NOTE: now ignoring this error at this location
...

% ls klee-last | grep err
test000014.external.err

% ktest-tool --write-ints klee-last/test000014.ktest
ktest file : 'klee-last/test000014.ktest'
args       : ['slre.bc']
num objects: 1
object    0: name: b's'
object    0: size: 6
object    0: data: b'5aaax\xff'
\end{lstlisting}

Это действительно корректная строка, а на месте терминирующего нулевого байта находится ``\textbackslash{}xff'',
но RE-движок, который мы используем, игнорирует последний нулевой байт, потому ему передается длина буфера в отдельном параметре.
Следовательно, KLEE не \textit{реконструирует} последний байт.

Можем ли получить еще?
Добавляем дополнительный констрайнт:

\begin{lstlisting}
int main(void)
{
	char s[6];
	klee_make_symbolic(s, sizeof s, "s");
	s[5]=0;
	if (slre_match("^\\d[a-c]+(x|y|z)", s, 5, NULL, 0, 0)==5 &&
			strcmp(s, "5aaax")!=0)
		klee_assert(0);
}
\end{lstlisting}

\begin{lstlisting}
% ktest-tool --write-ints klee-last/test000014.ktest
ktest file : 'klee-last/test000014.ktest'
args       : ['slre.bc']
num objects: 1
object    0: name: b's'
object    0: size: 6
object    0: data: b'7aaax\xff'
\end{lstlisting}

Скажем так, просто из прихоти, нам не нравится буква ``a'' на второй позиции (если считать начиная с нулевой):

\begin{lstlisting}
int main(void)
{
	char s[6];
	klee_make_symbolic(s, sizeof s, "s");
	s[5]=0;
	if (slre_match("^\\d[a-c]+(x|y|z)", s, 5, NULL, 0, 0)==5 &&
			strcmp(s, "5aaax")!=0 &&
			s[2]!='a')
		klee_assert(0);
}
\end{lstlisting}

KLEE нашел способ удовлетворить этот новый констрайнт:

\begin{lstlisting}
% ktest-tool --write-ints klee-last/test000014.ktest
ktest file : 'klee-last/test000014.ktest'
args       : ['slre.bc']
num objects: 1
object    0: name: b's'
object    0: size: 6
object    0: data: b'7abax\xff'
\end{lstlisting}

Попробуем также определить констрайнт, который KLEE не сможет удовлетворить:

\begin{lstlisting}
int main(void)
{
	char s[6];
	klee_make_symbolic(s, sizeof s, "s");
	s[5]=0;
	if (slre_match("^\\d[a-c]+(x|y|z)", s, 5, NULL, 0, 0)==5 &&
			strcmp(s, "5aaax")!=0 &&
			s[2]!='a' &&
			s[2]!='b' &&
			s[2]!='c')
		klee_assert(0);
}
\end{lstlisting}

Действительно не может, и KLEE заканчивает работу без сообщения о входе в \TT{klee\_assert()}.


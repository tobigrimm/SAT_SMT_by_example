\subsection{School-level equation}

Let's revisit school-level system of equations from (\ref{eq2_SMT}).

We will force KLEE to find a path, where all the constraints are satisfied:

\lstinputlisting{KLEE/klee_eq1.c}

% FIXME:
\begin{lstlisting}
\$ clang -emit-llvm -c -g klee_eq.c
...

\$ klee klee_eq.bc
KLEE: output directory is "/home/klee/klee-out-93"
KLEE: WARNING: undefined reference to function: klee_assert
KLEE: WARNING ONCE: calling external: klee_assert(0)
KLEE: ERROR: /home/klee/klee_eq.c:18: failed external call: klee_assert
KLEE: NOTE: now ignoring this error at this location

KLEE: done: total instructions = 32
KLEE: done: completed paths = 1
KLEE: done: generated tests = 1
\end{lstlisting}

Let's find out, where \TT{klee\_assert()} has been triggered:

% FIXME:
\begin{lstlisting}
\$ ls klee-last | grep err
test000001.external.err

\$ ktest-tool --write-ints klee-last/test000001.ktest
ktest file : 'klee-last/test000001.ktest'
args       : ['klee_eq.bc']
num objects: 3
object    0: name: b'circle'
object    0: size: 4
object    0: data: 5
object    1: name: b'square'
object    1: size: 4
object    1: data: 2
object    2: name: b'triangle'
object    2: size: 4
object    2: data: 1
\end{lstlisting}

This is indeed correct solution to the system of equations.

KLEE has intrinsic \TT{klee\_assume()} which tells KLEE to cut path if some constraint is not true.
So we can rewrite our example in such cleaner way:

\lstinputlisting{KLEE/klee_eq2.c}



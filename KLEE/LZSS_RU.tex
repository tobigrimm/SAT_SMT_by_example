\subsection{Декомпрессор LZSS}

Я погуглил в поисках очень простого декомпрессора \ac{LZSS} и остановился на этой странице:
\url{http://www.opensource.apple.com/source/boot/boot-132/i386/boot2/lzss.c}.

Сделаем вид, что мы смотрим на неизвестный алгоритм сжатия, к которому нет компрессора.
Можно ли будет ресконструировать сжатый фрагмент данных, так, что декомпрессор сгенерирует нужные нам данные?

Вот мой первый эксперимент:

\lstinputlisting{KLEE/klee_lzss1.c}

Я сменил размер кольцевого буфера с 4096 на 32, потому что если больше, KLEE начинает занимать всю память, что может.
Но я нашел что KLEE может кое-как работать с м\'{е}ньшим буфером.
Я также постепенно уменьшал \TT{COMPRESSED\_LEN}, чтобы проверить, сможет ли KLEE найти сжатый фрагмент данных, и он нашел:

\begin{lstlisting}
% clang -emit-llvm -c -g klee_lzss.c
...

% time klee klee_lzss.bc
KLEE: output directory is "/home/klee/klee-out-7"
KLEE: WARNING: undefined reference to function: klee_assert
KLEE: ERROR: /home/klee/klee_lzss.c:122: invalid klee_assume call (provably false)
KLEE: NOTE: now ignoring this error at this location
KLEE: ERROR: /home/klee/klee_lzss.c:47: memory error: out of bound pointer
KLEE: NOTE: now ignoring this error at this location
KLEE: ERROR: /home/klee/klee_lzss.c:37: memory error: out of bound pointer
KLEE: NOTE: now ignoring this error at this location
KLEE: WARNING ONCE: calling external: klee_assert(0)
KLEE: ERROR: /home/klee/klee_lzss.c:124: failed external call: klee_assert
KLEE: NOTE: now ignoring this error at this location

KLEE: done: total instructions = 41417919
KLEE: done: completed paths = 437820
KLEE: done: generated tests = 4

real    13m0.215s
user    11m57.517s
sys     1m2.187s

% ls klee-last | grep err
test000001.user.err
test000002.ptr.err
test000003.ptr.err
test000004.external.err

% ktest-tool --write-ints klee-last/test000004.ktest
ktest file : 'klee-last/test000004.ktest'
args       : ['klee_lzss.bc']
num objects: 1
object    0: name: b'input'
object    0: size: 15
object    0: data: b'\xffBuffalo \x01b\x0f\x03\r\x05'
\end{lstlisting}

KLEE занял $\approx 1GB$ памяти и работал $\approx 15$ минут (на моем ноутбуке с Intel Core i3-3110M 2.4GHz), 
но вот результат, 15 байт, которые, если расжать нашим скопированным алгоритмом, выдают нужный нам текст!

В процессе экспериментирования, я нашел что KLEE может сделать одну вещь даже еще круче, найти размер сжатого буфера:

\begin{lstlisting}
int main()
{
	uint8_t input[24];
	uint8_t plain[24];
	uint32_t size;
  
	klee_make_symbolic(input, sizeof input, "input");
	klee_make_symbolic(&size, sizeof size, "size");
	
	decompress_lzss(plain, input, size);

	for (int i=0; i<23; i++)
		klee_assume (plain[i]=="Buffalo buffalo Buffalo"[i]);

	klee_assert(0);
	
	return 0;
}
\end{lstlisting}

\dots но тогда KLEE работает намного медленнее, занимает намного больше памяти, и у меня получилось это даже с еще
м\'{е}ньшим размером желаемого текста.

Так как работает \ac{LZSS}? Без подглядывания в Википедию, мы можем сказать, что:
если компрессор \ac{LZSS} видит те же данные, что уже видел, он заменяет данные на ссылку на какое-то место в прошлом, с длиной.
Если он видит то, что еще пока не видел, он копирует данные, как есть.
Это теория.
И это то, что мы и получили. Желаемый текст это три слова ``Buffalo'', но первое и последнее одинаковы,
но второе \textit{почти} такое же, отличающееся только одним символом.

Вот что видим:

% FIXME: colored Buffalo, ``b'', slashes
\begin{lstlisting}
'\xffBuffalo \x01b\x0f\x03\r\x05'
\end{lstlisting}

Здесь есть какой-то управляющий байт (0xff), слово ``Buffalo'' скопировано \textit{как есть}, потом еще один управляющий байт
(0x01), 
затем мы видим начало второго слова (``b'') и далее управляющие байты, вероятно, ссылки на начало буфера.
Это команды лдя декомпрессора, на обычном русском языке, ``скопируй данные из буфера, которые мы уже копировали,
вот с этого места, по это место'', итд.

Интересно, можно ли вмешаться в этот фрагмент сжатых данных?
Сугубо из прихоти, мы можем заставить KLEE найти сжатые данные,
где \textit{как есть} будет помещен не только символ ``b'', но также и второй символ этого слова, т.е., ``bu''?

Я изменил ф-цию main() добавив \TT{klee\_assume()}: теперь 11-й байт входного (сжатого) буфера (прямо за байтом ``b'') должен
иметь ``u''.
С размером сжатых данных в 15 байт ничего не получилось, так что я увеличил до 16 байт:

\begin{lstlisting}
int main()
{
#define COMPRESSED_LEN 16
	uint8_t input[COMPRESSED_LEN];
	uint8_t plain[24];
	uint32_t size=COMPRESSED_LEN;
  
	klee_make_symbolic(input, sizeof input, "input");
	
	klee_assume(input[11]=='u');
	
	decompress_lzss(plain, input, size);

	for (int i=0; i<23; i++)
		klee_assume (plain[i]=="Buffalo buffalo Buffalo"[i]);

	klee_assert(0);
	
	return 0;
}
\end{lstlisting}

\dots и ура: KLEE нашел фрагмент сжатых данных, который удовлетворяет нашей прихоти:

\begin{lstlisting}
% time klee klee_lzss.bc
KLEE: output directory is "/home/klee/klee-out-9"
KLEE: WARNING: undefined reference to function: klee_assert
KLEE: ERROR: /home/klee/klee_lzss.c:97: invalid klee_assume call (provably false)
KLEE: NOTE: now ignoring this error at this location
KLEE: ERROR: /home/klee/klee_lzss.c:47: memory error: out of bound pointer
KLEE: NOTE: now ignoring this error at this location
KLEE: ERROR: /home/klee/klee_lzss.c:37: memory error: out of bound pointer
KLEE: NOTE: now ignoring this error at this location
KLEE: WARNING ONCE: calling external: klee_assert(0)
KLEE: ERROR: /home/klee/klee_lzss.c:99: failed external call: klee_assert
KLEE: NOTE: now ignoring this error at this location

KLEE: done: total instructions = 36700587
KLEE: done: completed paths = 369756
KLEE: done: generated tests = 4

real    12m16.983s
user    11m17.492s
sys     0m58.358s

% ktest-tool --write-ints klee-last/test000004.ktest
ktest file : 'klee-last/test000004.ktest'
args       : ['klee_lzss.bc']
num objects: 1
object    0: name: b'input'
object    0: size: 16
object    0: data: b'\xffBuffalo \x13bu\x10\x02\r\x05'
\end{lstlisting}

Так что теперь у нас есть фрагмент сжатых данных, где две строки были помещены \textit{как есть}: ``Buffalo'' и ``bu''.

% FIXME: colored Buffalo and bu
\begin{lstlisting}
'\xffBuffalo \x13bu\x10\x02\r\x05'
\end{lstlisting}

Обе фрагмента, если их подать на вход нашей ф-ции, выдадут текстовую строку ``Buffalo buffalo Buffalo''.

Нужно отметить, что у меня всё так же нет доступа к коду компрессора \ac{LZSS}, и я пока не вникал в детали декомпрессора.

К сожалению, всё не так оптимистично:
KLEE очень медленный и что-то и получилось только с короткими кусками текста, и также размер кольцевого буфера был сильно
уменьшен (оригинальный декомпрессор \ac{LZSS} с буфером в 4096 байт не может корректно разжать то, что мы здесь нашли).

Тем не менее, это всё еще очень впечатляет, учитывая тот факт, что мы не изучали устройство этого спефицичного декомпрессора.
В очередной раз, мы создали \textit{антипод} декомпрессора, или \textit{инверсную функцию}.

Также, как выясняется, KLEE пока еще не очень хорош с алгоритмами декомпрессии (ну а кто хорош?).
Я также пробовал различные декодеры JPEG/PNG/GIF (которые, конечно же, имеют декомпрессоры), начиная с простейшего,
но KLEE застрял.


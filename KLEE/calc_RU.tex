\subsection{Unit-тестирование: простой калькулятор}

Искал простой калькулятор, который принимает на вход выражение вроде ``2+2'' и выдает ответ.
Нашел один здесь: \url{http://stackoverflow.com/a/13895198}.
К сожалению, в нем не было ошибок, так что я добавил одну: буфер токенов (\TT{buf[]} на строке) короче чем входной буфер (\TT{input[]} на строке).

\lstinputlisting[numbers=left]{KLEE/calc.c}
( \url{https://github.com/dennis714/SAT_SMT_article/blob/master/KLEE/calc.c} )

KLEE легко нашел переполнение буфера (65 нулей + один символ табуляции):

\begin{lstlisting}
% ktest-tool --write-ints klee-last/test000468.ktest
ktest file : 'klee-last/test000468.ktest'
args       : ['calc.bc']
num objects: 1
object    0: name: b'input'
object    0: size: 128
object    0: data: b'0\t0000000000000000000000000000000000000000000000000000000000000000\xff\xff\xff\xff\xff\xff\xff\xff\xff\xff\xff\xff\xff\xff\xff\xff\xff\xff\xff\xff\xff\xff\xff\xff\xff\xff\xff\xff\xff\xff\xff\xff\xff\xff\xff\xff\xff\xff\xff\xff\xff\xff\xff\xff\xff\xff\xff\xff\xff\xff\xff\xff\xff\xff\xff\xff\xff\xff\xff\xff\xff\xff'
\end{lstlisting}

Трудно сказать, как в массив input[] попал символ табуляции (\TT{\textbackslash{}t})? но KLEE достиг желаемого: буфер переполнился.\\
\\
KLEE также нашла две строки с выражениями, которые приводят к делению на ноль (``0/0'' и ``0\%0''):

\begin{lstlisting}
% ktest-tool --write-ints klee-last/test000326.ktest
ktest file : 'klee-last/test000326.ktest'
args       : ['calc.bc']
num objects: 1
object    0: name: b'input'
object    0: size: 128
object    0: data: b'0/0\x00\xff\xff\xff\xff\xff\xff\xff\xff\xff\xff\xff\xff\xff\xff\xff\xff\xff\xff\xff\xff\xff\xff\xff\xff\xff\xff\xff\xff\xff\xff\xff\xff\xff\xff\xff\xff\xff\xff\xff\xff\xff\xff\xff\xff\xff\xff\xff\xff\xff\xff\xff\xff\xff\xff\xff\xff\xff\xff\xff\xff\xff\xff\xff\xff\xff\xff\xff\xff\xff\xff\xff\xff\xff\xff\xff\xff\xff\xff\xff\xff\xff\xff\xff\xff\xff\xff\xff\xff\xff\xff\xff\xff\xff\xff\xff\xff\xff\xff\xff\xff\xff\xff\xff\xff\xff\xff\xff\xff\xff\xff\xff\xff\xff\xff\xff\xff\xff\xff\xff\xff\xff\xff\xff\xff'

% ktest-tool --write-ints klee-last/test000557.ktest
ktest file : 'klee-last/test000557.ktest'
args       : ['calc.bc']
num objects: 1
object    0: name: b'input'
object    0: size: 128
object    0: data: b'0%0\x00\xff\xff\xff\xff\xff\xff\xff\xff\xff\xff\xff\xff\xff\xff\xff\xff\xff\xff\xff\xff\xff\xff\xff\xff\xff\xff\xff\xff\xff\xff\xff\xff\xff\xff\xff\xff\xff\xff\xff\xff\xff\xff\xff\xff\xff\xff\xff\xff\xff\xff\xff\xff\xff\xff\xff\xff\xff\xff\xff\xff\xff\xff\xff\xff\xff\xff\xff\xff\xff\xff\xff\xff\xff\xff\xff\xff\xff\xff\xff\xff\xff\xff\xff\xff\xff\xff\xff\xff\xff\xff\xff\xff\xff\xff\xff\xff\xff\xff\xff\xff\xff\xff\xff\xff\xff\xff\xff\xff\xff\xff\xff\xff\xff\xff\xff\xff\xff\xff\xff\xff\xff\xff\xff\xff'
\end{lstlisting}

Может это и не впечатляющий результат, тем не менее, это еще одно напоминание что операции деления и вычисления остатка должны быть обернуты как-то в продакшене, чтобы избежать возможного падения.


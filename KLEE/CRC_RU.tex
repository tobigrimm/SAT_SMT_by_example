\subsection{CRC (Циклический избыточный код)}

\subsubsection{Пример изменения буфера \#1}

Иногда нужно изменить часть данных, которые \textit{защищены} некоторой контрольной суммой или
\ac{CRC}, и вы не можете изменить контрольную сумму или значение CRC, но можете изменить часто данных, так что
сумма останется той же.

Представим, что у вас есть буфер со строкой ``Hello, world!'' в начале и строкой ``and goodbye'' в конце.
Мы можем изменить 14 символов в середине, но по каким-то причинам, они должны быть в пределах \textit{a..z}, но мы можем
вставить туда любые символы.
CRC64 всего блока должен быть \TT{0x12345678abcdef12}.

Посмотрим\footnote{Существуют несколько немного отличающихся реализаций CRC64, та, что я тут использую, тоже может немного
отличаться от более популярных.}:

\lstinputlisting{KLEE/klee_CRC64.c}

Так как наш код использует стандартную ф-цию Си/Си++ memcmp(), нужно добавить опцию \TT{--libc=uclibc}, так что
KLEE будет использовать свою собственную реализацию uClibc.
% \ref{} -> closed programs

\begin{lstlisting}
% clang -emit-llvm -c -g klee_CRC64.c

% time klee --libc=uclibc klee_CRC64.bc
\end{lstlisting}

Работает около минуты (на моем ноутбуке с Intel Core i3-3110M 2.4GHz) и вот что получаем:

\begin{lstlisting}
...
real    0m52.643s
user    0m51.232s
sys     0m0.239s
...
% ls klee-last | grep err
test000001.user.err
test000002.user.err
test000003.user.err
test000004.external.err

% ktest-tool --write-ints klee-last/test000004.ktest
ktest file : 'klee-last/test000004.ktest'
args       : ['klee_CRC64.bc']
num objects: 1
object    0: name: b'buf'
object    0: size: 46
object    0: data: b'Hello, world!.. qqlicayzceamyw ... and goodbye'
\end{lstlisting}

Может это и медленно, но точно быстрее брутфорса.
Действительно, $log_2{26^{14}} \approx 65.8$, что близко к 64-м битам.
Другими словами, нужно $\approx 14$ латинских символов, чтобы закодировать 64 бита.
И KLEE + \ac{SMT}-солверу нужно 64 бита в каком-то месте, которые он может изменить, чтобы сделать окончательную сумму
CRC64 равной той, что нужна нам.

Я попробовал уменьшить длину \textit{среднего блока} до 13-и символов: неудачно для KLEE, не хватает места.

\subsubsection{Пример изменения буфера \#2}

Проявил садизм: что если буфер должен содержать значение CRC64, которое, после вычисления CRC64, должно равняться тому же значению?
Удивительно, но KLEE может решиьт и это.
Буфер теперь будет иметь такой формат:

\begin{lstlisting}
§Hello, world! <8 байт (64-битное значение)> and goodbye <еще 6 байт>§
\end{lstlisting}

\begin{lstlisting}
int main()
{
#define HEAD_STR "Hello, world!.. "
#define HEAD_SIZE strlen(HEAD_STR)
#define TAIL_STR " ... and goodbye"
#define TAIL_SIZE strlen(TAIL_STR)
// 8 bytes for 64-bit value:
#define MID_SIZE 8
#define BUF_SIZE HEAD_SIZE+TAIL_SIZE+MID_SIZE+6

	char buf[BUF_SIZE];
  
	klee_make_symbolic(buf, sizeof buf, "buf");

	klee_assume (memcmp (buf, HEAD_STR, HEAD_SIZE)==0);

	klee_assume (memcmp (buf+HEAD_SIZE+MID_SIZE, TAIL_STR, TAIL_SIZE)==0);
	
	uint64_t mid_value=*(uint64_t*)(buf+HEAD_SIZE);
	klee_assume (crc64 (0, buf, BUF_SIZE)==mid_value);

	klee_assert(0);

	return 0;
}
\end{lstlisting}

Работает:

\begin{lstlisting}
% time klee --libc=uclibc klee_CRC64.bc
...
real    5m17.081s
user    5m17.014s
sys     0m0.319s

% ls klee-last | grep err
test000001.user.err
test000002.user.err
test000003.external.err

% ktest-tool --write-ints klee-last/test000003.ktest
ktest file : 'klee-last/test000003.ktest'
args       : ['klee_CRC64.bc']
num objects: 1
object    0: name: b'buf'
object    0: size: 46
object    0: data: b'Hello, world!.. T+]\xb9A\x08\x0fq ... and goodbye\xb6\x8f\x9c\xd8\xc5\x00'
\end{lstlisting}

8 байт меджу двумя строками это 64-битное значение, которое равно CRC64 всего блока.
И снова, это быстрее, чем использовать брутфорс для поиска.
Если уменьшить последний 6-байтный буфер до 4 байт или меньше, KLEE работает слишком долго, пришлось остановить.

\subsubsection{Восстановление входного буфера для заданного значения CRC32}

Всегда хотелось это сделать, но всё знают, чот это невозможно для входных буферов длинее 4-х байт.
Как показывают мои эксперименты, это все же возможно для очень маленьких буферов, содержимое которых как-то ограничено.

Значение CRC32 для 6-байтной строки ``SILVER'' известно: \TT{0xDFA3DFDD}.
KLEE может найти эту 6-байтную строку, если он знает, что каждый байт входного буфера находится в пределах \textit{A..Z}:

\lstinputlisting[numbers=left]{KLEE/klee_SILVER.c}

\begin{lstlisting}
% clang -emit-llvm -c -g klee_SILVER.c
...

% klee klee_SILVER.bc
...

% ls klee-last | grep err
test000013.external.err

% ktest-tool --write-ints klee-last/test000013.ktest
ktest file : 'klee-last/test000013.ktest'
args       : ['klee_SILVER.bc']
num objects: 1
object    0: name: b'str'
object    0: size: 6
object    0: data: b'SILVER'
\end{lstlisting}

Все же, никакой магии, если убрать условие в строках 23..25 (т.е., ослабить констрайнты),
KLEE выдаст какую-то другую строку, которая тоже будет правильна для этого значения CRC32.

Это работает, потому что 6 латинских символов в пределах \textit{A..Z} содержат $\approx 28.2$ бит:
$log_2{26^6} \approx 28.2$, что даже меньше чем 32.
Другими словами, конечное значение CRC32 содержит достаточно бит, чтобы восстановить $\approx 28.2$ бит входа.

Входной буфер может быть даже больше, если каждый его байт будет находится даже в еще более
жестких констрайнтах (десятичные цифры, двоичные цифры, итд).

\subsubsection{В сравнении с другими алгоритмами хэширования}

Всё это так просто еще и для других алгоритмом хэширования вроде контрольной суммы Флетчера,
но не для криптостойких (как MD5, SHA1, etc), они защищены от такого простого криптоанализа.
См.также: \ref{crypto}.


\section{KLEE}

\subsection{Инсталляция}

Процесс сборки KLEE из исходников сложный и запутанный.
Самый простой способ использовать KLEE, это инсталлировать docker
\footnote{\url{https://docs.docker.com/engine/installation/linux/ubuntulinux/}} а затем запустить образ KLEE для docker
\footnote{\url{http://klee.github.io/docker/}}.
Путь, где находятся файлы KLEE, может выглядет так:
\textbf{/var/lib/docker/aufs/mnt/(много шестнадцетиричных цифр)/home/klee}.

% subsections:
\input{KLEE/eq_RU.tex}
\subsection{Головоломка зебры}

Снова вернемся к головоломке зебры (\ref{xebra_SMT}).

Мы просто определяем все переменные и добавляем констрайнты:

\lstinputlisting{KLEE/klee_zebra1.c}

Я заставил KLEE находить отличные друг от друга значения для цветов, национальностей, сигарет, итд, точно также,
как я раннее сделал это для Судоку: (\ref{sudoku_SMT}).

Запускаем:

\begin{lstlisting}
% clang -emit-llvm -c -g klee_zebra1.c
...

% klee klee_zebra1.bc
KLEE: output directory is "/home/klee/klee-out-97"
KLEE: WARNING: undefined reference to function: klee_assert
KLEE: WARNING ONCE: calling external: klee_assert(0)
KLEE: ERROR: /home/klee/klee_zebra1.c:130: failed external call: klee_assert
KLEE: NOTE: now ignoring this error at this location

KLEE: done: total instructions = 761
KLEE: done: completed paths = 55
KLEE: done: generated tests = 55
\end{lstlisting}

Работает $\approx 7$ секунд на моем ноутбуке с Intel Core i3-3110M 2.4GHz.
Найдем путь, где был исполнен \TT{klee\_assert()}:

\begin{lstlisting}
% ls klee-last | grep err
test000051.external.err

% ktest-tool --write-ints klee-last/test000051.ktest | less

ktest file : 'klee-last/test000051.ktest'
args       : ['klee_zebra1.bc']
num objects: 25
object    0: name: b'Yellow'
object    0: size: 4
object    0: data: 1
object    1: name: b'Blue'
object    1: size: 4
object    1: data: 2
object    2: name: b'Red'
object    2: size: 4
object    2: data: 3
object    3: name: b'Ivory'
object    3: size: 4
object    3: data: 4

...

object   21: name: b'Horse'
object   21: size: 4
object   21: data: 2
object   22: name: b'Snails'
object   22: size: 4
object   22: data: 3
object   23: name: b'Dog'
object   23: size: 4
object   23: data: 4
object   24: name: b'Zebra'
object   24: size: 4
object   24: data: 5
\end{lstlisting}

Это действительно корректное решение.

В этот раз можно также использовать \TT{klee\_assume()}:

\lstinputlisting{KLEE/klee_zebra2.c}

\dots и эта версия работает немного быстрее ($\approx 5$ секунд),
может быть потому что KLEE знает об этой \textit{intrinsic} и обращается с ним особым образом?


\subsection{Головоломка Судоку}
\label{sudoku_SMT}

Головоломка Судоку это решетка 9*9, некоторые ячейки заполнены значениями, некоторые пустые:

% copypasted from http://www.texample.net/tikz/examples/sudoku/
\newcounter{row}
\newcounter{col}

\newcommand\setrow[9]{
  \setcounter{col}{1}
  \foreach \n in {#1, #2, #3, #4, #5, #6, #7, #8, #9} {
    \edef\x{\value{col} - 0.5}
    \edef\y{9.5 - \value{row}}
    \node[anchor=center] at (\x, \y) {\n};
    \stepcounter{col}
  }
  \stepcounter{row}
}

\begin{center}
\begin{tikzpicture}[scale=.7]
  \begin{scope}
    \draw (0, 0) grid (9, 9);
    \draw[very thick, scale=3] (0, 0) grid (3, 3);

    \setcounter{row}{1}
    \setrow { }{ }{5}  {3}{ }{ }  { }{ }{ }
    \setrow {8}{ }{ }  { }{ }{ }  { }{2}{ }
    \setrow { }{7}{ }  { }{1}{ }  {5}{ }{ }

    \setrow {4}{ }{ }  { }{ }{5}  {3}{ }{ }
    \setrow { }{1}{ }  { }{7}{ }  { }{ }{6}
    \setrow { }{ }{3}  {2}{ }{ }  { }{8}{ }

    \setrow { }{6}{ }  {5}{ }{ }  { }{ }{9}
    \setrow { }{ }{4}  { }{ }{ }  { }{3}{ }
    \setrow { }{ }{ }  { }{ }{9}  {7}{ }{ }

    \node[anchor=center] at (4.5, -0.5) {Нерешенная Судоку};
  \end{scope}
\end{tikzpicture}
\end{center}

Числа в каждом ряду должны быть уникальными, т.е., каждый ряд должен содержать 9 чисел в пределах 1..9 без повторений.
Та же история и для каждого столбца и каждого квадрата 3*3.

Головоломка представляет собой хороший кандидат, на котором можно попробовать \ac{SMT}-солвер, потому что это,
в общем-то, просто нерешенная система уравнений.

\subsubsection{Первая идея}

Всё что нужно решить, это как определять в одном выражении, содержат ли 9 переменных все 9 уникальных чисел?
Они ведь не упорядочены и не отсортированы, все-таки.

Из школьной арифметики, мы можем найти такую идею:

\begin{equation}
\underbrace{10^{i_1} + 10^{i_2} + \cdots + 10^{i_9}}_9 = 1111111110
\end{equation}

Берете каждую входную переменную, вычисляете $10^i$ и суммируете.
Если все входные значения уникальны, каждая найдет свое собственное место.
И даже более того: не будет дыр, т.е., не будет пропущенных значений.
Так что, в случае Судоку, число 1111111110 будет конечным результатом, означающим, что все входные
9 переменных уникальны, в пределах 1..9.

Возведение в степень это тяжелая операция, можно ли использовать двоичные операции? Да, просто замените 10 на 2:

\begin{equation}
\underbrace{2^{i_1} + 2^{i_2} + \cdots + 2^{i_9}}_9 = 1111111110_2
\end{equation}

Эффект тот же, но результат будет в двоичной системе а не в десятичной.

Вот рабочий пример:

\lstinputlisting{SMT/sudoku_plus.py}
( \url{https://github.com/dennis714/SAT_SMT_article/blob/master/SMT/sudoku_plus.py} )

\begin{lstlisting}
% time python sudoku_plus.py
1 4 5 3 2 7 6 9 8
8 3 9 6 5 4 1 2 7
6 7 2 9 1 8 5 4 3
4 9 6 1 8 5 3 7 2
2 1 8 4 7 3 9 5 6
7 5 3 2 9 6 4 8 1
3 6 7 5 4 2 8 1 9
9 8 4 7 6 1 2 3 5
5 2 1 8 3 9 7 6 4

real    0m11.717s
user    0m10.896s
sys     0m0.068s
\end{lstlisting}

И даже более того, можно заменить суммирование на логическое ИЛИ:

\begin{equation}
\underbrace{2^{i_1} \vee 2^{i_2} \vee \cdots \vee 2^{i_9}}_9 = 1111111110_2
\end{equation}

% FIXME: только часть исходника
\lstinputlisting{SMT/sudoku_or.py}
( \url{https://github.com/dennis714/SAT_SMT_article/blob/master/SMT/sudoku_or.py} )

Теперь работает намного быстрее. Наверное, Z3 лучше поддерживает операцию ИЛИ над битовыми векторами, чем сложение?

\begin{lstlisting}
% time python sudoku_or.py
1 4 5 3 2 7 6 9 8
8 3 9 6 5 4 1 2 7
6 7 2 9 1 8 5 4 3
4 9 6 1 8 5 3 7 2
2 1 8 4 7 3 9 5 6
7 5 3 2 9 6 4 8 1
3 6 7 5 4 2 8 1 9
9 8 4 7 6 1 2 3 5
5 2 1 8 3 9 7 6 4

real    0m1.429s
user    0m1.393s
sys     0m0.036s
\end{lstlisting}

Головоломка, которую я использовал как пример, известна как самая трудная из известных
\footnote{\url{http://www.mirror.co.uk/news/weird-news/worlds-hardest-sudoku-can-you-242294}} (по крайней мере для людей).
Для решения понадобилось $\approx 1.4$ секунды на моем ноутбуке Intel Core i3-3110M 2.4GHz.

\subsubsection{Вторая идея}

Мой первый подход далек от эффективного, я сделал то что первым пришло в голову, и оно заработало.
Другой подход это использовать команду \TT{distinct} в SMTLIB, которая говорит Z3, что некоторые переменные
должны быть отличны друг от друга (или уникальны).
Эта команда также имеется в питоновском интерфейсе к Z3.

Я переписал мой первый солвер Судоку, теперь он работает над \textit{sort}-ом 
\textit{Int}, и имеет команды \TT{distinct} вместо битовых операций,
и еще один констрайнт добавлен: значение каждой ячейки должно быть в пределах 1..9, потому что, иначе, Z3 предложит
(хотя и корректное) решение с очень большими и/или отрицательными числами.

% FIXME: только часть исходника
\lstinputlisting{SMT/sudoku2.py}
( \url{https://github.com/dennis714/SAT_SMT_article/blob/master/SMT/sudoku2.py} )

\begin{lstlisting}
% time python sudoku2.py
1 4 5 3 2 7 6 9 8
8 3 9 6 5 4 1 2 7
6 7 2 9 1 8 5 4 3
4 9 6 1 8 5 3 7 2
2 1 8 4 7 3 9 5 6
7 5 3 2 9 6 4 8 1
3 6 7 5 4 2 8 1 9
9 8 4 7 6 1 2 3 5
5 2 1 8 3 9 7 6 4

real    0m0.382s
user    0m0.346s
sys     0m0.036s
\end{lstlisting}

Это намного быстрее.

\subsubsection{Вывод}

\ac{SMT}-солверы настолько удобны, что в нашем солвере Судоку нет ничего больше ничего, мы просто определили
отношения между переменными (ячейками).

\subsubsection{Домашная работа}

Как видно, настоящая головоломка Судоку это та, для которой имеется только одно решение.
Фрагмент кода, который приведен здесь, показывает только первое.
Использая метод описанный раннее (\ref{SMTEnumerate}, также называемый ``подсчет моделей (model counting)''), 
попытайтесь найти больше решений, или доказать, что решение, которое вы нашли, единственное возможное.

\subsubsection{Дальнейшее чтение}

\url{http://www.norvig.com/sudoku.html}

\subsubsection{Судоку как \ac{SAT}-проблема}

Головоломку Судоку можно также представить как огромное \ac{CNF}-уравнение и использовать \ac{SAT}-солвер для поиска решения,
но это просто сложнее.

Некоторые статьи об этом:
\textit{Building a Sudoku Solver with SAT}\footnote{\url{http://ocw.mit.edu/courses/electrical-engineering-and-computer-science/6-005-elements-of-software-construction-fall-2011/assignments/MIT6_005F11_ps4.pdf}},
Tjark Weber, \textit{A SAT-based Sudoku Solver}\footnote{\url{https://www.lri.fr/~conchon/mpri/weber.pdf}},
Ines Lynce, Joel Ouaknine, \textit{Sudoku as a SAT Problem}\footnote{\url{http://sat.inesc-id.pt/~ines/publications/aimath06.pdf}},
Gihwon Kwon, Himanshu Jain, \textit{Optimized CNF Encoding for Sudoku Puzzles}\footnote{\url{http://www.cs.cmu.edu/~hjain/papers/sudoku-as-SAT.pdf}}.

\ac{SMT}-солвер также может использовать \ac{SAT}-солвер в своем ядре, так что он делает всю эту скучную работу
по трансляции.
Хотя и, как и компилятор, он может это делать не самым эффективным способом.


\subsection{Unit-тест: цвет в HTML/CSS}

Наиболее популярные способы представить цвет в HTML/CSS это или названием по-английски (например, ``red'') и используя шестнадцатеричное число из 6-и цифр (например, ``\#0077CC'').
Есть также третий, менее популярный способ: если каждый байт в шестнадцатеричном числе имеет 2 повторяющихся цифры, то можно вместо этого вписать \textit{аббревиатуру}, таким образом, 
``\#0077CC'' можно переписать просто как ``\#07C''.

Напишем ф-цию, которая будет конвертировать 3 компоненты цвета в название (если возможно, с первым приоритетом), 3-цифренную шестнадцатеричную форму (если возможно, с вторым приоритетом),
или в 6-цифренную шестнадцатеричную форму (как последний способ).

\lstinputlisting{KLEE/color.c}

В этой ф-ции есть 5 возможных путей, посмотрим, сможет ли KLEE найти их все?
И находит:

\begin{lstlisting}
% clang -emit-llvm -c -g color.c

% klee color.bc
KLEE: output directory is "/home/klee/klee-out-134"
KLEE: WARNING: undefined reference to function: sprintf
KLEE: WARNING: undefined reference to function: strcpy
KLEE: WARNING ONCE: calling external: strcpy(51867584, 51598960)
KLEE: ERROR: /home/klee/color.c:33: external call with symbolic argument: sprintf
KLEE: NOTE: now ignoring this error at this location
KLEE: ERROR: /home/klee/color.c:28: external call with symbolic argument: sprintf
KLEE: NOTE: now ignoring this error at this location

KLEE: done: total instructions = 479
KLEE: done: completed paths = 19
KLEE: done: generated tests = 5
\end{lstlisting}

Мы можем игнорировать вызовы strcpy() и sprintf(), потому что нам не очень-то интересно знать состояние переменной \TT{out}.

Так что здесь именно 5 путей:

\begin{lstlisting}
% ls klee-last
assembly.ll   run.stats            test000003.ktest     test000005.ktest
info          test000001.ktest     test000003.pc        test000005.pc
messages.txt  test000002.ktest     test000004.ktest     warnings.txt
run.istats    test000003.exec.err  test000005.exec.err
\end{lstlisting}

Первый набор входных значений приводит к строке ``red'':

\begin{lstlisting}
% ktest-tool --write-ints klee-last/test000001.ktest
ktest file : 'klee-last/test000001.ktest'
args       : ['color.bc']
num objects: 3
object    0: name: b'R'
object    0: size: 1
object    0: data: b'\xff'
object    1: name: b'G'
object    1: size: 1
object    1: data: b'\x00'
object    2: name: b'B'
object    2: size: 1
object    2: data: b'\x00'
\end{lstlisting}

Второй набор --- к строке ``green'':

\begin{lstlisting}
% ktest-tool --write-ints klee-last/test000002.ktest
ktest file : 'klee-last/test000002.ktest'
args       : ['color.bc']
num objects: 3
object    0: name: b'R'
object    0: size: 1
object    0: data: b'\x00'
object    1: name: b'G'
object    1: size: 1
object    1: data: b'\xff'
object    2: name: b'B'
object    2: size: 1
object    2: data: b'\x00'
\end{lstlisting}

Третий набор --- строка ``\#010000'':

\begin{lstlisting}
% ktest-tool --write-ints klee-last/test000003.ktest
ktest file : 'klee-last/test000003.ktest'
args       : ['color.bc']
num objects: 3
object    0: name: b'R'
object    0: size: 1
object    0: data: b'\x01'
object    1: name: b'G'
object    1: size: 1
object    1: data: b'\x00'
object    2: name: b'B'
object    2: size: 1
object    2: data: b'\x00'
\end{lstlisting}

Четвертый набор: строка ``blue'':

\begin{lstlisting}
% ktest-tool --write-ints klee-last/test000004.ktest
ktest file : 'klee-last/test000004.ktest'
args       : ['color.bc']
num objects: 3
object    0: name: b'R'
object    0: size: 1
object    0: data: b'\x00'
object    1: name: b'G'
object    1: size: 1
object    1: data: b'\x00'
object    2: name: b'B'
object    2: size: 1
object    2: data: b'\xff'
\end{lstlisting}

Пятый набор: строка ``\#F01'':

\begin{lstlisting}
% ktest-tool --write-ints klee-last/test000005.ktest
ktest file : 'klee-last/test000005.ktest'
args       : ['color.bc']
num objects: 3
object    0: name: b'R'
object    0: size: 1
object    0: data: b'\xff'
object    1: name: b'G'
object    1: size: 1
object    1: data: b'\x00'
object    2: name: b'B'
object    2: size: 1
object    2: data: b'\x11'
\end{lstlisting}

Эти 5 наборов входных значений могут послушить unit-тестом нашей ф-ции.


\subsection{Unit-тест: ф-ция strcmp()}

Стандартная ф-ция Си \TT{strcmp()} может возвращать 0, -1 или 1, в зависимости от результата сравнения.

Вот моя собственная реализация \TT{strcmp()}:

\lstinputlisting{KLEE/strcmp.c}

Попробуем узнать, способен ли KLEE найти все три пути?
Я специально сделал всё как можно проще для KLEE, ограничив входные массивы до двух байт, или до одного байта +
оконечивающий нулевой байт.

\begin{lstlisting}
% clang -emit-llvm -c -g strcmp.c

% klee strcmp.bc
KLEE: output directory is "/home/klee/klee-out-131"
KLEE: ERROR: /home/klee/strcmp.c:35: invalid klee_assume call (provably false)
KLEE: NOTE: now ignoring this error at this location
KLEE: ERROR: /home/klee/strcmp.c:36: invalid klee_assume call (provably false)
KLEE: NOTE: now ignoring this error at this location

KLEE: done: total instructions = 137
KLEE: done: completed paths = 5
KLEE: done: generated tests = 5

% ls klee-last
assembly.ll   run.stats            test000002.ktest     test000004.ktest
info          test000001.ktest     test000002.pc        test000005.ktest
messages.txt  test000001.pc        test000002.user.err  warnings.txt
run.istats    test000001.user.err  test000003.ktest
\end{lstlisting}

Первые две ошибки это о \TT{klee\_assume()}.
Это входные значения, на которых застряли вызовы \TT{klee\_assume()}.
Мы можем игнорировать их, либо же, из любопытства, посмотреть, что там:

\begin{lstlisting}
% ktest-tool --write-ints klee-last/test000001.ktest
ktest file : 'klee-last/test000001.ktest'
args       : ['strcmp.bc']
num objects: 2
object    0: name: b'input1'
object    0: size: 2
object    0: data: b'\x00\x00'
object    1: name: b'input2'
object    1: size: 2
object    1: data: b'\x00\x00'

% ktest-tool --write-ints klee-last/test000002.ktest
ktest file : 'klee-last/test000002.ktest'
args       : ['strcmp.bc']
num objects: 2
object    0: name: b'input1'
object    0: size: 2
object    0: data: b'a\xff'
object    1: name: b'input2'
object    1: size: 2
object    1: data: b'\x00\x00'
\end{lstlisting}

Остальные файлы это входные значения для каждого пути в моей реализации \TT{strcmp()}:

\begin{lstlisting}
% ktest-tool --write-ints klee-last/test000003.ktest
ktest file : 'klee-last/test000003.ktest'
args       : ['strcmp.bc']
num objects: 2
object    0: name: b'input1'
object    0: size: 2
object    0: data: b'b\x00'
object    1: name: b'input2'
object    1: size: 2
object    1: data: b'c\x00'

% ktest-tool --write-ints klee-last/test000004.ktest
ktest file : 'klee-last/test000004.ktest'
args       : ['strcmp.bc']
num objects: 2
object    0: name: b'input1'
object    0: size: 2
object    0: data: b'c\x00'
object    1: name: b'input2'
object    1: size: 2
object    1: data: b'a\x00'

% ktest-tool --write-ints klee-last/test000005.ktest
ktest file : 'klee-last/test000005.ktest'
args       : ['strcmp.bc']
num objects: 2
object    0: name: b'input1'
object    0: size: 2
object    0: data: b'a\x00'
object    1: name: b'input2'
object    1: size: 2
object    1: data: b'a\x00'
\end{lstlisting}

3-й это когда первый аргумент (``b'') меньше второго (``c'').
4-й это наоборот (``c'' и ``a'').
5-й это когда они равны (``a'' и ``a'').

Используя эти 3 теста, мы получаем полное покрытие (coverage) для нашей реализации \TT{strcmp()}.


\subsection{Дата и время в UNIX}

Дата и время в UNIX\footnote{\url{https://en.wikipedia.org/wiki/Unix_time}} это число секунд прошедших с
1-Jan-1970 00:00 UTC.
Ф-ция gmtime() в Си/Си++ используется для декодирования этого значения в строку, понятную человеку.

Вот фрагмент кода, который я скопипастил из древней версии ОС Minix:
(\url{http://www.cise.ufl.edu/~cop4600/cgi-bin/lxr/http/source.cgi/lib/ansi/gmtime.c}) и немного переработал:

\lstinputlisting[numbers=left]{KLEE/klee_time1.c}

Попробуем:

\begin{lstlisting}
% clang -emit-llvm -c -g klee_time1.c
...

% klee klee_time1.bc
KLEE: output directory is "/home/klee/klee-out-107"
KLEE: WARNING: undefined reference to function: printf
KLEE: ERROR: /home/klee/klee_time1.c:86: external call with symbolic argument: printf
KLEE: NOTE: now ignoring this error at this location
KLEE: ERROR: /home/klee/klee_time1.c:83: ASSERTION FAIL: 0
KLEE: NOTE: now ignoring this error at this location

KLEE: done: total instructions = 101579
KLEE: done: completed paths = 1635
KLEE: done: generated tests = 2
\end{lstlisting}

Ух ты, на строке 83 сработал assert(), почему?
Посмотрим, какое значение UNIX-времени привело к этому:

\begin{lstlisting}
% ls klee-last | grep err
test000001.exec.err
test000002.assert.err

% ktest-tool --write-ints klee-last/test000002.ktest
ktest file : 'klee-last/test000002.ktest'
args       : ['klee_time1.bc']
num objects: 1
object    0: name: b'time'
object    0: size: 4
object    0: data: 978278400
\end{lstlisting}

Попробуем декодировать это значение используя утилиту date в UNIX:

\begin{lstlisting}
% date -u --date='@978278400'
Sun Dec 31 16:00:00 UTC 2000
\end{lstlisting}

После изучения, я нашел что переменная \TT{month} может содержать неверное значение 12 (для которого максимальное это 11,
для декабря), 
потому что макрос LEAPYEAR() должен принимать на вход год как 2000, а не как 100.
Так что пока я переписывал ф-цию, я сделал ошибку, и KLEE нашла её!

Просто интересно, что будет если я заменю switch() на массив строк, как это обычно пишется в кратком коде на Си/Си++?

\begin{lstlisting}
	...

const char *_months[] =
{
	"January", "February", "March",
	"April", "May", "June",
	"July", "August", "September",
	"October", "November", "December"
};

	...

	while (dayno >= _ytab[LEAPYEAR(year)][month])
	{
		dayno -= _ytab[LEAPYEAR(year)][month];
		month++;
	}
	
	char *s=_months[month];

	printf ("%04d-%s-%02d %02d:%02d:%02d\n", YEAR0+year, s, dayno+1, hour, minutes, seconds);
	printf ("week day: %s\n", _days[wday]);	
	
	...

\end{lstlisting}

KLEE обнаруживает попытку прочитать за границами массива:

\begin{lstlisting}
% klee klee_time2.bc
KLEE: output directory is "/home/klee/klee-out-108"
KLEE: WARNING: undefined reference to function: printf
KLEE: ERROR: /home/klee/klee_time2.c:69: external call with symbolic argument: printf
KLEE: NOTE: now ignoring this error at this location
KLEE: ERROR: /home/klee/klee_time2.c:67: memory error: out of bound pointer
KLEE: NOTE: now ignoring this error at this location

KLEE: done: total instructions = 101716
KLEE: done: completed paths = 1635
KLEE: done: generated tests = 2
\end{lstlisting}

Это то же самое UNIX-время, которое мы уже видели:

\begin{lstlisting}
% ls klee-last | grep err
test000001.exec.err
test000002.ptr.err

% ktest-tool --write-ints klee-last/test000002.ktest
ktest file : 'klee-last/test000002.ktest'
args       : ['klee_time2.bc']
num objects: 1
object    0: name: b'time'
object    0: size: 4
object    0: data: 978278400
\end{lstlisting}

Так что, если этот фрагмент кода может быть выполнен на удаленном компьютере, с этим входным значением
(\textit{input of death}),
Так можно свалить процесс (хотя и с какой-то удачей).\\
\\
ОК, теперь я исправляю ошибку, перемещая выражение где отнимается год на строку 43, и теперь посмотрим,
какое UNIX-время соответствует некоторой красивой дате вроде 2022-February-2?

\lstinputlisting[numbers=left]{KLEE/klee_time3.c}

\begin{lstlisting}
% clang -emit-llvm -c -g klee_time3.c
...

% klee klee_time3.bc
KLEE: output directory is "/home/klee/klee-out-109"
KLEE: WARNING: undefined reference to function: klee_assert
KLEE: WARNING ONCE: calling external: klee_assert(0)
KLEE: ERROR: /home/klee/klee_time3.c:47: failed external call: klee_assert
KLEE: NOTE: now ignoring this error at this location

KLEE: done: total instructions = 101087
KLEE: done: completed paths = 1635
KLEE: done: generated tests = 1635

% ls klee-last | grep err
test000587.external.err

% ktest-tool --write-ints klee-last/test000587.ktest
ktest file : 'klee-last/test000587.ktest'
args       : ['klee_time3.bc']
num objects: 1
object    0: name: b'time'
object    0: size: 4
object    0: data: 1645488640

% date -u --date='@1645488640'
Tue Feb 22 00:10:40 UTC 2022
\end{lstlisting}

Успешно нашли, но часы/минуты/секунды выглядят как случайные --- они и правда случайные, потому что KLEE удовлетворило
все констрайнты нами добавленные, ничего более.
Мы ведь не просили выставить часы/минуты/секунды в нули.

Добавим также констрайнты для часом/минут/секунд:

\begin{lstlisting}
	...

	if (YEAR0+year==2022 && month==1 && dayno+1==22 && hour==22 && minutes==22 && seconds==22)
		klee_assert(0);
	
	...
\end{lstlisting}

Запустим и проверим \dots

% FIXME:
\begin{lstlisting}
% ktest-tool --write-ints klee-last/test000597.ktest
ktest file : 'klee-last/test000597.ktest'
args       : ['klee_time3.bc']
num objects: 1
object    0: name: b'time'
object    0: size: 4
object    0: data: 1645568542

% date -u --date='@1645568542'
Tue Feb 22 22:22:22 UTC 2022
\end{lstlisting}

Теперь всё точно.

Да, конечно, в библиотеках Си/Си++ есть ф-ции для кодирования строки с датой в UNIX-время, но то что мы тут получили,
это KLEE работающий как \textit{антипод} декодирующей ф-ции, \textit{инверсная ф-ция} в каком-то смысле.

\subsection{Обратная ф-ция base64-декодера}

Для KLEE нет никаких проблем реконструировать входную base64-строку, имея только код декодировщик base64, без соответствующего
кодировщика.
Я скопипастил код из
\url{http://www.opensource.apple.com/source/QuickTimeStreamingServer/QuickTimeStreamingServer-452/CommonUtilitiesLib/base64.c}.

Мы добавляем констрайнты (строки 84, 85) так что выходной буфер будет иметь байты от 0 до 15.
Мы также говорим KLEE что ф-ция Base64decode() должна вернуть 16 (т.е., размер выходного буфера в байтах, строка 82).

\lstinputlisting[numbers=left]{KLEE/klee_base64.c}

\begin{lstlisting}
% clang -emit-llvm -c -g klee_base64.c
...

% klee klee_base64.bc
KLEE: output directory is "/home/klee/klee-out-99"
KLEE: WARNING: undefined reference to function: klee_assert
KLEE: ERROR: /home/klee/klee_base64.c:99: invalid klee_assume call (provably false)
KLEE: NOTE: now ignoring this error at this location
KLEE: WARNING ONCE: calling external: klee_assert(0)
KLEE: ERROR: /home/klee/klee_base64.c:104: failed external call: klee_assert
KLEE: NOTE: now ignoring this error at this location
KLEE: ERROR: /home/klee/klee_base64.c:85: memory error: out of bound pointer
KLEE: NOTE: now ignoring this error at this location
KLEE: ERROR: /home/klee/klee_base64.c:81: memory error: out of bound pointer
KLEE: NOTE: now ignoring this error at this location
KLEE: ERROR: /home/klee/klee_base64.c:65: memory error: out of bound pointer
KLEE: NOTE: now ignoring this error at this location

...
\end{lstlisting}

Интересно посмотреть на вторую ошиюку, где сработала \TT{klee\_assert()}:

\begin{lstlisting}
% ls klee-last | grep err
test000001.user.err
test000002.external.err
test000003.ptr.err
test000004.ptr.err
test000005.ptr.err

% ktest-tool --write-ints klee-last/test000002.ktest
ktest file : 'klee-last/test000002.ktest'
args       : ['klee_base64.bc']
num objects: 1
object    0: name: b'input'
object    0: size: 32
object    0: data: b'AAECAwQFBgcICQoLDA0OD4\x00\xff\xff\xff\xff\xff\xff\xff\xff\x00'
\end{lstlisting}

Это действительно настоящая base64-строка, оканчивающаяся нулевым байтом, как и требуется по стандартам Си/Си++.
Последний нулевой байт на месте 31-го (считая с нулевого) это дело наших рук: так KLEE будет выдавать меньшее количество
ошибок.

Строка в base64 действительно корректна:

\begin{lstlisting}
% echo AAECAwQFBgcICQoLDA0OD4 | base64 -d | hexdump -C
base64: invalid input
00000000  00 01 02 03 04 05 06 07  08 09 0a 0b 0c 0d 0e 0f  |................|
00000010
\end{lstlisting}

Утилита для декодирования base64 из Linux которую я только что запустил, ругается: ``invalid input'' --- 
это означает что строка неверно дополнена выровнивающими символами (``='').
Дополним вручную, и декодер больше не будет ругаться:

\begin{lstlisting}
% echo AAECAwQFBgcICQoLDA0OD4== | base64 -d | hexdump -C
00000000  00 01 02 03 04 05 06 07  08 09 0a 0b 0c 0d 0e 0f  |................|
00000010
\end{lstlisting}

Причина, по которой наша base64-строка не была дополнена, в том, что декодеры base64 обычно игнорируют эти символы (``='')
в конце.
Другими словами, они не тербуют их, так же, как и наш декодер.
Так что, символы для выравнивания остались незамеченными для KLEE.

И снова мы сделали \textit{антипод} или \textit{инверсную функцию} декодера base64.


\subsection{CRC (Циклический избыточный код)}

\subsubsection{Пример изменения буфера \#1}

Иногда нужно изменить часть данных, которые \textit{защищены} некоторой контрольной суммой или
\ac{CRC}, и вы не можете изменить контрольную сумму или значение CRC, но можете изменить часто данных, так что
сумма останется той же.

Представим, что у вас есть буфер со строкой ``Hello, world!'' в начале и строкой ``and goodbye'' в конце.
Мы можем изменить 14 символов в середине, но по каким-то причинам, они должны быть в пределах \textit{a..z}, но мы можем
вставить туда любые символы.
CRC64 всего блока должен быть \TT{0x12345678abcdef12}.

Посмотрим\footnote{Существуют несколько немного отличающихся реализаций CRC64, та, что я тут использую, тоже может немного
отличаться от более популярных.}:

\lstinputlisting{KLEE/klee_CRC64.c}

Так как наш код использует стандартную ф-цию Си/Си++ memcmp(), нужно добавить опцию \TT{--libc=uclibc}, так что
KLEE будет использовать свою собственную реализацию uClibc.
% \ref{} -> closed programs

\begin{lstlisting}
% clang -emit-llvm -c -g klee_CRC64.c

% time klee --libc=uclibc klee_CRC64.bc
\end{lstlisting}

Работает около минуты (на моем ноутбуке с Intel Core i3-3110M 2.4GHz) и вот что получаем:

\begin{lstlisting}
...
real    0m52.643s
user    0m51.232s
sys     0m0.239s
...
% ls klee-last | grep err
test000001.user.err
test000002.user.err
test000003.user.err
test000004.external.err

% ktest-tool --write-ints klee-last/test000004.ktest
ktest file : 'klee-last/test000004.ktest'
args       : ['klee_CRC64.bc']
num objects: 1
object    0: name: b'buf'
object    0: size: 46
object    0: data: b'Hello, world!.. qqlicayzceamyw ... and goodbye'
\end{lstlisting}

Может это и медленно, но точно быстрее брутфорса.
Действительно, $log_2{26^{14}} \approx 65.8$, что близко к 64-м битам.
Другими словами, нужно $\approx 14$ латинских символов, чтобы закодировать 64 бита.
И KLEE + \ac{SMT}-солверу нужно 64 бита в каком-то месте, которые он может изменить, чтобы сделать окончательную сумму
CRC64 равной той, что нужна нам.

Я попробовал уменьшить длину \textit{среднего блока} до 13-и символов: неудачно для KLEE, не хватает места.

\subsubsection{Пример изменения буфера \#2}

Проявил садизм: что если буфер должен содержать значение CRC64, которое, после вычисления CRC64, должно равняться тому же значению?
Удивительно, но KLEE может решиьт и это.
Буфер теперь будет иметь такой формат:

\begin{lstlisting}
§Hello, world! <8 байт (64-битное значение)> and goodbye <еще 6 байт>§
\end{lstlisting}

\begin{lstlisting}
int main()
{
#define HEAD_STR "Hello, world!.. "
#define HEAD_SIZE strlen(HEAD_STR)
#define TAIL_STR " ... and goodbye"
#define TAIL_SIZE strlen(TAIL_STR)
// 8 bytes for 64-bit value:
#define MID_SIZE 8
#define BUF_SIZE HEAD_SIZE+TAIL_SIZE+MID_SIZE+6

	char buf[BUF_SIZE];
  
	klee_make_symbolic(buf, sizeof buf, "buf");

	klee_assume (memcmp (buf, HEAD_STR, HEAD_SIZE)==0);

	klee_assume (memcmp (buf+HEAD_SIZE+MID_SIZE, TAIL_STR, TAIL_SIZE)==0);
	
	uint64_t mid_value=*(uint64_t*)(buf+HEAD_SIZE);
	klee_assume (crc64 (0, buf, BUF_SIZE)==mid_value);

	klee_assert(0);

	return 0;
}
\end{lstlisting}

Работает:

\begin{lstlisting}
% time klee --libc=uclibc klee_CRC64.bc
...
real    5m17.081s
user    5m17.014s
sys     0m0.319s

% ls klee-last | grep err
test000001.user.err
test000002.user.err
test000003.external.err

% ktest-tool --write-ints klee-last/test000003.ktest
ktest file : 'klee-last/test000003.ktest'
args       : ['klee_CRC64.bc']
num objects: 1
object    0: name: b'buf'
object    0: size: 46
object    0: data: b'Hello, world!.. T+]\xb9A\x08\x0fq ... and goodbye\xb6\x8f\x9c\xd8\xc5\x00'
\end{lstlisting}

8 байт меджу двумя строками это 64-битное значение, которое равно CRC64 всего блока.
И снова, это быстрее, чем использовать брутфорс для поиска.
Если уменьшить последний 6-байтный буфер до 4 байт или меньше, KLEE работает слишком долго, пришлось остановить.

\subsubsection{Восстановление входного буфера для заданного значения CRC32}

Всегда хотелось это сделать, но всё знают, чот это невозможно для входных буферов длинее 4-х байт.
Как показывают мои эксперименты, это все же возможно для очень маленьких буферов, содержимое которых как-то ограничено.

Значение CRC32 для 6-байтной строки ``SILVER'' известно: \TT{0xDFA3DFDD}.
KLEE может найти эту 6-байтную строку, если он знает, что каждый байт входного буфера находится в пределах \textit{A..Z}:

\lstinputlisting[numbers=left]{KLEE/klee_SILVER.c}

\begin{lstlisting}
% clang -emit-llvm -c -g klee_SILVER.c
...

% klee klee_SILVER.bc
...

% ls klee-last | grep err
test000013.external.err

% ktest-tool --write-ints klee-last/test000013.ktest
ktest file : 'klee-last/test000013.ktest'
args       : ['klee_SILVER.bc']
num objects: 1
object    0: name: b'str'
object    0: size: 6
object    0: data: b'SILVER'
\end{lstlisting}

Все же, никакой магии, если убрать условие в строках 23..25 (т.е., ослабить констрайнты),
KLEE выдаст какую-то другую строку, которая тоже будет правильна для этого значения CRC32.

Это работает, потому что 6 латинских символов в пределах \textit{A..Z} содержат $\approx 28.2$ бит:
$log_2{26^6} \approx 28.2$, что даже меньше чем 32.
Другими словами, конечное значение CRC32 содержит достаточно бит, чтобы восстановить $\approx 28.2$ бит входа.

Входной буфер может быть даже больше, если каждый его байт будет находится даже в еще более
жестких констрайнтах (десятичные цифры, двоичные цифры, итд).

\subsubsection{В сравнении с другими алгоритмами хэширования}

Всё это так просто еще и для других алгоритмом хэширования вроде контрольной суммы Флетчера,
но не для криптостойких (как MD5, SHA1, etc), они защищены от такого простого криптоанализа.
См.также: \ref{crypto}.


\subsection{Декомпрессор LZSS}

Я погуглил в поисках очень простого декомпрессора \ac{LZSS} и остановился на этой странице:
\url{http://www.opensource.apple.com/source/boot/boot-132/i386/boot2/lzss.c}.

Сделаем вид, что мы смотрим на неизвестный алгоритм сжатия, к которому нет компрессора.
Можно ли будет ресконструировать сжатый фрагмент данных, так, что декомпрессор сгенерирует нужные нам данные?

Вот мой первый эксперимент:

\lstinputlisting{KLEE/klee_lzss1.c}

Я сменил размер кольцевого буфера с 4096 на 32, потому что если больше, KLEE начинает занимать всю память, что может.
Но я нашел что KLEE может кое-как работать с м\'{е}ньшим буфером.
Я также постепенно уменьшал \TT{COMPRESSED\_LEN}, чтобы проверить, сможет ли KLEE найти сжатый фрагмент данных, и он нашел:

\begin{lstlisting}
% clang -emit-llvm -c -g klee_lzss.c
...

% time klee klee_lzss.bc
KLEE: output directory is "/home/klee/klee-out-7"
KLEE: WARNING: undefined reference to function: klee_assert
KLEE: ERROR: /home/klee/klee_lzss.c:122: invalid klee_assume call (provably false)
KLEE: NOTE: now ignoring this error at this location
KLEE: ERROR: /home/klee/klee_lzss.c:47: memory error: out of bound pointer
KLEE: NOTE: now ignoring this error at this location
KLEE: ERROR: /home/klee/klee_lzss.c:37: memory error: out of bound pointer
KLEE: NOTE: now ignoring this error at this location
KLEE: WARNING ONCE: calling external: klee_assert(0)
KLEE: ERROR: /home/klee/klee_lzss.c:124: failed external call: klee_assert
KLEE: NOTE: now ignoring this error at this location

KLEE: done: total instructions = 41417919
KLEE: done: completed paths = 437820
KLEE: done: generated tests = 4

real    13m0.215s
user    11m57.517s
sys     1m2.187s

% ls klee-last | grep err
test000001.user.err
test000002.ptr.err
test000003.ptr.err
test000004.external.err

% ktest-tool --write-ints klee-last/test000004.ktest
ktest file : 'klee-last/test000004.ktest'
args       : ['klee_lzss.bc']
num objects: 1
object    0: name: b'input'
object    0: size: 15
object    0: data: b'\xffBuffalo \x01b\x0f\x03\r\x05'
\end{lstlisting}

KLEE занял $\approx 1GB$ памяти и работал $\approx 15$ минут (на моем ноутбуке с Intel Core i3-3110M 2.4GHz), 
но вот результат, 15 байт, которые, если расжать нашим скопированным алгоритмом, выдают нужный нам текст!

В процессе экспериментирования, я нашел что KLEE может сделать одну вещь даже еще круче, найти размер сжатого буфера:

\begin{lstlisting}
int main()
{
	uint8_t input[24];
	uint8_t plain[24];
	uint32_t size;
  
	klee_make_symbolic(input, sizeof input, "input");
	klee_make_symbolic(&size, sizeof size, "size");
	
	decompress_lzss(plain, input, size);

	for (int i=0; i<23; i++)
		klee_assume (plain[i]=="Buffalo buffalo Buffalo"[i]);

	klee_assert(0);
	
	return 0;
}
\end{lstlisting}

\dots но тогда KLEE работает намного медленнее, занимает намного больше памяти, и у меня получилось это даже с еще
м\'{е}ньшим размером желаемого текста.

Так как работает \ac{LZSS}? Без подглядывания в Википедию, мы можем сказать, что:
если компрессор \ac{LZSS} видит те же данные, что уже видел, он заменяет данные на ссылку на какое-то место в прошлом, с длиной.
Если он видит то, что еще пока не видел, он копирует данные, как есть.
Это теория.
И это то, что мы и получили. Желаемый текст это три слова ``Buffalo'', но первое и последнее одинаковы,
но второе \textit{почти} такое же, отличающееся только одним символом.

Вот что видим:

% FIXME: colored Buffalo, ``b'', slashes
\begin{lstlisting}
'\xffBuffalo \x01b\x0f\x03\r\x05'
\end{lstlisting}

Здесь есть какой-то управляющий байт (0xff), слово ``Buffalo'' скопировано \textit{как есть}, потом еще один управляющий байт
(0x01), 
затем мы видим начало второго слова (``b'') и далее управляющие байты, вероятно, ссылки на начало буфера.
Это команды лдя декомпрессора, на обычном русском языке, ``скопируй данные из буфера, которые мы уже копировали,
вот с этого места, по это место'', итд.

Интересно, можно ли вмешаться в этот фрагмент сжатых данных?
Сугубо из прихоти, мы можем заставить KLEE найти сжатые данные,
где \textit{как есть} будет помещен не только символ ``b'', но также и второй символ этого слова, т.е., ``bu''?

Я изменил ф-цию main() добавив \TT{klee\_assume()}: теперь 11-й байт входного (сжатого) буфера (прямо за байтом ``b'') должен
иметь ``u''.
С размером сжатых данных в 15 байт ничего не получилось, так что я увеличил до 16 байт:

\begin{lstlisting}
int main()
{
#define COMPRESSED_LEN 16
	uint8_t input[COMPRESSED_LEN];
	uint8_t plain[24];
	uint32_t size=COMPRESSED_LEN;
  
	klee_make_symbolic(input, sizeof input, "input");
	
	klee_assume(input[11]=='u');
	
	decompress_lzss(plain, input, size);

	for (int i=0; i<23; i++)
		klee_assume (plain[i]=="Buffalo buffalo Buffalo"[i]);

	klee_assert(0);
	
	return 0;
}
\end{lstlisting}

\dots и ура: KLEE нашел фрагмент сжатых данных, который удовлетворяет нашей прихоти:

\begin{lstlisting}
% time klee klee_lzss.bc
KLEE: output directory is "/home/klee/klee-out-9"
KLEE: WARNING: undefined reference to function: klee_assert
KLEE: ERROR: /home/klee/klee_lzss.c:97: invalid klee_assume call (provably false)
KLEE: NOTE: now ignoring this error at this location
KLEE: ERROR: /home/klee/klee_lzss.c:47: memory error: out of bound pointer
KLEE: NOTE: now ignoring this error at this location
KLEE: ERROR: /home/klee/klee_lzss.c:37: memory error: out of bound pointer
KLEE: NOTE: now ignoring this error at this location
KLEE: WARNING ONCE: calling external: klee_assert(0)
KLEE: ERROR: /home/klee/klee_lzss.c:99: failed external call: klee_assert
KLEE: NOTE: now ignoring this error at this location

KLEE: done: total instructions = 36700587
KLEE: done: completed paths = 369756
KLEE: done: generated tests = 4

real    12m16.983s
user    11m17.492s
sys     0m58.358s

% ktest-tool --write-ints klee-last/test000004.ktest
ktest file : 'klee-last/test000004.ktest'
args       : ['klee_lzss.bc']
num objects: 1
object    0: name: b'input'
object    0: size: 16
object    0: data: b'\xffBuffalo \x13bu\x10\x02\r\x05'
\end{lstlisting}

Так что теперь у нас есть фрагмент сжатых данных, где две строки были помещены \textit{как есть}: ``Buffalo'' и ``bu''.

% FIXME: colored Buffalo and bu
\begin{lstlisting}
'\xffBuffalo \x13bu\x10\x02\r\x05'
\end{lstlisting}

Обе фрагмента, если их подать на вход нашей ф-ции, выдадут текстовую строку ``Buffalo buffalo Buffalo''.

Нужно отметить, что у меня всё так же нет доступа к коду компрессора \ac{LZSS}, и я пока не вникал в детали декомпрессора.

К сожалению, всё не так оптимистично:
KLEE очень медленный и что-то и получилось только с короткими кусками текста, и также размер кольцевого буфера был сильно
уменьшен (оригинальный декомпрессор \ac{LZSS} с буфером в 4096 байт не может корректно разжать то, что мы здесь нашли).

Тем не менее, это всё еще очень впечатляет, учитывая тот факт, что мы не изучали устройство этого спефицичного декомпрессора.
В очередной раз, мы создали \textit{антипод} декомпрессора, или \textit{инверсную функцию}.

Также, как выясняется, KLEE пока еще не очень хорош с алгоритмами декомпрессии (ну а кто хорош?).
Я также пробовал различные декодеры JPEG/PNG/GIF (которые, конечно же, имеют декомпрессоры), начиная с простейшего,
но KLEE застрял.


\subsection{strtodx() из RetroBSD}

Нашел эту ф-цию в RetroBSD:
\url{https://github.com/RetroBSD/retrobsd/blob/master/src/libc/stdlib/strtod.c}.
Она конвертирует строку в число с плавающей точкой для заданной системы исчисления.

\lstinputlisting[numbers=left]{KLEE/strtodx.c}
( \url{https://github.com/dennis714/SAT_SMT_article/blob/master/KLEE/strtodx.c} )

Интересно, KLEE не поддерживает арифметику с плавающей точкой, но тем не менее, что-то нашел:

\begin{lstlisting}
...

KLEE: ERROR: /home/klee/klee_test.c:202: memory error: out of bound pointer

...

% ktest-tool klee-last/test003483.ktest
ktest file : 'klee-last/test003483.ktest'
args       : ['klee_test.bc']
num objects: 1
object    0: name: b'buf'
object    0: size: 10
object    0: data: b'-.0E-66\x00\x00\x00'
\end{lstlisting}

Как видно, строка ``-.0E-66'' приводит к чтению из массива за его пределами, в строке 202.
Во время дальнейшего изучения, я обнаружил что массив \TT{powersOf10[]} слишком короткий:
был прочитан 6-й элемент (считая с нулевого).
И мы видим что часть массива закомментирована (строка 79)!
Вероятно, чья-то ошибка?


\subsection{Unit-тестирование: простой калькулятор}

Искал простой калькулятор, который принимает на вход выражение вроде ``2+2'' и выдает ответ.
Нашел один здесь: \url{http://stackoverflow.com/a/13895198}.
К сожалению, в нем не было ошибок, так что я добавил одну: буфер токенов (\TT{buf[]} на строке) короче чем входной буфер (\TT{input[]} на строке).

\lstinputlisting[numbers=left]{KLEE/calc.c}
( \url{https://github.com/dennis714/SAT_SMT_article/blob/master/KLEE/calc.c} )

KLEE легко нашел переполнение буфера (65 нулей + один символ табуляции):

\begin{lstlisting}
% ktest-tool --write-ints klee-last/test000468.ktest
ktest file : 'klee-last/test000468.ktest'
args       : ['calc.bc']
num objects: 1
object    0: name: b'input'
object    0: size: 128
object    0: data: b'0\t0000000000000000000000000000000000000000000000000000000000000000\xff\xff\xff\xff\xff\xff\xff\xff\xff\xff\xff\xff\xff\xff\xff\xff\xff\xff\xff\xff\xff\xff\xff\xff\xff\xff\xff\xff\xff\xff\xff\xff\xff\xff\xff\xff\xff\xff\xff\xff\xff\xff\xff\xff\xff\xff\xff\xff\xff\xff\xff\xff\xff\xff\xff\xff\xff\xff\xff\xff\xff\xff'
\end{lstlisting}

Трудно сказать, как в массив input[] попал символ табуляции (\TT{\textbackslash{}t})? но KLEE достиг желаемого: буфер переполнился.\\
\\
KLEE также нашла две строки с выражениями, которые приводят к делению на ноль (``0/0'' и ``0\%0''):

\begin{lstlisting}
% ktest-tool --write-ints klee-last/test000326.ktest
ktest file : 'klee-last/test000326.ktest'
args       : ['calc.bc']
num objects: 1
object    0: name: b'input'
object    0: size: 128
object    0: data: b'0/0\x00\xff\xff\xff\xff\xff\xff\xff\xff\xff\xff\xff\xff\xff\xff\xff\xff\xff\xff\xff\xff\xff\xff\xff\xff\xff\xff\xff\xff\xff\xff\xff\xff\xff\xff\xff\xff\xff\xff\xff\xff\xff\xff\xff\xff\xff\xff\xff\xff\xff\xff\xff\xff\xff\xff\xff\xff\xff\xff\xff\xff\xff\xff\xff\xff\xff\xff\xff\xff\xff\xff\xff\xff\xff\xff\xff\xff\xff\xff\xff\xff\xff\xff\xff\xff\xff\xff\xff\xff\xff\xff\xff\xff\xff\xff\xff\xff\xff\xff\xff\xff\xff\xff\xff\xff\xff\xff\xff\xff\xff\xff\xff\xff\xff\xff\xff\xff\xff\xff\xff\xff\xff\xff\xff\xff'

% ktest-tool --write-ints klee-last/test000557.ktest
ktest file : 'klee-last/test000557.ktest'
args       : ['calc.bc']
num objects: 1
object    0: name: b'input'
object    0: size: 128
object    0: data: b'0%0\x00\xff\xff\xff\xff\xff\xff\xff\xff\xff\xff\xff\xff\xff\xff\xff\xff\xff\xff\xff\xff\xff\xff\xff\xff\xff\xff\xff\xff\xff\xff\xff\xff\xff\xff\xff\xff\xff\xff\xff\xff\xff\xff\xff\xff\xff\xff\xff\xff\xff\xff\xff\xff\xff\xff\xff\xff\xff\xff\xff\xff\xff\xff\xff\xff\xff\xff\xff\xff\xff\xff\xff\xff\xff\xff\xff\xff\xff\xff\xff\xff\xff\xff\xff\xff\xff\xff\xff\xff\xff\xff\xff\xff\xff\xff\xff\xff\xff\xff\xff\xff\xff\xff\xff\xff\xff\xff\xff\xff\xff\xff\xff\xff\xff\xff\xff\xff\xff\xff\xff\xff\xff\xff\xff\xff'
\end{lstlisting}

Может это и не впечатляющий результат, тем не менее, это еще одно напоминание что операции деления и вычисления остатка должны быть обернуты как-то в продакшене, чтобы избежать возможного падения.


\subsection{Регулярные выражения}

Я всегда хотел как-нибудь сгенерировать возможные строки для определенного регулярного выражения.
Это не очень трудно, если окунуться в теорию стояющую за матчером регулярных выражений, но можно ли заставить
этот матчер сделать это?

Я нашел самый легковесный RE-движок здесь: \url{https://github.com/cesanta/slre}, и написал вот это:

\begin{lstlisting}
int main(void)
{
	char s[6];
	klee_make_symbolic(s, sizeof s, "s");
	s[5]=0;
	if (slre_match("^\\d[a-c]+(x|y|z)", s, 5, NULL, 0, 0)==5)
		klee_assert(0);
}
\end{lstlisting}

Так что я хотел строку состоящую из цифры, буквы ``a'' или ``b'' или ``c'' (как минимум один символ) и ``x'' или ``y'' или ``z'' (oдин символ).
Вся строка должна иметь размер в 5 символов.

\begin{lstlisting}
% klee --libc=uclibc slre.bc
...
KLEE: ERROR: /home/klee/slre.c:445: failed external call: klee_assert
KLEE: NOTE: now ignoring this error at this location
...

% ls klee-last | grep err
test000014.external.err

% ktest-tool --write-ints klee-last/test000014.ktest
ktest file : 'klee-last/test000014.ktest'
args       : ['slre.bc']
num objects: 1
object    0: name: b's'
object    0: size: 6
object    0: data: b'5aaax\xff'
\end{lstlisting}

Это действительно корректная строка, а на месте терминирующего нулевого байта находится ``\textbackslash{}xff'',
но RE-движок, который мы используем, игнорирует последний нулевой байт, потому ему передается длина буфера в отдельном параметре.
Следовательно, KLEE не \textit{реконструирует} последний байт.

Можем ли получить еще?
Добавляем дополнительный констрайнт:

\begin{lstlisting}
int main(void)
{
	char s[6];
	klee_make_symbolic(s, sizeof s, "s");
	s[5]=0;
	if (slre_match("^\\d[a-c]+(x|y|z)", s, 5, NULL, 0, 0)==5 &&
			strcmp(s, "5aaax")!=0)
		klee_assert(0);
}
\end{lstlisting}

\begin{lstlisting}
% ktest-tool --write-ints klee-last/test000014.ktest
ktest file : 'klee-last/test000014.ktest'
args       : ['slre.bc']
num objects: 1
object    0: name: b's'
object    0: size: 6
object    0: data: b'7aaax\xff'
\end{lstlisting}

Скажем так, просто из прихоти, нам не нравится буква ``a'' на второй позиции (если считать начиная с нулевой):

\begin{lstlisting}
int main(void)
{
	char s[6];
	klee_make_symbolic(s, sizeof s, "s");
	s[5]=0;
	if (slre_match("^\\d[a-c]+(x|y|z)", s, 5, NULL, 0, 0)==5 &&
			strcmp(s, "5aaax")!=0 &&
			s[2]!='a')
		klee_assert(0);
}
\end{lstlisting}

KLEE нашел способ удовлетворить этот новый констрайнт:

\begin{lstlisting}
% ktest-tool --write-ints klee-last/test000014.ktest
ktest file : 'klee-last/test000014.ktest'
args       : ['slre.bc']
num objects: 1
object    0: name: b's'
object    0: size: 6
object    0: data: b'7abax\xff'
\end{lstlisting}

Попробуем также определить констрайнт, который KLEE не сможет удовлетворить:

\begin{lstlisting}
int main(void)
{
	char s[6];
	klee_make_symbolic(s, sizeof s, "s");
	s[5]=0;
	if (slre_match("^\\d[a-c]+(x|y|z)", s, 5, NULL, 0, 0)==5 &&
			strcmp(s, "5aaax")!=0 &&
			s[2]!='a' &&
			s[2]!='b' &&
			s[2]!='c')
		klee_assert(0);
}
\end{lstlisting}

Действительно не может, и KLEE заканчивает работу без сообщения о входе в \TT{klee\_assert()}.



\subsection{Еще примеры}

\url{https://feliam.wordpress.com/2010/10/07/the-symbolic-maze/}

\subsection{Упражнение}

Вот мой crackme/keygenme, который может быть очень запутанным, но его очень легко решить используя KLEE:
\url{http://challenges.re/74/}.



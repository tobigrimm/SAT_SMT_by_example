\subsection{Unit-тест: цвет в HTML/CSS}

Наиболее популярные способы представить цвет в HTML/CSS это или названием по-английски (например, ``red'') и используя шестнадцатеричное число из 6-и цифр (например, ``\#0077CC'').
Есть также третий, менее популярный способ: если каждый байт в шестнадцатеричном числе имеет 2 повторяющихся цифры, то можно вместо этого вписать \textit{аббревиатуру}, таким образом, 
``\#0077CC'' можно переписать просто как ``\#07C''.

Напишем ф-цию, которая будет конвертировать 3 компоненты цвета в название (если возможно, с первым приоритетом), 3-цифренную шестнадцатеричную форму (если возможно, с вторым приоритетом),
или в 6-цифренную шестнадцатеричную форму (как последний способ).

\lstinputlisting{KLEE/color.c}

В этой ф-ции есть 5 возможных путей, посмотрим, сможет ли KLEE найти их все?
И находит:

\begin{lstlisting}
% clang -emit-llvm -c -g color.c

% klee color.bc
KLEE: output directory is "/home/klee/klee-out-134"
KLEE: WARNING: undefined reference to function: sprintf
KLEE: WARNING: undefined reference to function: strcpy
KLEE: WARNING ONCE: calling external: strcpy(51867584, 51598960)
KLEE: ERROR: /home/klee/color.c:33: external call with symbolic argument: sprintf
KLEE: NOTE: now ignoring this error at this location
KLEE: ERROR: /home/klee/color.c:28: external call with symbolic argument: sprintf
KLEE: NOTE: now ignoring this error at this location

KLEE: done: total instructions = 479
KLEE: done: completed paths = 19
KLEE: done: generated tests = 5
\end{lstlisting}

Мы можем игнорировать вызовы strcpy() и sprintf(), потому что нам не очень-то интересно знать состояние переменной \TT{out}.

Так что здесь именно 5 путей:

\begin{lstlisting}
% ls klee-last
assembly.ll   run.stats            test000003.ktest     test000005.ktest
info          test000001.ktest     test000003.pc        test000005.pc
messages.txt  test000002.ktest     test000004.ktest     warnings.txt
run.istats    test000003.exec.err  test000005.exec.err
\end{lstlisting}

Первый набор входных значений приводит к строке ``red'':

\begin{lstlisting}
% ktest-tool --write-ints klee-last/test000001.ktest
ktest file : 'klee-last/test000001.ktest'
args       : ['color.bc']
num objects: 3
object    0: name: b'R'
object    0: size: 1
object    0: data: b'\xff'
object    1: name: b'G'
object    1: size: 1
object    1: data: b'\x00'
object    2: name: b'B'
object    2: size: 1
object    2: data: b'\x00'
\end{lstlisting}

Второй набор --- к строке ``green'':

\begin{lstlisting}
% ktest-tool --write-ints klee-last/test000002.ktest
ktest file : 'klee-last/test000002.ktest'
args       : ['color.bc']
num objects: 3
object    0: name: b'R'
object    0: size: 1
object    0: data: b'\x00'
object    1: name: b'G'
object    1: size: 1
object    1: data: b'\xff'
object    2: name: b'B'
object    2: size: 1
object    2: data: b'\x00'
\end{lstlisting}

Третий набор --- строка ``\#010000'':

\begin{lstlisting}
% ktest-tool --write-ints klee-last/test000003.ktest
ktest file : 'klee-last/test000003.ktest'
args       : ['color.bc']
num objects: 3
object    0: name: b'R'
object    0: size: 1
object    0: data: b'\x01'
object    1: name: b'G'
object    1: size: 1
object    1: data: b'\x00'
object    2: name: b'B'
object    2: size: 1
object    2: data: b'\x00'
\end{lstlisting}

Четвертый набор: строка ``blue'':

\begin{lstlisting}
% ktest-tool --write-ints klee-last/test000004.ktest
ktest file : 'klee-last/test000004.ktest'
args       : ['color.bc']
num objects: 3
object    0: name: b'R'
object    0: size: 1
object    0: data: b'\x00'
object    1: name: b'G'
object    1: size: 1
object    1: data: b'\x00'
object    2: name: b'B'
object    2: size: 1
object    2: data: b'\xff'
\end{lstlisting}

Пятый набор: строка ``\#F01'':

\begin{lstlisting}
% ktest-tool --write-ints klee-last/test000005.ktest
ktest file : 'klee-last/test000005.ktest'
args       : ['color.bc']
num objects: 3
object    0: name: b'R'
object    0: size: 1
object    0: data: b'\xff'
object    1: name: b'G'
object    1: size: 1
object    1: data: b'\x00'
object    2: name: b'B'
object    2: size: 1
object    2: data: b'\x11'
\end{lstlisting}

Эти 5 наборов входных значений могут послушить unit-тестом нашей ф-ции.

